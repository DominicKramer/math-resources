\documentclass{article}
\usepackage[T1]{fontenc}
\usepackage{textcomp}
\renewcommand{\rmdefault}{ptm}
\usepackage[scaled=0.92]{helvet}
\usepackage[psamsfonts]{amsfonts}
\usepackage{amsmath, amsbsy,verbatim}
\usepackage[dvips, bookmarks, colorlinks=true, plainpages = false,
  citecolor = blue, urlcolor = blue, filecolor = blue]{hyperref}
\newtheorem{corollary}{Corollary}
\newtheorem{definition}{Definition}
\newtheorem{lemma}{Lemma}
\newtheorem{theorem}{Theorem}
\newtheorem{example}{Example}
\newcommand{\proof}{\noindent{\sc\bf Proof}\quad }
\def\endproof{{\hfill \vbox{\hrule\hbox{%
 \vrule height1.3ex\hskip0.8ex\vrule}\hrule }}\par}
\newcommand{\bbox}{\phantom{1}\hfill{\rule{6pt}{6pt}}}
\newcommand{\pd}[2]{\frac{\partial{#1}}{\partial{#2}}}
\newcommand{\dst}{\displaystyle}
\newcommand{\place}{\bigskip\hrule\bigskip\noindent}
\newcommand{\set}[2]{\left\{#1\, \big|\, #2\right\}}

\newcommand{\exer}[1]{\par\noindent{\bf $#1$}.}
\newcommand{\boxit}[1]{\bigskip\noindent{\bf
#1}\\\vskip-6pt\hskip-\parindent}

\newcounter{lcal}
\newenvironment{alist}{\begin{list}{\bf (\alph{lcal})}
{\topsep 0pt\partopsep 0pt\labelwidth 14pt
\labelsep 8pt\leftmargin 22pt\itemsep 0pt
\usecounter{lcal}}}{\end{list}}

\newcounter{exercise}
\newenvironment{exerciselist}{\begin{list}{\bf \arabic{exercise}.}
{\topsep 10pt\partopsep 0pt\labelwidth 16pt
\labelsep 12pt\leftmargin 28pt
\itemsep 8pt\usecounter{exercise}}}{\end{list}}
\begin{document}

\thispagestyle{empty}
\bf
\begin{center}
{\Huge FUNCTIONS DEFINED BY\\ \medskip IMPROPER INTEGRALS}
\vspace{.5in}
\huge
\bigskip
\vspace{.75in}
\bf\huge
\href{http://ramanujan.math.trinity.edu/wtrench/index.shtml}
{William F. Trench}
\medskip
\\\large
Andrew G. Cowles Distinguished Professor Emeritus\\
Department of Mathematics\\
Trinity University \\
San Antonio, Texas, USA\\
\href{mailto:{wtrench@trinity.edu}}
{wtrench@trinity.edu}
\large
\vspace*{.75in}
\end{center}


\rm
\noindent
This is a supplement to the author's
\href{http://ramanujan.math.trinity.edu/wtrench/texts/TRENCH_REAL_ANALYSIS.PDF}
{\large Introduction to Real Analysis}.
It   has been judged to meet the evaluation criteria set by the
Editorial  Board
of the American Institute of Mathematics in connection with the Institute's
\href{http://www.aimath.org/textbooks/}
{Open
Textbook Initiative}.
It may be copied, modified, redistributed, translated,  and
built upon  subject to the Creative
Commons
      \href{http://creativecommons.org/licenses/by-nc-sa/3.0/deed.en_G}
{Attribution-NonCommercial-ShareAlike 3.0 Unported License}.
A complete instructor's solution manual is available by email to
\href{mailto:wtrench@trinity.edu}
{wtrench@trinity.edu},
 subject to verification of the requestor's
faculty status.



\newpage
\rm
\section{Foreword} \label{section:foreword}
This is a revised version of Section~7.5 of my \emph{Advanced Calculus}
(Harper \& Row, 1978).
It is a supplement to my textbook
\href{http://ramanujan.math.trinity.edu/wtrench/texts/TRENCH_REAL_ANALYSIS.PDF}
{\emph{Introduction to Real Analysis}}, which
is referenced several times here.
You
should review   Section~3.4  (Improper Integrals) of that book before
reading this document.

\section{Introduction}\label{section:introduction}
  In  Section~7.2 (pp. 462--484)
we considered functions of the form
$$
F(y)=\int_{a}^{b}f(x,y)\,dx, \quad c \le y \le d.
$$
We saw   that if $f$  is continuous on
$[a,b]\times
[c,d]$,  then
$F$  is continuous on $[c,d]$ (Exercise~7.2.3, p.~481) and that
we can reverse the order of integration  in
$$
\int_{c}^{d}F(y)\,dy=\int_{c}^{d}\left(\int_{a}^{b}f(x,y)\,dx\right)\,dy
$$
to evaluate it as
$$
\int_{c}^{d}F(y)\,dy=\int_{a}^{b}\left(\int_{c}^{d}f(x,y)\,dy\right)\,dx
$$
(Corollary~7.2.3, p.~466).

Here is  another important property of $F$.

\begin{theorem}  \label{theorem:1}
If $f$  and $f_{y}$ are continuous on  $[a,b]\times [c,d],$
then
\begin{equation} \label{eq:1}
F(y)=\int_{a}^{b}f(x,y)\,dx, \quad c \le y \le d,
\end{equation}
is continuously differentiable on $[c,d]$ and $F'(y)$ can be obtained
by differentiating \eqref{eq:1}
under the
integral sign with respect to $y;$ that is,
\begin{equation} \label{eq:2}
F'(y)=\int_{a}^{b}f_{y}(x,y)\,dx, \quad c \le y \le d.
\end{equation}
Here $F'(a)$ and $f_{y}(x,a)$ are  derivatives
from the right and $F'(b)$ and $f_{y}(x,b)$  are
derivatives from the left$.$
\end{theorem}

\proof
If $y$  and $y+\Delta y$  are in $[c,d]$ and $\Delta y\ne0$, then
\begin{equation} \label{eq:3}
\frac{F(y+\Delta y)-F(y)}{\Delta y}=
\int_{a}^{b}\frac{f(x,y+\Delta y)-f(x,y)}{\Delta y}\,dx.
\end{equation}
From the mean value theorem (Theorem~2.3.11, p.~83), if
$x\in[a,b]$ and $y$, $y+\Delta y\in[c,d]$, there is a
$y(x)$ between  $y$  and $y+\Delta y$ such that
$$
f(x,y+\Delta y)-f(x,y)=f_{y}(x,y)\Delta y=
f_{y}(x,y(x))\Delta y+(f_{y}(x,y(x)-f_{y}(x,y))\Delta y.
$$
From this and \eqref{eq:3},
\begin{equation} \label{eq:4}
\left|\frac{F(y+\Delta y)-F(y)}{\Delta
y}-\int_{a}^{b}f_{y}(x,y)\,dx\right|
\le \int_{a}^{b} |f_{y}(x,y(x))-f_{y}(x,y)|\,dx.
\end{equation}
Now suppose $\epsilon>0$. Since $f_{y}$  is uniformly continuous
on the compact set  $[a,b]\times [c,d]$
(Corollary~5.2.14, p.~314) and $y(x)$ is between $y$ and $y+\Delta y$,
there is a $\delta>0$    such that  if $|\Delta|<\delta$ then
$$
|f_{y}(x,y)-f_{y}(x,y(x))|<\epsilon,\quad
(x,y)\in[a,b]\times [c,d].
$$
This and \eqref{eq:4} imply that
$$
\left|\frac{F(y+\Delta y-F(y))}{\Delta
y}-\int_{a}^{b}f_{y}(x,y)\,dx\right|<\epsilon(b-a)
$$
if $y$ and $y+\Delta y$ are in $[c,d]$ and $0<|\Delta y|<\delta$.
This implies \eqref{eq:2}. Since
the integral in \eqref{eq:2} is  continuous
on $[c,d]$ (Exercise~7.2.3, p.~481, with $f$ replaced by $f_{y}$),  $F'$
is continuous on
$[c,d]$.
 \endproof

\begin{example}  \label{example:1}  \rm
Since
$$
f(x,y)=\cos xy\text{\quad and\quad} f_{y}(x,y)=-x\sin xy
$$
are continuous for all $(x,y)$,
Theorem~\ref{theorem:1} implies that if
\begin{equation} \label{eq:5}
F(y)=\int_{0}^{\pi} \cos xy\,dx,\quad -\infty<y<\infty,
\end{equation}
then
\begin{equation} \label{eq:6}
F'(y)=-\int_{0}^{\pi}x\sin xy\,dx,\quad -\infty<y<\infty.
\end{equation}
(In applying Theorem~\ref{theorem:1} for a specific value
of $y$, we take
$R=[0,\pi]\times [-\rho,\rho]$, where $\rho>|y|$.)  This provides a
convenient way to evaluate the integral in \eqref{eq:6}:
integrating the right side of  \eqref{eq:5} with respect to $x$ yields
$$
F(y)=\frac{\sin xy}{y}\bigg|_{x=0}^{\pi}=\frac{\sin\pi y}{y}, \quad
y\ne0.
$$
Differentiating this and using \eqref{eq:6}  yields
$$
\int_{0}^{\pi}x\sin xy\,dx =\frac{\sin \pi y}{y^{2}}-
\frac{\pi\cos \pi y}{y}, \quad y\ne0.
$$
To verify this, use integration by parts. \bbox
\end{example}



We will study the continuity,
differentiability, and integrability of
$$
F(y)=\int_{a}^{b}f(x,y)\,dx,\quad  y\in S,
$$
where $S$ is an interval or a union of intervals,
and $F$ is a convergent improper integral for each $y\in S$.
If the domain of $f$ is $[a,b)\times S$  where
$-\infty<a< b\le \infty$,
we say that $F$ is \emph{pointwise  convergent on $S$} or simply
\emph{convergent on $S$},  and write
\begin{equation} \label{eq:7}
\int_{a}^{b}f(x,y)\,dx=\lim_{r\to b-}\int_{a}^{r}f(x,y)\,dx
\end{equation}
  if,
 for each $y\in S$  and every
$\epsilon>0$,  there is an $r=r_{0}(y)$ (which also depends on $\epsilon$)
such that
\begin{equation} \label{eq:8}
\left|F(y)-\int_{a}^{r}f(x,y)\,dx\right|=
\left|\int_{r}^{b}f(x,y)\,dx\right|< \epsilon,
\quad  r_{0}(y)\le y<b.
\end{equation}
If the domain of $f$  is $(a,b]\times S$ where $-\infty\le a<b<\infty$,
  we replace  \eqref{eq:7} by
$$
\int_{a}^{b}f(x,y)\,dx=\lim_{r\to a+}\int_{r}^{b}f(x,y)\,dx
$$
and \eqref{eq:8} by
$$
\left|F(y)-\int_{r}^{b}f(x,y)\,dx\right|=
\left|\int_{a}^{r}f(x,y)\,dx\right|< \epsilon,
\quad a<r\le  r_{0}(y).
$$




In general, pointwise convergence of $F$ for all $y\in S$ does not imply
that $F$ is continuous or integrable on $[c,d]$, and the additional
assumptions that $f_{y}$ is continuous and $\int_{a}^{b}f_{y}(x,y)\,dx$
converges do not  imply \eqref{eq:2}.


\begin{example}  \label{example:2}  \rm
The function
$$
f(x,y)=ye^{-|y|x}
$$
is continuous on $[0,\infty)\times (-\infty,\infty)$ and
$$
F(y)=\int_{0}^{\infty}f(x,y)\,dx =\int_{0}^{\infty}ye^{-|y|x}\,dx
$$
converges for all $y$, with
$$
F(y)=
\begin{cases}
-1& y<0,\\
\phantom{-}0&y=0,\\
\phantom{-}1&y>0;\\
\end{cases}
$$
therefore, $F$ is discontinuous at $y=0$.
\end{example}

\begin{example}  \label{example:3}  \rm
The function
$$
f(x,y)=y^{3}e^{-y^{2}x}
$$
is continuous on $[0,\infty)\times (-\infty,\infty)$.
Let
$$
F(y)=\int_{0}^{\infty}f(x,y)\,dx=
\int_{0}^{\infty}y^{3}e^{-y^{2}x}\,dx  =y,\quad -\infty<y<\infty.
$$
 Then
$$
F'(y)=1, \quad -\infty<y<\infty.
$$
However,
$$
\int_{0}^{\infty}\pd{}{y}(y^{3}e^{-y^{2}x})\,dx
=\int_{0}^{\infty}(3y^{2}-2y^{4}x)e^{-y^{2}x}\,dx=
\begin{cases}
1,& y\ne0,\\
0,& y=0,
\end{cases}
$$
so
$$
F'(y)\ne\int_{0}^{\infty}\pd{f(x,y)}{y}\,dx\text{\quad if\quad}y=0.
$$
\end{example}

\section{Preparation} \label{section:preparation}


 We begin with two useful convergence criteria for improper integrals
that do not involve a parameter.
Consistent with the definition on p.~152, we say that $f$ is locally
integrable on
an interval $I$ if it is integrable on every finite closed subinterval of
$I$.

\begin{theorem}[
\href{http://www-history.mcs.st-and.ac.uk/Biographies/Cauchy.html}
{Cauchy}
Criterion for Convergence of an Improper
Integral I]
\label{theorem:2}
Suppose  $g$ is
locally integrable on  $[a,b)$ and denote
$$
G(r)=\int_{a}^{r}g(x)\,dx,\quad  a\le r<b.
$$
Then the improper integral $\int_{a}^{b}g(x)\,dx$ converges if and only
if$,$ for each
$\epsilon >0,$ there is an $r_{0}\in[a,b)$  such that
\begin{equation} \label{eq:9}
|G(r)-G(r_{1})|<\epsilon,\quad  r_{0}\le r,r_{1}<b.
\end{equation}
\end{theorem}

 \proof For necessity, suppose $\int_{a}^{b}g(x)\,dx=L$. By definition,
this  means that for each $\epsilon>0$ there is an $r_{0}\in [a,b)$
such that
$$
|G(r)-L|<\frac{\epsilon}{2}
\text{\quad and\quad}
|G(r_{1})-L|<\frac{\epsilon}{2},\quad
r_{0}\le r,r_{1}<b.
$$
Therefore
\begin{eqnarray*}
|G(r)-G(r_{1})|&=&|(G(r)-L)-(G(r_{1})-L)|\\
&\le& |G(r)-L|+|G(r_{1})-L|<
\epsilon,\quad  r_{0}\le r,r_{1}<b.
\end{eqnarray*}

For sufficiency, \eqref{eq:9} implies that
$$
|G(r)|= |G(r_{1})+(G(r)-G(r_{1}))|< |G(r_{1})|+|G(r)-G(r_{1})|\le
|G(r_{1})|+\epsilon,
$$
  $r_{0}\le r\le r_{1}<b$. Since $G$  is also bounded on the
compact set
$[a,r_{0}]$ (Theorem~5.2.11, p.~313),  $G$ is bounded on $[a,b)$. Therefore
the monotonic functions
$$
\overline{G}(r)=\sup\set{G(r_{1})}{r\le r_{1}<b} \text{\quad and\quad}
\underline{G}(r)=\inf\set{G(r_{1})}{r\le r_{1}<b}
$$
are well defined on $[a,b)$, and
$$
\lim_{r\to b-}\overline{G}(r)=\overline{L}
\text{\quad and\quad}
\lim_{r\to b-}\underline{G}(r)=\underline{L}
$$
both exist  and are finite (Theorem~2.1.11, p.~47).
From  \eqref{eq:9},
\begin{eqnarray*}
|G(r)-G(r_{1})|&=&|(G(r)-G(r_{0}))-(G(r_{1})-G(r_{0}))|\\
&\le &|G(r)-G(r_{0})|+|G(r_{1})-G(r_{0})|< 2\epsilon,
\end{eqnarray*}
so
$$
\overline{G}(r)-\underline{G}(r)\le 2\epsilon, \quad  r_{0}\le r, r_{1}<b.
$$
 Since
$\epsilon$  is an arbitrary positive number, this implies that
$$
\lim_{r\to b-}(\overline{G}(r)-\underline{G}(r))=0,
$$
so
$\overline{L}=\underline{L}$. Let $L=\overline{L}=\underline{L}$.
Since
$$
\underline{G}(r)\le G(r)\le \overline{G}(r),
$$
it follows that $\lim_{r\to b-} G(r)=L$. \endproof

We leave the proof of the following theorem to you
(Exercise~\ref{exer:2}).




\begin{theorem}[Cauchy Criterion for Convergence of an Improper
Integral II]
\label{theorem:3}
Suppose  $g$ is
locally integrable on  $(a,b]$ and denote
$$
G(r)=\int_{r}^{b}g(x)\,dx,\quad   a\le r<b.
$$
Then the improper integral $\int_{a}^{b}g(x)\,dx$ converges if and only
if$,$ for each
$\epsilon >0,$ there is an $r_{0}\in(a,b]$  such that
$$
|G(r)-G(r_{1})|<\epsilon,\quad  a<r,r_{1}\le r_{0}.
$$
\end{theorem}


To see why we associate Theorems~\ref{theorem:2} and~\ref{theorem:3} with
Cauchy, compare them with Theorem~4.3.5 (p.~204)

\section{Uniform convergence of improper integrals}\label{section:uniform}
\medskip

Henceforth we deal with functions $f=f(x,y)$  with domains
 $I\times S$, where $S$ is an interval or a union of intervals and   $I$ is
of one of the following forms:
\begin{itemize}
\item  $[a,b)$ with $-\infty<a<b\le \infty$;
\item $(a,b]$ with $-\infty\le a<b< \infty$;
\item   $(a,b)$ with $-\infty\le a\le b\le \infty$.
\end{itemize}
In all cases it is to be understood that $f$  is locally integrable with
respect to $x$ on  $I$.
 When we say that the
improper integral $\int_{a}^{b}f(x,y)\,dx$  has a stated property ``on
S'' we mean that it has the property for every $y\in S$.

\begin{definition}  \label{definition:1}
If the improper integral
\begin{equation} \label{eq:10}
\int_{a}^{b}f(x,y)\,dx=\lim_{r\to b-}\int_{a}^{r}f(x,y)\,dx
\end{equation}
converges on $S,$   it
 is said to converge
uniformly $($or be uniformly convergent$)$  on $S$ if$,$  for each
$\epsilon>0,$ there is an
$r_{0} \in [a,b)$
  such that
$$
\left|\int_{a}^{b}f(x,y)\,dx-\int_{a}^{r}f(x,y)\,dx\right|
< \epsilon,\quad  y\in S, \quad  r_{0}\le r<b,
$$
or$,$ equivalently$,$
\begin{equation} \label{eq:11}
\left|\int_{r}^{b}f(x,y)\,dx\right|< \epsilon, \quad
y\in S,\quad    r_{0}\le r<b.
\end{equation}
\end{definition}



The crucial difference between pointwise and uniform convergence is that
$r_{0}(y)$  in \eqref{eq:8} may depend upon the particular value of $y$,
while the
$r_{0}$ in \eqref{eq:11} does not: one choice must work for all $y\in S$.
Thus, uniform convergence
implies pointwise convergence, but pointwise convergence does not imply
uniform convergence.

\begin{theorem}{\bf$($Cauchy Criterion for  Uniform Convergence I$)$}
\label{theorem:4}
The improper integral in \eqref{eq:10}
converges uniformly on $S$ if and only if$,$ for each $\epsilon>0,$ there
is an
$r_{0} \in [a,b)$ such that
\begin{equation} \label{eq:12}
\left|\int_{r}^{r_{1}}f(x,y)\,dx\right|< \epsilon, \quad  y\in S,\quad
r_{0}\le  r,r_{1}<b.
\end{equation}
\end{theorem}

\proof  Suppose  $\int_{a}^{b} f(x,y)\,dx$ converges uniformly on
$S$ and $\epsilon>0$.
From Definition~\ref{definition:1},
there is an
$r_{0}\in [a,b)$  such that
\begin{equation} \label{eq:13}
\left|\int_{r}^{b}f(x,y)\,dx\right| <\frac{\epsilon}{2}
\text{\, and\,}
\left|\int_{r_{1}}^{b}f(x,y)\,dx\right|
<\frac{\epsilon}{2} ,\quad  y\in S, \quad r_{0}\le r,r_{1}<b.
\end{equation}
 Since
$$
\int_{r}^{r_{1}}f(x,y)\,dx=
\int_{r}^{b}f(x,y)\,dx-
\int_{r_{1}}^{b}f(x,y)\,dx,
$$
\eqref{eq:13} and the triangle inequality imply
\eqref{eq:12}.

For the converse, denote
$$
F(y)=\int_{a}^{r}f(x,y)\,dx.
$$
Since \eqref{eq:12} implies that
\begin{equation} \label{eq:14}
|F(r,y)-F(r_{1},y)|< \epsilon,\quad y\in S, \quad
r_{0}\le r, r_{1}<b,
\end{equation}
 Theorem~\ref{theorem:2} with $G(r)=F(r,y)$ ($y$ fixed
but arbitrary in $S$) implies that  $\int_{a}^{b} f(x,y)\,dx$
converges  pointwise for   $y\in S$.
 Therefore, if  $\epsilon>0$
then, for each $y\in S$,
there is an $r_{0}(y) \in [a,b)$ such that
\begin{equation} \label{eq:15}
\left|\int_{r}^{b}f(x,y)\,dx\right|< \epsilon, \quad  y\in S,\quad
r_{0}(y)\le r< b.
\end{equation}
For each $y\in S$, choose $r_{1}(y)\ge \max[{r_{0}(y),r_{0}}]$. (Recall
\eqref{eq:14}).  Then
$$
\int_{r}^{b}f(x,y)\,dx =
\int_{r}^{r_{1}(y)}f(x,y)\,dx+
\int_{r_{1}(y)}^{b}f(x,y)\,dx, \quad
$$
so \eqref{eq:12}, \eqref{eq:15}, and the triangle inequality imply
that
$$
\left|\int_{r}^{b} f(x,y)\,dx\right|< 2\epsilon, \quad  y\in S, \quad
r_{0}\le r<b.
$$
\endproof

In practice, we don't  explicitly exhibit $r_{0}$ for each  given
$\epsilon$.
It suffices to obtain estimates that clearly imply its existence.



\begin{example}  \label{example:4}  \rm
For the improper integral of Example~\ref{example:2},
$$
\left|\int_{r}^{\infty}f(x,y)\,dx\right|=
\int_{r}^{\infty} |y|e^{-|y|x}=e^{-r|y|}, \quad y\ne0.
$$
If $|y| \ge \rho$, then
$$
\left|\int_{r}^{\infty}f(x,y)\,dx\right| \le e^{-r\rho},
$$
so $\int_{0}^{\infty}f(x,y)\,dx$  converges uniformly on
$(-\infty,\rho]\cup[\rho,\infty)$ if $\rho>0$; however, it does not
converge uniformly  on any neighborhood of $y=0$, since, for any
$r>0$,
$e^{-r|y|}>\frac{1}{2}$ if $|y|$ is sufficiently small.
\end{example}



\begin{definition}  \label{definition:2}
If the  improper integral
$$
\int_{a}^{b}f(x,y)\,dx=\lim_{r\to a+}\int_{r}^{b}f(x,y)\,dx
$$
converges on $S,$ it
 is said to converge
uniformly $($or be uniformly convergent$)$  on $S$ if$,$  for each
$\epsilon>0,$ there is an
$r_{0} \in (a,b]$
  such that
$$
\left|\int_{a}^{b}f(x,y)\,dx-\int_{r}^{b}f(x,y)\,dx\right|
<\epsilon, \quad  y\in S,\quad
a<r\le  r_{0},
$$
or$,$ equivalently$,$
$$
\left|\int_{a}^{r} f(x,y)\,dx\right|< \epsilon, \quad  y\in S,\quad
  a<r\le  r_{0}.
$$
\end{definition}

We leave proof of the following theorem to you (Exercise~\ref{exer:3}).


\begin{theorem}{\bf $($Cauchy Criterion for Uniform Convergence II$)$}
\label{theorem:5}
The improper integral
$$
\int_{a}^{b}f(x,y)\,dx =\lim_{r\to a+}\int_{r}^{b}f(x,y)\,dx
$$
converges uniformly on $S$ if and only if$,$
 for each $\epsilon>0,$ there is
an  $r_{0}\in (a,b]$ such that
$$
\left|\int_{r_{1}}^{r}f(x,y)\,dx\right|< \epsilon,\quad
y\in S,\quad  a <r,r_{1}\le r_{0}.
$$
\end{theorem}

We need one more definition, as follows.

\begin{definition}  \label{definition:3}
Let $f=f(x,y)$ be defined on $(a,b) \times S,$ where $-\infty\le a<b\le
\infty.$ Suppose  $f$ is locally  integrable on
$(a,b)$ for all $y\in S$ and let
 $c$ be an arbitrary point in $(a,b).$
Then
$\int_{a}^{b}f(x,y)\,dx$ is said to converge
uniformly on  $S$ if  $\int_{a}^{c}f(x,y)\,dx$  and
$\int_{c}^{b}f(x,y)\,dx$ both converge uniformly on $S.$
\end{definition}


We leave it to you
(Exercise~\ref{exer:4}) to show that this definition is independent
 of $c$; that is,  if
  $\int_{a}^{c}f(x,y)\,dx$  and
$\int_{c}^{b}f(x,y)\,dx$ both converge uniformly  on $S$ for
some
$c\in(a,b)$, then they both converge uniformly on $S$ for every
$c \in (a,b)$.

We also leave it  you (Exercise~\ref{exer:5}) to show that if
$f$  is bounded
on $[a,b]\times [c,d]$ and  $\int_{a}^{b}f(x,y)\,dx$
exists as a proper integral for each $y\in [c,d]$, then it converges
uniformly  on $[c,d]$ according to all three
Definitions~\ref{definition:1}--\ref{definition:3}.


\begin{example}  \label{example:5}  \rm
Consider the improper integral
$$
F(y)=\int_{0}^{\infty}x^{-1/2}e^{-xy}\,dx,
$$
which diverges
if $y\le 0$ (verify).   Definition~\ref{definition:3}
applies if $y>0$, so we consider the improper
integrals
$$
F_{1}(y)=\int_{0}^{1}x^{-1/2}e^{-xy}\,dx
\text{\quad and\quad}
F_{2}(y)=\int_{1}^{\infty}x^{-1/2}e^{-xy}\,dx
$$
separately. Moreover, we could just as well   define
\begin{equation}\label{eq:16}
F_{1}(y)=\int_{0}^{c}x^{-1/2}e^{-xy}\,dx
\text{\quad and\quad}
F_{2}(y)=\int_{c}^{\infty}x^{-1/2}e^{-xy}\,dx,
\end{equation}
where $c$  is any positive number.


 Definition~\ref{definition:2} applies to $F_{1}$.
If $0<r_{1}<r$  and $y\ge 0$, then
$$
\left|\int_{r}^{r_{1}}x^{-1/2}e^{-xy}\,dx\right| <
\int_{r_{1}}^{r}x^{-1/2}\,dx<2r^{1/2},
$$
so $F_{1}(y)$  converges for uniformly on $[0,\infty)$.

Definition~\ref{definition:1} applies to $F_{2}$. Since
$$
\left|\int_{r}^{r_{1}}x^{-1/2}e^{-xy}\,dx\right| < r^{-1/2}
\int_{r}^{\infty}e^{-xy}\,dx = \frac{e^{-ry}}{yr^{1/2}},
$$
$F_{2}(y)$ converges uniformly on   $[\rho,\infty)$ if
 $\rho>0$. It does not converge uniformly on
$(0,\rho)$, since the change of variable $u=xy$ yields
$$
\int_{r}^{r_{1}}x^{-1/2}e^{-xy}\,dx=y^{-1/2}
\int_{ry}^{r_{1}y}u^{-1/2}e^{-u}\,du,
$$
which, for any fixed $r>0$, can be made arbitrarily large
by taking $y$ sufficiently small and $r=1/y$. Therefore we
conclude that $F(y)$ converges uniformly on $[\rho,\infty)$
if $\rho>0.$

Note that  that the constant $c$ in \eqref{eq:16} plays no role in this
argument.
\end{example}



\begin{example}  \label{example:6}  \rm
Suppose we take
\begin{equation} \label{eq:17}
\int_{0}^{\infty}\frac{\sin u}{u}\,du =\frac{\pi}{2}
\end{equation}
as given (Exercise~\ref{exer:31}{\bf(b)}).  Substituting $u=xy$ with $y>0$
yields
\begin{equation} \label{eq:18}
\int_{0}^{\infty}\frac{\sin xy}{x}\,dx=\frac{\pi}{2},\quad y>0.
\end{equation}
What about uniform convergence?
Since $(\sin xy)/x$ is continuous at $x=0$, Definition~\ref{definition:1}
and Theorem~\ref{theorem:4} apply here.
If $0<r<r_{1}$ and $y>0$, then
$$
\int_{r}^{r_{1}}\frac{\sin xy}{x}\,dx=-\frac{1}{y}
\left(\frac{\cos xy}{x}\biggr|_{r}^{r_{1}}+
\int_{r}^{r_{1}}\frac{\cos xy}{x^{2}}\,dx\right),
\text{\, so\quad}
\left|\int_{r}^{r_{1}}\frac{\sin xy}{x}\,dx\right|<\frac{3}{ry}.
$$
Therefore \eqref{eq:18} converges uniformly on
$[\rho,\infty)$ if $\rho>0$. On the other hand, from \eqref{eq:17},
there is a $\delta>0$  such that
$$
\int_{u_{0}}^{\infty}\frac{\sin u}{u}\,du>\frac{\pi}{4}, \quad
0 \le u_{0}<\delta.
$$
This and \eqref{eq:18} imply that
$$
\int_{r}^{\infty}\frac{\sin xy}{x}\,dx=\int_{yr}^{\infty}\frac{\sin
u}{u}\,du
>\frac{\pi}{4}
$$
for any $r>0$ if   $0 <y<\delta/r$. Hence, \eqref{eq:18}
does not converge uniformly  on any interval $(0,\rho]$  with $\rho>0$.
\end{example}

\section{ Absolutely Uniformly Convergent Improper
Integrals}\label{section:absolutely}

\begin{definition}{\bf$($Absolute Uniform Convergence I$)$}
\label{definition:4}
The improper integral
$$
\int_{a}^{b}f(x,y)\,dx=\lim_{r\to b-}\int_{a}^{r}f(x,y)\,dx
$$
is said to converge absolutely  uniformly on $S$ if the improper
integral
$$
\int_{a}^{b}|f(x,y)|\,dx=\lim_{r\to b-}\int_{a}^{r}|f(x,y)|\,dx
$$
converges uniformly on $S$; that is,
 if, for each $\epsilon>0$,
there is an $r_{0}\in [a,b)$  such that
$$
\left|\int_{a}^{b}|f(x,y)|\,dx-\int_{a}^{r}|f(x,y)|\,dx\right|
<\epsilon, \quad y\in S,\quad
 r_{0}<r<b.
$$
\end{definition}

To see that this definition makes sense, recall that if $f$ is
locally integrable on $[a,b)$ for all $y$ in $S$, then so is $|f|$
(Theorem~3.4.9, p.~161).
Theorem~\ref{theorem:4}  with $f$  replaced by $|f|$ implies that
$\int_{a}^{b}f(x,y)\,dx$ converges absolutely uniformly  on
$S$ if and only if, for each
$\epsilon>0$, there is an $r_{0}\in [a,b)$  such that
$$
\int_{r}^{r_{1}}|f(x,y)|\,dx<\epsilon,\quad  y\in S,\quad
r_{0}\le r<r_{1}<b .
$$
Since
$$
\left|\int_{r}^{r_{1}}f(x,y)\,dx\right| \le
\int_{r}^{r_{1}}|f(x,y)|\,dx,
$$
Theorem~\ref{theorem:4} implies that if $\int_{a}^{b}f(x,y)\,dx$
 converges absolutely uniformly on  $S$ then it converges
 uniformly   on $S$.

\begin{theorem}  \label{theorem:6}
{\bf$($
\href{http://www-history.mcs.st-and.ac.uk/Biographies/Weierstrass.html}
{Weierstrass}'s
Test for  Absolute Uniform Convergence I$)$}
Suppose $M=M(x)$ is nonnegative on $[a,b),$
$\int_{a}^{b}M(x)\,dx<\infty,$ and
\begin{equation} \label{eq:19}
|f(x,y)| \le M(x), \quad  y\in S,\quad
a\le x<b.
\end{equation}
  Then $\int_{a}^{b}f(x,y)\,dx$
converges absolutely uniformly on $S.$
\end{theorem}

\proof
Denote $\int_{a}^{b}M(x)\,dx=L<\infty$. By  definition,
for each $\epsilon>0$ there is an $r_{0}\in [a,b)$  such that
$$
L-\epsilon < \int_{a}^{r}M(x)\,dx \le L,\quad
r_{0}<r<b.
$$
Therefore, if $r_{0}< r\le  r_{1},$  then
$$
0\le \int_{r}^{r_{1}}M(x)\,dx=\left(\int_{a}^{r_{1}}M(x)\,dx
-L\right)-
\left(\int_{a}^{r}M(x)\,dx -L\right)<\epsilon
$$
   This and \eqref{eq:19} imply that
$$
\int_{r}^{r_{1}}|f(x,y)|\,dx\le
\int_{r}^{r_{1}} M(x)\,dx <\epsilon,\quad  y\in S, \quad
a\le  r_{0}<r<r_{1}<b.
$$
Now Theorem~\ref{theorem:4}
implies the stated conclusion.  \endproof


\begin{example}  \label{example:7}  \rm
Suppose $g=g(x,y)$ is locally integrable  on
$[0,\infty)$ for all $y\in S$
and, for some $a_{0}\ge 0$, there are  constants $K$ and  $p_{0}$ such that
$$
|g(x,y)| \le Ke^{p_{0}x},\quad y\in S, \quad x\ge a_{0}.
$$
If $p>p_{0}$ and $r\ge a_{0}$,  then
\begin{eqnarray*}
\int_{r}^{\infty}e^{-px} |g(x,y)|\,dx &=&
\int_{r}^{\infty} e^{-(p-p_{0})x}e^{-p_{0}x}|g(x,y)|\,dx\\
&\le& K\int_{r}^{\infty} e^{-(p-p_{0})x}\,dx= \frac{K
e^{-(p-p_{0})r}}{p-p_{0}},
\end{eqnarray*}
so
$\int_{0}^{\infty}e^{-px} g(x,y)\,dx $
converges absolutely on  $S$.
For example, since
$$
|x^{\alpha}\sin xy|<e^{p_{0}x}\text{\quad and \quad}
|x^{\alpha}\cos xy|<e^{p_{0}x}
$$
for $x$  sufficiently large if $p_{0}>0$, Theorem~\ref{theorem:4}
 implies that
$\int_{0}^{\infty}e^{-px}x^{\alpha}\sin xy\,dx$
and
$\int_{0}^{\infty}e^{-px}x^{\alpha}\cos xy\,dx$
converge absolutely uniformly on $(-\infty,\infty)$ if $p>0$
and $\alpha~\ge~0$.  As a matter of fact, $\int_{0}^{\infty}e^{-px}x^{\alpha}\sin xy\,dx$
converges absolutely on $(-\infty,\infty)$ if $p>0$  and
$\alpha>-1$. (Why?)
\end{example}




\begin{definition}{\bf$($Absolute Uniform Convergence II$)$}
\label{definition:5}
The improper integral
$$
\int_{a}^{b}f(x,y)\,dx=\lim_{r\to a+}\int_{r}^{b}f(x,y)\,dx
$$
is said to converge absolutely  uniformly on $S$ if the improper
integral
$$
\int_{a}^{b}|f(x,y)|\,dx=\lim_{r\to a+}\int_{r}^{b}|f(x,y)|\,dx
$$
converges uniformly on $S$; that is,
 if, for each $\epsilon>0$,
there is an $r_{0}\in (a,b]$  such that
$$
\left|\int_{a}^{b}|f(x,y)|\,dx-\int_{r}^{b}|f(x,y)|\,dx\right|
<\epsilon, \quad y\in S, \quad   a<r<r_{0}\le b.
$$
\end{definition}



We leave it to you (Exercise~\ref{exer:7}) to prove the following theorem.


\begin{theorem}  \label{theorem:7}
{\bf$($Weierstrass's Test for Absolute Uniform Convergence II$)$}
Suppose $M=M(x)$ is nonnegative on $(a,b],$ $\int_{a}^{b}M(x)\,dx<\infty,$
 and
$$
|f(x,y)| \le M(x), \quad  y\in S, \quad  x\in (a,b].
$$
Then $\int_{a}^{b}f(x,y)\,dx$    converges absolutely uniformly on $S$.
\end{theorem}



\begin{example}  \label{example:8}  \rm
If $g=g(x,y)$ is locally integrable on $(0,1]$ for all $y\in S$
and
$$
|g(x,y)| \le Ax^{-\beta}, \quad 0<x \le x_{0},
$$
for each $y \in S$, then
$$
\int_{0}^{1} x^{\alpha}g(x,y)\,dx
$$
converges absolutely uniformly on $S$  if $\alpha>\beta-1$. To
 see this, note that if $0<r< r_{1}\le x_{0}$, then
$$
\int_{r_{1}}^{r}x^{\alpha}|g(x,y)|\,dx \le A\int_{r_{1}}^{r}
x^{\alpha-\beta}\,dx=
\frac{Ax^{\alpha-\beta+1}}{\alpha-\beta+1}\biggr|_{r_{1}}^{r}<
\frac{Ar^{\alpha-\beta+1}}{\alpha-\beta+1}.
$$
Applying this with $\beta=0$ shows that
$$
F(y)=\int_{0}^{1} x^{\alpha}\cos xy\,dx
$$
converges absolutely uniformly on $(-\infty,\infty)$ if $\alpha>-1$
and
$$
G(y)=\int_{0}^{1}x^{\alpha}\sin xy \,dx
$$
converges absolutely uniformly on $(-\infty,\infty)$  if
$\alpha>-2$.
\end{example}


By recalling Theorem~4.4.15 (p.~246),
you can see why   we associate  Theorems~\ref{theorem:6}  and
\ref{theorem:7}
 with Weierstrass.

\section{Dirichlet's Tests} \label{section:dirichlet}
 Weierstrass's test is useful and important, but  it has a basic
shortcoming:
it applies only to absolutely uniformly convergent improper integrals.
 The next theorem applies in some cases
where $\int_{a}^{b}f(x,y)\,dx$   converges uniformly on  $S$,
but
$\int_{a}^{b}|f(x,y)|\,dx$  does not.



\begin{theorem}  \label{theorem:8}
$(${\bf
\href{http://www-history.mcs.st-and.ac.uk/Biographies/Dirichlet.html}
{Dirichlet}'s
Test for Uniform Convergence I}$)$
If $g,$  $g_{x},$ and $h$ are continuous on $[a,b)\times S,$  then
$$
\int_{a}^{b}g(x,y)h(x,y)\,dx
$$
converges uniformly on  $S$ if  the following
conditions are satisfied$:$
\begin{alist}
\item % a
$\dst{\lim_{x\to b-}\left\{\sup_{y\in S}|g(x,y)|\right\}=0};$
\item % b
There is a constant $M$ such that
$$
\sup_{y\in S}\left|\int_{a}^{x}h(u,y)\,du\right|< M, \quad
a\le x<b;
$$

\item % c
$\int_{a}^{b}|g_{x}(x,y)|\,dx$   converges uniformly on $S.$

\end{alist}
\end{theorem}

\proof
If
\begin{equation} \label{eq:20}
H(x,y)=\int_{a}^{x}h(u,y)\,du,
\end{equation}
then integration by parts yields
\begin{eqnarray}
\int_{r}^{r_{1}}g(x,y)h(x,y)\,dx&=&\int_{r}^{r_{1}}g(x,y)H_{x}(x,y)\,dx
\nonumber\\
&=&g(r_{1},y)H(r_{1},y)-g(r,y)H(r,y)\label{eq:21}\\
&&-\int_{r}^{r_{1}}g_{x}(x,y)H(x,y)\,dx.
\nonumber
\end{eqnarray}
Since assumption   {\bf(b)}  and \eqref{eq:20}  imply that
$|H(x,y)|\le M,$  $(x,y)\in (a,b]\times S$,
Eqn.~\eqref{eq:21}  implies that
\begin{equation} \label{eq:22}
\left|\int_{r}^{r_{1}}g(x,y)h(x,y)\,dx\right|<
M\left(2\sup_{x\ge
r}|g(x,y)|+\int_{r}^{r_{1}}|g_{x}(x,y)|\,dx\right)
\end{equation}
on $[r,r_{1}]\times S$.

Now suppose  $\epsilon>0$.    From assumption {\bf (a)}, there is an
$r_{0} \in [a,b)$ such that $|g(x,y)|<\epsilon$ on $S$ if
$r_{0}\le x <b$.
From assumption  {\bf(c)} and Theorem~\ref{theorem:6},  there is an
$s_{0}\in
[a,b)$ such that
$$
\int_{r}^{r_{1}}|g_{x}(x,y)|\,dx<\epsilon, \quad  y\in S, \quad
s_{0}<r<r_{1}<b.
$$
Therefore
\eqref{eq:22}  implies that
$$
\left|\int_{r}^{r_{1}}g(x,y)h(x,y)\right| < 3M\epsilon, \quad  y\in S, \quad
\max(r_{0},s_{0})<r<r_{1}<b.
$$
Now Theorem~\ref{theorem:4} implies the stated conclusion.
\endproof


The statement of this theorem is complicated, but applying it isn't;
just look for a factorization $f=gh$, where $h$ has a bounded
antderivative
  on $[a,b)$ and  $g$  is ``small'' near $b$. Then integrate by
parts and hope that something nice happens. A similar comment applies
to Theorem~9, which follows.




\begin{example}  \label{example:9}  \rm
Let
$$
I(y)=\int_{0}^{\infty}\frac{\cos xy}{x+y}\,dx,\quad y>0.
$$
The obvious inequality
$$
\left|\frac{\cos xy}{x+y}\right|\le \frac{1}{x+y}
$$
is useless here, since
$$
\int_{0}^{\infty}\frac{dx}{x+y}=\infty.
$$
However, integration by parts yields
\begin{eqnarray*}
\int_{r}^{r_{1}}\frac{\cos xy}{x+y}\,dx
&=& \frac{\sin xy}{y(x+y)}\biggr|_{r}^{r_{1}}+
\int_{r}^{r_{1}}\frac{\sin xy}{y(x+y)^{2}}\,dx\\
&=&\frac{\sin r_{1}y}{y(r_{1}+y)}-\frac{\sin ry}{y(r+y)}
+\int_{r}^{r_{1}}\frac{\sin xy}{y(x+y)^{2}}\,dx.
\end{eqnarray*}
Therefore, if $0< r<r_{1}$, then
\begin{eqnarray*}
\left|\int_{r}^{r_{1}}\frac{\cos xy}{x+y}\,dx\right|<
\frac{1}{y}\left(\frac{2}{r+y}+\int_{r}^{\infty}\frac{1}{(x+y)^{2}}\right)
\le \frac{3}{y(r+y)^{2}}\le \frac{3}{\rho(r+\rho)}
\end{eqnarray*}
if $y\ge \rho>0$. Now Theorem~\ref{theorem:4} implies that $I(y)$
converges uniformly on $[\rho,\infty)$ if $\rho>0$.
\end{example}

We leave the proof of the following theorem to you (Exercise~\ref{exer:10}).
\begin{theorem}  \label{theorem:9}
$(${\bf Dirichlet's Test for Uniform Convergence II}$)$
If $g,$  $g_{x},$ and $h$ are continuous on $(a,b]\times S,$  then
$$
\int_{a}^{b}g(x,y)h(x,y)\,dx
$$
converges uniformly on   $S$ if  the following
conditions are satisfied$:$
\begin{alist}
\item % a
$\dst{\lim_{x\to a+}\left\{\sup_{y\in S}|g(x,y)|\right\}=0};$
\item % b
There is a constant $M$ such that
$$
\sup_{y\in S}\left|\int_{x}^{b}h(u,y)\,du\right| \le M, \quad
a< x\le b;
$$

\item % c
$\int_{a}^{b}|g_{x}(x,y)|\,dx$   converges uniformly on $S$.
\end{alist}
\end{theorem}

By recalling Theorems~3.4.10 (p.~163), 4.3.20 (p.~217), and 4.4.16
(p.~248), you can see why   we associate  Theorems~\ref{theorem:8}  and
\ref{theorem:9}
 with Dirichlet.

\section{Consequences of uniform convergence}\label{section:consequences}

\begin{theorem}  \label{theorem:10}
If  $f=f(x,y)$ is continuous on either  $[a,b)\times [c,d]$ or
$(a,b]\times [c,d]$ and
\begin{equation} \label{eq:23}
F(y)=\int_{a}^{b}f(x,y)\,dx
\end{equation}
converges uniformly on $[c,d],$  then $F$ is continuous on
$[c,d].$ Moreover$,$
\begin{equation} \label{eq:24}
\int_{c}^{d}\left(\int_{a}^{b}f(x,y)\,dx\right)\,dy
=\int_{a}^{b}\left(\int_{c}^{d}f(x,y)\,dy\right)\,dx.
\end{equation}
\end{theorem}


\proof  We will assume that $f$ is continuous on $(a,b]\times [c,d]$.
You can consider the other case (Exercise~\ref{exer:14}).


We will first show that $F$ in \eqref{eq:23} is continuous on $[c,d]$.
Since $F$  converges uniformly on $[c,d]$,
Definition~\ref{definition:1}
(specifically, \eqref{eq:11})
   implies that  if $\epsilon>0$, there is an
$r \in [a,b)$ such that
$$
\left|\int_{r}^{b}f(x,y)\,dx\right|< \epsilon, \quad c \le y \le d.
$$
Therefore, if $c\le y, y_{0}\le d]$, then
\begin{eqnarray*}
|F(y)-F(y_{0})|&=&
\left|\int_{a}^{b}f(x,y)\,dx-\int_{a}^{b}f(x,y_{0})\,dx\right|\\
&\le&\left|\int_{a}^{r}[f(x,y)-f(x,y_{0})]\,dx\right|+
\left|\int_{r}^{b}f(x,y)\,dx\right|\\
&&+\left|\int_{r}^{b}f(x,y_{0})\,dx\right|,
\end{eqnarray*}
so
\begin{equation}\label{eq:25}
|F(y)-F(y_{0})|
\le  \int_{a}^{r}|f(x,y)-f(x,y_{0})|\,dx +2\epsilon.
\end{equation}


Since $f$  is uniformly continuous on the compact set $[a,r]\times [c,d]$
(Corollary~5.2.14, p.~314), there is a
$\delta>0$ such that
$$
|f(x,y)-f(x,y_{0})|<\epsilon
$$
if $(x,y)$ and $(x,y_{0})$  are in $[a,r]\times [c,d]$ and
$|y-y_{0}|<\delta$.    This and \eqref{eq:25}  imply that
$$
|F(y)-F(y_{0})|<(r-a)\epsilon +2\epsilon<(b-a+2)\epsilon
$$
if $y$ and $y_{0}$ are in $[c,d]$ and $|y-y_{0}|<\delta$. Therefore $F$
is continuous on $[c,d]$, so the integral on left side of
\eqref{eq:24} exists. Denote
\begin{equation} \label{eq:26}
I=
\int_{c}^{d}\left(\int_{a}^{b}f(x,y)\,dx\right)\,dy.
\end{equation}
 We will
show that the improper
integral on the right side of  \eqref{eq:24} converges to  $I$. To
this end, denote
$$
I(r)=
\int_{a}^{r}\left(\int_{c}^{d}f(x,y)\,dy\right)\,dx.
$$
Since we can reverse the order of integration of the
continuous function  $f$  over the rectangle $[a,r]\times [c,d]$
(Corollary~7.2.2, p.~466),
$$
I(r)=\int_{c}^{d}\left(\int_{a}^{r}f(x,y)\,dx\right)\,dy.
$$
From this and \eqref{eq:26},
$$
I-I(r)=\int_{c}^{d}\left(\int_{r}^{b}f(x,y)\,dx\right)\,dy.
$$
Now suppose $\epsilon>0$. Since $\int_{a}^{b}f(x,y)\,dx$ converges
uniformly on $[c,d]$, there is an $r_{0}\in (a,b]$ such that
$$
\left|\int_{r}^{b}f(x,y)\,dx\right|<\epsilon, \quad
r_{0}<r<b,
$$
so $|I-I(r)|<(d-c)\epsilon$ if $r_{0}<r<b$. Hence,
$$
\lim_{r\to b-}\int_{a}^{r}\left(\int_{c}^{d}f(x,y)\,dy\right)\,dx=
\int_{c}^{d}\left(\int_{a}^{b}f(x,y)\,dx\right)\,dy,
$$
which completes the proof of \eqref{eq:24}. \endproof

\begin{example}  \label{example:10}  \rm
It is straightforward to verify that
$$
\int_{0}^{\infty}e^{-xy}\,dx=\frac{1}{y}, \quad y>0,
$$
and  the convergence is uniform on $[\rho,\infty)$ if
$\rho>0$. Therefore Theorem~\ref{theorem:10} implies   that
if $0<y_{1}<y_{2}$, then
\begin{eqnarray*}
\int_{y_{1}}^{y_{2}}\frac{\,dy}{y}&=&
\int_{y_{1}}^{y_{2}}\left( \int_{0}^{\infty}e^{-xy}\,dx\right)\,dy
=\int_{0}^{\infty}\left(\int_{y_{1}}^{y_{2}}e^{-xy}\,dy\right)\,dy \\
&=&\int_{0}^{\infty}\frac{e^{-xy_{1}}-e^{-xy_{2}}}{x}\,dx.
\end{eqnarray*}
Since
$$
\int_{y_{1}}^{y_{2}}\frac{dy}{y}=
\log\frac{y_{2}}{y_{1}}, \quad y_{2} \ge y_{1}>0,
$$
it follows that
$$
\int_{0}^{\infty}\frac{e^{-xy_{1}}-e^{-xy_{2}}}{x}\,dx=
\log\frac{y_{2}}{y_{1}}, \quad y_{2} \ge y_{1}>0.
$$
\end{example}

\begin{example}  \label{example:11}  \rm
From Example~\ref{example:6},
$$
\int_{0}^{\infty}\frac{\sin xy}{x}\,dx=\frac{\pi}{2}, \quad y>0,
$$
and the convergence is uniform on $[\rho,\infty)$ if $\rho>0$. Therefore,
Theorem~\ref{theorem:10}  implies that if $0<y_{1}<y_{2}$, then
\begin{eqnarray}
\frac{\pi}{2}(y_{2}-y_{1})
&=&\int_{y_{1}}^{y_{2}}\left(\int_{0}^{\infty}\frac{\sin
xy}{x}\,dx\right)\,dy
=\int_{0}^{\infty}\left(\int_{y_{1}}^{y_{2}}\frac{\sin
xy}{x}\,dy\right)\,dx
\nonumber\\
&=&\int_{0}^{\infty}\frac{\cos xy_{1}-\cos xy_{2}}{x^{2}} \,dx.
\label{eq:27}
\end{eqnarray}
The last integral converges uniformly on $(-\infty,\infty)$
(Exercise 10{\bf(h)}), and is therefore continuous with respect to
$y_{1}$ on $(-\infty,\infty)$, by
Theorem~\ref{theorem:10}; in particular,
we can let $y_{1}\to0+$ in  \eqref{eq:27} and replace $y_{2}$
by  $y$ to obtain
$$
\int_{0}^{\infty} \frac{1-\cos xy}{x^{2}}\,dx=\frac{\pi y}{2}, \quad y
\ge 0.
 $$
\end{example}


The next theorem is analogous to  Theorem~4.4.20 (p.~252).


\begin{theorem}  \label{theorem:11}
Let   $f$  and $f_{y}$   be  continuous on either
$[a,b)\times [c,d]$ or $(a,b]\times [c,d].$ Suppose    that
the improper integral
$$
F(y)=\int_{a}^{b}f(x,y)\,dx
$$
converges for some $y_{0} \in [c,d]$ and
$$
G(y)=\int_{a}^{b}f_{y}(x,y)\,dx
$$
converges uniformly on $[c,d].$ Then $F$  converges
uniformly on $[c,d]$ and is given explicitly by
$$
F(y)=F(y_{0})+\int_{y_{0}}^{y} G(t)\,dt,\quad c\le y\le d.
$$
Moreover, $F$  is continuously differentiable on $[c,d]$; specifically,
\begin{equation} \label{eq:28}
F'(y)=G(y), \quad c \le y \le d,
\end{equation}
where  $F'(c)$ and $f_{y}(x,c)$ are derivatives
from the right, and $F'(d)$ and $f_{y}(x,d)$ are
 derivatives from the left$.$
\end{theorem}

\proof We will  assume that $f$ and $f_{y}$ are continuous
on $[a,b)\times [c,d]$. You can consider the other case
(Exercise~\ref{exer:15}).


Let
$$
F_{r}(y)=\int_{a}^{r}f(x,y)\,dx, \quad a\le r<b, \quad  c \le y \le d.
$$
Since $f$  and $f_{y}$  are continuous on $[a,r]\times [c,d]$,
Theorem~\ref{theorem:1} implies that
$$
F_{r}'(y)=\int_{a}^{r}f_{y}(x,y)\,dx, \quad c \le y \le d.
$$
Then
\begin{eqnarray*}
F_{r}(y)&=&F_{r}(y_{0})+\int_{y_{0}}^{y}\left(
\int_{a}^{r}f_{y}(x,t)\,dx\right)\,dt\\
&=&F(y_{0})+\int_{y_{0}}^{y}G(t)\,dt \\&&+(F_{r}(y_{0})-F(y_{0}))
-\int_{y_{0}}^{y}\left(\int_{r}^{b}f_{y}(x,t)\,dx\right)\,dt,
 \quad c \le y \le d.
\end{eqnarray*}
Therefore,
\begin{eqnarray}
\left|F_{r}(y)-F(y_{0})-\int_{y_{0}}^{y}G(t)\,dt\right|& \le &
|F_{r}(y_{0})-F(y_{0})|\nonumber\\
&&+\left|\int_{y_{0}}^{y}
\int_{r}^{b}f_{y}(x,t)\,dx\right|\,dt.
 \label{eq:29}
\end{eqnarray}
Now suppose $\epsilon>0$.  Since we have assumed that
$\lim_{r\to b-}F_{r}(y_{0})=F(y_{0})$ exists,
there is an $r_{0}$
in $(a,b)$ such that
$$
|F_{r}(y_{0})-F(y_{0})|<\epsilon,\quad  r_{0}<r<b.
$$
Since we have assumed that $G(y)$  converges for
$y\in[c,d]$, there is an $r_{1}  \in [a,b)$  such that
$$
\left|\int_{r}^{b}f_{y}(x,t)\,dx\right|<\epsilon, \quad
 t\in[c,d], \quad
r_{1}\le r<b.
$$
 Therefore, \eqref{eq:29}  yields
$$
\left|F_{r}(y)-F(y_{0})-\int_{y_{0}}^{y}G(t)\,dt\right|<
\epsilon(1+|y-y_{0}|) \le \epsilon(1+d-c)
$$
if $\max(r_{0},r_{1}) \le r <b$ and $t\in [c,d]$. Therefore $F(y)$
converges uniformly on   $[c,d]$  and
$$
F(y)=F(y_{0})+\int_{y_{0}}^{y}G(t)\,dt, \quad c \le y \le d.
$$
Since $G$  is continuous on $[c,d]$  by
Theorem~\ref{theorem:10}, \eqref{eq:28}
follows    from differentiating this (Theorem~3.3.11, p.~141).  \endproof

\begin{example}  \label{example:12}  \rm
Let
$$
I(y)=\int_{0}^{\infty}e^{-yx^{2}}\,dx, \quad y>0.
$$
Since
$$
\int_{0}^{r}e^{-yx^{2}}\,dx=\frac{1}{\sqrt{y}}
\int_{0}^{r\sqrt{y}} e^{-t^{2}}\,dt,
$$
it follows that
$$
I(y)=\frac{1}{\sqrt{y}}\int_{0}^{\infty}e^{-t^{2}}\,dt,
$$
and the convergence is uniform on $[\rho,\infty)$ if $\rho>0$
(Exercise~\ref{exer:8}{\bf(i)}).
To evaluate the last integral, denote
$J(\rho)=\int_{0}^{\rho}e^{-t^{2}}\,dt$;
then
$$
J^{2}(\rho)=\left(\int_{0}^{\rho}e^{-u^{2}}\,du\right)
\left(\int_{0}^{\rho}e^{-v^{2}}\,dv\right)
=\int_{0}^{\rho}\int_{0}^{\rho}e^{-(u^{2}+v^{2})}\,du\,dv.
$$
Transforming to polar coordinates $r=r\cos\theta$, $v=r\sin\theta$
 yields
$$
J^{2}(\rho)=\int_{0}^{\pi/2}\int_{0}^{\rho} re^{-r^{2}}\,dr\,d\theta
=\frac{\pi(1-e^{-\rho^{2}})}{4},
\text{\quad so\quad}
J(\rho)=\frac{\sqrt{\pi(1-e^{-\rho^{2}})}}{2}.
$$
Therefore
$$
\int_{0}^{\infty}e^{-t^{2}}\,dt=\lim_{\rho\to\infty}J(\rho)=
\frac{\sqrt{\pi}}{2}\text{\quad and\quad}
\int_{0}^{\infty}e^{-yx^{2}}\,dx= \frac{1}{2}\sqrt{\frac{\pi}{y}},
\quad y>0.
$$
 Differentiating this $n$ times with respect to
$y$ yields
$$
\int_{0}^{\infty}x^{2n}e^{-yx^{2}}\,dx=
\frac{1\cdot3\cdots(2n-1)\sqrt{\pi}}{2^{n}y^{n+1/2}}\quad y>0,\quad
n=1,2,3, \dots,
$$
where Theorem~\ref{theorem:11} justifies the differentiation for every
$n$, since all these integrals
 converge uniformly on $[\rho,\infty)$  if
$\rho>0$ (Exercise~\ref{exer:8}(i)).
\end{example}

Some advice for applying this theorem: Be sure to check first
that $F(y_{0})=\int_{a}^{b}f(x,y_{0})\,dx$ converges for at least one value
of
$y$. If so, differentiate $\int_{a}^{b}f(x,y)\,dx$ formally to obtain
$\int_{a}^{b}f_{y}(x,y)\,dx$. Then   $F'(y)=\int_{a}^{b}f_{y}(x,y)\,dx$
if $y$ is in some interval on which  this improper  integral converges
uniformly.


\place %

\section{Applications to Laplace transforms} \label{section:laplace}
\medskip

The
\href{http://www-history.mcs.st-and.ac.uk/Biographies/Laplace.html}
{\emph{Laplace}}
\emph{transform}  of a function $f$ locally integrable
   on $[0,\infty)$ is
$$
F(s)=\int_{0}^{\infty}e^{-sx}f(x)\,dx
$$
for all   $s$ such that integral converges. Laplace
transforms are widely applied in mathematics, particularly in solving
differential equations.

We leave it to you to prove the following theorem (Exercise~\ref{exer:26}).

\begin{theorem} \label{theorem:12}
Suppose $f$ is locally integrable on $[0,\infty)$  and
$|f(x)|\le M e^{s_{0}x}$  for sufficiently large $x$.
Then the Laplace
transform of $F$  converges uniformly on $[s_{1},\infty)$ if $s_{1}>s_{0}$.
\end{theorem}


\begin{theorem} \label{theorem:13}
If $f$ is continuous on $[0,\infty)$ and
$H(x)=\int_{0}^{\infty}e^{-s_{0}u}f(u)\,du$
is bounded on $[0,\infty),$  then the Laplace transform of $f$
converges uniformly on $[s_{1},\infty)$ if $s_{1}>s_{0}.$
\end{theorem}

\proof  If $0\le r\le r_{1}$,
$$
\int_{r}^{r_{1}}e^{-sx}f(x)\,dx
=\int_{r}^{r_{1}}e^{-(s-s_{0})x}e^{-s_{0}x}f(x)\,dt
=\int_{r}^{r_{1}}e^{-(s-s_{0})t}H'(x)\,dt.
$$
Integration by parts yields
$$
\int_{r}^{r_{1}}e^{-sx}f(x)\,dt=e^{-(s-s_{0})x}H(x)\biggr|_{r}^{r_{1}}
+(s-s_{0})\int_{r}^{r_{1}}e^{-(s-s_{0})x} H(x)\,dx.
$$
Therefore, if $|H(x)|\le M$, then
\begin{eqnarray*}
\left|\int_{r}^{r_{1}}e^{-sx}f(x)\,dx\right|&\le&
M\left|e^{-(s-s_{0})r_{1}}
+e^{-(s-s_{0})r} +(s-s_{0})\int_{r}^{r_{1}}e^{-(s-s_{0})x}\,dx\right|\\
&\le &3Me^{-(s-s_{0})r}\le 3Me^{-(s_{1}-s_{0})r},\quad s\ge s_{1}.
\end{eqnarray*}
Now Theorem~\ref{theorem:4} implies that $F(s)$ converges uniformly
on $[s_{1},\infty)$.

The following theorem draws a considerably stonger conclusion from
the same assumptions.

\begin{theorem}  \label{theorem:14}
If $f$  is continuous on $[0,\infty)$ and
$$
H(x)=\int_{0}^{x}e^{-s_{0}u}f(u)\,du
$$
is bounded on $[0,\infty),$  then the Laplace transform of $f$
is infinitely differentiable on $(s_{0},\infty),$ with
\begin{equation} \label{eq:30}
F^{(n)}(s)=(-1)^{n}\int_{0}^{\infty} e^{-sx} x^{n}f(x)\,dx;
\end{equation}
that is, the $n$-th derivative of the Laplace transform of $f(x)$ is the
Laplace transform of $(-1)^{n}x^{n}f(x)$.
\end{theorem}

\proof
First we will
 show that the integrals
$$
I_{n}(s)=\int_{0}^{\infty}e^{-sx}x^{n}f(x)\,dx,\quad n=0,1,2, \dots
$$
all converge uniformly on   $[s_{1},\infty)$ if
$s_{1}>s_{0}$.  If $0<r<r_{1}$, then
$$
\int_{r}^{r_{1}}e^{-sx}x^{n}f(x)\,dx=
\int_{r}^{r_{1}}e^{-(s-s_{0})x}e^{-s_{0}x}x^{n}f(x)\,dx
=\int_{r}^{r_{1}}e^{-(s-s_{0})x}x^{n}H'(x)\,dx.
$$
Integrating by parts yields
\begin{eqnarray*}
\int_{r}^{r_{1}}e^{-sx}x^{n}f(x)\,dx
&=&r_{1}^{n}e^{-(s-s_{0})r_{1}}H(r)-r^{n}e^{-(s-s_{0})r}H(r)\\
&&-\int_{r}^{r_{1}}H(x)\left(e^{-(s-s_{0})x}x^{n}\right)'\,dx,
\end{eqnarray*}
where $'$ indicates differentiation with respect to $x$. Therefore, if
 $|H(x)|\le M\le \infty$ on $[0,\infty)$, then
$$
\left|\int_{r}^{r_{1}}e^{-sx}x^{n}f(x)\,dx\right|\le
M\left(e^{-(s-s_{0})r}r^{n}+e^{-(s-s_{0})r}r^{n}
+\int_{r}^{\infty}|(e^{-(s-s_{0})x})x^{n})'|\,dx\right).
$$
Therefore, since $e^{-(s-s_{0})r}r^{n}$ decreases monotonically on
$(n,\infty)$  if $s>s_{0}$
(check!),
$$
\left|\int_{r}^{r_{1}}e^{-sx}x^{n}f(x)\,dx\right|<3Me^{-(s-s_{0})r}r^{n},\quad
n<r<r_{1},
$$
so  Theorem~\ref{theorem:4}  implies that $I_{n}(s)$ converges
uniformly  $[s_{1},\infty)$ if $s_{1}>s_{0}$. Now
Theorem~\ref{theorem:11} implies
that $F_{n+1}=-F_{n}'$, and an easy induction proof yields \eqref{eq:30}
(Exercise~\ref{exer:25}).
\endproof

\begin{example}  \label{example:13}  \rm
Here we  apply Theorem~\ref{theorem:12} with $f(x)=\cos ax$ ($a\ne0$) and
$s_{0}=0$. Since
$$
\int_{0}^{x}\cos au\,du=\frac{\sin ax}{a}
$$
is bounded on $(0,\infty)$, Theorem~\ref{theorem:12} implies that
$$
F(s)=\int_{0}^{\infty}e^{-sx}\cos ax\,dx
$$
converges and
\begin{equation} \label{eq:31}
F^{(n)}(s)=(-1)^{n}\int_{0}^{\infty}e^{-sx}x^{n}\cos ax\,dx, \quad s>0.
\end{equation}
(Note that this is also true if $a=0$.) Elementary integration
yields
$$
F(s)=\frac{s}{s^{2}+a^{2}}.
$$
Hence, from \eqref{eq:31},
$$
\int_{0}^{\infty}e^{-sx}x^{n}\cos ax=(-1)^{n}\frac{d^n}{ds^n}
\frac{s}{s^{2}+a^{2}}, \quad n=0,1, \dots.
$$

\end{example}
\newpage

\section{Exercises}

\begin{exerciselist}

\item\label{exer:1}
Suppose $g$ and $h$ are differentiable on $[a,b]$, with
$$
a \le g(y) \le b \text{\quad and\quad}  a \le h(y) \le b, \quad
c \le y \le d.
$$
Let $f$ and $f_{y}$  be continuous on $[a,b]\times [c,d]$. Derive
\emph{Liebniz's rule}:
\begin{eqnarray*}
\frac{d}{dy}\int_{g(y)}^{h(y)}f(x,y)\,dx
&=&f(h(y),y)h'(y)-f(g(y),y)g'(y)\\&&+\int_{g(y)}^{h(y)}f_{y}(x,y)\,dx.
\end{eqnarray*}
(Hint: Define $H(y,u,v)=\int_{u}^{v}f(x,y)\,dx$ and use the chain
rule.)

\item\label{exer:2}
Adapt the proof of Theorem~\ref{theorem:2} to prove
Theorem~\ref{theorem:3}.

\item\label{exer:3}
Adapt the proof of Theorem~\ref{theorem:4} to prove
Theorem~\ref{theorem:5}.


\item\label{exer:4}
Show that  Definition~\ref{definition:3} is independent
 of $c$; that is,  if
  $\int_{a}^{c}f(x,y)\,dx$  and
$\int_{c}^{b}f(x,y)\,dx$ both converge uniformly on $S$ for
some
$c\in (a,b)$, then they both converge uniformly on $S$
and every
$c\in
(a,b)$.

\item\label{exer:5}
\begin{alist}
\item % a
Show that if $f$ is bounded on $[a,b]\times [c,d]$  and
$\int_{a}^{b}f(x,y)\,dx$  exists as a proper integral for each
$y\in [c,d]$, then it converges uniformly on $[c,d]$
according to all of
Definition~\ref{definition:1}--\ref{definition:3}.

\item % b
Give an example to show that the boundedness of $f$ is essential
in {\bf(a)}.
\end{alist}


\item\label{exer:6}
Working directly from Definition~\ref{definition:1}, discuss uniform
convergence of  the following integrals:

\begin{tabular}{ll}
{\bf(a)}
$\dst{\int_{0}^{\infty}\frac{dx}{1+y^{2}x^{2}}\,dx}$  &
{\bf(b)} $\dst{\int_{0}^{\infty}e^{-xy}x^{2}\,dx}$  \\ \\
{\bf(c)} $\dst{\int_{0}^{\infty}x^{2n}e^{-yx^{2}}\,dx}$  &
{\bf(d)} $\dst{\int_{0}^{\infty}\sin xy^{2}\,dx}$  \\\\
{\bf(e)} $\dst{\int_{0}^{\infty}(3y^{2}-2xy)e^{-y^{2}x}\,dx}$  &
{\bf(f)} $\dst{\int_{0}^{\infty}(2xy-y^{2}x^{2})e^{-xy}\,dx}$
\end{tabular}

\item\label{exer:7}
Adapt the proof of Theorem~\ref{theorem:6} to prove
Theorem~\ref{theorem:7}.

\item\label{exer:8}
Use Weierstrass's test to show that the integral converges uniformly
on $S:$

\begin{alist}
\item % a
$\dst{\int_{0}^{\infty}e^{-xy}\sin x\,dx}$,\quad
$S=[\rho,\infty)$,\quad $\rho>0$

\item % b
$\dst{\int_{0}^{\infty}\dst{\frac{\sin x}{x^{y}}}\,dx}$,\quad
$S=[c,d]$, \quad $1<c<d<2$

\item % c
$\dst{\int_{1}^{\infty}e^{-px}\dst{\frac{\sin xy}{x}}\,dx}$,\quad
$p>0$,\quad
$S=(-\infty,\infty)$

\item % d
$\dst{\int_{0}^{1}\frac{e^{xy}}{(1-x)^{y}}}\,dx$, \quad
$S=(-\infty,b)$,\quad  $b<1$

\item % e
$\dst{\int_{-\infty}^{\infty}\frac{\cos xy}{1+x^{2}y^{2}}}\,dx$,\quad
$S=(-\infty,-\rho]\cup[\rho,\infty)$,\quad $\rho>0$.

\item % f
$\dst{\int_{1}^{\infty}e^{-x/y}\,dx}$,\quad
$S=[\rho,\infty)$,\quad $\rho>0$

\item % g
$\dst{\int_{-\infty}^{\infty}e^{xy}e^{-x^{2}}\,dx}$,\quad
$S=[-\rho,\rho]$,\quad $\rho>0$

\item % h
$\dst{\int_{0}^{\infty}\frac{\cos xy-\cos ax}{x^{2}}\,dx}$,\quad
$S=(-\infty,\infty)$

\item % i
$\dst{\int_{0}^{\infty}x^{2n}e^{-yx^{2}}\,dx}$,\quad
$S=[\rho,\infty)$,\quad $\rho>0$, \quad  $n=0$, $1$, $2$,\dots
\end{alist}

\item\label{exer:9}
\begin{alist}
\item % a
Show that
$$
\Gamma(y)=\int_{0}^{\infty} x^{y-1}e^{-x}\,dx
$$
converges if $y>0$, and uniformly on $[c,d]$ if $0<c<d<\infty$.

\item % b
Use integration by parts to show that
$$
\Gamma(y)=\frac{\Gamma(y+1)}{y},\quad   y \ge 0,
$$
and then show by induction that
$$
\Gamma(y)=\frac{\Gamma(y+n)}{y(y+1)\cdots(y+n-1)},  \quad y>0, \quad
n=1,2,3, \dots.
$$
How can this be used to define $\Gamma(y)$ in a natural way for all
$y\ne0$, $-1$, $-2$, \dots? (This function is called the \emph{gamma
function}.)

\item % c
Show that $\Gamma(n+1)=n!$  if $n$ is a positive integer.

\item % d
Show that
$$
\int_{0}^{\infty}e^{-st}t^{\alpha}\,dt =s^{-\alpha-1}\Gamma(\alpha+1), \quad
 \alpha>-1, \quad  s>0.
$$
\end{alist}

\item\label{exer:10}
Show that Theorem~\ref{theorem:8} remains valid with
assumption {\bf(c)} replaced
by the assumption that $|g_{x}(x,y)|$ is monotonic with respect to $x$
for all $y\in S$.

\item\label{exer:11}
Adapt the proof of Theorem~\ref{theorem:8} to prove
Theorem~\ref{theorem:9}.

\item\label{exer:12}
Use Dirichlet's test   to show
that the following
integrals converge uniformly on  $S=[\rho,\infty)$ if $\rho>0$:

\begin{tabular}{ll}
{\bf(a)} $\dst{\int_{1}^{\infty}\frac{\sin xy}{x^{y}}\,dx}$&
{\bf(b)} $\dst{\int_{2}^{\infty}\frac{\sin xy}{\log x}\,dx}$\\\\
{\bf(c)} $\dst{\int_{0}^{\infty}\frac{\cos xy}{x+y^{2}}\,dx}$&
{\bf(d)} $\dst{\int_{1}^{\infty}\frac{\sin xy}{1+xy}\,dx}$
\end{tabular}

\item\label{exer:13}
 Suppose $g,$ $g_{x}$ and $h$ are continuous on $[a,b)\times
S,$ and denote $H(x,y)=\int_{a}^{x}h(u,y)\,du,$ $a\le x<b.$   Suppose also
that
$$
\lim_{x\to b-} \left\{\sup_{y\in S}|g(x,y)H(x,y)|\right\}=0
\text{\quad and \quad}\int_{a}^{b}g_{x}(x,y)H(x,y)\,dx
$$
 converges uniformly on $S.$  Show
that $\int_{a}^{b}g(x,y)h(x,y)\,dx$  converges uniformly on $S$.


\item\label{exer:14}
 Prove Theorem~\ref{theorem:10} for the case where  $f=f(x,y)$
is continuous on $(a,b]\times [c,d]$.


\item\label{exer:15}
 Prove Theorem~\ref{theorem:11} for the case where  $f=f(x,y)$
is continuous on $(a,b]\times [c,d]$.


\item\label{exer:16}
Show that
$$
C(y)=\int_{-\infty}^{\infty}f(x)\cos xy\,dx
\text{\quad and\quad}
S(y)=\int_{-\infty}^{\infty}f(x)\sin xy\,dx
$$
are continuous on $(-\infty,\infty)$  if
$$
\int_{-\infty}^{\infty}|f(x)|\,dx<\infty.
$$


\item\label{exer:17}
Suppose $f$ is continuously differentiable on $[a,\infty)$,
$\lim_{x\to\infty}f(x)=0$, and
$$
\int_{a}^{\infty}|f'(x)|\,dx<\infty.
$$
 Show that the functions
$$
C(y)=\int_{a}^{\infty}f(x)\cos xy\,dx
\text{\quad and\quad}
S(y)=\int_{a}^{\infty}f(x)\sin xy\,dx
$$
are continuous for all $y\ne0$. Give an example showing that they need
not be continuous at $y=0$.




\item\label{exer:18}
Evaluate $F(y)$ and use Theorem~\ref{theorem:11} to
evaluate $I$:

\begin{alist}
\item % a
$F(y)=\dst{\int_{0}^{\infty}\frac{dx}{1+y^{2}x^{2}}}$,
$y\ne0$;\quad
$I=\dst{\int_{0}^{\infty}\frac{\tan^{-1}ax-\tan^{-1}bx}{x}\,dx}$,\quad
$a$, $b>0$

\item % b
$F(y)=\dst{\int_{0}^{\infty}x^{y}\,dx}$,
$y>-1$;\quad
$I=\dst{\int_{0}^{\infty}\frac{x^{a}-x^{b}}{\log x}\,dx}$,
\quad $a$, $b>-1$

\item % c
$F(y)=\dst{\int_{0}^{\infty}e^{-xy}\cos x\,dx}$,\quad
$y>0$

$I=\dst{\int_{0}^{\infty}\frac{e^{-ax}-e^{-bx}}{x}\cos x\,dx}$,\quad
$a$, $b>0$


\item % d
$F(y)=\dst{\int_{0}^{\infty}e^{-xy}\sin x\,dx}$, \quad
$y>0$

$I=\dst{\int_{0}^{\infty}\frac{e^{-ax}-e^{-bx}}{x}\sin x\,dx}$,
\quad $a$, $b>0$


\item % e
$F(y)=\dst{\int_{0}^{\infty}e^{-x}\sin xy\,dx}$;\,
$I=\dst{\int_{0}^{\infty}e^{-x}\dst\frac{1-\cos ax}{x}}\,dx$


\item % f
$F(y)=\dst{\int_{0}^{\infty}e^{-x}\cos xy\,dx}$;\,
$I=\dst{\int_{0}^{\infty}e^{-x}\dst\frac{\sin ax}{x}}\,dx$
\end{alist}


\item\label{exer:19}
Use Theorem~\ref{theorem:11} to evaluate:
\begin{alist}
\item % a
$\dst{\int_{0}^{1}(\log x)^{n}x^{y}\,dx}$, \quad  $y>-1$,\quad  $n=0$, $1$,
$2$,\dots .


\item % b
$\dst{\int_{0}^{\infty}\dst{\frac{dx}{(x^{2}+y)^{n+1}}}\,dx}$,\quad
$y>0$,\quad
$n=0$,
$1$, $2$, \dots.



\item % c
$\dst{\int_{0}^{\infty}x^{2n+1}e^{-yx^{2}}\,dx}$, \quad  $y>0$, \quad
$n=0$,
$1$,
$2$,\dots.

\item % e
$\dst{\int_{0}^{\infty}xy^{x}\,dx}$, \quad $0<y<1$.
\end{alist}


\item\label{exer:20}
\begin{alist}
\item % a
Use Theorem~\ref{theorem:11} and integration by parts
to show that
$$
F(y)=\int_{0}^{\infty}e^{-x^{2}}\cos 2xy\,dx
$$
satisfies
$$
F'+2y F=0.
$$
\item % b
Use part {\bf(a)} to show that
$$
F(y)=\frac{\sqrt{\pi}}{2} e^{-y^{2}}.
$$
\end{alist}


\item\label{exer:21}
Show that
$$
\int_{0}^{\infty}e^{-x^{2}}\sin 2xy\,dx =e^{-y^{2}}\int_{0}^{y}
e^{u^{2}}\,du.
$$
(Hint: See Exercise~~\ref{exer:20}.)



\item\label{exer:22}
State a condition implying  that
$$
C(y)=\int_{a}^{\infty}f(x)\cos xy\,dx
\text{\quad and\quad}
S(y)=\int_{a}^{\infty}f(x)\sin xy\,dx
$$
are $n$ times differentiable on for all $y\ne0$.
(Your condition should imply the hypotheses of Exercise~\ref{exer:16}.)


\item\label{exer:23}
Suppose $f$  is continuously differentiable on $[a,\infty)$,
$$
\int_{a}^{\infty}|(x^{k}f(x))'|\,dx<\infty,\quad 0\le k\le n,
$$
and $\lim_{x\to\infty}x^{n}f(x)=0$.  Show that if
$$
C(y)=\int_{a}^{\infty}f(x)\cos xy\,dx
\text{\quad and\quad}
S(y)=\int_{a}^{\infty}f(x)\sin xy\,dx,
$$
then
$$
C^{(k)}(y)=\int_{a}^{\infty}x^{k}f(x)\cos xy\,dx
\text{\quad and\quad}
S^{(k)}(y)=\int_{a}^{\infty}x^{k}f(x)\sin xy\,dx,
$$
$0\le k\le n$.



\item\label{exer:24}
Differentiating
$$
F(y)=\int_{1}^{\infty}\cos\frac{y}{x}\,dx
$$
under the integral sign yields
$$
-\int_{1}^{\infty}\frac{1}{x}\sin\frac{y}{x}\,dx,
$$
which converges uniformly  on any finite interval.
(Why?) Does this imply  that  $F$ is differentiable for all $y$?



\item\label{exer:25}
Show that Theorem~\ref{theorem:11} and induction imply
Eq.~\eqref{eq:30}.


\item\label{exer:26}
Prove Theorem~\ref{theorem:12}.


\item\label{exer:27} Show that if  $F(s)=\int_{0}^{\infty}e^{-sx}f(x)\,dx$
converges for $s=s_{0}$, then it converges uniformly on $[s_{0},\infty)$.
(What's  the difference between this and Theorem~\ref{theorem:13}?)



\item\label{exer:28}
Prove: If $f$  is continuous on  $[0,\infty)$ and
 $\int_{0}^{\infty}e^{-s_{0}x}f(x)\,dx$ converges, then
$$
\lim_{s\to s_{0}+}\int_{0}^{\infty}e^{-sx}f(x)\,dx=
\int_{0}^{\infty}e^{-s_{0}x}f(x)\,dx.
$$
(Hint: See the proof of Theorem~4.5.12, p.~273.)

\item\label{exer:29} Under the assumptions of Exercise~\ref{exer:28},
show that
$$
\lim_{s\to s_{0}+}\int_{r}^{\infty}e^{-sx}f(x)\,dx=
\int_{r}^{\infty}e^{-s_{0}x}f(x)\,dx,\quad r>0.
$$

\item\label{exer:30}
Suppose $f$ is continuous on $[0,\infty)$ and
$$
F(s)=\int_{0}^{\infty}e^{-sx}f(x)\,dx
$$
converges for $s = s_{0}$. Show that $\lim_{s\to\infty}F(s)=0$.
(Hint: Integrate by parts.)

\item\label{exer:31}
\begin{alist}
\item % a
Starting from the  result of
Exercise~\ref{exer:18}{\bf(d)}, let $b\to\infty$
and invoke Exercise~\ref{exer:30}  to evaluate
$$
\int_{0}^{\infty}e^{-ax} \frac{\sin x}{x}\,dx,  \quad a>0.
$$

\item % b
Use  {\bf(a)} and Exercise~\ref{exer:28} to show
that
$$
\int_{0}^{\infty} \frac{\sin x}{x}\,dx  =\frac{\pi}{2}.
$$
\end{alist}

\item\label{exer:32}
\begin{alist}
\item % a
Suppose $f$ is continuously differentiable on $[0,\infty)$ and
$$
|f(x)| \le Me^{s_{0}x}, \quad 0\le x\le \infty.
$$
Show that
$$
G(s)=\int_{0}^{\infty} e^{-sx}f'(x)\,dx
$$
converges uniformly on   $[s_{1},\infty)$ if $s_{1}>s_{0}$.
(Hint: Integrate by parts.)

\item % b
Show from part {\bf(a)} that
$$
G(s)=\int_{0}^{\infty} e^{-sx}xe^{x^{2}}\sin e^{x^{2}}\,dx
$$
converges uniformly on $[\rho,\infty)$ if $\rho>0$. (Notice
that
this does not follow from  Theorem~\ref{theorem:6} or \ref{theorem:8}.)
\end{alist}


\item\label{exer:33}
Suppose $f$  is continuous on $[0,\infty)$,
$$
\lim_{x\to0+}\frac{f(x)}{x}
$$
exists, and
$$
F(s)=\int_{0}^{\infty}e^{-sx}f(x)\,dx
$$
converges for $s=s_{0}$. Show that
$$
\int_{s_{0}}^{\infty}F(u)\,du=\int_{0}^{\infty}e^{-s_{0}x}\frac{f(x)}{x}\,dx.
$$

\end{exerciselist}

\newpage

\bigskip

\section{Answers to selected exercises}\label{section:answers}
\bigskip

\noindent
{\bf\ref{exer:5}. (b)} If $f(x,y)=1/y$  for $y\ne0$  and $f(x,0)=1$, then
$\int_{a}^{b}f(x,y)\,dx$  does not converge uniformly on
$[0,d]$ for any $d>0$.


\bigskip

\noindent
{\bf\ref{exer:6}.}
{\bf(a)}, {\bf(d)}, and {\bf(e)} converge uniformly on
$(-\infty,\rho]\cup[\rho,\infty)$ if $\rho>0$;\, {\bf(b)}, {\bf(c)},
and {\bf(f)}  converge uniformly on $[\rho,\infty)$ if
$\rho>0$.

\bigskip

\noindent
{\bf\ref{exer:17}.}
Let $C(y)=\dst{\int_{1}^{\infty}\frac{\cos xy}{x}\,dx}$ and
 $S(y)=\dst{\int_{1}^{\infty}\frac{\sin xy}{x}\,dx}$. Then
$C(0)=\infty$ and $S(0)=0$, while $S(y)=\pi/2$ if $y\ne0$.


\bigskip

\noindent
{\bf\ref{exer:18}.}
{\bf(a)}
$F(y)=\dst{\frac{\pi}{2|y|}}$;\quad $I=\dst{\frac{\pi}{2}\log\frac{a}{b}}$
\quad
 {\bf(b)} $F(y)=\dst{\frac{1}{y+1}}$;\quad $I=\dst{\log\frac{a+1}{b+1}}$


\bigskip
{\bf(c)}
$F(y)=\dst{\frac{y}{y^{2}+1}}$;\quad
$I=\dst{\frac{1}{2}\,\frac{b^{2}+1}{a^{2}+1}}$


{\bf(d)}
$F(y)=\dst{\frac{1}{y^{2}+1}}$;\quad   $I=\tan^{-1}b-\tan^{-1}a$



{\bf(e)}
$F(y)=\dst{\frac{y}{y^{2}+1}}$;\quad   $I=\dst{\frac{1}{2}}\log(1+a^{2})$


{\bf(f)}
$F(y)=\dst{\frac{1}{y^{2}+1}}$;\quad  $I=\tan^{-1}a$


\bigskip


\noindent
{\bf\ref{exer:19}.}
{\bf(a)} $(-1)^{n}n!(y+1)^{-n-1}$  \quad
{\bf(b)} $\pi2^{-2n-1}\dst{\binom{2n}{n}}y^{-n-1/2}$

{\bf(c)} $\dst{\frac{n!}{2y^{n+1}}}$ $(\log y)^{-2}$
{\bf(d)} $\dst{\frac{1}{(\log x)^{2}}}$

\noindent
{\bf\ref{exer:22}.}
$\dst{\int_{-\infty}^{\infty}|x^{n}f(x)|\,dx<\infty}$


\bigskip
\noindent
{\bf\ref{exer:24}.}
No; the integral defining $F$  diverges for all $y$.


\bigskip

\noindent
{\bf\ref{exer:31}.}
{\bf(a)}\,  $\dst{\frac{\pi}{2}}-\tan^{-1}a$


\end{document}
\newpage

\setlength{\parindent}{0pt}
\centerline{\large Beginning of manual}

{\bf 1.}
If
$H(y,u,v)=\dst{\int_{u}^{v}f(x,y)\,dx}$
then
$$
  H_{u}(y,u,v)=-f(u,y),
\quad H_{v}(y,u,v)=f(v,y),
$$
 and, by Theorem~1,
$H_{y}(u,v,y) =\dst{\int_{u}^{v}f_{y}(x,y)\,dx}$.
If
$$
F(y)=H(y,g(y),h(y))=\int_{g(y)}^{h(y)}f(x,y)\,dx,
$$
then
\begin{eqnarray*}
F'(y)&=&H_{v}(y, g(y),h(y))h'(y)+H_{u}(y,g(y),h(y))g'(y)+
H_{y}(y,g(y),h(y))\\
&=& f(h(y),y)h'(y)-f(g(y),y)g'(y)
+\int_{g(y)}^{h(y)} f_{y}(x,y)\,dx.
\end{eqnarray*}

\medskip


{\bf 2.}
{\bf Theorem 3 (Cauchy Criterion for Convergence of an Improper
Integral II)} \it
Suppose  $g$ is
integrable on every finite closed subinterval of  $(a,b]$ and denote
$$
G(r)=\int_{r}^{b}g(x)\,dx,\quad  a< r\le b.
$$
Then the improper integral $\int_{a}^{b}g(x)\,dx$ converges if and only
if$,$ for each
$\epsilon >0,$ there is an $r_{0}\in(a,b]$  such that
\begin{equation}\tag{A}
|G(r)-G(r_{1})|\le\epsilon,\quad  a<r,r_{1}\le r_{0}.
\end{equation} \rm



 \proof For necessity, suppose $\int_{a}^{b}g(x)\,dx=L$. By definition,
this  means that for each $\epsilon>0$ there is an $r_{0}\in (a,b]$
such that
$$
|G(r)-L|<\frac{\epsilon}{2}
\text{\quad and\quad}
|G(r_{1})-L|<\frac{\epsilon}{2}, \quad
a< r,r_{1}\le r_{0}.
$$
Therefore,
\begin{eqnarray*}
|G(r)-G(r_{1})|&=&|(G(r)-L)-(G(r_{1})-L)|\\
&\le& |G(r)-L|+|G(r_{1})-L|\le
\epsilon,\quad
a< r,r_{1}\le r_{0}.
\end{eqnarray*}

For sufficiency, (A) implies that
$$
|G(r)|= |G(r_{1})+(G(r)-G(r_{1}))|\le |G(r_{1})|+|G(r)-G(r_{1})|\le
|G(r_{1})|+\epsilon,
$$
  $a< r_{1}\le r_{0}$. Since $G$  is also bounded on the
compact set
$[r_{0},b]$ (Theorem~5.2.11, p.~313),  $G$ is bounded on $(a,b]$.
Therefore the monotonic functions
$$
\overline{G}(r)=\sup\set{G(r_{1})}{a<r_{1}\le r} \text{\quad and\quad}
\underline{G}(r)=\inf\set{G(r_{1})}{a<r_{1}\le r}
$$
are well defined on $(a,b]$, and
$$
\lim_{r\to a+}\overline{G}(r)=\overline{L}
\text{\quad and\quad}
\lim_{r\to a+}\underline{G}(r)=\underline{L}
$$
both exist  and are finite (Theorem~2.1.11, p.~47).
From  (A),
\begin{eqnarray*}
|G(r)-G(r_{1})|&=&|(G(r)-G(r_{0}))-(G(r_{1})-G(r_{0}))|\\
&\le &|G(r)-G(r_{0})|+|G(r_{1})-G(r_{0})|\le 2\epsilon,
\end{eqnarray*}
so
$\overline{G}(r)-\underline{G}(r)\le 2\epsilon$.
 Since
$\epsilon$  is an arbitrary positive number, this implies that
$$
\lim_{r\to a+}(\overline{G}(r)-\underline{G}(r))=0,
$$
so $\overline{L}=\underline{L}$. Let $L=\overline{L}=\underline{L}$.
Since
$$
\underline{G}(r)\le G(r)\le \overline{G}(r),
$$
it follows that $\lim_{r\to a+} G(r)=L$.

\medskip

{\bf 3.}
{\bf Theorem~5 $($Cauchy Criterion for Uniform
Convergence II$)$} \it
The improper integral
$$
\int_{a}^{b}f(x,y)\,dx =\lim_{r\to a+}\int_{r}^{b}f(x,y)\,dx
$$
converges uniformly on $S$ if and only if$,$
 for each $\epsilon>0,$ there is
an  $r_{0}\in (a,b]$ such that
\begin{equation}\tag{A}
\left|\int_{r_{1}}^{r}f(x,y)\,dx\right|< \epsilon, \quad y\in S,
\quad a <r,r_{1}\le r_{0}.
\end{equation}
\rm

\proof  Suppose  $\int_{a}^{b} f(x,y)\,dx$ converges uniformly on
$S$ and $\epsilon>0$.
From Definition~2,
there is an
$r_{0}\in (a,b]$  such that
\begin{equation} \tag{B}
\left|\int_{a}^{r}f(x,y)\,dx\right| <\frac{\epsilon}{2}
\text{\quad and\quad}
\left|\int_{a}^{r_{1}}f(x,y)\,dx\right|
<\frac{\epsilon}{2},\, y\in S, \,   a< r,r_{1}\le r_{0}.
\end{equation}
Since
$$
\int_{r_{1}}^{r}f(x,y)\,dx=
\int_{r_{1}}^{b}f(x,y)\,dx-
\int_{r}^{b}f(x,y)\,dx
$$
(B) and the triangle inequality imply (A).

For the converse, denote
$$
F(y)=\int_{r}^{b}f(x,y)\,dx.
$$
Since (A) implies that
$$
|F(r,y)-F(r_{1},y)|\le \epsilon, \quad y\in S, \quad
a< r, r_{1}\le r_{0},
$$
 Theorem~2 with $G(r)=F(r,y)$ ($y$ fixed
but arbitrary in $S$) implies that  $\int_{a}^{b} f(x,y)\,dx$
converges  pointwise for   $y\in S$.
 Therefore, if  $\epsilon>0$
then, for each $y\in S$,
there is an $r_{0}(y) \in (a,b]$ such that
\begin{equation} \tag{C}
\left|\int_{a}^{r}f(x,y)\,dx\right|\le \epsilon,
\quad y\in S,\quad
a<r\le r_{0}(y).
\end{equation}
For each $y\in S$, choose $r_{1}(y)\le \min[{r_{0}(y),r_{0}}]$. Then
$$
\int_{a}^{r}f(x,y)\,dx =
\int_{a}^{r_{1}(y)}f(x,y)\,dx+
\int_{r_{1}(y)}^{r}f(x,y)\,dx, \quad
$$
so (A), (C),  and the triangle inequality imply
that
$$
\left|\int_{a}^{r} f(x,y)\,dx\right|\le 2\epsilon,\quad y\in S,\quad a<r\le
r_{0}
$$

\medskip

{\bf 4.}
From Definition~3, $\int_{a}^{b}f(x,y)\,dx$
 converges uniformly on  $S$  if and  only if
$\int_{a}^{c}f(x,y)\,dx$  and $\int_{c}^{b}f(x,y)\,dx$ both converge
uniformly on $S$, where $c\in(a,b)$.
From Theorems~4 and  Theorem~5, this is true if and only if, for any
$\epsilon>0$ there are points $r_{0}$ and $s_{0}$ in $(a,b)$ such that
$$
\left|\int_{r}^{r_{1}}f(x,y)\,dx\right|\le \epsilon,\quad y\in S,\quad
r_{0}\le r,r_{1}<b
$$
and
$$
\left|\int_{s_{1}}^{s}f(x,y)\,dx\right|\le \epsilon,\quad y\in S,\quad
a< s,s_{1}<s_{0}.
$$
These conditions are independent of $c$.


\medskip

{\bf 5. (a)}
If $|f(x,y)|\le M$ on $[a,b]\times [c,d]$ then
$$
\left|\int_{r_{1}}^{r_{2}}f(x,y)\,dx\right|\le M|r_{2}-r_{1}|
$$
so the Cauchy convergence theorems imply the conclusion.


\medskip
{\bf (b)} Define
 $f=f(x,y)$ on $[0,1]\times [0,1]$ by
$$
f(x,y)= \begin{cases}\dst\frac{1}{y} &\text{if\quad} 0<y\le 1,\\
1&\text{if\quad} y=0.
\end{cases}
$$
Then
$$
\int_{r_{1}}^{r_{2}}f(x,y)\,dx= \begin{cases}\dst\frac{r_{2}-r_{1}}{y}
&\text{if\quad}
0<y\le 1,\\
r_{2}-r_{1}&\text{if\quad} y=0.
\end{cases}
$$
Therefore $f$ does not satisfy the requirements of Cauchy's convergence
theorems.


\medskip

{\bf 6.}
In all parts
$I(y)$ denotes the given integral.

\medskip {\bf(a)}
$I(0)=\infty$. If $y\ne0$ let
 $u=xy$; then
$I(y)=\dst{\frac{1}{y}\int_{0}^{\infty}\frac{du}{1+u^{2}}}$.
 If $\rho>0$ and $\epsilon >0$,  choose $r$  so that
$\dst{\int_{r}^{\infty}\frac{du}{1+u^{2}}}< \rho\epsilon$.
Then
$\dst{\frac{1}{|y|}\int_{r}^{\infty}\frac{du}{1+u^{2}}}<\epsilon$ if
$|y|\ge
\rho$, so $I(y)$ converges uniformly on
$(-\infty, \rho]\bigcup [\rho,\infty)$ if $\rho>0$.


\medskip {\bf(b)}
$I(y)=\infty$  if $y\le0$. If $y>0$
let $u=xy$; then
$I(y)=\dst{\frac{1}{y^{3}}\int_{0}^{\infty}e^{-u}u^{2}\,du}$.
 If $\rho>0$ and $\epsilon >0$, choose $r$  so that
$\dst{\int_{r}^{\infty}e^{-u}u^{2}\,du}<\rho^{3}\epsilon$.
Then
$\dst{\frac{1}{y^{3}}\int_{r}^{\infty}e^{-u}u^{2}\,du}<\epsilon$ if $y\ge
\rho$, so $I(y)$ converges uniformly on
$[\rho,\infty)$  if $\rho>0$.


\medskip {\bf(c)}
$I(y)=\infty$ if $y\le0$. If $y>0$ let $u=xy^{1/2}$; then
$I(y)=\dst{y^{-n-1/2}\int_{0}^{\infty}u^{2n}e^{-u}\,du}$.
 If $\rho>0$ and $\epsilon >0$, we can choose $r$  so that
$\dst{\int_{r}^{\infty}u^{2n}e^{-u}\,du}<\epsilon
\rho^{n+1/2}$. Then
$y^{-n-1/2}\dst{\int_{r}^{\infty}u^{2n}e^{-u}\,du}<\epsilon$ if $y\ge
\rho$,
 so $I(y)$ converges uniformly on
$S=[\rho,\infty)$ if $\rho>0$.


\medskip {\bf(d)}
Since $I(-y)=-I(y)$, it suffices to assume that $y>0$. If $u=yx^{2}$ then
$I(y)=\dst{\frac{1}{2\sqrt{y}}\int_{0}^{\infty}\frac{\sin
u\,du}{\sqrt{u}}}$. From Example 3.4.14 (p.~162), this integral
converges conditionally.
 If $\rho>0$ and $\epsilon >0$, we can choose $r$  so that
$\dst{\left|\int_{r}^{\infty}\frac{\sin
u\,du}{\sqrt{u}}\right|}<2\epsilon\sqrt{\rho}$, so
$I(y)$ converges uniformly on
$(-\infty,-\rho]\bigcup[\rho,\infty)$ if $\rho>0.$


\medskip {\bf(e)}
If $u=y^{2}x$ then
$\dst{I(y)=3\int_{0}^{\infty}e^{-u}\,du
-\frac{2}{y^{3}}\int_{0}^{\infty} ue^{-u}\,du}$.
If $\rho>0$, we can choose $r$  so that
$\dst{3\int_{r}^{\infty}e^{-u}\,du<\frac{\epsilon}{2}}$
and
$\dst{\int_{r}^{\rho}ue^{-u}\,du<\frac{\rho^{3}\epsilon}{2}}$.
Then
$$
\left|3\int_{r}^{\infty}e^{-u}\,du
-\frac{3}{y^{3}}\int_{r}^{\infty} ue^{-u}\,du\right|<\epsilon
\text{\quad if\quad} |y|\ge \rho,
$$
 so $I(y)$ converges uniformly  on
$(-\infty, \rho]\bigcup [\rho,\infty)$ if $\rho>0$.


\medskip {\bf(f)}
$I(y)=-\infty$ if $y\le0$. If $y>0$, let $u=xy$; then
$I(y)=\dst{\frac{1}{y}\int_{0}^{\infty}(2u-u^{2})e^{-u}\,du}$.
 If $\rho>0$ and $\epsilon >0$, we can choose $r$  so that
$\dst{\int_{r}^{\infty}|2u-u^{2}|e^{-u}\,du}<\epsilon \rho$,
 so $I(y)$ converges uniformly on
$S=[\rho,\infty)$  if $\rho>0$.


\medskip

{\bf 7.}
{\bf Theorem~7 $($Weierstrass's Test for Absolute Uniform Convergence
II$)$} Suppose $f=f(x,y)$ is locally integrable
$(a,b]$  and, for some $b_{0}\in (a,b],$
\begin{equation}\tag{A}
|f(x,y)| \le M(x), \:
y\in S, \:  x\in (a,b_{0}],
\end{equation}
 where
$$
\int_{a}^{b_{0}}M(x)\,dx<\infty.
$$
Then $\int_{a}^{b}f(x,y)\,dx$    converges absolutely uniformly on $S.$

\proof
Denote $\int_{a}^{b_{0}}M(x)\,dx=L<\infty$. By  definition,
for each $\epsilon>0$ there is an $r_{0}\in (a,b_{0}]$  such that
$$
L-\epsilon \le \int_{r}^{b_{0}}M(x)\,dx \le L,\quad
a<r\le r_{0}.
$$
Therefore, if $a<r_{1}< r\le r_{0}$, then
$$
0\le \int_{r_{1}}^{r}M(x)\,dx=\left(\int_{r_{1}}^{b_{0}}M(x)\,dx
-L\right)-
\left(\int_{r}^{b_{0}}M(x)\,dx -L\right)<\epsilon.
$$
  This and (A) imply that
$$
\int_{r_{1}}^{r}|f(x,y)|\,dx\le
\int_{r_{1}}^{r} M(x)\,dx <\epsilon, \: y\in S,\:
a<r_{1}\le r_{0}\le b.
$$
Now Theorem~5
implies the stated conclusion.

\medskip
{\bf 8. (a)}
$|e^{-xy}\sin x|\le e^{-\rho x}$ if $y\ge\rho$ and
$\int_{\rho}^{\infty}e^{-\rho x}\,dx<\infty$.

\medskip
{\bf(b)}
$\dst{\int_{0}^{\infty}\frac{\sin x\,dx}{x^{y}}}=I_{1}(y)+I_{2}(y)$,
 where
$$
I_{1}(y)=\int_{0}^{1}\frac{\sin x\,dx}{x^{y}}
\text{\quad and\quad}
I_{2}(y)=\int_{1}^{\infty}\frac{\sin x\,dx}{x^{y}}
$$
are both improper integrals.
Since
$$
\sin x= x-\left(\frac{x^{3}}{3!}-\frac{x^{5}}{5!}\right)
-\left(\frac{x^{7}}{7!}-\frac{x^{9}}{9!}\right)+ \cdots <x,\quad 0\le x
\le 1.
$$
If  $0\le 1$ and $y\le d\le 2$, then
$$
\left|\frac{\sin x}{x^{y}}\right|\le x^{1-y}\le x^{1-d} \text{\:so\:}
\int_{0}^{1}x^{-1+y}\,dx<\int_{0}^{1}x^{-1+d}\,dx=\frac{1}{2-d}
$$
so $I_{1}(y)$ converges absolutely uniformly on $S$.
Since  $c>1$,
$$
\frac{|\sin x|}{x^{y}}\le x^{-c} \text{\quad and\quad}
\int_{1}^{\infty}x^{-c} \,dx=\frac{1}{c-1}\text{\quad if \quad}
$$
$I_{2}(y)$  converges absolutely uniformly on $S$.


\medskip
{\bf (c)}
If $x\ge 1$ then
$\dst{e^{-px}\left|\frac{\sin xy}{x}\right|\le e^{-px}}$ for all $y$ and
$\dst{\int_{1}^{\infty}e^{-px}\,dx<\infty}$, since $p>0$.


\medskip {\bf(d)}
$\dst{\frac{e^{xy}}{(1-x)^{y}}}\le  \dst{\frac{e^{b}}{(1-x)^{b}}}$, if
$0\le
x<1$ and $ y\le b$, and  $\dst{\int_{0}^{1}(1-x)^{-b}\,dx}<\infty$ if
$b<1$.


\medskip  {\bf(e)}
If $|y|\ge \rho>0$ then $\dst{\left|\frac{\cos xy}{1+x^{2}y^{2}}\right|\le
\frac{1}{1+\rho^{2}x^{2}}}$ for all $x$, and
$\dst{\int_{0}^{\infty}\frac{dx}{1+\rho^{2}x^{2}}}<\infty$.




\medskip {\bf(f)}
If $y\ge \rho>0$ then $e^{-x/y}\le e^{-x/\rho}$ and
$\dst{\int_{0}^{\infty}e^{-x/\rho}\,dx<\infty}$.


\medskip  {\bf(g)}
If $|y|\le \rho$ then $e^{xy}e^{-x^{2}}\le e^{x\rho}e^{-x^{2}}$ and
$\dst{\int_{-\infty}^{\infty}e^{x\rho}e^{-x^{2}}\,dx}<\infty$.


\medskip  {\bf(h)}
If $|x|\ge 1$ then $|\cos xy-\cos ax|\le 2$ and
$\dst{\int_{1}^{\infty}\frac{2\,dx}{x^{2}}}<\infty$.


\medskip \noindent {\bf(i)}
If $y\ge \rho>0$ then $e^{-yx^{2}}\le e^{-\rho x^{2}}$ and
$\dst{\int_{0}^{\infty} x^{2n}e^{-\rho x^{2}}\,dx<\infty}$.


\medskip

{\bf 9. (a)}
If $0<x<1$ then $|x^{y-1}e^{-x}|<x^{c-1}$. Therefore,  since
$\dst{\int_{0}^{1}x^{c-1}\,dx}<\infty$ if $c>0$,
$\int_{0}^{1}x^{y-1}e^{-x}\,dx$
converges uniformly on  $[c,\infty)$ if $c>0$, by Theorem~7. If $x>1$ then
$|x^{y-1}e^{-x}|\le x^{d-1}e^{-x}$ if $y\le d$. Therefore, since
$\dst{\int_{1}^{\infty}x^{d-1}e^{-x}}\,dx<\infty$,
$\dst{\int_{1}^{\infty}x^{y-1}e^{-x}}\,dx$  converges uniformly on
$(-\infty,d]$   for every $d$,
 by Theorem~6. Hence, $\dst{\int_{0}^{\infty}x^{y-1}e^{-x}\,dx}$
converges uniformly on  $[c,d]$ if $c>0$.


\medskip
{\bf (b)}
If $y>0$ then
\begin{equation}
\Gamma(y)=\int_{0}^{\infty}x^{y-1}e^{-x}\,dx
=\frac{x^{y}e^{-x}}{y}\biggr|_{0}^{\infty}+
\frac{1}{y}\int_{0}^{\infty}x^{y}e^{-x}\,dx
=\frac{\Gamma(y+1)}{y}.
\tag{A}
\end{equation}
Therefore
\begin{equation}
\Gamma(y)=\frac{\Gamma(y+n)}{y(y+1)\cdots (y+n-1)}
\tag{B}
\end{equation}
is true when $n=1$. Now suppose it is true for given positive integer $n$,
and replace $y$ by $y+n$ in (A):
$$
\Gamma(y+n)=\frac{\Gamma(y+n+1)}{y+n}.
$$
Substituting this into (B) yields
$$
\Gamma(y)=\frac{\Gamma(y+n+1)}{y(y+1)\cdots (y+n)},
$$
which completes the induction.

 If  $-n <y<-n+1$ with $n\le 1$ then $0<y+n<1$ and we can compute
$\Gamma(y+n)$ from the definition in  part {\bf (a)}:
$$
\Gamma(y+n)=\int_{0}^{\infty}x^{y+n-1}e^{-x}\,dx.
$$
Then we can define $\Gamma(y)$ by (B).

\medskip
{\bf(c)}
The assertion is true if $n=1$, since
$$
\Gamma(2)=\int_{0}^{\infty}xe^{-x}\,dx =
-xe^{-x}\biggr|_{0}^{\infty}+\int_{0}^{\infty} e^{-x}\,dx = 1.
$$
If $\Gamma(n+1)=n!$  for some $n\ge 1$, then
\begin{eqnarray*}
\Gamma(n+2)&=&\int_{0}^{\infty}x^{n+1}e^{-x}\,dx= -e^{-x}
x^{n+1}\biggr|_{0}^{\infty}+(n+1)
\int_{0}^{\infty}x^{n+1}e^{-x}\,dx\\
&=&(n+1)\Gamma(n+1)=(n+1)n!=(n+1)!,
\end{eqnarray*}
which completes the induction proof.

\medskip
{\bf (d)}
The change of variable $x=st$ yields
$$
\int_{0}^{\infty}e^{-st}t^{\alpha}\,dt=\frac{1}{s^{\alpha+1}}
\int_{0}^{\infty}x^{\alpha}e^{-x}\,dx=\frac{1}{s^{\alpha+1}}
\Gamma(\alpha+1),
$$
from the definition of the Gamma  function.
\medskip

\medskip
{\bf 10.}
Since $|g_{x}(x,y)|$  is monotonic with respect to $x$,
\begin{equation}\tag{A}
\int_{r}^{r_{1}}|g_{x}(x,y)|\,dx=|g(r_{1},y)-g(r,y)|,\quad a\le r<r_{1}<b.
\end{equation}
From  Assumption {\bf(a)} of Theorem~8, if $\epsilon>0$  there is an
$r_{0}\in
[a,b)$ such that
$$
 |g(s,y)|\le \epsilon,\quad  y\in S,\quad  r_{0}\le s<b.
$$
Therefore, (A)  implies that
$$
\int_{r}^{r_{1}}|g_{x}(x,y)\,dx\le 2\epsilon,\quad y\in S,\quad r_{0}\le r\le
r_{1}<b.
$$
Now Theorem~4 implies that $\int_{a}^{b}|g(x,y)|\,dx$  converges uniformly
on $S$, which is assumption    {\bf(c)}  of Theorem~8.



\medskip
{\bf 11.}
{\bf Theorem~9 $($Dirichlet's Test for Uniform Convergence II}$)$
If $g,$  $g_{x},$ and $h$ are continuous on $(a,b]\times S$  then
$$
\int_{a}^{b}g(x,y)h(x,y)\,dx
$$
converges uniformly on $S$ if  the following
conditions are satisfied$:$
\begin{alist}
\item % a
$\dst{\lim_{x\to a+}\left\{\sup_{y\in S}|g(x,y)|\right\}=0};$
\item % b
There is a constant $M$ such that
$$
\sup_{y\in S}\left|\int_{x}^{b}h(u,y)\,du\right| \le M, \quad
a< x\le b;
$$

\item % c
$\int_{a}^{b}|g_{x}(x,y)|\,dx$   converges uniformly on $S$.
\end{alist}  \rm
\medskip

\proof
If
\begin{equation} \tag{A}
H(x,y)=\int_{x}^{b}h(u,y)\,du
\end{equation}
then integration by parts yields
\begin{eqnarray*}
\int_{r_{1}}^{r}g(x,y)h(x,y)\,dx&=&-\int_{r_{1}}^{r}g(x,y)H_{x}(x,y)\,dx
\\
&=&-g(r,y)H(r,y)+g(r_{1},y)H(r_{1},y)\\
&&+\int_{r_{1}}^{r}g_{x}(x,y)H(x,y)\,dx.
\end{eqnarray*}
Therefore,
since assumption   {\bf(b)}  and (A) imply that
$|H(x,y)|\le M$, $(x,y)\in (a,b]\times S$,
\begin{equation}\tag{B}
\left|\int_{r_{1}}^{r}g(x,y)h(x,y)\,dx\right|\le
M\left(2\sup_{a<x\le r}
|g(x,y)|+\int_{r_{1}}^{r}|g_{x}(x,y)|\,dx\right)
\end{equation}
on $[r_{1},r]\times S$.
Now suppose  $\epsilon>0$.    From assumption {\bf (a)}, there is an
$r_{0} \in [a,b)$ such that $|g(x,y)|<\epsilon$ on $S$ if
$a< x \le r_{0} \le b$.
From assumption  {\bf(c)} and Theorem~5,  there is an
$s_{0}\in
(a,b]$ such that
$$
\int_{r_{1}}^{r}|g_{x}(x,y)|\,dx<\epsilon,\quad
y\in S,\quad
a<r_{1}<r\le  s_{0}.
$$
Therefore
(B)  implies that
$$
\left|\int_{r_{1}}^{r}g(x,y)h(x,y)\right| < 3M\epsilon,\quad y\in S,\quad
a<r_{1}<r\min(r_{0},s_{0})
$$
Now Theorem~5 implies the stated conclusion.


\medskip
{\bf 12. (a)}
Denote
$F(y)=\dst{\int_{1}^{\infty}\frac{\sin xy}{x^{y}}}$
and, with $1\le r< r_{1}$,
\begin{eqnarray*}
F(r,r_{1},y)=\int_{r}^{r_{1}}\frac{\sin xy}{x^{y}}\,dx &=&
-\frac{\cos xy}{yx^{y}}\biggr|_{r}^{r_{1}}-
\int_{r}^{r_{1}}\frac{\cos xy}{x^{y+1}}\,dx\\
&=&
\frac{\cos r y}{yr^{y}}-\frac{\cos r_{1}y}{yr_{1}^{y}}-
\int_{r}^{r_{1}}\frac{\cos xy}{x^{y+1}}\,dx.
\end{eqnarray*}
Therefore
$$
|F(r,r_{1},y)|\le
\frac{2}{yr^{y}}+\int_{r}^{r_{1}}x^{-y-1}\,dx<\frac{3}{yr^{y}}, \quad
r,y>0.
$$
Now Theorem~4 implies that  $F(y)$ converges uniformly
on $[\rho,\infty)$ if $\rho>0$.

\medskip

{\bf (b)}
Denote
$F(y)=\dst{\int_{2}^{\infty}\frac{\sin xy}{\log x}\,dx}$
and, with $2\le r< r_{1}$,
$$
F(r,r_{1},y)=\int_{r}^{r_{1}}\frac{\sin xy}{\log x}\,dx
=-\frac{\cos xy}{y\log x}\biggr|_{r}^{r_{1}}-
\frac{1}{y}\int_{r}^{r_{1}}\frac{\cos xy}{x(\log x)^{2}}\,dx.
$$
Therefore
$$
|F(r,r_{1},y)|\le \frac{1}{y}\left|\frac{2}{\log r}+\int_{r}^{r_{1}}
\frac{dx}{x(\log x)^{2}}\right|\le \frac{3}{y\log r}.
$$
Now Theorem~4 implies that  $F(y)$ converges uniformly
on $[\rho,\infty)$ if $\rho>0$.


\medskip

{\bf (c)}
Denote
$F(y)=\dst{\int_{0}^{\infty}\frac{\cos xy}{x+y^{2}}}$,
and, with $0<r<r_{1}$,
$$
F(r,r_{1},y)=\int_{r}^{r_{1}}\frac{\cos xy}{x+y^{2}}\,dx=
\frac{1}{y}\left(\frac{\sin xy}{x+y^{2}}\biggr|_{r}^{r_{1}}
+\int_{r}^{r_{1}}\frac{\sin xy}{(x+y^{2})^{2}}\,dx\right),
$$
so
$$
|F(r,r_{1},y)|\le \frac{3}{y(r+y^{2})}.
$$
Now Theorem~4 implies that  $F(y)$ converges uniformly
on $[\rho,\infty)$ if $\rho>0$.


\medskip


{\bf (d)}
Denote
$F(y)=\dst{\int_{0}^{\infty}\frac{\sin xy}{1+xy}}$,
and, with $0<r<r_{1}$,
$$
F(r,r_{1},y)=\int_{r}^{r_{1}}\frac{\sin xy}{1+xy}\,dx=
-\frac{\cos xy}{y(1+xy)}\biggr|_{r}^{r_{1}}-\int_{r}^{r_{1}}
\frac{\cos xy}{y^{2}(1+xy)^{2}}\,dx,
$$
so
$$
|F(r,r_{1},y)|\le \frac{3}{y(1+ry)}.
$$
Now Theorem~4 implies that  $F(y)$ converges uniformly
on $[\rho,\infty)$ if $\rho>0$.

\medskip

{\bf 13.}
Integration by parts yields
\begin{eqnarray*}
\int_{r}^{r_{1}}g(x,y)h(x,y)\,dx&=&\int_{r}^{r_{1}}g(x,y)H_{x}(x,y)\,dx\\
&=&g(r_{1},y)H(r_{1},y)-g(r,y)H(r,y)\\
&&-\int_{r}^{r_{1}}g_{x}(x,y)H(x,y)\,dx,
\end{eqnarray*}
so
$$
\left|\int_{r}^{r_{1}}g(x,y)h(x,y)\,dx\right|\le
2\sup_{x\ge r}\left\{\left\{\sup_{y\in S}|g(x,y)H(x,y)|\right\}\right\}+
\left|\int_{r}^{r_{1}}g_{x}(x,y)H(x,y)\,dx\right|.
$$
Now suppose $\epsilon\ge 0$. From our first assmption, there is an
$s_{0}\in [a,b)$ such that
$$
\sup_{x\ge r}\left\{\left\{\sup_{y\in
S}|g(x,y)H(x,y)|\right\}\right\}<\epsilon, \quad  s_{0}\le r<b.
$$
Since $\int_{a}^{b}g_{x}(x,y)H(x,y)\,dx$
 converges uniformly  on $S$,  Theorem~4 implies that there is an $r_{0}\in
[a,b)$ such that
$$
\left|\int_{r}^{r_{1}}g_{x}(x,y)H(x,y)\,dx\right|\le \epsilon, \quad y\in S,
\quad  r_{0}\le r<r_{1}<b.
$$
Therefore,
$$
\left|\int_{r}^{r_{1}}g(x,y)h(x,y)\,dx\right|\le 2\epsilon, \quad y\in S, \quad
\max(r_{0},s_{0})\le r<r_{1}<b.
$$
Now Theorem~4  implies that  $\int_{a}^{b}g_{x}(x,y)h(x,y)\,dx$
converges  uniformly  on $S$.

\medskip


{\bf 14. Theorem 10} \it
If  $f=f(x,y)$ is continuous on
$(a,b]\times [c,d]$ and
\begin{equation} \tag{A}
F(y)=\int_{a}^{b}f(x,y)\,dx
\end{equation}
converges uniformly on $[c,d],$  then $F$ is continuous on
$[c,d].$ Moreover$,$
\begin{equation} \tag{B}
\int_{c}^{d}\left(\int_{a}^{b}f(x,y)\,dx\right)\,dy
=\int_{a}^{b}\left(\int_{c}^{d}f(x,y)\,dy\right)\,dx.
\end{equation}   \rm



We will first show that $F$ in (A) is continuous on $[c,d]$.
Since $F$  converges uniformly on $[c,d]$,
Definition~1
   implies that  if $\epsilon>0$, there is an
$r \in (a,b]$ such that
$$
\left|\int_{a}^{r}f(x,y)\,dx\right|\le \epsilon, \quad c \le y \le d.
$$
Therefore, if $y$ and $y_{0}$ are in $[c,d]$, then
\begin{eqnarray*}
|F(y)-F(y_{0})|&=&
\left|\int_{a}^{b}f(x,y)\,dx-\int_{a}^{b}f(x,y_{0})\,dx\right|\\
&\le&\left|\int_{r}^{b}[f(x,y)-f(x,y_{0})]\,dx\right|+
\left|\int_{a}^{r}f(x,y)\,dx\right|\\
&&+\left|\int_{a}^{r}f(x,y_{0})\,dx\right|\\
\end{eqnarray*}
so
\begin{equation}\tag{C}
|F(y)-F(y_{0})|
\le  \int_{r}^{b}|f(x,y)-f(x,y_{0})|\,dx +2\epsilon.
\end{equation}
Since $f$  is uniformly continuous on the compact set $[r,b]\times [c,d]$
(Corollary~5.2.14, p.~314), there is a
$\delta>0$ such that
$$
|f(x,y)-f(x,y_{0})|<\epsilon
$$
if $(x,y)$ and $(x,y_{0})$  are in $[r,b]\times [c,d]$ and
$|y-y_{0}|<\delta$.    This and (C)  imply that
$$
|F(y)-F(y_{0})|<(r-a)\epsilon +2\epsilon<(b-a+2)\epsilon
$$
if $y$ and $y_{0}$ are in $[c,d]$ and $|y-y_{0}|<\delta$. Therefore $F$
is continuous on $[c,d]$, so the integral on left side of
(B) exists. Denote
\begin{equation}\tag{D}
I=
\int_{c}^{d}\left(\int_{a}^{b}f(x,y)\,dx\right)\,dy.
\end{equation}
 We will
show that the improper
integral on the right side of  (B) converges to  $I$. To
this end, denote
$$
I(r)=
\int_{r}^{b}\left(\int_{c}^{d}f(x,y)\,dy\right)\,dx.
$$
Since we can reverse the order of integration of the
continuous function  $f$  over the rectangle $[r,b]\times [c,d]$
(Corollary~7.2.2, p.~466),
$$
I(r)=\int_{c}^{d}\left(\int_{r}^{b}f(x,y)\,dx\right)\,dy.
$$
From this and (D),
$$
I-I(r)=\int_{c}^{d}\left(\int_{a}^{r}f(x,y)\,dx\right)\,dy.
$$
Now suppose $\epsilon>0$. Since $\int_{a}^{b}f(x,y)\,dx$ converges
uniformly on $[c,d]$, there is an $r_{0}\in (a,b]$ such that
$$
\left|\int_{a}^{r}f(x,y)\,dx\right|<\epsilon, \quad
a<r<r_{0},
%###
$$
so $|I-I(r)|<(d-c)\epsilon$,  $a<r<r_{0}$. Hence,
$$
\lim_{r\to a+}\int_{r}^{b}\left(\int_{c}^{d}f(x,y)\,dy\right)\,dx=
\int_{c}^{d}\left(\int_{a}^{b}f(x,y)\,dx\right)\,dy,
$$
which completes the proof of (B).


\medskip

{\bf 15. Theorem~11} \it
Let   $f$  and $f_{y}$   be  continuous on
 $(a,b]\times [c,d],$ and suppose    that
$$
F(y)=\int_{a}^{b}f(x,y)\,dx
$$
converges for some $y_{0} \in [c,d]$ and
$$
G(y)=\int_{a}^{b}f_{y}(x,y)\,dx
$$
converges uniformly on $[c,d].$ Then $F$  converges
uniformly on $[c,d]$ and is given explicitly by
$$
F(y)=F(y_{0})+\int_{y_{0}}^{y} G(t)\,dt,\quad c\le y\le d.
$$
Moreover, $F$  is continuously differentiable on $[c,d]$ and
\begin{equation} \tag{A}
F'(y)=G(y), \quad c \le y \le d,
\end{equation}
where  $F'(c)$ and $f_{y}(x,c)$ are derivatives
from the right, and $F'(d)$ and $f_{y}(x,d)$ are
 derivatives from the left$.$ \rm



\proof Let
$$
F_{r}(y)=\int_{r}^{b}f(x,y)\,dx, \quad a\le r<b,\quad  c \le y \le d.
$$
Since $f$  and $f_{y}$  are continuous on $[r,b]\times [c,d]$,
Theorem~1 implies that
$$
F_{r}'(y)=\int_{r}^{b}f_{y}(x,y)\,dx, \quad c \le y \le d.
$$
Therefore
\begin{eqnarray*}
F_{r}(y)&=&F_{r}(y_{0})+\int_{y_{0}}^{y}\left(
\int_{r}^{b}f_{y}(x,t)\,dx\right)\,dt\\
&=&F(y_{0})+\int_{y_{0}}^{y}G(t)\,dt \\&&+(F_{r}(y_{0})-F(y_{0}))
-\int_{y_{0}}^{y}\left(\int_{a}^{r}f_{y}(x,t)\,dx\right)\,dt,
 \quad c \le y \le d.
\end{eqnarray*}
Therefore,
\begin{equation}\tag{B}
\left|F_{r}(y)-F(y_{0})-\int_{y_{0}}^{y}G(t)\,dt\right| \le
|F_{r}(y_{0})-F(y_{0})|
+\left|\int_{y_{0}}^{y}
\int_{a}^{r}f_{y}(x,t)\,dx\right|\,dt.
\end{equation}
Now suppose $\epsilon>0$.  Since we have assumed that
$\lim_{r\to a+}F_{r}(y_{0})=F(y_{0})$ exists,
there is an $r_{0}$
in $(a,b)$ such that
$$
|F_{r}(y_{0})-F(y_{0})|<\epsilon,\quad a<r< r_{0}.
$$
Since we have assumed that $G(y)$  converges for
$y\in[c,d]$, there is an $r_{1}  \in (a,b]$  such that
$$
\left|\int_{a}^{b}f_{y}(x,t)\,dx\right|<\epsilon, \quad
 t\in[c,d], \quad
a<r\le r_{1}.
$$
 Therefore, (B)  yields
$$
\left|F_{r}(y)-F(y_{0})-\int_{y_{0}}^{y}G(t)\,dt\right|<
\epsilon(1+|y-y_{0}|) \le \epsilon(1+d-c)
$$
if $a<r<\min(r_{0},r_{1})$ and $t\in[c,d]$. Therefore
$F(y)$ converges uniformly on   $[c,d]$  and
$$
F(y)=F(y_{0})+\int_{y_{0}}^{y}G(t)\,dt, \quad c \le y \le d.
$$
Since $G$  is continuous on $[c,d]$  by
Theorem~10, (A)
follows    from differentiating this (Theorem~3.3.11, p.~141).

\medskip


{\bf 16.}
Since
$$
|f(x)\cos xy|\le |f(x)|,\quad
|f(x)\sin xy|\le |f(x)|,\text{\quad and\quad}
\int_{-\infty}^{\infty} |f(x)|\,dx<\infty,
$$
Theorems~6 and 7 imply that $\int_{-\infty}^{\infty}f(x)\cos xy\,dx$
and $\int_{-\infty}^{\infty}f(x) \sin xy\,dx$   converge uniformly
on $(-\infty,\infty)$, so Theorem~10 implies that $C(y)$
and $S(y)$ are continuous on $(-\infty,\infty)$.


\medskip
{\bf 17.}
If $y\ne0$, integrating by parts yields
\begin{eqnarray*}
C(y)&=&f(x)\frac{\sin xy}{y}\biggr|_{a}^{\infty}-\frac{1}{y}
\int_{a}^{\infty}f'(x)\sin xy \,dx\\
&=&-f(a)\frac{\sin ay}{y}
-\frac{1}{y}\int_{a}^{\infty}f'(x)\sin xy \,dx
\end{eqnarray*}
and
\begin{eqnarray*}
S(y)&=&-f(x)\frac{\cos xy}{y}\biggr|_{a}^{\infty}+\frac{1}{y}
\int_{a}^{\infty}f'(x)\cos xy \,dx \\
&=&f(a)\frac{\cos ay}{y}+
\frac{1}{y}\int_{a}^{\infty}f'(x)\cos xy \,dx,
\end{eqnarray*}
since $\lim_{x\to\infty} f(x)=0$.  From Exercise \ref{exer:17}  with
$f$ replaced by
$f'$,
$\int_{1}^{\infty}f'(x)\cos xy \,dx$  and
$\int_{1}^{\infty}f'(x)\cos xy\,dx$
are continuous on $(-\infty,\infty)$. Therefore $C(y)$  and $S(y)$  are
continuous on $(-\infty,0)\cup(0,\infty)$.

To see that $C$  and $S$ are not necessarily  continuous at  $y=0$, let
$a=1$ and
$f(x)=1/x$, so
$$
 \lim_{x\to\infty}f(x)=0\text{\quad and\quad}
\int_{1}^{\infty}|f'(x)|=\int_{1}^{\infty}\frac{dx}{x^{2}}=1.
$$
Then
$$
C(y)=\lim_{r\to\infty}\int_{1}^{r}\frac{\cos xy}{x}\,dx
\text{\quad and\quad}
S(y)=\lim_{r\to\infty}\int_{1}^{r}\frac{\sin xy}{x}\,dx,\quad y\ne0.
$$
If $y>0$ make the change of variable $u=xy$ to see that
$$
C(y)=\lim_{r\to\infty}\int_{y}^{ry}\frac{\cos u}{u}\,du=
\int_{y}^{\infty}\frac{\cos u}{u}\,du
$$
and
$$
S(y)=\lim_{r\to\infty}\int_{y}^{ry}\frac{\sin u}{u}\,du.
S(y)=\int_{y}^{\infty}\frac{\sin u}{u}\,du.
$$
Therefore $\lim_{y\to 0+}C(y)=\infty$, so $C$ is not continuous at $y=0$.
Since  $S(0)=0$
and
$\lim_{y\to 0+}S(y)=
\dst{\int_{0}^{\infty}\frac{\sin u}{u}\,du}\ne 0$,  $S$ is not continuous
at
$y=0$.


\medskip
{\bf 18. (a)}
The integral diverges if $y=0$. If $y\ne0$ substitute
 $u=|y|x$ to obtain
\begin{equation}
F(y)=\int_{0}^{\infty}\frac{dx}{1+x^{2}y^{2}}=
\frac{1}{|y|}\int_{0}^{\infty}\frac{du}{1+u^{2}}
=\frac{1}{|y|}\tan^{-1}u\biggr|_{0}^{\infty}=\frac{\pi}{2|y|},
\tag{A}
\end{equation}
so $F(y)$  converges for all $y\ne0$.
To test for uniform convergence,
suppose  $|y|>0$ and $0<r<r_{1}$. Then
$$
\int_{r}^{r_{1}}\frac{dx}{1+x^{2}y^{2}}
=\frac{1}{|y|}\int_{r|y|}^{r_{1}|y|} \frac{du}{1+u^{2}}
<\frac{1}{\rho}\int_{r\rho}^{\infty}\frac{du}{1+u^{2}}
$$
if $|y|\ge \rho$.   If $\epsilon>0$ there is an $\alpha>0$ such that
$\dst{\frac{1}{\rho}\int_{\alpha}^{\infty}\frac{du}{1+u^{2}}}<\epsilon$.
Therefore $\dst{\int_{r}^{r_{1}}\frac{dx}{1+x^{2}y^{2}}}<\epsilon$ if
 $\alpha/\rho<r<r_{1}$. Now Theorem~4  implies that $F(y)$  converges
uniformly on  $(-\infty,-\rho]\cup[\rho,\infty)$  if $\rho>0$.

\medskip
To evaluate
$$
I=\dst{\int_{0}^{\infty}\frac{\tan^{-1}ax-\tan^{-1}bx}{x}\,dx},
$$
we note that
$$
\frac{\tan^{-1}ax-\tan^{-1}bx}{x}=\int_{b}^{a}\frac{dy}{1+x^{2}y^{2}}.
$$
Therefore
$$
I=\int_{0}^{\infty}\,dx \int_{b}^{a}\frac{dy}{1+x^{2}y^{2}}
=\int_{b}^{a}\,dy\int_{0}^{\infty}\frac{dx}{1+x^{2}y^{2}}
=\frac{\pi}{2}\int_{b}^{a}\frac{dy}{y}=\frac{\pi}{2}\log\frac{a}{b},
$$
where the second equality is valid because of the uniform convergence
of $F(y)$ on the closed interval with endpoints $a$ and $b$, and the
third equality follows from (A).

\medskip
{\bf (b)}
$F(y)$  is a proper integral if $y\ge 0$ and it diverges if $y\le -1$.
If $-1<y<0$,  then
\begin{equation}
F(y)=\int_{0}^{1}x^{y}\,dx=\frac{x^{y+1}}{y+1}\biggr|_{0}^{1}=\frac{1}{y+1}
\tag{A}
\end{equation}
is convergent.
Since
$$
\int_{0}^{r}x^{y}\,dx=\frac{x^{y+1}}{y+1}\biggr|_{0}^{r}=
\frac{r^{y+1}}{y+1}.
$$
and
$$
\frac{\partial}{\partial y}\left  (\frac{r^{y+1}}{y+1}\right)
=\frac{r^{y+1}}{y+1}\left(\log r-\frac{1}{y+1}\right)<0
\text{\quad if \quad} 0<r\le 1
\text{\quad and\quad} y>-1,
$$
it follows that
$$
\left|\int_{0}^{r}x^{y}\,dx\right|\le \frac{r^{\rho+1}}{\rho +1}
\text{\quad if\quad} 0<r\le 1\text{\quad and\quad} -1<\rho\le y.
$$
Therefore, Theorem~5 implies that $F(y)$  converges uniformly on
$[\rho,\infty)$  if $\rho>-1$.

\medskip
Now  Theorem~11  implies that
$$
I=\dst{\int_{0}^{1}\frac{x^{a}-x^{b}}{\log x}\,dx}
= \int_{0}^{1}\,dx \int_{b}^{a}x^{y}\,dy
=\int_{b}^{a}\,dy\int_{0}^{1}x^{y}\,dx
=\int_{b}^{a}\frac{dy}{y+1}=\log\frac{a+1}{b+1}.
$$

\medskip
{\bf (c)}
$\dst{F(y)=\int_{0}^{\infty} e^{-yx}\cos x \,dx=\frac{y}{y^{2}+1}}$.
Since
$$
\left|\int_{r}^{\infty}e^{-yx} \cos x\,dx\right|\le
\int_{r}^{\infty}e^{-xy}\,dx=\frac{e^{-yr}}{y},
$$
Theorem~4 (or Theorem~6) implies that $F(y)$  converges uniformly on
  $[\rho,\infty)$ if $\rho>0$.
Therefore, Theorem implies that if $a$, $b>0$  then
\begin{eqnarray*}
I&=&\int_{0}^{\infty}\frac{e^{-ax}-e^{-bx}}{x}\cos x\,dx
=\int_{0}^{\infty}\cos x\,dx\int_{a}^{b}e^{-yx}\,dy \\
&=&\int_{a}^{b} \,dy\int_{0}^{\infty}e^{-yx}\cos x\,dx
=\int_{a}^{b}\frac{y}{y^{2}+1}\,dy=\frac{1}{2}\log\frac{b^{2}+1}{a^{2}+1}.
\end{eqnarray*}


\medskip
{\bf (d)}
$\dst{F(y)=\int_{0}^{\infty} e^{-yx}\sin x \,dx=\frac{1}{y^{2}+1}}$.
Since
$$
\left|\int_{r}^{\infty}e^{-yx} \sin x\,dx\right|\le
\int_{r}^{\infty}e^{-yx}\,dx=\frac{e^{-yr}}{y},
$$
Theorem~4 (or Theorem~6) implies that    $F(y)$
converges  uniformly
 on every $[\rho,\infty)$ if $\rho>0$.
Therefore, if $a$, $b>0$  then  Theorem~11 implies that
\begin{eqnarray*}
I&=&\int_{0}^{\infty}\frac{e^{-ax}-e^{-bx}}{x}\sin x\,dx
=\int_{0}^{\infty}\sin x\,dx\int_{a}^{b}e^{-yx}\,dy \\
&=&\int_{a}^{b} \,dy\int_{0}^{\infty}e^{-yx}\sin x\,dx
=\int_{a}^{b}\frac{1}{y^{2}+1}\,dy=\tan^{-1}b-\tan^{-1}a.
\end{eqnarray*}

\medskip


{\bf(e)}
$\dst{F(y)=\int_{0}^{\infty} e^{-x}\sin xy \,dx=\frac{y}{y^{2}+1}}$.
Since
$$
\left|\int_{r}^{\infty}e^{-x} \sin xy\,dx\right|\le
\int_{r}^{\infty}e^{-x}\,dx=e^{-r},
$$
Theorem~4 (or Theorem~6) implies that $F(y)$ converges uniformly
 on  $[\rho,\infty)$  if $\rho>0$. Therefore Theorem~11 implies that
\begin{eqnarray*}
I&=&\int_{0}^{\infty}e^{-x}\frac{1-\cos ax}{x}\,dx
=\int_{0}^{\infty}e^{-x}\,dx\int_{0}^{a}\sin xy\,dy\\
&=&\int_{0}^{a}\,dy\int_{0}^{\infty}e^{-x}\sin xy\,dx
=\int_{0}^{a} \frac{y}{y^{2}+1}\,dy=\frac{1}{2}\log(1+a^{2}).
\end{eqnarray*}

\medskip
{\bf(f)}
$\dst{F(y)=\int_{0}^{\infty} e^{-x}\cos xy \,dx=\frac{1}{y^{2}+1}}$.
Since
$$
\left|\int_{r}^{\infty}e^{-x} \cos xy\,dx\right|\le
\int_{r}^{\infty}e^{-x}\,dx=e^{-r},
$$
Theorem~4 (or Theorem~6) implies that $F(y)$ converges uniformly
 on  $[\rho,\infty)$  if $\rho>0$. Therefore Theorem~11 implies that
\begin{eqnarray*}
I&=&\int_{0}^{\infty}e^{-x}\sin ax\,dx
=\int_{0}^{\infty}e^{-x}\,dx\int_{0}^{a}\cos xy\,dy\\
&=&\int_{0}^{a}\,dy\int_{0}^{\infty}e^{-x}\cos xy\,dx
=\int_{0}^{a} \frac{1}{y^{2}+1}\,dy=\tan^{-1}a.
\end{eqnarray*}

\medskip

{\bf 19. (a)}
We start with
\begin{equation}
F(y)=\int_{0}^{1} x^{y}\,dx =\frac{1}{y+1}\quad y>-1.
\tag{A}
\end{equation}
Formally differentiating this yields
\begin{equation}
F^{(n)}(y)=\int_{0}^{1}(\log x)^{n} x^{y}\,dx
=\frac{(-1)^{n}n!}{(y+1)^{n+1}},\quad y>-1.
\tag{B}
\end{equation}
To justify this we will show by induction that the improper integrals
$$
I_{n}(y)=\int_{0}^{1}(\log x)^{n} x^{y}\,dx,\quad     n=0,1,2,
\dots
$$
converge uniformly on $[\rho,\infty)$ if $\rho>-1$.  We begin with $n=0$:.
$$
\int_{0}^{r}x^{y}\,dx =
\frac{x^{y+1}}{y+1}\biggr|_{r_{1}}^{r}=\frac{r^{y+1}}{y+1}\le
\frac{r^{y+1}}{\rho+1},\quad -1<\rho\le y.
$$
so $I_{0}(y)=F(y)$ converges uniformly on $[\rho,\infty)$ if $\rho>-1$.
Now suppose that $I_{n}(y)$  converges uniformly on $[\rho,\infty)$.
Integrating by parts  yields
\begin{eqnarray*}
\int_{r_{1}}^{r}(\log x)^{n+1}x^{y}\,dx&=&
\frac{r^{y+1}(\log r)^{n+1}-r_{1}^{y+1}(\log r_{1})^{n+1}}
{y+1}\\ &&-\frac{n+1}{y+1}
\int_{r_{1}}^{r}(\log x)^{n}x^{y}\,dx, \quad -1<y<\infty.
\end{eqnarray*}
Letting $r_{1}\to 0$  yields
\begin{equation}\tag{C}
\int_{0}^{r}(\log x)^{n+1} x^{y}\,dx =\frac{r^{y+1}(\log r)^{n+1}}{y+1}
-\frac{n+1}{y+1}\int_{0}^{r}(\log x)^{n}x^{y}\,dx.
\end{equation}
Since the integral on the right converges, it follows that the integral
on the left converges; in fact
$$
\int_{0}^{1}(\log x)^{n+1}x^{y}\,dx=
-\frac{n+1}{y+1}
\int_{0}^{1}(\log x)^{n}x^{y}\,dx.
$$
We must still show that the integral on the left converges uniformly on
$[\rho,\infty)$ if \\$\rho>-1$.  To this end, note from (C) that
\begin{equation}\tag{D}
\left|\int_{0}^{r}(\log x)^{n+1} x^{y}\,dx\right| \le
\left|\frac{r^{\rho+1}(\log r)^{n+1}}{\rho+1}\right|
+\frac{n+1}{\rho+1}\left|\int_{0}^{r}(\log x)^{n}x^{y}\,dx\right|
\end{equation}
if $y\ge \rho$,
Now suppose $\epsilon>0$. Since $\dst{\lim_{r\to0+}r^{\rho+1}(\log
r)^{n+1}=0}$, there is an $r_{1}\in (0,1)$  such that
$$
\left|\frac{r^{\rho+1}(\log r)^{n+1}}{\rho+1}\right| \le \frac{\epsilon}{2}
\text{\quad if \quad} 0<r<r_{1}.
$$
Since $\int_{0}^{r}(\log x)^{n}x^{y}\,dx$  is uniformly convergent
(by our induction assumption), there is $r_{2}\in (0,1)$ such that
$$
\frac{n+1}{\rho+1}\left|\int_{0}^{r}(\log x)^{n}x^{y}\,dx\right|\le
\frac{\epsilon}{2},\quad  y\ge \rho,
$$
Now (D) implies that
$$
\left|\int_{0}^{r}(\log x)^{n+1} x^{y}\,dx\right|\epsilon,\quad
y\in [\rho,\infty),\quad 0<r<\min(r_{1},r_{2}).
$$
This,  Theorem~11, and an easy  induction argument  imply (B).

\medskip

{\bf (b)}
Substituting  $x=u\sqrt{y}$ yields
\begin{equation}
F(y)=\int_{0}^{\infty}\frac{dx}{x^{2}+y}=\frac{1}{\sqrt{y}}\int_{0}^{\infty}
\frac{du}{u^{2}+1}
=\frac{\pi}{2\sqrt{y}},\quad y>0.
\tag{A}
\end{equation}
Formally differentiating this yields  yields
\begin{eqnarray*}
\int_{0}^{\infty}\frac{dx}{(x^{2}+y)^{n+1}}
&=&\frac{\pi}{2n+1}1\cdot 3\cdots(2n-1)y^{-n-1/2}
=\frac{\pi}{2^{2n+1}}\frac{(2n)!}{n!}y^{-n-1/2}\\
&=&\frac{\pi}{2^{2n+1}}\binom{2n}{n}y^{-n-1/2},\quad y>0.
\end{eqnarray*}
Theorem~11 implies that
the  formal differentiation is legitimate, since, if $y\ge 0$
and $r>0$, then
$$
\int_{r}^{\infty}\frac{dx}{(x^{2}+y)^{n+1}}\le
\int_{r}^{\infty}x^{-2n-2}dx=\frac{r^{-2n-1}}{(2n-1)};
$$
hence,
 the improper integrals
$\dst{\int_{0}^{\infty}\frac{dx}{(x^{2}+y)^{n+1}}}$,
$n=0$, $1$, $2$, \dots
converge uniformly on $[0,\infty)$.



\medskip
{\bf (c)}
Denote $I_{n}(y)=\dst{\int_{0}^{\infty}x^{2n+1}e^{-yx^{2}}\,dx}$.   Then
$$
I_{0}(y)=\int_{0}^{\infty}xe^{-yx^{2}}=
\frac{1}{2}\int_{0}^{\infty}2xe^{-yx^{2}}\,dx
=-\frac{1}{2y}e^{-yx^{2}}\biggr|_{0}^{\infty}=\frac{1}{2y}.
$$
Since
$$
\int_{r}^{\infty}x^{2n+1}e^{-yx^{2}} \,dx\le
\int_{r}^{\infty}x^{2n+1}e^{-\rho x^{2}} \,dx\text{\quad if\quad}
0<\rho\le r,
$$
if $n\ge 0$, we can differentiate $I_{n}$  formally with respect to
$y\in (0,\infty)$ to obtain
$$
I_{n}(y)=(-1)^{n}I_{0}^{(n)}=\frac{n!}{2y^{n+1}}.
$$

{\bf (d)}
Denote
\begin{eqnarray*}
I(y)&=&\int_{0}^{\infty}y^{x}\,dx =\int_{0}^{\infty}e^{x\log y}\,dx
=\frac{1}{\log y}\int_{0}^{\infty}(\log y) y^{x}\,dx  \\
&=&\frac{y^{x}}{\log y}\biggr|_{0}^{\infty}=-\frac{1}{\log y}\quad
0<y<1.
\end{eqnarray*}
Formally differentiating this yields
$I'(y)=\dst{\int_{0}^{\infty}xy^{x-1}\,dx}$.
There are two improper integrals here:
$J_{1}(y)=\dst{\int_{0}^{1}xy^{x-1}\,dx}$ and
$J_{2}(y)=\dst{\int_{1}^{\infty}xy^{x-1}\,dx}$.
If $r<1$ then
$$
\int_{0}^{r}xy^{x-1}\,dx=\frac{1}{y}\int_{0}^{r}xy^{x}\,dx
\le \frac{1}{y}\int_{0}^{r}x\,dx=\frac{r^{2}}{2y}\le
\frac{r^{2}}{2\rho_{1}}, \quad
 0<\rho_{1}\le y\le 1.
$$
Therefore $J_{1}(y)$ converges uniformly on $[\rho_{1},1]$.
If $x>r>1$ and  $\rho_{2}<1$ then
$$
\int_{r}^{\infty}xy^{x-1}\,dx<\int_{r}^{\infty}x\rho_{2}^{x-1}\,dx
=\frac{1}{\rho_{2}}\int_{r}^{\infty}x\rho_{2}^{x}\,dx,
$$
Since
$$
\lim_{x\to\infty}\frac{1}{\rho_{2}}\int_{r}^{\infty}x\rho_{2}^{x}\,dx =0
$$
Theorem~7 implies that $J_{2}(y)$ converges uniformly on $[0,\rho_{2}]$.
Therefore, if $0<\rho_{1}<\rho_{2}<1$ then
$\dst{\int_{0}^{\infty}xy^{x-1}}$
converges uniformly on $[\rho_{1},\rho_{2}]$. Now Theorem~11 implies that
\begin{equation}\tag{A}
I'(y)=\int_{0}^{\infty}xy^{x-1}\,dx,\quad 0<y<1.
\end{equation}
 However, since
$I(y)=-\dst{\frac{1}{\log y}}$, we know that
$I'(y)=\dst{\frac{1}{y(\log y)^{2}}}$.   This and (A)  imply that
$\dst{\int_{0}^{\infty}xy^{x}\,dx}=\dst{\frac{1}{(\log x)^{2}}}$.


\medskip


{\bf 20.}
Here $F(y)=\dst{\int_{0}^{\infty} e^{-x^{2}}\cos 2xy\,dx}$, so
Theorem~11 implies that
\begin{equation}
F'(y)=-2\int_{0}^{\infty}xe^{-x^{2}}\sin 2xy\,dx,
\tag{A}
\end{equation}
since the integral on the right converges uniformly on $(-\infty,\infty)$,
by Theorem~6.

Integration by parts yields
\begin{eqnarray*}
F(y)&=&
=\frac{1}{2y}\int_{0}^{\infty}e^{-x^{2}}(2y\cos 2xy)\,dx\\
&=&\frac{1}{2y}\left(e^{-x^{2}}\sin 2xy\,dx\biggr|_{0}^{\infty}
+2\int_{0}^{\infty}xe^{-x^{2}}\sin 2xy\,dx\right)  \\
&=&\frac{1}{y}
\int_{0}^{\infty}xe^{-x^{2}}\sin 2xy\,dx
=-\frac{1}{2y} F'(y).
\end{eqnarray*}
From this and (A),   $F'(y)+2yF(y)=0$, so $\dst{\frac{F'(y)}{F(y)}}=-2y$,
$\log F(y)=-y^{2}+\log F(0)$, and  $F(y)=F(0)e^{-y^{2}}$.
Since
$F(0)=\dst{\int_{0}^{\infty}e^{-x^{2}}\,dx}=\dst{\frac{\sqrt{\pi}}{2}}$
(Example~12),
$F(y)=\dst{\frac{\sqrt{\pi}}{2}}e^{-y^{2}}$.


\medskip
{\bf 21.}
Here $F(y)=\dst{\int_{0}^{\infty} e^{-x^{2}}\sin 2xy\,dx}$, so
Theorem~11 implies that
\begin{equation}
F'(y)=2\int_{0}^{\infty}xe^{-x^{2}}\cos 2xy\,dx,
\tag{A}
\end{equation}
since the integral on the right converges uniformly on $(-\infty,\infty)$.
 Integrating this by parts yields
\begin{eqnarray*}
F'(y)
&=&-e^{-x^{2}}\cos 2xy\biggr|_{0}^{\infty}-
2y\int_{0}^{\infty} e^{-x^{2}}\sin 2xy\,dx \\
&=&1-2y F(y),
\end{eqnarray*}
so $F'(y)+2yF(y)=1$,
$e^{y^{2}}F'(y)+2e^{y^{2}}yF(y)=e^{y^{2}}$, and
 $\dst{\left(e^{y^{2}}F(y)\right)'=e^{y^{2}}}$.
Therefore, since $F(0)=0$,
$F(y)=\dst{e^{-y^{2}}\int_{0}^{y}e^{u^{2}}\,du}$.

{\bf 22.}
Theorems~6 and 11 imply that $S$ and $C$  are $n$ times continuously
differentiable on $(-\infty,\infty)$ if
 $\dst{\int_{-\infty}^{\infty}|x^{n}f(x)|\,dx<\infty}$.




\medskip


{\bf 23.}
We will show first  that
$$
C_{k}(y)=\int_{a}^{\infty} x^{k}f(x) \cos xy \,dx
 \text{\: and\:}
S_{k}(y)=\int_{a}^{\infty}x^{k}f(x)\sin xy\,dx,\: 0\le k\le n,
$$
converge uniformly on
$U_{\rho}=(-\infty,-\rho]\cup[\rho,\infty)$ if $\rho>0$.
Note that if $\lim_{x\to\infty} x^{n}f(x)=0$,  then
$\lim_{x\to\infty} x^{k}f(x)=0$, $k=0$, $1$, $2$,\dots $n$.
If $0\le k\le n$, then
\begin{equation}
\tag{A}
\int_{r}^{r_{1}}x^{k}f(x)\cos xy\,dx=
\frac{1}{y}\left[x^{k}f(x)\sin xy\biggr|_{r}^{r_{1}}-
\int_{r}^{r_{1}}(x^{k}f(x))'\sin xy\,dx\right].
\end{equation}
Henceforth $k$ is fixed.
Our assumptions imply that
if $\rho>0$ and $\epsilon>0$ then there is an $r_{0}\in [a,\infty)$  such
that
$$
\int_{r_{0}}^{\infty}|(x^{k}f(x))'|\,dx<\rho\epsilon
\text{\quad and \quad} |x^{k}f(x)|<\rho\epsilon,\quad  x\ge r_{0}.
$$
Therefore (A) implies that
$$
\left|\int_{r}^{r_{1}}x^{k}f(x)\cos xy\,dx\right|<3\epsilon,\quad
r_{0}\le r<r_{1},\:  y\in (-\infty,-\rho]\cup[\rho,\infty).
$$
Now Theorem~4  implies that $C_{0}$, $C_{1}$,\dots, $C_{k}$  converge
uniformly on  $(-\infty,-\rho]\cup[\rho,\infty)$.   Since every $y\ne0$
is in such an interval, Theorem~11 now implies that  that if $y\ne 0$
then
$$
C^{(k)}(y)=\int_{a}^{\infty}x^{k}f(x)\sin xy\,dx,\quad
0\le k\le n.
$$

A similar argument applies  to  $S$, $S'$,\dots $S^{(n)}$.

\medskip
{\bf 24.}
Let $I(y;r,r_{1})=\dst{\int_{r}^{r_{1}}\frac{1}{x}\sin\frac{y}{x}\,dx}$.
Assume for the moment that $y\ge 0$.
Substituting $u=y/x$ yields
$$
I(y;r,r_{1})=\int_{y/r_{1}}^{y/r}\left(\frac{u}{y}\right)\sin u
\left(-\frac{y}{u^{2}}\right)\,du =
\int_{y/r_{1}}^{y/r}\frac{\sin u}{u}\,du.
$$
Therefore, since $\dst{\left|\frac{\sin u}{u}\right|}\le 1$ for all $u$,
$|I(y;r,r_{1})|\le y/r$,\quad  $1\le r\le r_{1}$. In fact, since
$I(-y;r,r_{1})=-I(y;r,r_{1})$, we can write $|I(y;r,r_{1})\le |y|/r$,
\quad $1\le r\le r_{1}$. Therefore, Theorem~4 implies that
 $\dst{\int_{1}^{\infty}\frac{1}{x}\sin\frac{y}{x}\,dx}$   converges
uniformly on every finite interval.

Now
denote $F_{r}(y)=\dst{\int_{1}^{r}\cos\frac{y}{x}\,dx}$.
substituting $u=y/x$ yields
$F_{r}(y)=y\dst{\int_{y/r}^{y}\frac{\cos u}{u^{2}}\,du}$,
so $\lim_{r\to\infty}F_{r}(y)=\infty$  for all $y\ge 0$. Since
$F_{r}(-y)=F_{r}(y)$, it follows that $\lim_{r\to\infty}F_{r}(y)=\infty$
for all $y$, so the answer to the question is ``no.''



\medskip
{\bf 25.}
Let $P_{n}$ be the induction assumption
$$
 F^{(n)}(s)=(-1)^{n}\int_{0}^{\infty}e^{-sx}x^{n}f(x)\,dx,\quad    s>s_{0},
$$
which is true by the definition of $F$ for $n=0$. If $P_{n}$ is true, then
Theorems~11 and 13 imply that
\begin{eqnarray*}
F^{(n+1)}(s)&=&(-1)^{n}\frac{d}{ds}
\int_{0}^{\infty}e^{-sx}x^{n}f(x)\,dx=(-1)^{n}
\int_{0}^{\infty}\frac{d}{ds}\left(e^{-sx}x^{n}f(x)\right)\,dx\\
&=&(-1)^{n+1}\int_{0}^{\infty}e^{-sx}x^{n+1}f(x)\,dx,
\end{eqnarray*}
so $P_{n}$ implies $P_{n+1}$, which completes the induction proof.



\medskip
{\bf 26.}
Let $G(x)=\dst{\int_{0}^{x}e^{-s_{0}t}f(t)\,dt}$. If $s>s_{0}$  then
\begin{equation}\tag{A}
F(s)=\int_{0}^{\infty}e^{-sx}f(x)\,dx
=\int_{0}^{\infty}e^{-(s-s_{0})x}G'(x)\,dx
=(s-s_{0})\int_{0}^{\infty}e^{-(s-s_{0})x}G(x)\,dx
\end{equation}
(integration by parts). Since
$\dst{(s-s_{0})\int_{0}^{\infty}e^{-(s-s_{0})x}\,dx=1}$, (A) implies that
\begin{equation}\tag{B}
F(s)-F(s_{0})=(s-s_{0})\int_{0}^{\infty}e^{-(s-s_{0})x}(G(x)-F(s_{0}))\,dx.
\end{equation}
Now suppose $\epsilon>0$.
Since
$F(s_{0})=\dst{\int_{0}^{\infty}e^{-s_{0}t}
f(t)\,dt}=\lim_{t\to\infty}G(x)$, there is an $r$ such that
$|G(x)-F(s_{0})|<\epsilon$ if $x\ge r$; hence, from (B), then
\begin{eqnarray*}
|F(s)-F(s_{0})|&\le& (s-s_{0})\int_{0}^{r}e^{-(s-s_{0})x}
|G(x)-F(s_{0})|+\epsilon(s-s_{0})\int_{r}^{\infty}
e^{-(s-s_{0})x}\,dx\\
&<&
 (s-s_{0})\int_{0}^{r}e^{-(s-s_{0})x}
|G(x)-F(s_{0})|+\epsilon.
\end{eqnarray*}
Since $r$ is fixed, we can let $s\to s_{0}^{+}$ to conclude that
$\limsup_{s\to s_{0}+}|F(s)-F(s_{0})|\le \epsilon$, which implies that
$\lim_{s\to S_{0}+}F(s)=F(s_{0})$.


\medskip
{\bf 26.}
If  $s\ge s_{1}>s_{0}$  then
$$
|e^{-sx}f(x)|= |e^{-(s-s_{0})x}e^{s_{0}x}f(x)|\le M e^{-(s-s_{0})x}
\le M e^{-(s_{1}-s_{0})x}.
$$
Since
$$
\int_{0}^{\infty}Me^{-(s_{1}-s_{0})x}\,dx=\frac{M}{s_{1}-s_{0}}<\infty,
$$
Theorem~6 implies the stated conclusion.




\medskip
{\bf 27.}
In Theorem~13 we assumed only that $\int_{0}^{x}e^{-s_{0}u}f(u)\,du$
is bounded; here we are assuming that
$\int_{0}^{\infty}e^{-s_{0}u}f(u)\,du$ is convergent.

\medskip

 Let
$$
G(x)=\int_{x}^{\infty}e^{-s_{0}t}f(t)\,dt
\text{\quad and\quad} H(x)=\sup\set{|G(t)|}{t\ge x}.
$$
 Then
\begin{equation}\tag{A}
|G(x)|\le H(x)\text{\quad and \quad}
 \lim_{x\to\infty}G(x)=\lim_{x\to\infty}H(x)=0,
\end{equation}
since $\int_{0}^{\infty}e^{-s_{0}x}f(x)\,dx$ converges.
  Since  $f$ is continuous on $[0,\infty)$,
$G'(x)=-e^{-s_{0}x}f(x)$. Integration by parts yields
\begin{eqnarray*}
\int_{r}^{\infty}e^{-sx}f(x)\,dx&=&
\int_{r}^{\infty}e^{-(s-s_{0})x}(e^{-s_{0}x}f(x))\,dx
=-\int_{0}^{\infty}e^{-(s-s_{0})x}G'(x)\,dx\\
&=&-e^{-(s-s_{0})x}G(x)\biggr|_{r}^{\infty}
+(s-s_{0})\int_{r}^{\infty}e^{-(s-s_{0})x}G(x)\,dx\\
&=&e^{-(s-s_{0})r}G(r)+(s-s_{0})\int_{r}^{\infty}e^{-(s-s_{0})x}G(x)\,dx,\quad s\ge s_{0}.
\end{eqnarray*}
Therefore
\begin{eqnarray*}
\left|\int_{r}^{\infty}e^{-sx}f(x)\,dx\right|&\le&
|G(r)|e^{-(s-s_{0})r}+H(r)(s-s_{0})\int_{r}^{\infty}e^{-(s-s_{0})x}\,dx\\
&=&(G(r)+H(r))e^{-(s-s_{0})r}\le 2H(r)e^{-(s-s_{0})},  \quad  s\ge s_{0},
\end{eqnarray*}
so (A) implies that $F(s)$ converges uniformly on $[s_{0},\infty)$.


\medskip
{\bf 28.}
  From Theorem~13,
$F(s)=\dst{\int_{0}^{\infty}e^{-sx}f(x)\, dx}$ converges for all $s>s_{0}$.
Denote $G(x)=\dst{\int_{0}^{x}e^{-s_{0}t}f(t)\,dt}$, $x\ge 0$.  Then
\begin{eqnarray*}
F(s)&=&\int_{0}^{\infty}e^{-(s-s_{0})x}e^{-s_{0}x}f(x)\,dx=
\int_{0}^{\infty}e^{-(s-s_{0})x}G'(x)\,dx \\
&=&(s-s_{0})\int_{0}^{\infty}e^{-(s-s_{0})x}G(x)\,dx
\end{eqnarray*}
(integration by parts). Since
$\dst{(s-s_{0})\int_{0}^{\infty}e^{-(s-s_{0})x}\,dx}=1$,
$$
F(s)-F(s_{0})=\int_{0}^{\infty}e^{-(s-s_{0})x}(G(x)-F(s_{0}))\,dx
$$
If $\epsilon>0$ there is an $R$ such that $|G(x)-F(s_{0})|<\epsilon$ if
$x\ge R$. Therefore, if $s>s_{0}$ then
\begin{eqnarray*}
|F(s)-F(s_{0})|&<&
(s-s_{0})\int_{0}^{R}e^{-(s-s_{0})x}|G(x)-F(s_{0})|\,dx+\epsilon\\
&<&(s-s_{0})\int_{0}^{R}|G(x)-F(s_{0})|\,dx+\epsilon.
\end{eqnarray*}
Hence
$\limsup_{s\to s_{0}+}|F(s)-F(s_{0})|\le \epsilon$. Since $\epsilon$
is arbitrary, this implies that \\ $\lim_{s\to s_{0}+}|F(s)-F(s_{0})|=0$.


\medskip
{\bf 29.}
The assumptions of  Exercise~28 imply that
 $\int_{r}^{\infty}e^{-s_{0}x}f(x)\,dx$ converges for every $r>0$. Since
$$
\int_{r}^{\infty}e^{-s_{0}x}f(x)\,dx=\int_{0}^{\infty}e^{-s(r+x)}f(x+r)\,dx
=e^{-sr}\int_{0}^{\infty}e^{-sx}f(x+r)\,dx,
$$
we can apply the result of Exercise~30 with $f(x)$ replaced by $f(x+r)$, to
conclude that
\begin{eqnarray*}
\lim_{s\to s_{0}+}\int_{r}^{\infty}e^{-sx}f(x)\,dx&=&
e^{-s_{0}r}\int_{0}^{\infty}e^{-s_{0}x}f(x+r)\,dx\\
&=&\int_{0}^{\infty}e^{-s_{0}(x+r)}f(x+r)\,dx\\
&=&\int_{r}^{\infty}e^{-s_{0}x}f(x)\,dx.
\end{eqnarray*}


\medskip
{\bf 30.}
If  $G(x)=\dst{\int_{0}^{x}e^{-s_{0}t}f(t)\,dt}$, then $|G(x)|\le M$
on $[0,\infty)$ for some $M$. If $\epsilon>0$, there is an $r>0$ such that
\begin{equation}\tag{A}
\int_{0}^{r}e^{-s_{0}x}|f(x)|\,dx <\epsilon.
\end{equation}
If $s>s_{0}$, then
\begin{eqnarray*}
\int_{r}^{\infty}e^{-sx}f(x)\,dx&=&\int_{r}^{\infty}e^{-(s-s_{0})x}G'(x)\,dx \\
&=&e^{-(s-s_{0})x}G(x)\biggr|_{r}^{\infty}
+(s-s_{0})\int_{r}^{\infty}G(x)e^{-(s-s_{0})x}\,dx\\
&=&-e^{-(s-s_{0})r}G(r)
+(s-s_{0})\int_{r}^{\infty}G(x)e^{-(s-s_{0})x}\,dx.
\end{eqnarray*}
Therefore, since $|G(x)|\le M$,
\begin{eqnarray*}
\left|\int_{r}^{\infty}e^{-sx}f(x)\,dx\right|
&\le&Me^{-(s-s_{0})r}+M(s-s_{0})\int_{r}^{\infty}e^{-(s-s_{0})x}\,dx\\
&=&M\left(e^{-(s-s_{0})r}-e^{-(s-s_{0})x}\biggr|_{r}^{\infty}\right)
=2Me^{-(s-s_{0})r}.
\end{eqnarray*}
This and (A) imply that
$$
\left|\int_{0}^{\infty}e^{-sx}f(x)\,dx\right|\le
\epsilon+2Me^{-(s-s_{0})r}.
$$
Therefore,
$$
\limsup_{s\to\infty} \left|\int_{0}^{\infty}e^{-sx}f(x)\,dx\right|\le
\epsilon.
$$
 Since $\epsilon$ is arbitrary, this implies that
$$
\limsup_{s\to\infty}\int_{0}^{\infty}e^{-sx}f(x)\,dx=0,
$$

\medskip
{\bf 31. (a)}
From Exercise~18{\bf(d)},
$\dst{\int_{0}^{\infty}\frac{e^{-ax}-e^{-bx}}{x}\sin x\,dx}
=\tan^{-1}b-\tan^{-1}a.$ From Exercise~30, letting $b\to\infty$ yields
$$
\int_{0}^{\infty}e^{-ax}\frac{\sin x}{x}\,dx=
\frac{\pi}{2}-\tan^{-1}a,
 \text{\quad  so \bf{(b)}\quad}
\int_{0}^{\infty}\frac{\sin x}{x}\,dx=\frac{\pi}{2}.
$$



\medskip
{\bf 32. (a)}
Integrating by parts yields
\begin{equation}\tag{A}
\int_{0}^{r}e^{-sx}f'(x)\,dt
=e^{-sr}f(r)-f(0)
+\int_{0}^{r}se^{-sx}f(x)\,dx.
\end{equation}
Suppose $s\ge s_{1}>s_{0}$.
Since $|f(x)|\le  Me^{s_{0}x}$, $e^{-sr}|f(r)|\le Me^{-(s_{1}-s_{0})r}$.
Therefore
$e^{-sr}f(r)=0$  converges uniformly to zero on
$[s_{1},\infty)$. Since
\begin{eqnarray*}
\left|\int_{r}^{\infty}se^{-sx}f(x)\,dx \right|&\le&
M|s|\int_{r}^{\infty}e^{-(s-s_{0})x}\,dx\le
\frac{M|s|e^{-(s_{1}-s_{0})r}}{s-s_{0}}\\
&\le&\frac{M(s-s_{0}+|s_{0}|)e^{-(s_{1}-s_{0})r}}{s-s_{0}}
\le M\left(1+\frac{|s_{0}|}{s_{1}-s_{0}}\right)e^{-(s_{1}-s_{0})r},
\end{eqnarray*}
it follows  that
$\dst{\int_{r}^{\infty}se^{-sx}f(x)\,dx}$  converges to zero
uniformly on $[s_{1},\infty)$. Since this implies that
$\dst{\int_{0}^{r}se^{-sx}f(x)\,dx}$ converges uniformly on
$[s_{1},\infty)$,
(A) implies that
$G(s)$ converges uniformly on $[s_{1},\infty)$.

\medskip


{\bf(b)}
In this case let $f'(x)=xe^{x^{2}}\sin e^{x^{2}}$,
so $f(x)=-\dst{\frac{1}{2}}\cos e^{x^{2}}$. Since $|\cos e^{x^{2}}|\le 1$
for all $x$, the hypotheses stated in (a) hold with $s_{0}=0$. Therefore
$G(s)$ converges uniformly on $[\rho,\infty)$ if $\rho>0$.

\medskip

{\bf 33.}
We will first show that
$\dst{\int_{0}^{\infty}e^{-s_{0}x}\frac{f(x)}{x}\,dx}$  converges.
Denote $G(x)=\dst{\int_{0}^{x}e^{-s_{0}t}f(t)\,dt}$. Since $F(s_{0})$
is convergent; say $|G(x)|\le M$, $0\le x<\infty$.
If $0<r<r_{1}$ then
$$
\int_{r}^{r_{1}}e^{-s_{0}x}\frac{f(x)}{x}\,dx=
\int_{r}^{r_{1}}\frac{G'(x)}{x}\,dx
=\frac{G(r)}{r}-\frac{G(r_{1})}{r_{1}}-\int_{r}^{r_{1}}\frac{G(x)}{x^{2}}\,dx.
$$
Therefore
$$
\left|\int_{r}^{r_{1}}e^{-s_{0}x}\frac{f(x)}{x}\,dx\right|\le
\frac{3M}{\rho},\quad \rho<r<r_{1}
$$
so Theorem~2  implies that
$H(s)=\dst{\int_{0}^{\infty}e^{-st}\frac{f(x)}{x}\,dx}$
converge when $s=s_{0}$. Therefore  Exercise~27  implies that it converges
uniformly on
$[s_{0},\infty)$,
Therefore Theorem~10 implies that
\begin{eqnarray*}
\int_{s_{0}}^{s}F(u)\,du &=&\int_{s_{0}}^{s}\left(\int_{0}^{\infty}
e^{-ux}f(x)\,dx\right)\,du
=\int_{0}^{\infty}\left(\int_{s_{0}}^{s}e^{-ux}\,du\right)f(x)\,dx\\
&=&\int_{0}^{\infty}\left(e^{-s_{0}x}-e^{-sx}\right)\frac{f(x)}{x}\,dx
\end{eqnarray*}
From Exercise~30 (with $f(x)$ replaced by $f(x)/x$),
$\dst{\lim_{s\to\infty}\int_{0}^{\infty}e^{-sx}\frac{f(x)}{x}\,dx}=0$,
which implies the stated conclusion.



\end{document}

