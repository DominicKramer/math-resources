
\subsection{Riemann-Stieltjes integral}

FIXME: we'd need to redo a bunch of things from Riemann integral.
Perhaps useful, but those are missing below and sort of make this more and
more out of scope of the book.

A common useful generalization of the Riemann integral is the
Riemann-Stieltjes integral\footnote{Named for ...}.
If we think of the Riemann integral as a sum where all terms are weighted
equally, it is natural that we may want to do a weigthed sum.
That is, we may wish to give some points ``more weight''
than to other points.  A particular simple example of what we might want to
accomplish is an integral which evaluates a function at a point.  You may
have seen this concept in your calculus class as the
delta function.

We will again define this integral using the Darboux approach for
simplicity.

\begin{defn}
Let $f \colon [a,b] \to \R$ be a bounded function and let
$\alpha \colon [a,b] \to \R$ be a monotone increasing function.
Let $P$ be a partition of $[a,b]$, then define
\begin{align*}
& m_i := \inf \{ f(x) : x_{i-1} \leq x \leq x_i \} , \\
& M_i := \sup \{ f(x) : x_{i-1} \leq x \leq x_i \} , \\
& L(P,f,\alpha) :=
\sum_{i=1}^n m_i \bigl( \alpha(x_i) - \alpha(x_{i-1}) \bigr) , \\
& U(P,f,\alpha) :=
\sum_{i=1}^n M_i \bigl( \alpha(x_i) - \alpha(x_{i-1}) \bigr) .
\end{align*}
We call $L(P,f,\alpha)$ the \emph{\myindex{lower Darboux sum}} and
$U(P,f,\alpha)$ the \emph{\myindex{upper Darboux sum}}\index{Darboux sum}.
Then define
\begin{align*}
& \underline{\int_a^b} f~d\alpha := \sup \{ L(P,f,\alpha) : P \text{ a
partition of $[a,b]$} \} , \\
& \overline{\int_a^b} f~d\alpha := \inf \{ U(P,f,\alpha) : P \text{ a
partition of $[a,b]$} \} .
\end{align*}
And we call $\underline{\int}$ the \emph{\myindex{lower Darboux integral}} and
$\overline{\int}$ the \emph{\myindex{upper Darboux integral}}.
Finally, if 
\begin{equation*}
\underline{\int_a^b} f~d\alpha = \overline{\int_a^b} f~d\alpha .
\end{equation*}
Then we say that $f$ is \emph{\myindex{Riemann-Stieltjes integrable}}
with respect to $\alpha$.
\end{defn}

When we need to specify the variable of integration we may write
\begin{equation*}
\int_a^b f(x) ~d\alpha(x) .
\end{equation*}

When we set $\alpha(x) := x$ we recover the Riemann integral.  The notation
$d\alpha$ suggests derivative, in this case $\alpha'(x) = 1$ and as we
said, the Riemann integral is when all points are weighted equally.

\begin{prop}
If $\alpha(x) := x$, then a bounded function $f \colon [a,b] \to \R$
is Riemann integrable if and only if it is Riemann-Stieltjes integrable
with respect to $\alpha$.  In this case
\begin{equation*}
\int_a^b f = \int_a^b f~d\alpha .
\end{equation*}
\end{prop}

\begin{proof}
Simply plug in $\alpha(x) = x$ into the definition and note that
the definition is now precisely the same as for the Riemann integral.
\end{proof}

\begin{example}
Suppose that $f \colon [a,b] \to \R$ is continuous.
Given $c \in (a,b)$, let 
\begin{equation*}
\alpha(x) :=
\begin{cases}
1 & \text{if $x \geq c$,} \\
0 & \text{if $x < c$.}
\end{cases}
\end{equation*}
We claim that $f$ is Riemann-Stieltjes differentiable with respect to
$\alpha$ and that
\begin{equation*}
\int_a^b f~d\alpha = f(c) .
\end{equation*}

Proof: Given $\epsilon > 0$ take $\delta > 0$ such
that $\abs{f(x)-f(c)} < \epsilon$ for all $x \in [a,b]$
with $\abs{x-c} < \delta$.
Take the partition $P = \{ a , c-\delta,
c+\delta, b \}$.  Then
\begin{equation*}
\begin{split}
L(P,f,\alpha)
& =
m_1 \bigl( \alpha(c-\delta) - \alpha(a) \bigr)
+
m_2 \bigl( \alpha(c+\delta) - \alpha(c-\delta) \bigr)
+
m_3 \bigl( \alpha(b) - \alpha(c+\delta) \bigr)
\\
& =
m_2 \bigl( 1 - 0 ) = m_2 = \inf \{ f(x) : x \in [c-\delta,c+\delta] \}
\\
& >
f(c) - \epsilon .
\end{split}
\end{equation*}
Similarly $U(P,f,\alpha) < f(c)+\epsilon$.  Therefore
\begin{equation}
U(P,f,\alpha)-L(P,f,\alpha) < 2 \epsilon .
\end{equation}
\end{example}


\begin{example}
The notion of of integrability really does depend on $\alpha$.  For a
very trivial example,
it is not difficult to see that if $\alpha(x) = 0$, then
all bounded functions $f$ on $[a,b]$ are integrable with
respect to this $\alpha$ and
\begin{equation*}
\int_a^b f~d \alpha = 0.
\end{equation*}
\end{example}

If $\alpha$ is very nice, we can recover the Riemann-Stieltjes integral
using the Riemann integral.

\begin{prop}
Suppose that $f \colon [a,b] \to \R$ is Riemann integrable
and $\alpha \colon [a,b] \to \R$ is a continuously differentiable
increasing function.
Then $f$ is Riemann-Stieltjes integrable with respect to $\alpha$
and
\begin{equation*}
\int_a^b f(x)~d\alpha(x) = \int_a^b f(x) \alpha'(x)~dx .
\end{equation*}
\end{prop}

\begin{proof}
FIXME
\end{proof}

\subsection{Exercises}

\begin{exercise}
Directly from the definition of the Riemann-Stieltjes integral prove that if
$\alpha(x) = px$ for some $p \geq 0$, then 
If $f$ is Riemann integrable, then it is Riemann-Stieltjes integrable
with respect to $\alpha$ and
$p \int_a^b f = \int_a^b f~d\alpha$.
\end{exercise}

\begin{exercise}
Let $\alpha \colon [a,b] \to \R$
and $\beta \colon [a,b] \to \R$ be increasing functions and
suppose that $\alpha(x) = \beta(x) + C$ for some constant $C$.
If $f \colon [a,b] \to \R$
is integrable with respect to $\alpha$, show that it is
integrable with respect to $\beta$ and $\int_a^b f~d\alpha = \int_a^b f~d\beta$.
\end{exercise}



