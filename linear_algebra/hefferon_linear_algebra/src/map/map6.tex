% Chapter 3, Section 6 _Linear Algebra_ Jim Hefferon
%  http://joshua.smcvt.edu/linalg.html
%  2001-Jun-12
\section{Projection}
\index{projection}
\noindent\textit{This section is optional.
     It is a prerequisite only for 
     the final two sections of Chapter Five, and some Topics.}

We have described projection from $\Re^3$ into its
\( xy \)-plane subspace as a shadow map.
This shows why
but it also shows that some shadows fall upward.
\begin{center}  \small
  \raisebox{20.8163bp}{\includegraphics{map/mp/ch3.25}}
  \hspace*{4em}
  \raisebox{0bp}{\includegraphics{map/mp/ch3.26}}
\end{center}
\vspace*{1ex}  % graphic is poorly drawn and sticks down into text
So perhaps a better description is:~the projection of $\vec{v}$
is the vector $\vec{p}$ in the plane with the property that 
someone standing on $\vec{p}$ and looking straight up or 
down\Dash that is, looking orthogonally to the plane\Dash
sees the tip of $\vec{v}$.
In this section we will generalize this to other projections,
orthogonal and non-orthogonal.

% The first subsection develops orthogonal projection of a vector
% into a line, which is used often in applications.
% The second subsection shows how the definition of orthogonal
% projection into a line allows us to calculate especially convenient bases
% for vector spaces, again something that often appears in applications.
% The final subsection completely generalizes projection, orthogonal or not, 
% into any subspace at all.








\subsectionoptional{Orthogonal Projection Into a Line}
We first consider orthogonal projection 
of a vector \( \vec{v} \) into a line $\ell$.
% We darken a point on the line if someone at that point
% looking straight up or down (that is, looking orthogonal to the line)
% will see \( \vec{v} \).
%<*PictureOrthogonalProjectionIntoALine0>
This shows a figure walking out on the line 
to a point $\vec{p}\/$ such that
the tip of $\vec{v}$ is directly above them, where ``above'' does not mean
parallel to the $y$-axis but instead means orthogonal to the line.
%</PictureOrthogonalProjectionIntoALine0>
\begin{center}  \small
  \includegraphics{map/mp/ch3.28}
\end{center}
%<*PictureOrthogonalProjectionIntoALine1>
Since the line is the span of some
vector $\ell=\set{c\cdot\vec{s}\suchthat c\in\Re}$,
we have a coefficient $c_{\vec{p}}\,$
with the property that $\vec{v}-c_{\vec{p}}\vec{s}\,$ is orthogonal
to $c_{\vec{p}}\vec{s}$.
%</PictureOrthogonalProjectionIntoALine1>
\begin{center}  \small
  \includegraphics{map/mp/ch3.29}
\end{center}
%<*PictureOrthogonalProjectionIntoALine2>
To solve for this coefficient, observe that because 
$\vec{v}-c_{\vec{p}}\vec{s}\,$ is orthogonal to a scalar multiple 
of $\vec{s}$, it must be orthogonal to $\vec{s}\,$ itself.
Then 
$(\vec{v}-c_{\vec{p}}\vec{s})\dotprod \vec{s}=0$ gives that
$c_{\vec{p}}=\vec{v}\dotprod \vec{s}/\vec{s}\dotprod \vec{s}$.
%</PictureOrthogonalProjectionIntoALine2>

\begin{definition}  \label{def:ProjIntoLine}
%<*df:ProjIntoLine>
The \definend{orthogonal projection of \( \vec{v} \) into the line
spanned by a nonzero \( \vec{s}\, \)}\index{projection!orthogonal}%
\index{projection!into a line} is this vector.
\begin{equation*}
  \proj{\vec{v}}{\spanof{\vec{s}\,}}=
  \frac{ \vec{v}\dotprod\vec{s} }{ \vec{s}\dotprod\vec{s} }\cdot\vec{s}
\end{equation*}
%</df:ProjIntoLine>
\end{definition}

% \noindent \nearbyexercise{exer:LineProjFormIndOfVec} checks that the 
% outcome of the calculation depends only on the line and not on which vector 
% \( \vec{s}\, \) we used to describe that line.
\noindent (That says `spanned by \( \vec{s}\, \)'
instead the more formal `span of the set \( \set{\vec{s}\,} \)'.
This more casual phrase is common.)

\begin{example}
To orthogonally project
the vector $\binom{2}{3}$ into the line \( y=2x \),
first pick a direction vector for the line.
\begin{equation*}
  \vec{s}=\colvec[r]{1 \\ 2}
\end{equation*}
The calculation is easy.
\begin{center}  \small
  \begin{tabular}{@{}c@{}}\includegraphics{map/mp/ch3.30}\end{tabular}
  \qquad
  $\frac{ \colvec[r]{2 \\ 3}\dotprod\colvec[r]{1 \\ 2} }{% 
          \colvec[r]{1 \\ 2}\dotprod\colvec[r]{1 \\ 2} }
     \cdot\colvec[r]{1 \\ 2}
    ={\displaystyle \frac{8}{5}}\cdot\colvec[r]{1 \\ 2}=\colvec[r]{8/5 \\ 16/5}$
\end{center}
\end{example}

\begin{example}
In \( \Re^3 \), the orthogonal projection of a general vector
\begin{equation*}
  \colvec{x \\ y \\ z}
\end{equation*}
into the \( y \)-axis is
\begin{equation*}
  \frac{ \colvec{x \\ y \\ z}\dotprod\colvec[r]{0 \\ 1 \\ 0} }{
         \colvec[r]{0 \\ 1 \\ 0}\dotprod\colvec[r]{0 \\ 1 \\ 0} }
          \cdot\colvec[r]{0 \\ 1 \\ 0}=\colvec{0 \\ y \\ 0}
\end{equation*}
which matches our intuitive expectation.
\end{example}

The picture above showing the figure walking out on the line until 
$\vec{v}$'s tip is overhead is one way to think of the orthogonal projection of
a vector into a line.
We finish this subsection with two other ways.


\begin{example} \label{exam:RailCar}
A railroad car left on an east-west track without its brake is pushed by
a wind blowing toward the northeast at fifteen miles per hour;
what speed will the car reach?
\begin{center}  \small
  \includegraphics{map/mp/ch3.31}
\end{center}
For the wind we use a vector of length $15$ that points toward
the northeast.
\begin{equation*}
  \vec{v}=\colvec[r]{15\sqrt{1/2} \\ 15\sqrt{1/2}}
\end{equation*}
The car is only affected by the part of the wind blowing in the
east-west direction\Dash the part of $\vec{v}$ in the direction
of the $x$-axis is this (the picture has the same perspective as the
railroad car picture above).
\begin{center}
  \begin{tabular}{@{}c@{}}\includegraphics{map/mp/ch3.32}\end{tabular}
  \qquad
  $\displaystyle \vec{p}=\colvec{15\sqrt{1/2} \\ 0}$
\end{center}
So the car will reach a velocity of 
\( 15\sqrt{1/2} \) miles per hour toward the east.
\end{example}

Thus, another way to think
of the picture that precedes the definition is that it shows
$\vec{v}$ as decomposed into two parts, the part $\vec{p}$ with the line, 
and the part that is orthogonal to the line
(shown above on the north-south axis).
These two are non-interacting 
in the sense that the east-west car is not at all affected by the
north-south part of the wind (see \nearbyexercise{exer:PerpAreInd}).
So we can think of the orthogonal projection of \( \vec{v} \) 
into the line spanned by \( \vec{s}\, \)  as 
the part of \( \vec{v}\, \) that lies in the direction of \( \vec{s} \).

Still another useful way to think of orthogonal projection into a line
is to have the person stand on the vector, not the line.
This person holds a rope looped over the line.
As they pull, the loop slides on the line.
\begin{center}  \small
  \includegraphics{map/mp/ch3.33}
\end{center}
When it is tight, the rope is orthogonal to the line.
That is, we can think of the projection \( \vec{p}\, \) as being the vector 
in the line that is closest to \( \vec{v} \)
(see \nearbyexercise{exer:ProjMinimizesDist}). 

\begin{example}
A submarine is tracking a ship moving along the line \( y=3x+2 \).
Torpedo range is one-half mile.
If the sub stays where it is, at the origin on the chart below, 
will the ship pass within range?
\begin{center}  \small
  \includegraphics{map/mp/ch3.34}
\end{center} 
The formula for projection
into a line does not immediately apply because the line doesn't pass through
the origin, and so isn't the span of any $\vec{s}$.
To adjust for this, we start by shifting the entire map down two units.
Now the line is $y=3x$, a subspace. 
We project to get
the point $\vec{p}$ on the 
line closest to 
\begin{equation*}
  \vec{v}=\colvec[r]{0 \\ -2}
\end{equation*}
the sub's shifted position.
\begin{equation*}
  \vec{p}=\frac{\colvec[r]{0 \\ -2}\dotprod\colvec[r]{1 \\ 3}}{
                \colvec[r]{1 \\ 3}\dotprod\colvec[r]{1 \\ 3}}
          \cdot \colvec[r]{1 \\ 3}=\colvec[r]{-3/5 \\ -9/5}
\end{equation*}
The distance between  $\vec{v}$ and $\vec{p}$ is about
\( 0.63 \)~miles. 
The ship will never be in range.
\end{example}


\begin{exercises}
  \recommended \item
    Project the first vector orthogonally
    into the line spanned by the second vector.
    \begin{exparts*}
      \partsitem \mbox{\( \colvec[r]{2 \\ 1} \), \( \colvec[r]{3 \\ -2} \)}
      \partsitem \mbox{\( \colvec[r]{2 \\ 1} \), \( 
            \colvec[r]{3 \\ 0} \)}
      \partsitem \mbox{\( \colvec[r]{1 \\ 1 \\ 4} \), \( 
         \colvec[r]{1 \\ 2 \\ -1} \)}
      \partsitem \mbox{\( \colvec[r]{1 \\ 1 \\ 4} \), \( 
         \colvec[r]{3 \\ 3 \\ 12} \)}
    \end{exparts*}
    \begin{answer}
       % Each is a straightforward application of the formula from 
       % \nearbydefinition{def:ProjIntoLine}.
       \begin{exparts}
        \partsitem 
          $\displaystyle \frac{\colvec[r]{2 \\ 1}\dotprod\colvec[r]{3 \\ -2}}{%
                               \colvec[r]{3 \\ -2}\dotprod\colvec[r]{3 \\ -2}}
             \cdot\colvec[r]{3 \\ -2}
           =\frac{4}{13}
             \cdot\colvec[r]{3 \\ -2}
           =\colvec[r]{12/13 \\ -8/13}$
        \partsitem 
          $\displaystyle \frac{\colvec[r]{2 \\ 1}\dotprod\colvec[r]{3 \\ 0}}{%
                               \colvec[r]{3 \\ 0}\dotprod\colvec[r]{3 \\ 0}}
             \cdot\colvec[r]{3 \\ 0}
           =\frac{2}{3}
             \cdot\colvec[r]{3 \\ 0}
           =\colvec[r]{2 \\ 0}$
        \partsitem 
          $\displaystyle 
             \frac{\colvec[r]{1 \\ 1 \\ 4}\dotprod\colvec[r]{1 \\ 2 \\ -1}}{%
                   \colvec[r]{1 \\ 2 \\ -1}\dotprod\colvec[r]{1 \\ 2 \\ -1}}
             \cdot\colvec[r]{1 \\ 2 \\ -1}
           =\frac{-1}{6}
             \cdot\colvec[r]{1 \\ 2 \\ -1}
           =\colvec[r]{-1/6 \\ -1/3 \\ 1/6}$
        \partsitem 
          $\displaystyle 
             \frac{\colvec[r]{1 \\ 1 \\ 4}\dotprod\colvec[r]{3 \\ 3 \\ 12}}{%
                   \colvec[r]{3 \\ 3 \\ 12}\dotprod\colvec[r]{3 \\ 3 \\ 12}}
             \cdot\colvec[r]{3 \\ 3 \\ 12}
           =\frac{1}{3}
             \cdot\colvec[r]{3 \\ 3 \\ 12}
           =\colvec[r]{1 \\ 1 \\ 4}$
      \end{exparts}  
    \end{answer}
  \recommended \item 
    Project the vector orthogonally into the line.
    \begin{exparts*}
      \partsitem 
       $\colvec[r]{2 \\ -1 \\ 4},~\set{c\colvec[r]{-3 \\ 1 \\ -3}\suchthat c\in\Re}$
      \partsitem $\colvec[r]{-1 \\ -1}$,~the line $y=3x$
    \end{exparts*}
    \begin{answer}
      \begin{exparts}
        \partsitem 
          $\displaystyle
             \frac{\colvec[r]{2 \\ -1 \\ 4}\dotprod\colvec[r]{-3 \\ 1 \\ -3}}{%
                   \colvec[r]{-3 \\ 1 \\ -3}\dotprod\colvec[r]{-3 \\ 1 \\ -3}}
             \cdot\colvec[r]{-3 \\ 1 \\ -3}
           =\frac{-19}{19}\cdot\colvec[r]{-3 \\ 1 \\ -3}
           =\colvec[r]{3 \\ -1 \\ 3}$
         \partsitem Writing the line as
           \begin{equation*} 
             \set{c\cdot \colvec[r]{1 \\ 3}\suchthat c\in\Re}
           \end{equation*}
           gives this projection.
           \begin{equation*}
             \frac{\colvec[r]{-1 \\ -1}\dotprod\colvec[r]{1 \\ 3}}{%
                   \colvec[r]{1 \\ 3}\dotprod\colvec[r]{1 \\ 3}}
              \cdot\colvec[r]{1 \\ 3}
             =\frac{-4}{10}\cdot\colvec[r]{1 \\ 3}
             =\colvec[r]{-2/5 \\ -6/5}
           \end{equation*}
      \end{exparts}
    \end{answer}
  \item 
    Although pictures guided our 
    development of \nearbydefinition{def:ProjIntoLine}, 
    we are not restricted to spaces that 
    we can draw.
    In $\Re^4$ project this vector into this line.
    \begin{equation*}
      \vec{v}=\colvec[r]{1 \\ 2 \\ 1 \\ 3}
      \qquad
      \ell=\set{c\cdot\colvec[r]{-1 \\ 1 \\ -1 \\ 1}\suchthat c\in\Re}
    \end{equation*}
    \begin{answer}
      $\displaystyle 
         \frac{\colvec[r]{1 \\ 2 \\ 1 \\ 3}\dotprod\colvec[r]{-1 \\ 1 \\ -1 \\ 1}}{%
               \colvec[r]{-1 \\ 1 \\ -1 \\ 1}\dotprod\colvec[r]{-1 \\ 1 \\ -1 \\ 1}}
         \cdot\colvec[r]{-1 \\ 1 \\ -1 \\ 1}
       =\frac{3}{4}
         \cdot\colvec[r]{-1 \\ 1 \\ -1 \\ 1}
       =\colvec[r]{-3/4 \\ 3/4 \\ -3/4 \\ 3/4}$ 
    \end{answer}
  \recommended \item 
    \nearbydefinition{def:ProjIntoLine} uses two vectors $\vec{s}$ and
    $\vec{v}$.
    Consider the transformation of $\Re^2$ resulting from fixing
    \begin{equation*}
      \vec{s}=\colvec[r]{3 \\ 1}
    \end{equation*}
    and projecting $\vec{v}$ into the line that is the span of $\vec{s}$.
    Apply it to these vectors.
    \begin{exparts*}
      \partsitem $\colvec[r]{1 \\ 2}$
      \partsitem $\colvec[r]{0 \\ 4}$ 
    \end{exparts*}
    Show that in general the projection transformation is this.
    \begin{equation*}
      \colvec{x_1 \\ x_2}
      \mapsto
      \colvec{(9x_1+3x_2)/10 \\ (3x_1+x_2)/10}
    \end{equation*}
    Express the action of this transformation with a matrix.
    \begin{answer}
      \begin{exparts}
        \partsitem $\displaystyle
          \frac{\colvec[r]{1 \\ 2}\dotprod\colvec[r]{3 \\ 1}}{%
                 \colvec[r]{3 \\ 1}\dotprod\colvec[r]{3 \\ 1}}
          \cdot \colvec[r]{3 \\ 1}
          =\frac{1}{2}\cdot\colvec[r]{3 \\ 1}
          =\colvec[r]{3/2 \\ 1/2}$
        \partsitem $\displaystyle
          \frac{\colvec[r]{0 \\ 4}\dotprod\colvec[r]{3 \\ 1}}{%
                 \colvec[r]{3 \\ 1}\dotprod\colvec[r]{3 \\ 1}}
          \cdot \colvec[r]{3 \\ 1}
          =\frac{2}{5}\cdot\colvec[r]{3 \\ 1}
          =\colvec[r]{6/5 \\ 2/5}$
      \end{exparts}
      \noindent In general the projection is this.
      \begin{equation*}
          \frac{\colvec{x_1 \\ x_2}\dotprod\colvec[r]{3 \\ 1}}{%
                 \colvec[r]{3 \\ 1}\dotprod\colvec[r]{3 \\ 1}}
          \cdot \colvec[r]{3 \\ 1}
          =\frac{3x_1+x_2}{10}\cdot\colvec[r]{3 \\ 1}
          =\colvec{(9x_1+3x_2)/10 \\ (3x_1+x_2)/10}
      \end{equation*}
      The appropriate matrix is this.
      \begin{equation*}
        \begin{mat}[r]
          9/10  &3/10  \\
          3/10  &1/10
        \end{mat}
      \end{equation*}  
     \end{answer}
  \item \label{exer:PerpAreInd}
    \nearbyexample{exam:RailCar} suggests that projection breaks $\vec{v}$ into
    two parts, $\proj{\vec{v}\,}{\spanof{\vec{s}\,}}$ and 
    $\vec{v}-\proj{\vec{v}\,}{\spanof{\vec{s}\,}}$, that are 
    non-interacting.
    Recall that the two are orthogonal.
    Show that any two nonzero orthogonal vectors make up a linearly
    independent set.
    \begin{answer}
      Suppose that $\vec{v}_1$ and $\vec{v}_2$ are nonzero and orthogonal.
      Consider the linear relationship
      $c_1\vec{v}_1+c_2\vec{v}_2=\zero$.
      Take the dot product of both sides 
      of the equation with $\vec{v}_1$ to get that
      \begin{multline*}
        \vec{v}_1\dotprod(c_1\vec{v}_1+c_2\vec{v}_2)
        =c_1\cdot(\vec{v}_1\dotprod\vec{v}_1)
           +c_2\cdot(\vec{v}_1\dotprod\vec{v}_2)            \\
        =c_1\cdot (\vec{v}_1\dotprod\vec{v}_1)+c_2\cdot 0
        =c_1\cdot (\vec{v}_1\dotprod\vec{v}_1)
      \end{multline*}
      is equal to $\vec{v}_1\dotprod\zero=\zero$.
      With the assumption that $\vec{v}_1$ is nonzero, this gives that
      $c_1$ is zero.
      Showing that $c_2$ is zero is similar.
    \end{answer}
  \item    
    \begin{exparts}
      \partsitem What is the orthogonal projection of \( \vec{v} \) 
        into a line if \( \vec{v} \) is a member of that line?
      \partsitem Show that 
        if $\vec{v}$ is not a member of the line
        then the set 
        $\set{\vec{v},\vec{v}-\proj{\vec{v}\,}{\spanof{\vec{s}\,}}}$
        is linearly independent.
    \end{exparts}
    \begin{answer}
      \begin{exparts}
       \partsitem If the vector $\vec{v}\,$ is in the line then the
         orthogonal projection is \( \vec{v} \).
         To verify this
         by calculation, note that since $\vec{v}\,$ is in the line
         we have that \( \vec{v}=c_{\vec{v}}\cdot\vec{s}\, \) for some scalar
         $c_{\vec{v}}$.
         \begin{equation*}
           \frac{\vec{v}\dotprod\vec{s}}{\vec{s}\dotprod\vec{s}}\cdot\vec{s}
           =\frac{c_{\vec{v}}\cdot \vec{s}\dotprod\vec{s}}{%
                         \vec{s}\dotprod\vec{s}}\cdot\vec{s}
           =c_{\vec{v}}\cdot\frac{\vec{s}\dotprod\vec{s}}{%
                                  \vec{s}\dotprod\vec{s}}\cdot\vec{s}
           =c_{\vec{v}}\cdot 1 \cdot\vec{s}
           =\vec{v}
         \end{equation*}
          (\textit{Remark}.
          If we assume that $\vec{v}\,$ is nonzero then we can simplify 
          the above by taking $\vec{s}\,$ to be $\vec{v}$.)
       \partsitem Write $c_{\vec{p}}\vec{s}\,$ for the projection
         $\proj{\vec{v}}{\spanof{\vec{s}\,}}$. 
         Note that, by the assumption that $\vec{v}$ is not in the line,
         both $\vec{v}$ and $\vec{v}-c_{\vec{p}}\vec{s}\,$ are nonzero.
         Note also that if $c_{\vec{p}}\,$ is zero then we are actually 
         considering the one-element set $\set{\vec{v}\,}$, 
         and with $\vec{v}$ nonzero, this set 
         is necessarily linearly independent.
         Therefore, we are left considering the case that
         $c_{\vec{p}}$ is nonzero.

         Setting up a linear relationship
         \begin{equation*}
           a_1(\vec{v})+a_2(\vec{v}-c_{\vec{p}}\vec{s})=\zero
         \end{equation*}
         leads to the equation 
         $(a_1+a_2)\cdot\vec{v}=a_2c_{\vec{p}}\cdot\vec{s}$.
         Because $\vec{v}\,$ isn't in the line, the scalars 
         $a_1+a_2$ and $a_2 c_{\vec{p}}$ must both be zero.
         We handled the $c_{\vec{p}}=0$ case above, so 
         the remaining case is that $a_2=0$, and 
         this gives that $a_1=0$ also.
         Hence the set is linearly independent.
     \end{exparts}
    \end{answer}
  \item 
    \nearbydefinition{def:ProjIntoLine} requires that $\vec{s}\,$ be
    nonzero.
    Why?
    What is the right definition of the orthogonal projection 
    of a vector into the (degenerate) line spanned by the zero vector?
    \begin{answer}
      If $\vec{s}\,$ is the zero vector then the expression
          \begin{equation*}
            \proj{\vec{v}}{\spanof{\vec{s}\,}}=
            \frac{ \vec{v}\dotprod\vec{s} }{%
                    \vec{s}\dotprod\vec{s} }\cdot\vec{s}
          \end{equation*}
      contains a division by zero, and so is undefined.
      As for the right definition,
      for the projection to lie in the span of the zero vector, it 
      must be defined to be \( \zero \). 
    \end{answer}
  \item 
    Are all vectors the projection of some other vector into some line?
    \begin{answer}
      Any vector in \( \Re^n \) is the projection of some other into a
      line, provided that the dimension \( n \) is greater than one.
      (Clearly, any vector is the projection of itself
      into a line containing itself; the question is to 
      produce some vector other 
      than $\vec{v}$ that projects to \( \vec{v} \).)

      Suppose that \( \vec{v}\in\Re^n \) with \( n>1 \).
      If \( \vec{v}\neq\zero \) then we consider the line
      \( \ell=\set{c\vec{v}\suchthat c\in\Re} \) and if \( \vec{v}=\zero \)
      we take \( \ell \) to be any (non-degenerate) line at all
      (actually, we needn't distinguish between these two cases\Dash see
      the prior exercise).
      Let \( v_1,\dots,v_n \) be the components of \( \vec{v} \);
      since \( n>1 \), there are at least two.
      If some \( v_i \) is zero then the vector \( \vec{w}=\vec{e}_i \) is
      perpendicular to \( \vec{v} \).
      If none of the components is zero then
      the vector \( \vec{w}\, \) whose components are
      \( v_2,-v_1,0,\dots ,0 \) is perpendicular to \( \vec{v} \).
      In either case, observe that \( \vec{v}+\vec{w} \) does not equal
      \( \vec{v} \), and that \( \vec{v} \) is the projection of
      \( \vec{v}+\vec{w} \) into \( \ell \).
      \begin{equation*}
        \frac{ (\vec{v}+\vec{w})\dotprod\vec{v} }{ 
               \vec{v}\dotprod\vec{v} }\cdot\vec{v}
        =\bigl(\frac{ \vec{v}\dotprod\vec{v} }{ 
                     \vec{v}\dotprod\vec{v} }+
               \frac{ \vec{w}\dotprod\vec{v} }{ \
                      \vec{v}\dotprod\vec{v} }\bigr)\cdot\vec{v}
        =\frac{ \vec{v}\dotprod\vec{v} }{ \vec{v}\dotprod\vec{v} }\cdot\vec{v}
        =\vec{v}
      \end{equation*}

      We can dispose of the remaining $n=0$ and $n=1$ cases.
      The dimension \( n=0 \) case is the trivial vector space, here
      there is only one vector and so it cannot be expressed as the projection
      of a different vector.
      In the dimension $n=1$ case there is only one (non-degenerate) line,
      and every vector is in it, hence every vector is the projection only
      of itself.
    \end{answer}
  \item 
    Show that the projection of \( \vec{v}\, \) into the line spanned by
    \( \vec{s}\, \) has length equal to the absolute value of the number
    \( \vec{v}\dotprod\vec{s}\, \) divided by the length of the vector
    \( \vec{s}\, \).
    \begin{answer}
      The proof is simply a calculation.
      \begin{equation*}
        \norm{\frac{\vec{v}\dotprod\vec{s}}{\vec{s}\dotprod\vec{s}}
              \cdot\vec{s}\, }
        =
        \absval{\frac{\vec{v}\dotprod\vec{s}}{
                      \vec{s}\dotprod\vec{s}}}\cdot \norm{\vec{s}\, }
        =
        \frac{\absval{\vec{v}\dotprod\vec{s}\,}}{
                      \norm{\vec{s}\,}^2}\cdot\norm{\vec{s}\,}
        =
        \frac{\absval{\vec{v}\dotprod\vec{s}\,}}{\norm{\vec{s}\,}}
      \end{equation*} 
    \end{answer}
  \item
    Find the formula for the distance from a point to a line.
    \begin{answer}
      Because the projection of \( \vec{v} \) into the line spanned by
      \( \vec{s} \) is
      \begin{equation*}
        \frac{\vec{v}\dotprod\vec{s}}{\vec{s}\dotprod\vec{s}}\cdot\vec{s}
      \end{equation*}
      the distance squared from the point to the line is this
      (we write a vector dotted with itself $\vec{w}\dotprod\vec{w}$
      as $\vec{w}^2$).
      \begin{align*}
        \norm{\vec{v}-
                \frac{\vec{v}\dotprod\vec{s}}{\vec{s}\dotprod\vec{s}}
                \cdot\vec{s}\, }^2
        &=\vec{v}\dotprod\vec{v}
        -\vec{v}\dotprod(
              \frac{\vec{v}\dotprod\vec{s}}{\vec{s}\dotprod\vec{s}}\cdot\vec{s}
           )
        -(
              \frac{\vec{v}\dotprod\vec{s}}{\vec{s}\dotprod\vec{s}}
               \cdot\vec{s}\,
         )\dotprod\vec{v}
        +(\frac{\vec{v}\dotprod\vec{s}}{
                \vec{s}\dotprod\vec{s}}\cdot\vec{s}\,)^2     \\   
        &=\vec{v}\dotprod\vec{v}
        -2\cdot(\frac{\vec{v}\dotprod\vec{s}}{\vec{s}\dotprod\vec{s}})
           \cdot\vec{v}\dotprod\vec{s}
        +(\frac{\vec{v}\dotprod\vec{s}}{
                \vec{s}\dotprod\vec{s}})\cdot\vec{s}\dotprod\vec{s}  \\
        &=\frac{(\vec{v}\dotprod\vec{v}\,)\cdot(\vec{s}\dotprod\vec{s}\,)
                 -2\cdot(\vec{v}\dotprod\vec{s}\,)^2
                 +(\vec{v}\dotprod\vec{s}\,)^2}{%
               \vec{s}\dotprod\vec{s}}                               \\     
        &=\frac{(\vec{v}\dotprod\vec{v}\,)(\vec{s}\dotprod\vec{s}\,)
                -(\vec{v}\dotprod\vec{s}\,)^2}{\vec{s}\dotprod\vec{s}}
      \end{align*}  
    \end{answer}
  \item 
    Find the scalar \( c \) such that the point \( (cs_1,cs_2) \)
    is a minimum distance from the point \( (v_1,v_2) \)
    by using Calculus (i.e., consider the distance function, set the
    first derivative equal to zero, and solve). 
    Generalize to \( \Re^n \).
    \begin{answer}
      Because square root is a strictly increasing function, we can
      minimize \( d(c)=(cs_1-v_1)^2+(cs_2-v_2)^2 \) instead of the square root
      of \( d \).
      The derivative is 
      $dd/dc=2(cs_1-v_1)\cdot s_1 +2(cs_2-v_2)\cdot s_2$.
      Setting it equal to zero 
      $2(cs_1-v_1)\cdot s_1 +2(cs_2-v_2)\cdot s_2
        =c\cdot(2s_1^2+2s_2^2)-(v_1s_1+v_2s_2)=0$
      gives the only critical point.
      \begin{equation*}
        c=\frac{v_1s_1+v_2s_2}{{s_1}^2+{s_2}^2}
         =\frac{\vec{v}\dotprod\vec{s}}{\vec{s}\dotprod\vec{s}}
      \end{equation*}
      Now the second derivative with respect to $c$
      \begin{equation*}
        \frac{d^2\,d}{dc^2}=2{s_1}^2+2{s_2}^2
      \end{equation*}
      is strictly positive (as long as neither  \( s_1 \) nor \( s_2 \) 
      is zero, in which case the question is trivial) 
      and so the critical point is a minimum. 

      The generalization to $\Re^n$ is straightforward.
      Consider $d_n(c)=(cs_1-v_1)^2+\dots+(cs_n-v_n)^2$, take the derivative,
      etc. 
    \end{answer}
  \recommended \item \label{exer:ProjMinimizesDist}
    Let $\vec{p}$ be the orthogonal projection of 
    $\vec{v}\in\Re^n$ onto a line~$\ell$.
    Show that $\vec{p}$ is the point in the line closest to~$\vec{v}$.
    \begin{answer}
      Suppose $\vec{w}\in\ell$.
      Because this is orthogonal projection, 
      the two vectors $\vec{v}-\vec{p}$ and $\vec{p}-\vec{w}$ are at 
      a right angle.
      The Triangle Inequality applies and the hypotenuse $\vec{v}-\vec{w}$
      is therefore at least as long as $\vec{v}-\vec{p}$. 
    \end{answer}
  \item
    Prove that the orthogonal projection of a vector into a line has length
    less than or equal to that of the vector.
    \begin{answer}
      The Cauchy-Schwarz inequality 
      $\absval{\vec{v}\dotprod\vec{s}\,}
         \leq\norm{\vec{v}\,}\cdot\norm{\vec{s}\,}$ 
      gives that this fraction
      \begin{equation*}
        \norm{\frac{\vec{v}\dotprod\vec{s}}{\vec{s}\dotprod\vec{s}}
              \cdot\vec{s}\, }
        =\absval{\frac{\vec{v}\dotprod\vec{s}}{
                       \vec{s}\dotprod\vec{s}}}
         \cdot\norm{\vec{s}\, }
        =
        \frac{\absval{\vec{v}\dotprod\vec{s}\,}}{
              \norm{\vec{s}\,}^2}\cdot\norm{\vec{s}\,}
        =
        \frac{\absval{\vec{v}\dotprod\vec{s}\,}}{\norm{\vec{s}\,}}
      \end{equation*}
      when divided by \( \norm{\vec{v}\,} \) is less than or equal to one.
      That is, \( \norm{\vec{v}\,} \) is larger than or equal to the fraction.
    \end{answer}
  \recommended \item  \label{exer:LineProjFormIndOfVec}
    Show that the definition of orthogonal projection into a line 
    does not depend
    on the spanning vector:~if \( \vec{s} \) is a nonzero multiple
    of \( \vec{q} \) then
    \( (\vec{v}\dotprod\vec{s}/\vec{s}\dotprod\vec{s}\,)\cdot\vec{s} \) equals
    \( (\vec{v}\dotprod\vec{q}/\vec{q}\dotprod\vec{q}\,)\cdot\vec{q} \).
    \begin{answer}
       Write \( c\vec{s} \) for \( \vec{q} \), and
       calculate:
       \( (\vec{v}\dotprod c\vec{s}/c\vec{s}\dotprod c\vec{s}\,)\cdot c\vec{s}=
          (\vec{v}\dotprod \vec{s}/\vec{s}\dotprod \vec{s}\,)\cdot \vec{s} \).
    \end{answer}
  \item
    Consider the function mapping the plane to itself that takes 
    a vector to its projection into the line \( y=x \).
    These two each show that the map is linear, the first one in a way that
    is coordinate-bound (that is, it fixes a basis and then computes)
    and the second in a way that is more conceptual.
    \begin{exparts} 
      \partsitem Produce a matrix that describes the function's action.
      \partsitem Show that we can obtain this map by first rotating 
         everything in the plane \( \pi/4 \)~radians  clockwise, 
        then projecting into the \( x \)-axis,
        and then rotating \( \pi/4 \)~radians counterclockwise.
    \end{exparts}
    \begin{answer}
      \begin{exparts}
        \partsitem Fixing
          \begin{equation*}
            \vec{s}=\colvec[r]{1 \\ 1}
          \end{equation*}
          as the vector whose span is the line, the formula gives this
          action,
          \begin{equation*}
            \colvec{x \\ y}
             \mapsto
            \frac{\colvec{x \\ y}\dotprod\colvec[r]{1 \\ 1}}{%
                  \colvec[r]{1 \\ 1}\dotprod\colvec[r]{1 \\ 1}}\cdot\colvec[r]{1 \\ 1}
            =\frac{x+y}{2}\cdot\colvec[r]{1 \\ 1}
            =\colvec{(x+y)/2 \\ (x+y)/2}
          \end{equation*} 
          which is the effect of this matrix.
          \begin{equation*}
            \begin{mat}[r]
              1/2  &1/2  \\
              1/2  &1/2
            \end{mat}
          \end{equation*}
        \partsitem Rotating the entire plane $\pi/4$~radians clockwise
          brings the $y=x$~line to lie on the $x$-axis.
          Now projecting and then rotating back has the desired effect.
      \end{exparts}
    \end{answer}
  \item 
    For \( \vec{a},\vec{b}\in\Re^n \) let \( \vec{v}_1 \) be the
    projection of \( \vec{a} \) into the line spanned by \( \vec{b} \), let
    \( \vec{v}_2 \) be the projection of \( \vec{v}_1 \) into the line spanned
    by \( \vec{a} \), let \( \vec{v}_3 \) be the projection of \( \vec{v}_2 \)
    into the line spanned by \( \vec{b} \), etc., 
    back and forth between the spans of $\vec{a}$ and $\vec{b}$.
    That is, $\vec{v}_{i+1}$ is the projection of $\vec{v}_i$ into the
    span of $\vec{a}$ if $i+1$ is even, and  
    into the span of $\vec{b}$ if $i+1$ is odd.
    Must that sequence of vectors eventually settle down\Dash must
    there be a sufficiently large~$i$ such that 
    \( \vec{v}_{i+2} \) equals \( \vec{v}_{i} \) 
    and \( \vec{v}_{i+3} \) equals \( \vec{v}_{i+1} \)?
    If so, what is the earliest such~$i$?
    \begin{answer}
      The sequence need not settle down.
      With
      \begin{equation*}
        \vec{a}=\colvec[r]{1 \\ 0}
        \qquad
        \vec{b}=\colvec[r]{1 \\ 1}
      \end{equation*}
      the projections are these.
      \begin{equation*}
        \vec{v}_1=\colvec[r]{1/2 \\ 1/2},
        \quad
        \vec{v}_2=\colvec[r]{1/2 \\ 0},
        \quad
        \vec{v}_3=\colvec[r]{1/4 \\ 1/4},
        \quad
        \ldots
      \end{equation*}
      This sequence doesn't repeat.
      \begin{center}  \small
        \includegraphics{map/mp/ch3.82}
      \end{center}
      \end{answer}
\end{exercises}
















\subsectionoptional{Gram-Schmidt Orthogonalization}
\index{Gram-Schmidt process|(}
% \noindent\textit{This subsection is optional.
%      It is a prerequisite only for the final two sections
%      of Chapter Five.
%      Also, this subsection requires material from the previous subsection, 
%      which itself was optional.}

The prior subsection suggests
that projecting $\vec{v}$ into the line spanned by \( \vec{s} \)
decomposes that vector into two parts
\begin{center}  \small
  \vcenteredhbox{\includegraphics{map/mp/ch3.35}}
  % \begin{tabular}{@{}c@{}}\includegraphics{map/mp/ch3.35}\end{tabular}
   \qquad
   $\displaystyle \vec{v}=\proj{\vec{v}}{\spanof{\vec{s}\,}}
             \,+\,\left(\vec{v}-\proj{\vec{v}}{\spanof{\vec{s}\,}}\right)$
\end{center}
that are orthogonal and so are ``non-interacting.''
We now develop that suggestion.

\begin{definition}  \label{df:MutuallyOrthogonal}
%<*df:MutuallyOrthogonal>
Vectors \( \vec{v}_1,\dots,\vec{v}_k\in\Re^n \) are
\definend{mutually orthogonal\/}\index{orthogonal!mutually}
when any two are orthogonal:~if \( i\neq j \) then
the dot product \( \vec{v}_i\dotprod\vec{v}_j \) is zero.
%</df:MutuallyOrthogonal>
\end{definition}

\begin{theorem} \label{th:OrthoIsInd}
%<*th:OrthoIsInd>
If the vectors in a set \( \set{\vec{v}_1,\dots,\vec{v}_k}\subset\Re^n \)
are mutually orthogonal and nonzero then that set is linearly
independent.
%</th:OrthoIsInd>
\end{theorem}

\begin{proof}
%<*pf:OrthoIsInd>
Consider
\( \zero=c_1\vec{v}_1+c_2\vec{v}_2+\dots+c_k\vec{v}_k \).
For $i\in \set{1,..\,,k}$, taking the dot product of \( \vec{v}_i \)  
with both sides of the equation
$\vec{v}_i\dotprod (c_1\vec{v}_1+c_2\vec{v}_2+\dots+c_k\vec{v}_k)
   =\vec{v}_i\dotprod\zero$,
which gives   
$c_i\cdot(\vec{v}_i\dotprod\vec{v}_i)=0$,
shows that \( c_i= 0 \) since \( \vec{v}_i\neq\zero \). 
%</pf:OrthoIsInd>
\end{proof}

\begin{corollary}  \label{cor:OrthAndBigEnoughIsBasis}
%<*co:OrthAndBigEnoughIsBasis>
In a $k$~dimensional vector space,
if the vectors in a size~\( k \) set
are mutually orthogonal
and nonzero then that set is a basis for the space.
%</co:OrthAndBigEnoughIsBasis>
\end{corollary}

\begin{proof}
%<*pf:OrthAndBigEnoughIsBasis>
Any linearly independent size~\( k \) subset of a
$k$~dimensional space is a basis. 
%</pf:OrthAndBigEnoughIsBasis>
\end{proof}

Of course, the converse of \nearbycorollary{cor:OrthAndBigEnoughIsBasis} 
does not hold\Dash not every basis of every subspace
of $\Re^n$ has mutually orthogonal vectors.
However, we can get the partial converse
that for every subspace of $\Re^n$ there is at least one basis
consisting of mutually orthogonal vectors.

\begin{example}
The members $\vec{\beta}_1$ and $\vec{\beta}_2$ of this basis for $\Re^2$
are not orthogonal.
\begin{center}  \small
  $\displaystyle 
      B=\sequence{
            \colvec[r]{4 \\ 2},
            \colvec[r]{1 \\ 3} }$
  \qquad %\hspace*{3em}
  \begin{tabular}{@{}c@{}}\includegraphics{map/mp/ch3.36}\end{tabular}
\end{center}
We will derive from $B$
a new basis for the space $\sequence{\vec{\kappa}_1,\vec{\kappa}_2}$
consisting of mutually orthogonal vectors.
The first member of the new basis is just $\vec{\beta}_1$.
\begin{equation*}
  \vec{\kappa}_1=\colvec[r]{4 \\ 2}
\end{equation*}
For the second member of the new basis,  
we subtract from $\vec{\beta}_2$ the part in the direction of
$\vec{\kappa}_1$.
This leaves the part
of $\vec{\beta}_2$ that is orthogonal to $\vec{\kappa}_1$.
% (it is orthogonal by the definition of the projection into the span of 
% $\vec{\kappa}_1$).
\begin{center}  \small
   $\displaystyle \vec{\kappa}_2=\colvec[r]{1 \\ 3}
    -\proj{\colvec[r]{1 \\ 3}}{\scriptstyle\spanof{\vec{\kappa}_1}}
    =\colvec[r]{1 \\ 3}-\colvec[r]{2 \\ 1}=\colvec[r]{-1 \\ 2}$
   \qquad
  \begin{tabular}{@{}c@{}}\includegraphics{map/mp/ch3.37}\end{tabular}
\end{center} 
By the corollary $\sequence{\vec{\kappa}_1,\vec{\kappa}_2}$
is a basis for $\Re^2$.
\end{example}

\begin{definition}  \label{df:OrthogonalBasis}
%<*df:OrthogonalBasis>
An \definend{orthogonal basis\/}%
\index{orthogonal!basis}\index{basis!orthogonal}
for a vector space is a basis of mutually orthogonal vectors.
%</df:OrthogonalBasis>
\end{definition}

% The next result gives a way to produce an orthogonal basis 
% from any given starting basis.
% We first see an example.

\begin{example} \label{ex:OrthoBasisForReThree}
To produce from this basis for $\Re^3$
\begin{equation*}
  B=\sequence{
           \colvec[r]{1 \\ 1 \\ 1},
           \colvec[r]{0 \\ 2 \\ 0},
           \colvec[r]{1 \\ 0 \\ 3}
            }
\end{equation*}
an orthogonal basis, start by taking the first vector unchanged.
\begin{equation*}
   \vec{\kappa}_1=\colvec[r]{1 \\ 1 \\ 1}
\end{equation*}
Get \( \vec{\kappa}_2 \) by subtracting from
$\vec{\beta}_2$ 
its part in the direction of \( \vec{\kappa}_1 \).
\begin{equation*}
   \vec{\kappa}_2=\colvec[r]{0 \\ 2 \\ 0}
                   -\proj{\colvec[r]{0 \\ 2 \\ 0}}{\spanof{\vec{\kappa}_1}}
                 =\colvec[r]{0 \\ 2 \\ 0}-\colvec[r]{2/3 \\ 2/3 \\ 2/3}
                 =\colvec[r]{-2/3 \\ 4/3 \\ -2/3}
\end{equation*}
Find $\vec{\kappa}_3$  
by subtracting from $\vec{\beta}_3$ the part 
in the direction of \( \vec{\kappa}_1 \) and also the part 
in the direction of \( \vec{\kappa}_2 \).
\begin{equation*}
   \vec{\kappa}_3=\colvec[r]{1 \\ 0 \\ 3}
                   -\proj{\colvec[r]{1 \\ 0 \\ 3}}{\spanof{\vec{\kappa}_1}}
                   -\proj{\colvec[r]{1 \\ 0 \\ 3}}{\spanof{\vec{\kappa}_2}}
                 =\colvec[r]{-1 \\ 0 \\ 1}
\end{equation*}
As above, the corollary gives that the result is a basis for $\Re^3$.
\begin{equation*}
  \sequence{
           \colvec[r]{1 \\ 1 \\ 1},
           \colvec[r]{-2/3 \\ 4/3 \\ -2/3},
           \colvec[r]{-1 \\ 0 \\ 1}
           }
\end{equation*}
\end{example}


\begin{theorem}[Gram-Schmidt orthogonalization]
\index{orthogonalization}\index{basis!orthogonalization}
\label{th:GramSchmidt}
%<*th:GramSchmidt>
If \( \sequence{\vec{\beta}_1,\ldots\vec{\beta}_k} \)
is a basis for a subspace of \( \Re^n \) then the vectors
\begin{align*}
  \vec{\kappa}_1
  &=\vec{\beta}_1    \\
  \vec{\kappa}_2
  &=\vec{\beta}_2-\proj{\vec{\beta}_2}{\spanof{\vec{\kappa}_1}}    \\
  \vec{\kappa}_3
  &=\vec{\beta}_3
  -\proj{\vec{\beta}_3}{\spanof{\vec{\kappa}_1}}
  -\proj{\vec{\beta}_3}{\spanof{\vec{\kappa}_2}}    \\
  &\vdotswithin{=}           \\
  \vec{\kappa}_k
  &=\vec{\beta}_k
  -\proj{\vec{\beta}_k}{\spanof{\vec{\kappa}_1}}-\cdots
  -\proj{\vec{\beta}_k}{\spanof{\vec{\kappa}_{k-1}}}
\end{align*}
form an orthogonal basis for the
same subspace.
%</th:GramSchmidt>
\end{theorem}

\begin{remark}
This is restricted to $\Re^n$ only because we have not given a 
definition of orthogonality for other spaces.
\end{remark}

\begin{proof}
%<*pf:GramSchmidt0>
We will use induction to check that each \( \vec{\kappa}_i \) is nonzero,
is in the span of $\sequence{\vec{\beta}_1,\ldots\,\vec{\beta}_i}$,
and is orthogonal to all preceding vectors
\( \vec{\kappa}_1\dotprod\vec{\kappa}_i= \cdots
   =\vec{\kappa}_{i-1}\dotprod\vec{\kappa}_{i}=0 \).
Then 
\nearbycorollary{cor:OrthAndBigEnoughIsBasis} gives that
\( \sequence{\vec{\kappa}_1,\ldots\,\vec{\kappa}_k} \) 
is a basis for the same space as is the starting basis.
%</pf:GramSchmidt0>

%<*pf:GramSchmidt1>
We shall only cover the cases up to \( i=3 \), to give the
sense of the argument.
The full argument is \nearbyexercise{exer:GramSchmidt}.
%</pf:GramSchmidt1>

%<*pf:GramSchmidt2>
The \( i=1 \) case is trivial; taking \( \vec{\kappa}_1 \) to be 
\( \vec{\beta}_1 \) 
makes it a nonzero vector since $\vec{\beta}_1$ is a member of a basis, 
it is obviously in the span of $\sequence{\vec{\beta}_1}$,
and the `orthogonal to all preceding vectors' condition
is satisfied vacuously.
%</pf:GramSchmidt2>

%<*pf:GramSchmidt3>
In the \( i=2 \) case the expansion
\begin{equation*}
  \vec{\kappa}_2=\vec{\beta}_2
      -\proj{\vec{\beta}_2}{\spanof{\vec{\kappa}_1}}=
  \vec{\beta}_2
  -\frac{\vec{\beta}_2\dotprod\vec{\kappa}_1}{
         \vec{\kappa}_1\dotprod\vec{\kappa}_1}
   \cdot\vec{\kappa}_1
  =
  \vec{\beta}_2
  -\frac{\vec{\beta}_2\dotprod\vec{\kappa}_1}{
         \vec{\kappa}_1\dotprod\vec{\kappa}_1}
   \cdot\vec{\beta}_1
\end{equation*}
shows that $\vec{\kappa}_2\neq \zero$ or else this
would be a non-trivial linear dependence among the \( \vec{\beta} \)'s
(it is nontrivial because the coefficient of $\vec{\beta}_2$ is $1$).
It also shows that 
$\vec{\kappa}_2$ is in the span of $\sequence{\vec{\beta}_1,\vec{\beta}_2}$.
And, $\vec{\kappa}_2$ is orthogonal to the only preceding vector
\begin{equation*}
   \vec{\kappa}_1\dotprod\vec{\kappa}_2=
   \vec{\kappa}_1\dotprod(\vec{\beta}_2
      -\proj{\vec{\beta}_2}{\spanof{\vec{\kappa}_1}})=0
\end{equation*}
because this projection is orthogonal.
%</pf:GramSchmidt3>

%<*pf:GramSchmidt4>
The \( i=3 \) case is the same as the $i=2$ case except for one detail.
As in the $i=2$ case, expand the definition. 
\begin{align*}
  \vec{\kappa}_3
  &=\vec{\beta}_3
  -\frac{\vec{\beta}_3\dotprod\vec{\kappa}_1}{
         \vec{\kappa}_1\dotprod\vec{\kappa}_1}
   \cdot\vec{\kappa}_1
  -\frac{\vec{\beta}_3\dotprod\vec{\kappa}_2}{
         \vec{\kappa}_2\dotprod\vec{\kappa}_2}
   \cdot\vec{\kappa}_2  \\
  &=\vec{\beta}_3
  -\frac{\vec{\beta}_3\dotprod\vec{\kappa}_1}{
         \vec{\kappa}_1\dotprod\vec{\kappa}_1}
   \cdot\vec{\beta}_1
  -\frac{\vec{\beta}_3\dotprod\vec{\kappa}_2}{
         \vec{\kappa}_2\dotprod\vec{\kappa}_2}
  \cdot\bigl(\vec{\beta}_2
  -\frac{\vec{\beta}_2\dotprod\vec{\kappa}_1}{
         \vec{\kappa}_1\dotprod\vec{\kappa}_1}
   \cdot\vec{\beta}_1\bigr)
\end{align*}
By the first line $\vec{\kappa}_3\neq\zero$,
since $\vec{\beta}_3$ isn't in the 
span~$\spanof{\vec{\beta}_1,\vec{\beta}_2}$ and 
therefore by the inductive hypothesis it isn't in the
span~$\spanof{\vec{\kappa}_1,\vec{\kappa}_2}$.
By the second line $\vec{\kappa}_3$ is in the span of the first three
$\vec{\beta}$'s.
Finally, 
the calculation below shows that 
$\vec{\kappa}_3$ is orthogonal to $\vec{\kappa}_1$.
%</pf:GramSchmidt4>
%<*pf:GramSchmidt5>
\begin{align*}
   \vec{\kappa}_1\dotprod\vec{\kappa}_3
   &=\vec{\kappa}_1\dotprod\bigl(\;\vec{\beta}_3
      -\proj{\vec{\beta}_3}{\spanof{\vec{\kappa}_1}}
      -\proj{\vec{\beta}_3}{\spanof{\vec{\kappa}_2}}\;\bigr) \\
   &=\vec{\kappa}_1\dotprod\bigl(\vec{\beta}_3
          -\proj{\vec{\beta}_3}{\spanof{\vec{\kappa}_1}}\bigr)
    -\vec{\kappa}_1\dotprod\proj{\vec{\beta}_3}{\spanof{\vec{\kappa}_2}} \\
   &=0
\end{align*}
(Here is the difference with the $i=2$ case:
as happened for $i=2$ 
the first term is~$0$ because this projection is orthogonal,
but here the second term in the second line
is~$0$ because \( \vec{\kappa}_1 \) is orthogonal
to \( \vec{\kappa}_2 \) and so is orthogonal to any vector
in the line spanned by \( \vec{\kappa}_2 \).)
A similar check shows that $\vec{\kappa}_3$ is also 
orthogonal to $\vec{\kappa}_2$.
%</pf:GramSchmidt5>
\end{proof}

In addition to having the vectors in the basis be orthogonal, we can also
\definend{normalize}\index{normalize, vector} each vector by dividing
by its length, to end with an 
\definend{orthonormal basis}.\index{basis!orthonormal}%
\index{orthonormal basis}. 

\begin{example}
From the orthogonal basis of
\nearbyexample{ex:OrthoBasisForReThree}, normalizing
produces this orthonormal basis.
\begin{equation*}
  \sequence{
           \colvec[r]{1/\sqrt{3} \\ 1/\sqrt{3} \\ 1/\sqrt{3}},
           \colvec[r]{-1/\sqrt{6} \\ 2/\sqrt{6} \\ -1/\sqrt{6}},
           \colvec[r]{-1/\sqrt{2} \\ 0 \\ 1/\sqrt{2}}
           }
\end{equation*}
\end{example}

\noindent Besides its intuitive appeal, and its analogy with the 
standard basis $\stdbasis_n$ for $\Re^n$, an orthonormal basis also simplifies
some computations.
\nearbyexercise{exer:OrthoRepEasy} is an example.




\begin{exercises}
  \item Normalize the lengths of these vectors.
    \begin{exparts*}
      \partsitem $\colvec{1 \\ 2}$
      \partsitem $\colvec{-1 \\ 3 \\ 0}$
      \partsitem $\colvec{1 \\ -1}$
    \end{exparts*}
    \begin{answer}
      \begin{exparts}
        \partsitem $\colvec{1/\sqrt{5} \\ 2/\sqrt{5}}
             =\frac{1}{\sqrt{5}}\cdot\colvec{1 \\ 2}$
        \partsitem $\frac{1}{\sqrt{10}}\cdot\colvec{-1 \\ 3 \\ 0}$
        \partsitem $\frac{1}{\sqrt{2}}\cdot\colvec{1 \\ -1}$
      \end{exparts}
    \end{answer}
  \recommended \item Perform Gram-Schmidt on this basis for~$\Re^2$.
    \begin{equation*}
      \sequence{\colvec{1 \\ 1}, \colvec{-1 \\ 2}}
    \end{equation*}
    Check that the resulting vectors are orthogonal.
    \begin{answer}
      Call the given basis $B=\sequence{\vec{\beta}_1,\vec{\beta}_2}$.
      First, $\vec{\kappa}_1=\vec{\beta}_1$.
      For the other, $\vec{\kappa}_2=\vec{\beta}_2-\proj{\vec{\beta}_2}{\spanof{\vec{\kappa}_1}}$.
      \begin{equation*}
        \vec{\kappa}_2=
        \colvec{-1 \\ 2}-\frac{\colvec{-1 \\ 2}\dotprod\colvec{1 \\ 1}}{\colvec{1 \\ 1}\dotprod\colvec{1 \\ 1}}\cdot\colvec{1 \\ 1}
        =\colvec{-3/2 \\ 3/2}
      \end{equation*}
      To check that they are orthogonal, just note that their dot product is 
      zero.
    \end{answer}
  \recommended \item 
    Perform the Gram-Schmidt process on this basis for $\Re^3$.
    \begin{equation*}
      \sequence{
                         \colvec{1 \\ 2 \\ 3},
                         \colvec{2 \\ 1 \\ -3},
                         \colvec{3 \\ 3 \\ 3}
                         }  
    \end{equation*}
    \begin{answer}
      Call the given basis 
      $B=\sequence{\vec{\beta}_1,\vec{\beta}_2,\vec{\beta}_3}$.
      First, $\vec{\kappa}_1=\vec{\beta}_1$.

      Next,
      $\vec{\kappa}_2
          =\vec{\beta}_2-\proj{\vec{\beta}_2}{\spanof{\vec{\kappa}_1}}$.
      \begin{equation*}
        \vec{\kappa}_2=
        \colvec{2 \\ 1 \\ -3}-\frac{\colvec{2 \\ 1 \\ -3}\dotprod\colvec{1 \\ 2 \\ 3}}{\colvec{1 \\ 2 \\ 3}\dotprod\colvec{1 \\ 2 \\ 3}}\cdot\colvec{1 \\ 2 \\ 3}
        =\colvec{33/14 \\ 24/14 \\ -27/14}
      \end{equation*}

      For the third, this is the formula
      \begin{equation*}
         \vec{\kappa}_3
          =\vec{\beta}_3
           -\proj{\vec{\beta}_3}{\spanof{\vec{\kappa}_1}}
           -\proj{\vec{\beta}_3}{\spanof{\vec{\kappa}_2}}    
      \end{equation*} 
      and here is the calculation.
      \begin{equation*}
        \colvec{3 \\ 3 \\ 3}
        -\frac{\colvec{3 \\ 3 \\ 3}\dotprod\colvec{1 \\ 2 \\ 3}}{\colvec{1 \\ 2 \\ 3}\dotprod\colvec{1 \\ 2 \\ 3}}\cdot\colvec{1 \\ 2 \\ 3}
        -\frac{\colvec{3 \\ 3 \\ 3}\dotprod\colvec{33/14 \\ 24/14 \\ -27/14}}{\colvec{33/14 \\ 24/14 \\ -27/14}\dotprod\colvec{33/14 \\ 24/14 \\ -27/14}}\cdot\colvec{33/14 \\ 24/14 \\ -27/14}
       =\colvec{9/19 \\ -9/19 \\ 3/19}
      \end{equation*}
      % 1 - 18/14-(0/14)/(1274/196) times \colvec{3 \\ 3 \\ 3}
    \end{answer}
  \recommended \item 
    Perform Gram-Schmidt on each of these bases
    for $\Re^2$.
    \begin{exparts*}
      \partsitem \( \sequence{
                         \colvec[r]{1 \\ 1},
                         \colvec[r]{2 \\ 1}
                         }  \)
      \partsitem \( \sequence{
                         \colvec[r]{0 \\ 1},
                         \colvec[r]{-1 \\ 3}
                         }  \)
      \partsitem \( \sequence{
                         \colvec[r]{0 \\ 1},
                         \colvec[r]{-1 \\ 0}
                         }  \)
    \end{exparts*}
    Then turn those orthogonal bases into orthonormal bases.
    \begin{answer}
      \begin{exparts}
       \partsitem 
        \begin{align*}
          \vec{\kappa}_1 &= \colvec[r]{1 \\ 1}           \\
          \vec{\kappa}_2
            &=
            \colvec[r]{2 \\ 1}
            -\proj{\colvec[r]{2 \\ 1}}{\spanof{\vec{\kappa}_1}}  
            =
            \colvec[r]{2 \\ 1}
            -\frac{\colvec[r]{2 \\ 1}\dotprod\colvec[r]{1 \\ 1}}{%
                    \colvec[r]{1 \\ 1}\dotprod\colvec[r]{1 \\ 1}}
            \cdot\colvec[r]{1 \\ 1}                                \\
            &\quad=
            \colvec[r]{2 \\ 1}
            -\frac{3}{2}
            \cdot\colvec[r]{1 \\ 1}                                
            =\colvec[r]{1/2 \\ -1/2}
        \end{align*}
        This is the corresponding orthonormal basis.
        \begin{equation*}
          \sequence{
              \colvec[r]{1/\sqrt{2} \\ 1/\sqrt{2}},
              \colvec[r]{\sqrt{2}/2 \\ -\sqrt{2}/2}
                }
        \end{equation*}
       \partsitem 
        \begin{align*}
          \vec{\kappa}_1 &= \colvec[r]{0 \\ 1}           \\
          \vec{\kappa}_2
            &=
            \colvec[r]{-1 \\ 3}
            -\proj{\colvec[r]{-1 \\ 3}}{\spanof{\vec{\kappa}_1}}  
            =
            \colvec[r]{-1 \\ 3}
            -\frac{\colvec[r]{-1 \\ 3}\dotprod\colvec[r]{0 \\ 1}}{%
                    \colvec[r]{0 \\ 1}\dotprod\colvec[r]{0 \\ 1}}
            \cdot\colvec[r]{0 \\ 1}                                \\
            &\quad=
            \colvec[r]{-1 \\ 3}
            -\frac{3}{1}
            \cdot\colvec[r]{0 \\ 1}                              
            =\colvec[r]{-1 \\ 0}
        \end{align*}
        Here is the orthonormal basis.
        \begin{equation*}
          \sequence{
                \colvec[r]{0 \\ 1},
                \colvec[r]{-1 \\ 0}
                }
        \end{equation*}
       \partsitem 
        \begin{align*}
          \vec{\kappa}_1 &= \colvec[r]{0 \\ 1}           \\
          \vec{\kappa}_2
            &=
            \colvec[r]{-1 \\ 0}
            -\proj{\colvec[r]{-1 \\ 0}}{\spanof{\vec{\kappa}_1}}  
            =
            \colvec[r]{-1 \\ 0}
            -\frac{\colvec[r]{-1 \\ 0}\dotprod\colvec[r]{0 \\ 1}}{%
                    \colvec[r]{0 \\ 1}\dotprod\colvec[r]{0 \\ 1}}
            \cdot\colvec[r]{0 \\ 1}                               \\ 
            &\quad=
            \colvec[r]{-1 \\ 0}
            -\frac{0}{1}
            \cdot\colvec[r]{0 \\ 1}                                
            =\colvec[r]{-1 \\ 0}
        \end{align*}
      \end{exparts}
      This is the associated orthonormal basis.
      \begin{equation*}
        \sequence{
              \colvec[r]{0 \\ 1},
              \colvec[r]{-1 \\ 0}
              }
      \end{equation*}
    \end{answer}
  \item 
    Perform the Gram-Schmidt process on each of these bases
    for $\Re^3$.
    \begin{exparts*}
      \partsitem \( \sequence{
                         \colvec[r]{2 \\ 2 \\ 2},
                         \colvec[r]{1 \\ 0 \\ -1},
                         \colvec[r]{0 \\ 3 \\ 1}
                         }  \)
      \partsitem \( \sequence{
                         \colvec[r]{1 \\ -1 \\ 0},
                         \colvec[r]{0 \\ 1 \\ 0},
                         \colvec[r]{2 \\ 3 \\ 1}
                         }  \)
    \end{exparts*}
    Then turn those orthogonal bases into orthonormal bases.
    \begin{answer} 
      \begin{exparts}
       \partsitem The first basis vector is unchanged.
        \begin{equation*}
          \vec{\kappa}_1 = \colvec[r]{2 \\ 2 \\ 2}          
        \end{equation*}
        The second one comes from this calculation.
        \begin{align*}
          \vec{\kappa}_2
            &=
            \colvec[r]{1 \\ 0 \\ -1}
            -\proj{\colvec[r]{1 \\ 0 \\ -1}}{\spanof{\vec{\kappa}_1}}  
            =
            \colvec[r]{1 \\ 0 \\ -1}
            -\frac{\colvec[r]{1 \\ 0 \\ -1}\dotprod\colvec[r]{2 \\ 2 \\ 2}}{%
                    \colvec[r]{2 \\ 2 \\ 2}\dotprod\colvec[r]{2 \\ 2 \\ 2}}
            \cdot\colvec[r]{2 \\ 2 \\ 2}                                   \\
            &=
            \colvec[r]{1 \\ 0 \\ -1}
            -\frac{0}{12}
            \cdot\colvec[r]{2 \\ 2 \\ 2}                                
            =\colvec[r]{1 \\ 0 \\ -1}                              
        \end{align*}
        For the third the arithmetic is uglier but it is a straightforward
        calculation.
        \begin{align*}
          \vec{\kappa}_3
            &=
            \colvec[r]{0 \\ 3 \\ 1}
            -\proj{\colvec[r]{0 \\ 3 \\ 1}}{\spanof{\vec{\kappa}_1}}  
            -\proj{\colvec[r]{0 \\ 3 \\ 1}}{\spanof{\vec{\kappa}_2}}   \\ 
            &=
            \colvec[r]{0 \\ 3 \\ 1}
            -\frac{\colvec[r]{0 \\ 3 \\ 1}\dotprod\colvec[r]{2 \\ 2 \\ 2}}{%
                    \colvec[r]{2 \\ 2 \\ 2}\dotprod\colvec[r]{2 \\ 2 \\ 2}}
            \cdot\colvec[r]{2 \\ 2 \\ 2}                                
            -\frac{\colvec[r]{0 \\ 3 \\ 1}\dotprod\colvec[r]{1 \\ 0 \\ -1}}{%
                    \colvec[r]{1 \\ 0 \\ -1}\dotprod\colvec[r]{1 \\ 0 \\ -1}}
            \cdot\colvec[r]{1 \\ 0 \\ -1}                                   \\
            &\hbox{}\quad=
            \colvec[r]{0 \\ 3 \\ 1}
            -\frac{8}{12}
            \cdot\colvec[r]{2 \\ 2 \\ 2}                                
            -\frac{-1}{2}
            \cdot\colvec[r]{1 \\ 0 \\ -1}                              
            =\colvec[r]{-5/6 \\ 5/3 \\ -5/6}
        \end{align*}
        This is the orthonormal basis.
        \begin{equation*}
          \sequence{
              \colvec[r]{1/\sqrt{3} \\ 1/\sqrt{3} \\ 1/\sqrt{3}},
              \colvec[r]{1/\sqrt{2} \\ 0 \\ -1/\sqrt{2}},
              \colvec[r]{-1/\sqrt{6} \\ 2/\sqrt{6} \\ -1/\sqrt{6}}
                }
        \end{equation*}
       \partsitem The first basis vector is what was given.
        \begin{equation*}
          \vec{\kappa}_1 = \colvec[r]{1 \\ -1 \\ 0}           
        \end{equation*}
        The second is here.
        \begin{align*}
          \vec{\kappa}_2
            &=
            \colvec[r]{0 \\ 1 \\ 0}
            -\proj{\colvec[r]{0 \\ 1 \\ 0}}{\spanof{\vec{\kappa}_1}}  
            =
            \colvec[r]{0 \\ 1 \\ 0}
            -\frac{\colvec[r]{0 \\ 1 \\ 0}\dotprod\colvec[r]{1 \\ -1 \\ 0}}{%
                    \colvec[r]{1 \\ -1 \\ 0}\dotprod\colvec[r]{1 \\ -1 \\ 0}}
            \cdot\colvec[r]{1 \\ -1 \\ 0}                                   \\ 
            &=
            \colvec[r]{0 \\ 1 \\ 0}
            -\frac{-1}{2}
            \cdot\colvec[r]{1 \\ -1 \\ 0}                                
            =\colvec[r]{1/2 \\ 1/2 \\ 0}                                  
         \end{align*}
         Here is the third.
        \begin{align*}
          \vec{\kappa}_3
            &=
            \colvec[r]{2 \\ 3 \\ 1}
            -\proj{\colvec[r]{2 \\ 3 \\ 1}}{\spanof{\vec{\kappa}_1}}  
            -\proj{\colvec[r]{2 \\ 3 \\ 1}}{\spanof{\vec{\kappa}_2}}       \\ 
            &=
            \colvec[r]{2 \\ 3 \\ 1}
            -\frac{\colvec[r]{2 \\ 3 \\ 1}\dotprod\colvec[r]{1 \\ -1 \\ 0}}{%
                    \colvec[r]{1 \\ -1 \\ 0}\dotprod\colvec[r]{1 \\ -1 \\ 0}}
            \cdot\colvec[r]{1 \\ -1 \\ 0}                                
            -\frac{\colvec[r]{2 \\ 3 \\ 1}\dotprod\colvec[r]{1/2 \\ 1/2 \\ 0}}{%
                    \colvec[r]{1/2 \\ 1/2 \\ 0}\dotprod\colvec[r]{1/2 \\ 1/2 \\ 0}}
            \cdot\colvec[r]{1/2 \\ 1/2 \\ 0}                   \\
            &=
            \colvec[r]{2 \\ 3 \\ 1}
            -\frac{-1}{2}
            \cdot\colvec[r]{1 \\ -1 \\ 0}                                
            -\frac{5/2}{1/2}
            \cdot\colvec[r]{1/2 \\ 1/2 \\ 0} 
            =\colvec[r]{0 \\ 0 \\ 1}
        \end{align*}
        Here is the associated orthonormal basis.
        \begin{equation*}
          \sequence{
              \colvec[r]{1/\sqrt{2} \\ -1/\sqrt{2} \\ 0},
              \colvec[r]{1/\sqrt{2} \\ 1/\sqrt{2} \\ 0}
              \colvec[r]{0 \\ 0 \\ 1}
                }
        \end{equation*}
      \end{exparts}
    \end{answer}
   \recommended \item 
       Find an orthonormal basis for this subspace of $\Re^3$:~the
       plane $x-y+z=0$.
       \begin{answer}
         We can parametrize the given space can in this way.
         \begin{equation*}
           \set{\colvec{x \\ y \\ z} \suchthat x=y-z}
           =\set{\colvec[r]{1 \\ 1 \\ 0}\cdot y+\colvec[r]{-1 \\ 0 \\ 1}\cdot z 
                     \suchthat y,z\in\Re}
         \end{equation*}
         So we take the basis
         \begin{equation*}
           \sequence{\colvec[r]{1 \\ 1 \\ 0},
                     \colvec[r]{-1 \\ 0 \\ 1}}
         \end{equation*}
         apply the Gram-Schmidt process to get this first basis vector
         \begin{equation*}
           \vec{\kappa}_1 = \colvec[r]{1 \\ 1 \\ 0}             
         \end{equation*}
         and this second one.
         \begin{align*}
           \vec{\kappa}_2
           &=
           \colvec[r]{-1 \\ 0 \\ 1}-\proj{\colvec[r]{-1 \\ 0 \\ 1}}{%
                                        \spanof{\vec{\kappa}_1}} 
           =
           \colvec[r]{-1 \\ 0 \\ 1}
            -\frac{\colvec[r]{-1 \\ 0 \\ 1}\dotprod\colvec[r]{1 \\ 1 \\ 0}}{%
                    \colvec[r]{1 \\ 1 \\ 0}\dotprod\colvec[r]{1 \\ 1 \\ 0}}
              \cdot\colvec[r]{1 \\ 1 \\ 0}                                \\
           &=
           \colvec[r]{-1 \\ 0 \\ 1}
            -\frac{-1}{2}\cdot\colvec[r]{1 \\ 1 \\ 0}                  
           =\colvec[r]{-1/2 \\ 1/2 \\ 1}
         \end{align*}
         and then normalize.
         \begin{equation*}
           \sequence{
                    \colvec[r]{1/\sqrt{2} \\ 1/\sqrt{2} \\ 0},
                    \colvec[r]{-1/\sqrt{6} \\ 1/\sqrt{6} \\ 2/\sqrt{6}}
                    }
         \end{equation*}
       \end{answer}
   \item 
     Find an orthonormal basis for this subspace of $\Re^4$.
     \begin{equation*}
       \set{\colvec{x \\ y \\ z \\ w}\suchthat x-y-z+w=0\text{\ and\ }x+z=0}
     \end{equation*}
     \begin{answer}
       Reducing the linear system
       \begin{equation*}
         \begin{linsys}{4}
           x  &-  &y  &-  &z  &+  &w  &=  &0  \\
           x  &   &   &+  &z  &   &   &=  &0
         \end{linsys}
         \grstep{-\rho_1+\rho_2}
         \begin{linsys}{4}
           x  &-  &y  &-  &z  &+  &w  &=  &0  \\
              &   &y  &+  &2z &-  &w  &=  &0
         \end{linsys}
       \end{equation*}
       and parametrizing gives this description of the subspace.
       \begin{equation*}
         \set{\colvec[r]{-1 \\ -2 \\ 1 \\ 0}\cdot z
              +\colvec[r]{0 \\ 1 \\ 0 \\ 1}\cdot w
              \suchthat z,w\in\Re}
       \end{equation*}
       So we take the basis,
       \begin{equation*}
         \sequence{
                  \colvec[r]{-1 \\ -2 \\ 1 \\ 0},
                  \colvec[r]{0 \\ 1 \\ 0 \\ 1}
                  }
       \end{equation*}
       go through the Gram-Schmidt process
       with the first 
       \begin{equation*}
         \vec{\kappa}_1 = \colvec[r]{-1 \\ -2 \\ 1 \\ 0}   
       \end{equation*}
       and second basis vectors
       \begin{align*}
         \vec{\kappa}_2
          &=
          \colvec[r]{0 \\ 1 \\ 0 \\ 1}
          -\proj{\colvec[r]{0 \\ 1 \\ 0 \\ 1}}{\spanof{\vec{\kappa}_1}}   
          =
          \colvec[r]{0 \\ 1 \\ 0 \\ 1}
         -\frac{\colvec[r]{0 \\ 1 \\ 0 \\ 1}\dotprod\colvec[r]{-1 \\ -2 \\ 1 \\ 0}}{%
             \colvec[r]{-1 \\ -2 \\ 1 \\ 0}\dotprod\colvec[r]{-1 \\ -2 \\ 1 \\ 0}} 
          \cdot\colvec[r]{-1 \\ -2 \\ 1 \\ 0}                            \\
          &=
          \colvec[r]{0 \\ 1 \\ 0 \\ 1}
         -\frac{-2}{6} 
          \cdot\colvec[r]{-1 \\ -2 \\ 1 \\ 0}  
          =\colvec[r]{-1/3 \\ 1/3 \\ 1/3 \\ 1}
       \end{align*}
       and finish by normalizing.
       \begin{equation*}
         \sequence{
               \colvec[r]{-1/\sqrt{6} \\ -2/\sqrt{6} \\ 1/\sqrt{6} \\ 0},
               \colvec[r]{-\sqrt{3}/6 \\ \sqrt{3}/6 \\ \sqrt{3}/6 \\ \sqrt{3}/2}
                  }
       \end{equation*}
     \end{answer}
  \item 
     Show that any linearly independent subset of \( \Re^n \) can be
     orthogonalized without changing its span.
     \begin{answer} 
       A linearly independent subset of $\Re^n$ is a basis for its 
       own span.
       Apply \nearbytheorem{th:GramSchmidt}.

       \textit{Remark}.
       Here's why the phrase `linearly independent' is in the question.
       Dropping the phrase would require us to worry about two things.
       The first thing to worry about is that
       when we do the Gram-Schmidt process on a linearly dependent 
       set then we get some zero vectors.
       For instance, with
       \begin{equation*}
         S=\set{\colvec[r]{1 \\ 2},\colvec[r]{3 \\ 6}}
       \end{equation*}
       we would get this.
       \begin{equation*}
         \vec{\kappa}_1   =   \colvec[r]{1 \\ 2}
         \qquad 
         \vec{\kappa}_2
           =
           \colvec[r]{3 \\ 6}-\proj{\colvec[r]{3 \\ 6}}{\spanof{\vec{\kappa}_1}} 
           =
           \colvec[r]{0 \\ 0}
       \end{equation*}
       This first thing is not so bad because
       the zero vector is by definition orthogonal to every other vector,
       so we could accept this situation 
       as yielding an orthogonal set (although
       it of course can't be normalized), or we just could modify the
       Gram-Schmidt procedure to throw out any zero vectors.
       The second thing to worry about if we drop the phrase 
       `linearly independent' from the question is that the set might
       be infinite.
       Of course, any subspace of the finite-dimensional $\Re^n$
       must also be finite-dimensional so only finitely many of its
       members are linearly independent, but nonetheless,
       a ``process'' that examines the vectors in an infinite set one at a
       time would at least require some more elaboration in this question. 
       A linearly independent subset of $\Re^n$ is automatically
       finite\Dash in fact, of
       size~$n$ or less\Dash so the `linearly independent' 
       phrase obviates these concerns.
     \end{answer}
  \item What happens if we try to apply the Gram-Schmidt process to
    a finite set that is not a basis?
    \begin{answer}
      If that set is not linearly independent, then we get a zero vector.
      Otherwise (if our set is linearly independent but does not span the
      space), we are doing Gram-Schmidt on a set that is a basis for a
      subspace and so we get an orthogonal basis for a subspace.  
    \end{answer}
   \recommended \item
     What happens if we apply the Gram-Schmidt process to 
     a basis that is already orthogonal?
     \begin{answer}
       The process leaves the basis unchanged.  
     \end{answer}
  \item 
     Let $\sequence{\vec{\kappa}_1,\dots,\vec{\kappa}_k}$
     be a set of mutually orthogonal vectors in $\Re^n$.
     \begin{exparts}
       \partsitem Prove that for any $\vec{v}$ in the space, the vector
         $\vec{v}-(\proj{\vec{v}\,}{\spanof{\vec{\kappa}_1}}
           +\dots+\proj{\vec{v}\,}{\spanof{\vec{v}_k}})$
         is orthogonal to each of $\vec{\kappa}_1$, \ldots, $\vec{\kappa}_k$.
       \partsitem Illustrate the prior item in $\Re^3$ by using $\vec{e}_1$ as
         $\vec{\kappa}_1$, using $\vec{e}_2$ as $\vec{\kappa}_2$, and
         taking $\vec{v}$ to have components $1$, $2$, and $3$.
       \partsitem Show that $\proj{\vec{v}\,}{\spanof{\vec{\kappa}_1}}
         +\dots+\proj{\vec{v}\,}{\spanof{\vec{v}_k}}$ is the vector in the
         span of the set of $\vec{\kappa}$'s that is closest to $\vec{v}$.
         \textit{Hint}.  To the illustration done for the prior part,
         add a vector $d_1\vec{\kappa}_1+d_2\vec{\kappa}_2$
         and apply the Pythagorean Theorem to the resulting triangle.
     \end{exparts}
     \begin{answer}
       \begin{exparts}
         \partsitem The argument is as in the $i=3$ case of the proof
           of \nearbytheorem{th:GramSchmidt}.
           The dot product
           \begin{equation*}
             \vec{\kappa}_i\dotprod
             \left(\vec{v}-\proj{\vec{v}\,}{\spanof{\vec{\kappa}_1}}
               -\dots-\proj{\vec{v}\,}{\spanof{\vec{v}_k}}\right)
           \end{equation*}
           can be written as the sum of terms of the form
           $-\vec{\kappa}_i\dotprod\proj{\vec{v}\,}{\spanof{\vec{\kappa}_j}}$ 
           with $j\neq i$, 
           and the term 
           $\vec{\kappa}_i\dotprod(\vec{v}-
                                 \proj{\vec{v}\,}{\spanof{\vec{\kappa}_i}})$.
           The first kind of term equals zero because the $\vec{\kappa}$'s
           are mutually orthogonal.
           The other term is zero because this projection is orthogonal
           (that is, the projection definition makes it zero:
           $\vec{\kappa}_i\dotprod(\vec{v}
               -\proj{\vec{v}\,}{\spanof{\vec{\kappa}_i}})
            =\vec{\kappa}_i\dotprod\vec{v}-
              \vec{\kappa}_i\dotprod
               \left((\vec{v}\dotprod\vec{\kappa}_i)/(%
                      \vec{\kappa}_i\dotprod\vec{\kappa}_i)\right)
                   \cdot\vec{\kappa}_i$
           equals, after all of the cancellation is done, zero).
         \partsitem The vector $\vec{v}$ is in black and the
           vector $\proj{\vec{v}\,}{\spanof{\vec{\kappa}_1}}
                    +\proj{\vec{v}\,}{\spanof{\vec{v}_2}}
                   =1\cdot\vec{e}_1+2\cdot\vec{e}_2$ is in gray.
           \begin{center}  \small
             \includegraphics{map/mp/ch3.83}
              \end{center}
              The vector
              $\vec{v}-(\proj{\vec{v}\,}{\spanof{\vec{\kappa}_1}}
                    +\proj{\vec{v}\,}{\spanof{\vec{v}_2}})$
              lies on the dotted line connecting the black vector to the 
              gray one, that is, it is orthogonal to the $xy$-plane.
          \partsitem We get this diagram by following the hint.
           \begin{center}  \small
             \includegraphics{map/mp/ch3.84}
            \end{center}
            The dashed triangle has a right angle where 
            the gray vector $1\cdot\vec{e}_1+2\cdot\vec{e}_2$
            meets the vertical dashed line
            $\vec{v}-(1\cdot\vec{e}_1+2\cdot\vec{e}_2)$; this is what
            first item of this question proved.
            The Pythagorean theorem then gives that the hypotenuse\Dash the
            segment from $\vec{v}$ to any other vector\Dash is longer than
            the vertical dashed line.
       
            More formally, writing  $\proj{\vec{v}\,}{\spanof{\vec{\kappa}_1}}
                 +\dots+\proj{\vec{v}\,}{\spanof{\vec{v}_k}}$ as
            $c_1\cdot\vec{\kappa}_1+\dots+c_k\cdot\vec{\kappa}_k$, 
            consider any other vector in the span
            $d_1\cdot\vec{\kappa}_1+\dots+d_k\cdot\vec{\kappa}_k$. 
            Note that
            \begin{multline*}
              \vec{v}-(d_1\cdot\vec{\kappa}_1+\dots+d_k\cdot\vec{\kappa}_k)  \\
              \begin{aligned}
              &=
              \bigl(\vec{v}
                    -(c_1\cdot\vec{\kappa}_1+\dots+c_k\cdot\vec{\kappa}_k)
              \bigr)                                                     \\
              &\quad+\bigl((c_1\cdot\vec{\kappa}_1+\dots+c_k\cdot\vec{\kappa}_k)
                     -(d_1\cdot\vec{\kappa}_1+\dots+d_k\cdot\vec{\kappa}_k)
               \bigr)
              \end{aligned}
            \end{multline*}
            and that
            $
              \bigl(\vec{v}
                 -(c_1\cdot\vec{\kappa}_1+\dots+c_k\cdot\vec{\kappa}_k)
              \bigr)
              \dotprod
              \bigl((c_1\cdot\vec{\kappa}_1+\dots+c_k\cdot\vec{\kappa}_k)
                     -(d_1\cdot\vec{\kappa}_1+\dots+d_k\cdot\vec{\kappa}_k)
              \bigr)                                                      
              =0
            $
            (because the first item shows the $\vec{v}
                 -(c_1\cdot\vec{\kappa}_1+\dots+c_k\cdot\vec{\kappa}_k)$
            is orthogonal to each $\vec{\kappa}$ and so it is orthogonal 
            to this linear combination of the $\vec{\kappa}$'s).
            Now apply the Pythagorean Theorem (i.e., the Triangle Inequality).
       \end{exparts}
     \end{answer}
   \item 
     Find a nonzero vector in \( \Re^3 \) that is orthogonal to both of these.
     \begin{equation*}
       \colvec[r]{1 \\ 5 \\ -1}
       \qquad
       \colvec[r]{2 \\ 2 \\ 0}
     \end{equation*}
     \begin{answer} 
       One way to proceed is to find a third vector so that the three together
       make a basis for $\Re^3$, e.g.,
       \begin{equation*}
         \vec{\beta}_3=\colvec[r]{1 \\ 0 \\ 0}  
       \end{equation*} 
       (the second vector is not dependent on the third because it has a
       nonzero second component, and the first is not dependent on the second
       and third because of its nonzero third component),
       and then apply the Gram-Schmidt process.
       The first element of the new basis is this.
       \begin{equation*}
         \vec{\kappa}_1 = \colvec[r]{1 \\ 5 \\ -1}  
       \end{equation*}
       And this is the second element.
       \begin{align*}
         \vec{\kappa}_2
           &=
           \colvec[r]{2 \\ 2 \\ 0}
           -\proj{\colvec[r]{2 \\ 2 \\ 0}}{\spanof{\vec{\kappa}_1}}
           =\colvec[r]{2 \\ 2 \\ 0}
           -\frac{\colvec[r]{2 \\ 2 \\ 0}\dotprod\colvec[r]{1 \\ 5 \\ -1}}{%
                  \colvec[r]{1 \\ 5 \\ -1}\dotprod\colvec[r]{1 \\ 5 \\ -1}}
            \cdot\colvec[r]{1 \\ 5 \\ -1}                               \\
           &\hbox{}\quad=
           \colvec[r]{2 \\ 2 \\ 0}
           -\frac{12}{27}\cdot\colvec[r]{1 \\ 5 \\ -1}
           =\colvec[r]{14/9 \\ -2/9 \\ 4/9}                        
       \end{align*}
       Here is the final element.
       \begin{align*}
        \vec{\kappa}_3    
           &=  
           \colvec[r]{1 \\ 0 \\ 0}
             -\proj{\colvec[r]{1 \\ 0 \\ 0}}{\spanof{\vec{\kappa}_1}}
             -\proj{\colvec[r]{1 \\ 0 \\ 0}}{\spanof{\vec{\kappa}_2}}       \\
           &\hbox{}\quad=
           \colvec[r]{1 \\ 0 \\ 0}
           -\frac{\colvec[r]{1 \\ 0 \\ 0}\dotprod\colvec[r]{1 \\ 5 \\ -1}}{%
                    \colvec[r]{1 \\ 5 \\ -1}\dotprod\colvec[r]{1 \\ 5 \\ -1}}
             \cdot\colvec[r]{1 \\ 5 \\ -1}
           -\frac{\colvec[r]{1 \\ 0 \\ 0}\dotprod\colvec[r]{14/9 \\ -2/9 \\  4/9}}{%
            \colvec[r]{14/9 \\ -2/9 \\ 4/9}\dotprod\colvec[r]{14/9 \\ -2/9 \\ 4/9}}
            \cdot\colvec[r]{14/9 \\ -2/9 \\ 4/9}              \\
         &=
         \colvec[r]{1 \\ 0 \\ 0}
           -\frac{1}{27}\cdot\colvec[r]{1 \\ 5 \\ -1}
           -\frac{7}{12}\cdot\colvec[r]{14/9 \\ -2/9 \\ 4/9}
         =\colvec[r]{1/18 \\ -1/18  \\ -4/18}
       \end{align*}
       The result $\vec{\kappa}_3$ is orthogonal to both $\vec{\kappa}_1$
       and $\vec{\kappa}_2$.
       It is therefore orthogonal to every vector in the span
       of the set $\set{\vec{\kappa}_1,\vec{\kappa}_2}$,
       including the two vectors given in the question.
     \end{answer}
  \recommended \item \label{exer:OrthoRepEasy} 
    One advantage of orthogonal bases is that they simplify finding the
    representation of a vector with respect to that basis.
    \begin{exparts}
      \partsitem For this vector and this non-orthogonal basis for $\Re^2$
        \begin{equation*}
          \vec{v}=\colvec[r]{2 \\ 3}
          \qquad
          B=\sequence{\colvec[r]{1 \\ 1},
                    \colvec[r]{1 \\ 0}}
        \end{equation*}
        first represent the vector with respect to the basis. 
        Then project the vector into the span of each basis vector
        $\spanof{\smash{\vec{\beta}_1}}$ and $\spanof{\smash{\vec{\beta}_2}}$.
      \partsitem With this orthogonal basis for $\Re^2$
        \begin{equation*}
          K=\sequence{\colvec[r]{1 \\ 1},
                    \colvec[r]{1 \\ -1}}
        \end{equation*}
        represent the same vector $\vec{v}$ with respect to the basis. 
        Then project the vector into the span of each basis vector.
        Note that the coefficients in the representation and the projection
        are the same.
      \partsitem Let
        $K=\sequence{\vec{\kappa}_1,\ldots,\vec{\kappa}_k}$
        be an orthogonal basis for some subspace of $\Re^n$.
        Prove that for any $\vec{v}$ in the subspace,
        the $i$-th component of the representation $\rep{\vec{v}\,}{K}$
        is the scalar coefficient $(\vec{v}\dotprod\vec{\kappa}_i)/
               (\vec{\kappa}_i\dotprod\vec{\kappa}_i)$
        from $\proj{\vec{v}\,}{\spanof{\vec{\kappa}_i}}$.
      \partsitem Prove that 
        $\vec{v}=\proj{\vec{v}\,}{\spanof{\vec{\kappa}_1}}
                   +\dots+\proj{\vec{v}\,}{\spanof{\vec{\kappa}_k}}$.
    \end{exparts}
    \begin{answer}
      \begin{exparts}
        \partsitem We can do the representation by eye.
          \begin{equation*}
            \colvec[r]{2 \\ 3}=3\cdot\colvec[r]{1 \\ 1}+(-1)\cdot\colvec[r]{1 \\ 0}
            \qquad
            \rep{\vec{v}\,}{B}=\colvec[r]{3 \\ -1}_B
          \end{equation*}
          The two projections are also easy.
          \begin{align*}
            \proj{\colvec[r]{2 \\ 3}}{\spanof{\vec{\beta}_1}}
            &=\frac{\colvec[r]{2 \\ 3}\dotprod\colvec[r]{1 \\ 1}}{%
                   \colvec[r]{1 \\ 1}\dotprod\colvec[r]{1 \\ 1}}
              \cdot\colvec[r]{1 \\ 1}
            =\frac{5}{2}\cdot\colvec[r]{1 \\ 1}            \\
            \proj{\colvec[r]{2 \\ 3}}{\spanof{\vec{\beta}_2}}
            &=\frac{\colvec[r]{2 \\ 3}\dotprod\colvec[r]{1 \\ 0}}{%
                   \colvec[r]{1 \\ 0}\dotprod\colvec[r]{1 \\ 0}}
              \cdot\colvec[r]{1 \\ 0}
            =\frac{2}{1}\cdot\colvec[r]{1 \\ 0}
          \end{align*}
        \partsitem As above, we can do the representation by eye
          \begin{equation*}
            \colvec[r]{2 \\ 3}=(5/2)\cdot\colvec[r]{1 \\ 1}
                             +(-1/2)\cdot\colvec[r]{1 \\ -1}
          \end{equation*}
          and the two projections are easy.
          \begin{align*}
            \proj{\colvec[r]{2 \\ 3}}{\spanof{\vec{\beta}_1}}
            &=\frac{\colvec[r]{2 \\ 3}\dotprod\colvec[r]{1 \\ 1}}{%
                   \colvec[r]{1 \\ 1}\dotprod\colvec[r]{1 \\ 1}}
              \cdot\colvec[r]{1 \\ 1}
            =\frac{5}{2}\cdot\colvec[r]{1 \\ 1}                    \\
            \proj{\colvec[r]{2 \\ 3}}{\spanof{\vec{\beta}_2}}
            &=\frac{\colvec[r]{2 \\ 3}\dotprod\colvec[r]{1 \\ -1}}{%
                   \colvec[r]{1 \\ -1}\dotprod\colvec[r]{1 \\ -1}}
              \cdot\colvec[r]{1 \\ -1}
            =\frac{-1}{2}\cdot\colvec[r]{1 \\ -1}
          \end{align*}
          Note the recurrence of the $5/2$ and the $-1/2$.
        \partsitem Represent $\vec{v}$ with respect to the basis
          \begin{equation*}
            \rep{\vec{v}\,}{K}=\colvec{r_1 \\ \vdots \\ r_k}
          \end{equation*}
          so that
          $\vec{v}=r_1\vec{\kappa}_1+\dots+r_k\vec{\kappa}_k$.
          To determine $r_i$,
          take the dot product of both sides with $\vec{\kappa}_i$.
          \begin{equation*}
            \vec{v}\dotprod\vec{\kappa}_i
               =\left(r_1\vec{\kappa}_1+\dots+r_k\vec{\kappa}_k\right)
                 \dotprod\vec{\kappa}_i
               =r_1\cdot 0+\dots+r_i\cdot(\vec{\kappa}_i\dotprod\vec{\kappa}_i)
                    +\dots+r_k\cdot 0
          \end{equation*}
          Solving for $r_i$ yields the desired coefficient.
        \partsitem This is a restatement of the prior item.
      \end{exparts}
    \end{answer}
  \item 
     \textit{Bessel's Inequality}. 
     Consider these orthonormal sets
     \begin{equation*}
          B_1=\set{\vec{e}_1}
          \quad
          B_2=\set{\vec{e}_1,\vec{e}_2}
          \quad
          B_3=\set{\vec{e}_1,\vec{e}_2,\vec{e}_3}
          \quad
          B_4=\set{\vec{e}_1,\vec{e}_2,\vec{e}_3,\vec{e}_4}
     \end{equation*}
     along with the vector $\vec{v}\in\Re^4$ whose components are
     $4$, $3$, $2$, and $1$.
    \begin{exparts}
      \partsitem 
        Find the coefficient $c_1$ for the projection of $\vec{v}$ into
        the span of the vector in $B_1$.
        Check that $\norm{\vec{v}\,}^2\geq \absval{c_1}^2$.
      \partsitem 
        Find the  coefficients $c_1$ and $c_2$ for the projection of $\vec{v}$
        into the spans of the two vectors in $B_2$.
        Check that $\norm{\vec{v}\,}^2\geq \absval{c_1}^2+\absval{c_2}^2$.
      \partsitem Find $c_1$, $c_2$, and $c_3$ associated with the vectors in 
        $B_3$, and $c_1$, $c_2$, $c_3$, and $c_4$ for the vectors in $B_4$. 
        Check that 
        $\norm{\vec{v}\,}^2\geq \absval{c_1}^2+\dots+\absval{c_3}^2$ and
        that
        $\norm{\vec{v}\,}^2\geq \absval{c_1}^2+\dots+\absval{c_4}^2$.
    \end{exparts}
    Show that this holds in general:~where 
    $\set{\vec{\kappa}_1,\ldots,\vec{\kappa}_k}$
    is an orthonormal set and $c_i$ is coefficient of
    the projection of a vector $\vec{v}$ from the space
    then 
    $\norm{\vec{v}\,}^2\geq \absval{c_1}^2+\dots+\absval{c_k}^2$.
    \textit{Hint}.~~One way is to look at the inequality 
      $0\leq\norm{\vec{v}-(c_1\vec{\kappa}_1+\dots+c_k\vec{\kappa}_k)}^2$
      and expand the $c$'s.
    \begin{answer}
      First, $\norm{\vec{v}\,}^2=4^2+3^2+2^2+1^2=50$.
      \begin{exparts*}
        \partsitem $c_1=4$
        \partsitem $c_1=4$, $c_2=3$
        \partsitem $c_1=4$, $c_2=3$, $c_3=2$, $c_4=1$
      \end{exparts*}
      For the proof, we will do only the $k=2$ case 
      because the completely general case is messier but no more enlightening.
      We follow the hint
      (recall that for any vector $\vec{w}$ we have 
      $\norm{\vec{w}\,}^2=\vec{w}\dotprod\vec{w}$).
      \begin{align*}
        0
         &\leq
         \left(\vec{v}-\bigl(\frac{\vec{v}\dotprod\vec{\kappa}_1}{%
                              \vec{\kappa}_1\dotprod\vec{\kappa}_1}
                          \cdot\vec{\kappa}_1
                         +\frac{\vec{v}\dotprod\vec{\kappa}_2}{%
                              \vec{\kappa}_2\dotprod\vec{\kappa}_2}
                          \cdot\vec{\kappa}_2
                       \bigr)
         \right)
         \dotprod
         \left(\vec{v}-\bigl(\frac{\vec{v}\dotprod\vec{\kappa}_1}{%
                              \vec{\kappa}_1\dotprod\vec{\kappa}_1}
                          \cdot\vec{\kappa}_1
                         +\frac{\vec{v}\dotprod\vec{\kappa}_2}{%
                              \vec{\kappa}_2\dotprod\vec{\kappa}_2}
                          \cdot\vec{\kappa}_2
                       \bigr)
         \right)                                                       \\
         &=\begin{aligned}[t]
            &\vec{v}\dotprod\vec{v}
             -2\cdot\vec{v}
              \dotprod\left(\frac{\vec{v}\dotprod\vec{\kappa}_1}{%
                              \vec{\kappa}_1\dotprod\vec{\kappa}_1}
                          \cdot\vec{\kappa}_1
                       +\frac{\vec{v}\dotprod\vec{\kappa}_2}{%
                              \vec{\kappa}_2\dotprod\vec{\kappa}_2}
                          \cdot\vec{\kappa}_2\right)               \\
            &+\left(\frac{\vec{v}\dotprod\vec{\kappa}_1}{%
                              \vec{\kappa}_1\dotprod\vec{\kappa}_1}
                          \cdot\vec{\kappa}_1
                       +\frac{\vec{v}\dotprod\vec{\kappa}_2}{%
                              \vec{\kappa}_2\dotprod\vec{\kappa}_2}
                          \cdot\vec{\kappa}_2\right)
           \dotprod
           \left(\frac{\vec{v}\dotprod\vec{\kappa}_1}{%
                              \vec{\kappa}_1\dotprod\vec{\kappa}_1}
                          \cdot\vec{\kappa}_1
                       +\frac{\vec{v}\dotprod\vec{\kappa}_2}{%
                              \vec{\kappa}_2\dotprod\vec{\kappa}_2}
                          \cdot\vec{\kappa}_2\right)               
          \end{aligned}                                              \\
         &=
          \vec{v}\dotprod\vec{v}
          -2\cdot
             \left(\frac{\vec{v}\dotprod\vec{\kappa}_1}{%
                              \vec{\kappa}_1\dotprod\vec{\kappa}_1}
                          \cdot(\vec{v}\dotprod\vec{\kappa}_1)
                       +\frac{\vec{v}\dotprod\vec{\kappa}_2}{%
                              \vec{\kappa}_2\dotprod\vec{\kappa}_2}
                          \cdot(\vec{v}\dotprod\vec{\kappa}_2)\right)  \\
          &\quad+\left((\frac{\vec{v}\dotprod\vec{\kappa}_1}{%
                              \vec{\kappa}_1\dotprod\vec{\kappa}_1})^2
                          \cdot(\vec{\kappa}_1\dotprod\vec{\kappa}_1)
                       +(\frac{\vec{v}\dotprod\vec{\kappa}_2}{%
                              \vec{\kappa}_2\dotprod\vec{\kappa}_2})^2
                          \cdot(\vec{\kappa}_2\dotprod\vec{\kappa}_2)\right)
      \end{align*}
      (The two mixed terms in the third part of the third line are zero because
      $\vec{\kappa}_1$ and $\vec{\kappa}_2$ are orthogonal.)
      The result now follows on gathering like terms and on recognizing that
      $\vec{\kappa}_1\dotprod\vec{\kappa}_1=1$ and           
      $\vec{\kappa}_2\dotprod\vec{\kappa}_2=1$ because these vectors are
      members of an orthonormal set.           
    \end{answer}
  \item
    Prove or disprove:~every vector in \( \Re^n \) is in some orthogonal
    basis.
    \begin{answer}
      It is true, except for the zero vector.
      Every vector in \( \Re^n \) except the zero vector is in a basis, and
      that basis can be orthogonalized.
     \end{answer}
  \item 
    Show that the columns of an \( \nbyn{n} \) matrix form an orthonormal
    set if and only if the inverse of the matrix is its transpose.
    Produce such a matrix.
    \begin{answer} 
      The $\nbyn{3}$ case gives the idea.
      The set 
      \begin{equation*}
        \set{\colvec{a \\ d \\ g},\colvec{b \\ e \\ h},\colvec{c \\ f \\ i}}
      \end{equation*}
      is orthonormal if and only if these nine conditions all hold
      \begin{equation*}
        \begin{array}{ccc}
          \rowvec{a &d &g}\dotprod\colvec{a \\ d \\ g}=1
            &\rowvec{a &d &g}\dotprod\colvec{b \\ e \\ h}=0
            &\rowvec{a &d &g}\dotprod\colvec{c \\ f \\ i}=0    \\
          \rowvec{b &e &h}\dotprod\colvec{a \\ d \\ g}=0
            &\rowvec{b &e &h}\dotprod\colvec{b \\ e \\ h}=1
            &\rowvec{b &e &h}\dotprod\colvec{c \\ f \\ i}=0    \\
          \rowvec{c &f &i}\dotprod\colvec{a \\ d \\ g}=0
            &\rowvec{c &f &i}\dotprod\colvec{b \\ e \\ h}=0
            &\rowvec{c &f &i}\dotprod\colvec{c \\ f \\ i}=1    
        \end{array}
      \end{equation*}
      (the three conditions in the lower left are redundant but nonetheless 
      correct).
      Those, in turn, hold if and only if 
      \begin{equation*}
        \begin{mat}
          a &d &g \\
          b &e &h \\
          c &f &i
        \end{mat}
        \begin{mat}
          a &b &c  \\
          d &e &f  \\
          g &h &i
        \end{mat}
        =
        \begin{mat}
          1 &0 &0  \\
          0 &1 &0  \\
          0 &0 &1
        \end{mat}
      \end{equation*}
      as required.

      This is an example, the inverse of this matrix is its transpose.
      \begin{equation*}
        \begin{mat}[r]
          1/\sqrt{2}  &1/\sqrt{2}  &0  \\
         -1/\sqrt{2}  &1/\sqrt{2}  &0  \\
          0           &0           &1
        \end{mat}
      \end{equation*}
    \end{answer}
  \item 
    Does the proof of \nearbytheorem{th:OrthoIsInd} fail to consider the
    possibility that the set of vectors is empty (i.e., that $k=0$)?
    \begin{answer}
      If the set is empty then the summation on the left side is the
      linear combination of the empty set of vectors, 
      which by definition adds to the zero vector. 
      In the second sentence, there is not such $i$, so the 
      `if \ldots then \ldots' implication is vacuously true.
    \end{answer}
  \item 
    \nearbytheorem{th:GramSchmidt} describes a change of basis 
    from any basis \( B=\sequence{\vec{\beta}_1,\ldots,\vec{\beta}_k} \)
    to one that is orthogonal 
    \( K=\sequence{\vec{\kappa}_1,\ldots,\vec{\kappa}_k} \).
    Consider the change of basis matrix $\rep{\identity}{B,K}$. 
    \begin{exparts}
      \partsitem Prove that the matrix $\rep{\identity}{K,B}$
        changing bases in the direction opposite to that of the theorem 
        has an upper triangular shape\Dash all
        of its entries below the main diagonal are zeros.
      \partsitem Prove that the inverse of an upper triangular matrix is 
        also upper triangular (if the matrix is invertible, that is).
        This shows that the matrix $\rep{\identity}{B,K}$  changing bases 
        in the direction described in the theorem is upper triangular.
    \end{exparts}
    \begin{answer} 
      \begin{exparts}
        \partsitem Part of the induction argument proving 
          \nearbytheorem{th:GramSchmidt} checks that 
          $\vec{\kappa}_i$ is in the span of 
          $\sequence{\vec{\beta}_1,\dots,\vec{\beta}_i}$.
          (The $i=3$ case in the proof illustrates.)
          Thus, in the change of basis matrix $\rep{\identity}{K,B}$, 
          the $i$-th column $\rep{\vec{\kappa}_i}{B}$
          has components $i+1$ through $k$ that are zero.
        \partsitem One way to see this is to recall the computational
          procedure that we use to find the inverse.
          We write the matrix, write the identity matrix next to
          it, and then we do Gauss-Jordan reduction.
          If the matrix starts out upper triangular then the Gauss-Jordan
          reduction involves only the Jordan half and these steps,
          when performed on the identity, will result in an upper triangular
          inverse matrix.
      \end{exparts}
    \end{answer}
 \item \label{exer:GramSchmidt} 
    Complete the induction argument in the proof of  
    \nearbytheorem{th:GramSchmidt}.
    \begin{answer}
      For the inductive step, we assume that for all $j$ in~$[1..i]$, 
      these three conditions are true of each $\vec{\kappa}_j$:
      (i)~each $\vec{\kappa}_j$ is nonzero, 
      (ii)~each $\vec{\kappa}_j$ is a linear combination of 
         the vectors $\vec{\beta}_1,\dots,\vec{\beta}_j$,
      and (iii)~each $\vec{\kappa}_j$ is orthogonal to all of the 
         $\vec{\kappa}_m$'s prior to it (that is, with $m<j$).
      With those inductive hypotheses, consider $\vec{\kappa}_{i+1}$.
      \begin{align*}
        \vec{\kappa}_{i+1}
         &=
        \vec{\beta}_{i+1}
          -\proj{\beta_{i+1}}{\spanof{\vec{\kappa}_1}}
          -\proj{\beta_{i+1}}{\spanof{\vec{\kappa}_2}}
          -\dots
          -\proj{\beta_{i+1}}{\spanof{\vec{\kappa}_i}}  \\
         &=
        \vec{\beta}_{i+1}
          -\frac{\beta_{i+1}\dotprod\vec{\kappa}_1}{%
                  \vec{\kappa}_1\dotprod\vec{\kappa}_1}
              \cdot\vec{\kappa}_1
          -\frac{\beta_{i+1}\dotprod\vec{\kappa}_2}{%
                  \vec{\kappa}_2\dotprod\vec{\kappa}_2}
              \cdot\vec{\kappa}_2
          -\dots
          -\frac{\beta_{i+1}\dotprod\vec{\kappa}_i}{%
                  \vec{\kappa}_i\dotprod\vec{\kappa}_i}
              \cdot\vec{\kappa}_i
       \end{align*}
       By the inductive assumption~(ii) we can expand each $\vec{\kappa}_j$
       into a linear combination of $\vec{\beta}_1,\ldots,\vec{\beta}_j$
       \begin{align*}
         &=
        \vec{\beta}_{i+1}
          -\frac{\vec{\beta}_{i+1}\dotprod\vec{\kappa}_1}{%
                  \vec{\kappa}_1\dotprod\vec{\kappa}_1}
              \cdot\vec{\beta}_1                          \\
          &\quad-\frac{\vec{\beta}_{i+1}\dotprod\vec{\kappa}_2}{%
                  \vec{\kappa}_2\dotprod\vec{\kappa}_2}
              \cdot
                 \left(\text{
                   linear combination of $\vec{\beta}_1,\vec{\beta}_2$}
                 \right)                                                 \\
          &\quad-\;\cdots\;
          -\frac{\vec{\beta}_{i+1}\dotprod\vec{\kappa}_i}{%
                  \vec{\kappa}_i\dotprod\vec{\kappa}_i}
              \cdot
                 \left(\text{
                   linear combination of $\vec{\beta}_1,\dots,\vec{\beta}_i$}
                 \right)
       \end{align*}
       The fractions are scalars so this is a linear combination
       of linear combinations of $\vec{\beta}_1,\dots,\vec{\beta}_{i+1}$.
       It is therefore just a 
       linear combination of $\vec{\beta}_1,\dots,\vec{\beta}_{i+1}$.
       Now, (i)~it cannot sum to the zero vector because the equation would
       then describe a nontrivial linear relationship among the $\vec{\beta}$'s
       that are given as members of a basis
       (the relationship is nontrivial because the coefficient of 
       $\vec{\beta}_{i+1}$ is $1$).
       Also, (ii)~the equation gives $\vec{\kappa}_{i+1}$ as a combination of 
       $\vec{\beta}_1,\dots,\vec{\beta}_{i+1}$.
       Finally, for~(iii), consider $\vec{\kappa}_j\dotprod\vec{\kappa}_{i+1}$;
       as in the $i=3$ case, the dot product of $\vec{\kappa}_j$ with 
       $\vec{\kappa}_{i+1}=\vec{\beta}_{i+1}
          -\proj{\vec{\beta}_{i+1}}{\spanof{\vec{\kappa}_1}}
          -\dots
          -\proj{\vec{\beta}_{i+1}}{\spanof{\vec{\kappa}_i}}$
        can be rewritten to give two kinds of terms,
       $\vec{\kappa}_j\dotprod
           \left(\vec{\beta}_{i+1}
                  -\proj{\vec{\beta}_{i+1}}{\spanof{\vec{\kappa}_j}}
           \right)$
       (which is zero because the projection is orthogonal)
       and 
       $\vec{\kappa}_j\dotprod
               \proj{\vec{\beta}_{i+1}}{\spanof{\vec{\kappa}_m}}$
       with $m\neq j$ and $m<i+1$ 
       (which is zero because by the hypothesis~(iii)  
       the vectors $\vec{\kappa}_j$ and $\vec{\kappa}_m$ are orthogonal).
     \end{answer}
\index{Gram-Schmidt process|)}
\end{exercises}



















\subsectionoptional{Projection Into a Subspace}
\noindent\textit{This subsection 
     uses material from the 
     optional earlier subsection on Combining Subspaces.}

The prior subsections
project a vector into a line by decomposing it into two parts:~the
part in the line $\proj{\vec{v}\,}{\spanof{\vec{s}\,}}$ 
and the rest $\vec{v}-\proj{\vec{v}\,}{\spanof{\vec{s}\,}}$.
To generalize projection to arbitrary subspaces
we will follow this decomposition idea.

\begin{definition}
\label{def:ProjIntoMAlongN}
Let a vector space be a direct sum \( V=M\directsum N \).
Then for any \( \vec{v}\in V \) 
with $\vec{v}=\vec{m}+\vec{n}$ where \( \vec{m}\in M,\,\vec{n}\in N \),
the \definend{projection of \( \vec{v} \) into \( M \) along \( N \)}%
\index{projection!into a subspace}\index{projection!along a subspace}
is
$\proj{\vec{v}\,}{M,N}=\vec{m}$.
\end{definition}

This definition applies in spaces where we don't have a ready definition
of orthogonal.
(Definitions of orthogonality for spaces other than the $\Re^n$
are perfectly possible but we haven't seen any in this book.)

\begin{example} \label{ex:OrthProjOne}
The space \( \matspace_{\nbyn{2}} \) of $\nbyn{2}$ matrices is the
direct sum of these two.
\begin{equation*}
   M=\set{\begin{mat}
              a  &b  \\
              0  &0
            \end{mat} \suchthat a,b\in\Re }
  \qquad
   N=\set{\begin{mat}
              0  &0  \\
              c  &d
            \end{mat} \suchthat c,d\in\Re }
\end{equation*}
To project
\begin{equation*}
  A=\begin{mat}[r]
          3  &1  \\
          0  &4
  \end{mat}
\end{equation*}
into $M$ along $N$, we first fix bases for the two subspaces.
\begin{equation*}
  B_{M}=\sequence{
    \begin{mat}[r]
      1  &0  \\
      0  &0      
    \end{mat},
    \begin{mat}[r]
      0  &1  \\
      0  &0
    \end{mat}
                   }
  \qquad
  B_{N}=\sequence{
    \begin{mat}[r]
      0  &0  \\
      1  &0      
    \end{mat},
    \begin{mat}[r]
      0  &0  \\
      0  &1
    \end{mat}
                   }
\end{equation*}
Their concatenation
\begin{equation*}
  B=\cat{B_{M}}{B_{N}}=\sequence{
    \begin{mat}[r]
      1  &0  \\
      0  &0      
    \end{mat},
    \begin{mat}[r]
      0  &1  \\
      0  &0
    \end{mat},
    \begin{mat}[r]
      0  &0  \\
      1  &0      
    \end{mat},
    \begin{mat}[r]
      0  &0  \\
      0  &1
    \end{mat}
                   }
\end{equation*}
is a basis for the entire space because \( \matspace_{\nbyn{2}} \) 
is the direct sum.
So we can use it to represent $A$.
\begin{equation*}
  \begin{mat}[r]
          3  &1  \\
          0  &4
  \end{mat}=
    3\cdot\begin{mat}[r]
      1  &0  \\
      0  &0      
    \end{mat}
    +1\cdot\begin{mat}[r]
      0  &1  \\
      0  &0
    \end{mat}
    +0\cdot\begin{mat}[r]
      0  &0  \\
      1  &0      
    \end{mat}
    +4\cdot\begin{mat}[r]
      0  &0  \\
      0  &1
    \end{mat}
\end{equation*}
The projection of $A$ into $M$ along $N$ keeps the 
$M$ part and drops the $N$ part.
\begin{equation*}
  \proj{\begin{mat}[r]
          3  &1  \\
          0  &4
  \end{mat} }{M,N}
  =
    3\cdot\begin{mat}[r]
      1  &0  \\
      0  &0      
    \end{mat}
    +1\cdot\begin{mat}[r]
      0  &1  \\
      0  &0
    \end{mat}
    =\begin{mat}[r]
      3  &1  \\
      0  &0      
    \end{mat}
\end{equation*}
\end{example}

\begin{example}  \label{ex:ProjIntoMAlongNGener}
Both subscripts on $\proj{\vec{v}\,}{M,N}$ are significant.
The first subscript $M$ matters because the result of
the projection is a member of $M$.
For an example showing that the second one matters,  
fix this plane subspace of $\Re^3$ and its basis. 
\begin{equation*}
  M=\set{\colvec{x \\ y \\ z}\suchthat y-2z=0}
  \qquad
  B_M=\sequence{
                \colvec[r]{1 \\ 0 \\ 0},
                \colvec[r]{0 \\ 2 \\ 1} }
\end{equation*}
We will compare the projections of this element of~$\Re^3$
\begin{equation*}
  \vec{v}=\colvec[r]{2 \\ 2 \\ 5}
\end{equation*}
into $M$ along these two subspaces
(verification that \( \Re^3=M\directsum N \) and 
\( \Re^3=M\directsum \hat{N} \) is routine).
\begin{equation*}
  N=\set{k\colvec[r]{0 \\ 0 \\ 1}\suchthat k\in\Re}
  \qquad
  \hat{N}=\set{k\colvec[r]{0 \\ 1 \\ -2}\suchthat k\in\Re}
\end{equation*}
Here are natural bases for $N$ and $\hat{N}$.
\begin{equation*}
  B_N=\sequence{
                \colvec[r]{0 \\ 0 \\ 1} }
  \qquad
  B_{\hat{N}}=\sequence{
                \colvec[r]{0 \\ 1 \\ -2} }
\end{equation*}
To project into $M$ along~$N$,
represent $\vec{v}$ with respect to the concatenation $\cat{B_M}{B_N}$
\begin{equation*}
  \colvec[r]{2 \\ 2 \\ 5}=
  2\cdot\colvec[r]{1 \\ 0 \\ 0}+
  1\cdot\colvec[r]{0 \\ 2 \\ 1}+
  4\cdot\colvec[r]{0 \\ 0 \\ 1}
\end{equation*}
and drop the $N$ term. 
\begin{equation*}
  \proj{\vec{v}\,}{M,N}
  =2\cdot\colvec[r]{1 \\ 0 \\ 0}+
  1\cdot\colvec[r]{0 \\ 2 \\ 1}
  =\colvec[r]{2 \\ 2 \\ 1}
\end{equation*}
To project into~$M$ along~$\hat{N}$
represent $\vec{v}$ with respect to $\cat{B_M}{B_{\hat{N}}}$
\begin{equation*}
  \colvec[r]{2 \\ 2 \\ 5}=
  2\cdot\colvec[r]{1 \\ 0 \\ 0}
  +(9/5)\cdot\colvec[r]{0 \\ 2 \\ 1}
  -(8/5)\cdot\colvec[r]{0 \\ 1 \\ -2}
\end{equation*}
and omit the $\hat{N}$ part. 
\begin{equation*}
  \proj{\vec{v}\,}{M,\hat{N}}
  =
  2\cdot\colvec[r]{1 \\ 0 \\ 0}
  +(9/5)\cdot\colvec[r]{0 \\ 2 \\ 1}
  =\colvec[r]{2 \\ 18/5 \\ 9/5}
\end{equation*}
So projecting along different subspaces can give different results.

These pictures compare the two maps.
Both show that the projection is indeed `into' the plane and
`along' the line.
\begin{center}  \small
  \begin{tabular}{@{}c@{}}\includegraphics{map/mp/ch3.38}\end{tabular}
  \qquad
  \begin{tabular}{@{}c@{}}\includegraphics{map/mp/ch3.39}\end{tabular}
\end{center}
Notice that the projection along $N$ is not orthogonal since there are
members of the plane $M$ that are not orthogonal to the dotted line.
But the projection along $\hat{N}$ is orthogonal. 
\end{example}

We have seen two projection operations,
orthogonal projection into a line
as well as this subsections's projection into an~$M$ and along an~$N$, and  
we naturally ask whether they are related.
The right-hand picture above suggests the 
answer\Dash orthogonal projection into
a line is a special case of this subsection's projection; 
it is projection along a subspace perpendicular to the line.
\begin{center}  \small
  \includegraphics{map/mp/ch3.40}
\end{center}
% In addition to pointing out that projection along a subspace is a 
% generalization, this scheme shows how to define orthogonal projection into
% any subspace of $\Re^n$, of any dimension. 

\begin{definition} \label{def:OrthComp}
The \definend{orthogonal complement\/}\index{orthogonal!complement}%
\index{complementary subspaces!orthogonal}\index{perp, of a subspace}
of a subspace \( M \) of $\Re^n$ is
\begin{equation*}
  M^\perp=\set{\vec{v}\in\Re^n\suchthat
                 \vec{v} \text{ is perpendicular to all vectors in }M}
\end{equation*}
(read ``\( M \) perp'').
The 
\definend{orthogonal projection}\index{orthogonal!projection}%
\index{projection!orthogonal} \( \proj{\vec{v}\,}{M} \) 
of a vector is its projection into $M$ along \( M^\perp \).
\end{definition}


\begin{example} \label{ex:OrthoCompOne}
In \( \Re^3 \), to find the orthogonal complement of the plane
\begin{equation*}
   P=\set{\colvec{x \\ y \\ z}\suchthat 3x+2y-z=0 }
\end{equation*}
we start with a basis for \( P \).
\begin{equation*}
   B=\sequence{\colvec[r]{1 \\ 0 \\ 3},
               \colvec[r]{0 \\ 1 \\ 2} }
\end{equation*}
Any \( \vec{v} \) perpendicular to every vector in $B$
is perpendicular to every vector in the span of $B$
(the proof of this is \nearbyexercise{exer:PerpOnBasisImplPerp}).
Therefore, the subspace $P^\perp$ consists of the vectors that satisfy
these two conditions.
\begin{equation*}
  \colvec[r]{1 \\ 0 \\ 3}\dotprod\colvec{v_1 \\ v_2 \\ v_3}=0
  \qquad
  \colvec[r]{0 \\ 1 \\ 2}\dotprod\colvec{v_1 \\ v_2 \\ v_3}=0
\end{equation*}
Those conditions give a linear system.
\begin{equation*}
  P^\perp
    =\set{\colvec{v_1 \\ v_2 \\ v_3}\suchthat \begin{mat}[r]
                                                      1  &0  &3  \\
                                                      0  &1  &2
                                                    \end{mat}
                                                    \colvec{v_1 \\ v_2 \\ v_3}
                                                    =\colvec[r]{0 \\ 0} }
\end{equation*}
We are thus left with finding the null space of the map represented 
by the matrix, that is, with 
calculating the solution set of the homogeneous linear system.
\begin{equation*}
  \begin{linsys}{3}
    v_1 &  &    &+ &3v_3 &= &0 \\
        &  &v_2 &+ &2v_3 &= &0
  \end{linsys}
  \quad\Longrightarrow\quad
  P^\perp=\set{k\colvec[r]{-3 \\ -2 \\ 1}\suchthat k\in\Re}
\end{equation*}
% Instead of the term orthogonal complement,
% this is sometimes called the line \definend{normal}\index{normal} %
% to the plane.
\end{example}

\begin{example} \label{ex:OrthoCompTwo}
Where \( M \) is the \( xy \)-plane subspace of \( \Re^3 \),
what is \( M^\perp \)?
A common first reaction is that \( M^\perp \) is the \( yz \)-plane but
that's not right because
some vectors from the \( yz \)-plane are not perpendicular to every
vector in the \( xy \)-plane.
\begin{center}  \small
    {\small $\colvec[r]{1 \\ 1 \\ 0} \not\perp \colvec[r]{0 \\ 3 \\ 2}$}
   \quad
  \begin{tabular}{@{}c@{}}\includegraphics{map/mp/ch3.41}\end{tabular}
   \quad
   {\small $\displaystyle 
       \theta=\arccos(\frac{1\cdot 0+1\cdot 3+0\cdot 2}{
                            \sqrt{2}\cdot\sqrt{13}})
    \approx\text{$0.94$~rad}$}
\end{center}
Instead \( M^\perp \) is the \( z \)-axis, since proceeding as in the
prior example and taking the natural basis for the $xy$-plane gives this.
\begin{equation*}
  M^\perp
  =
  \set{\colvec{x \\ y \\ z}\suchthat \begin{mat}[r]
                                         1  &0  &0  \\
                                         0  &1  &0
                                       \end{mat}
                                       \colvec{x \\ y \\ z}=
                                       \colvec[r]{0 \\ 0} \;}
  =
  \set{\colvec{x \\ y \\ z}\suchthat x=0 \text{\ and\ }y=0  }
\end{equation*}
\end{example}

% The two examples that we've seen since \nearbydefinition{def:OrthComp}
% illustrate the first sentence in that definition.
% The next result justifies the second sentence.

\begin{lemma} \label{le:OrthoProjWellDefd}
If $M$ is a subspace of $\Re^n$ then its
orthogonal complement $M^\perp$ is also a subspace.
The space is the direct sum of the two \( \Re^n=M\directsum M^\perp \).
For any $\vec{v}\in\Re^n$
the vector $\vec{v}-\proj{\vec{v}\,}{M}$ is perpendicular to every
vector in $M$. 
\end{lemma}

\begin{proof}
First, the orthogonal complement $M^\perp$ is a subspace of $\Re^n$ because
it is a null space,
namely the null space of the orthogonal projection map.
 
To show that the space~$\R^n$ is the direct sum of the two,
start with  any basis \( B_M=\sequence{\vec{\mu}_1,\dots,\vec{\mu}_k} \) 
for \( M \). 
Expand it to a basis
%\( B=\sequence{\vec{\mu}_1,\dots,\vec{\mu}_k,
%   \vec{\nu}_1,\dots,\vec{\nu}_{n-k} } \)
for the entire space and then
apply the Gram-Schmidt process to get an orthogonal basis
\( K=\sequence{\vec{\kappa}_1,\dots,\vec{\kappa}_n} \) for \( \Re^n \).
This $K$ is the concatenation of two bases: 
\( \sequence{\vec{\kappa}_1,\dots,\vec{\kappa}_k} \) 
with the same number of members,~$k$, as $B_M$, and
\( D=\sequence{\vec{\kappa}_{k+1},\dots,\vec{\kappa}_n} \).
The first is a basis for $M$ so if we show that the second is a
basis for \( M^\perp \) then we will have that
the entire space is the direct sum. 

\nearbyexercise{exer:OrthoRepEasy} from the prior subsection 
proves this about any orthogonal basis:~each vector $\vec{v}$ 
in the space is the sum of its orthogonal projections
into the lines spanned by the basis vectors.
\begin{equation*}
  \vec{v}=\proj{\vec{v}\,}{\spanof{\vec{\kappa}_1}}
           +\dots+\proj{\vec{v}\,}{\spanof{\vec{\kappa}_n}}
\tag*{($*$)}\end{equation*}
To check this, represent the vector as
$\vec{v}=r_1\vec{\kappa}_1+\dots+r_n\vec{\kappa}_n$,
apply $\vec{\kappa}_i$ to both sides
$
   \vec{v}\dotprod\vec{\kappa}_i
      =\left(r_1\vec{\kappa}_1+\dots+r_n\vec{\kappa}_n\right)
          \dotprod\vec{\kappa}_i
      =r_1\cdot 0+\dots+r_i\cdot(\vec{\kappa}_i\dotprod\vec{\kappa}_i)
            +\dots+r_n\cdot 0
$,
and solve to get
$r_i=(\vec{v}\dotprod\vec{\kappa}_i)/(\vec{\kappa}_i\dotprod\vec{\kappa}_i)$,
as desired.

Any member of the span of
\( D \) %\sequence{\vec{\kappa}_{k+1},\dots,\vec{\kappa}_n} \)
is orthogonal to any vector in $M$ 
so the span of \( D \) is a subset of \( M^\perp \).
To show that \( D \) is a basis for $M^\perp$
we need only show the other 
containment, that any $\vec{w}\in M^\perp$ is an element of
the span of~\( D \).
The prior paragraph works for this.
Any $\vec{w}\in M^\perp$ gives this
on projections into basis vectors from $M$:
$\proj{\vec{w}\,}{\spanof{\vec{\kappa}_1}}=\zero,
\dots,\,\proj{\vec{w}\,}{\spanof{\vec{\kappa}_k}}=\zero$.
Therefore equation~($*$) gives that $\vec{w}$ is a linear combination of 
\( \vec{\kappa}_{k+1},\dots,\vec{\kappa}_n \).
Thus \( D \) is a basis for $M^\perp$ and
$\Re^n$ is the direct sum of the two.

The final sentence of the lemma
is proved in much the same way.
Write $\vec{v}=\proj{\vec{v}\,}{\spanof{\vec{\kappa}_1}}
           +\dots+\proj{\vec{v}\,}{\spanof{\vec{\kappa}_n}}$.
Then $\proj{\vec{v}\,}{M}$ keeps only the $M$ part and
drops the $M^\perp$ part:
$\proj{\vec{v}\,}{M}=\proj{\vec{v}\,}{\spanof{\vec{\kappa}_{1}}}
                       +\dots+\proj{\vec{v}\,}{\spanof{\vec{\kappa}_k}}$.
Therefore $\vec{v}-\proj{\vec{v}\,}{M}$ consists of a linear combination
of elements of $M^\perp$ and so is perpendicular to every vector in $M$.
\end{proof}

Given a subspace, we could compute the orthogonal projection into 
that subspace by following the steps of that proof: finding a basis,
expanding it to a basis for the entire space, applying Gram-Schmidt
to get an orthogonal basis, and projecting into each linear subspace. 
However we will instead use a convenient formula.

\begin{theorem} \label{th:OrthProjMat}
Let \( M \) be a subspace of \( \Re^n \) with basis
\( \sequence{\vec{\beta}_1,\dots,\vec{\beta}_k} \) and 
let \( A \) be the matrix whose columns are the \( \vec{\beta} \)'s.
Then for any $\vec{v}\in\Re^n$ the orthogonal projection is 
$\proj{\vec{v}\,}{M}=c_1\vec{\beta}_1+\dots+c_k\vec{\beta}_k$,
where the coefficients $c_i$ are the entries of the vector
$(\trans{A}A)^{-1}\trans{A}\cdot\vec{v}$.
That is, 
$\proj{\vec{v}\,}{M}=A(\trans{A}A)^{-1}\trans{A}\cdot\vec{v}$.
\end{theorem}

\begin{proof}
The vector \( \proj{\vec{v}}{M} \) is a member of \( M \) and so is a
linear combination of basis vectors
\( c_1\cdot\vec{\beta}_1+\dots+c_k\cdot\vec{\beta}_k \).
Since $A$'s columns are the $\vec{\beta}$'s, 
there is a \( \vec{c}\in\Re^k \) such that
\( \proj{\vec{v}\,}{M}=A\vec{c} \).
% (this is expressed compactly with matrix
% multiplication as in
% \nearbyexample{ex:OrthoCompOne} and~\ref{ex:OrthoCompTwo}).
To find $\vec{c}$ note that
the vector \( \vec{v}-\proj{\vec{v}\,}{M} \) is perpendicular to each member
of the basis so 
 % (again, expressed compactly).
\begin{equation*}
  \zero
  =\trans{A}\bigl( \vec{v}-A\vec{c} \bigr) 
  =\trans{A}\vec{v}-\trans{A}A\vec{c}
\end{equation*}
and solving gives this
(showing that \( \trans{A}A \) is invertible is an
exercise).
\begin{equation*}
  \vec{c}=\bigl( \trans{A}A\bigr)^{-1}\trans{A}\cdot\vec{v}
\end{equation*}
Therefore 
$\proj{\vec{v}\,}{M}=A\cdot\vec{c}=A(\trans{A}A)^{-1}\trans{A}\cdot\vec{v}$,
as required.
\end{proof}

\begin{example}
To orthogonally project this vector into this subspace
\begin{equation*}
  \vec{v}=\colvec[r]{1 \\ -1 \\ 1}
  \qquad
  P=\set{\colvec{x \\ y \\ z}\suchthat x+z=0}
\end{equation*}
first make a matrix whose columns are a basis for the subspace
\begin{equation*}
  A=\begin{mat}[r]
      0  &1  \\
      1  &0  \\
      0  &-1
    \end{mat}
\end{equation*}
and then compute.
\begin{align*}
  A\bigl(\trans{A}A\bigr)^{-1}\trans{A}
  &=\begin{mat}[r]
      0  &1  \\
      1  &0  \\
      0  &-1
    \end{mat}
    \begin{mat}[r]
      1    &0  \\
      0    &1/2
    \end{mat}
    \begin{mat}[r]
      0   &1  &0  \\
      1   &0  &-1 
    \end{mat}             \\
    &=
    \begin{mat}[r]
      1/2     &0  &-1/2  \\
      0     &1    &0   \\
      -1/2  &0    &1/2
    \end{mat}
\end{align*}
With the matrix, calculating the orthogonal projection of any vector into $P$
is easy.
\begin{equation*}
  \proj{\vec{v}}{P}=
    \begin{mat}[r]
      1/2     &0  &-1/2  \\
      0     &1    &0   \\
      -1/2  &0    &1/2
    \end{mat}
    \colvec[r]{1 \\ -1 \\ 1}
    =\colvec[r]{0 \\ -1 \\ 0}
\end{equation*}
Note, as a check, that this result is indeed in $P$.
\end{example}

\begin{exercises}
  \recommended \item
    Project the vectors into \( M \) along \( N \).
     \begin{exparts}
       \partsitem \( \colvec[r]{3 \\ -2},\quad
                M=\set{\colvec{x \\ y}\suchthat x+y=0},\quad
                N=\set{\colvec{x \\ y}\suchthat -x-2y=0} \)
       \partsitem \( \colvec[r]{1 \\ 2},\quad
                M=\set{\colvec{x \\ y}\suchthat x-y=0},\quad
                N=\set{\colvec{x \\ y}\suchthat 2x+y=0} \)
       \partsitem \( \colvec[r]{3 \\ 0 \\ 1},\quad
                M=\set{\colvec{x \\ y \\ z}\suchthat x+y=0},\quad
                N=\set{c\cdot\colvec[r]{1 \\ 0 \\ 1}\suchthat c\in\Re} \)
     \end{exparts}
     \begin{answer}
       \begin{exparts}
         \partsitem When bases for the subspaces
           \begin{equation*}
             B_M=\sequence{\colvec[r]{1 \\ -1}}
             \qquad
             B_N=\sequence{\colvec[r]{2 \\ -1}}
           \end{equation*}
           are concatenated 
           \begin{equation*}
             B=\cat{B_M}{B_N}=\sequence{\colvec[r]{1 \\ -1},
                                        \colvec[r]{2 \\ -1}}
           \end{equation*}
           and the given vector is represented
           \begin{equation*}
             \colvec[r]{3 \\ -2}=1\cdot\colvec[r]{1 \\ -1}+1\cdot\colvec[r]{2 \\ -1}
           \end{equation*}
           then the answer comes from retaining the $M$~part and dropping the 
           $N$~part.
           \begin{equation*}
             \proj{\colvec[r]{3 \\ -2}}{M,N}
             =\colvec[r]{1 \\ -1}
           \end{equation*}
         \partsitem When the bases
           \begin{equation*}
             B_M=\sequence{\colvec[r]{1 \\ 1}}
             \qquad
             B_N\sequence{\colvec[r]{1 \\ -2}}
           \end{equation*}
           are concatenated, and the vector is represented,
           \begin{equation*}
             \colvec[r]{1 \\ 2}
               =(4/3)\cdot\colvec[r]{1 \\ 1}
                -(1/3)\cdot\colvec[r]{1 \\ -2}
           \end{equation*}
           then retaining only the $M$~part gives this answer.
           \begin{equation*}
             \proj{\colvec[r]{1 \\ 2}}{M,N}=\colvec[r]{4/3 \\ 4/3}
           \end{equation*}
         \partsitem With these bases
           \begin{equation*}
             B_M=\sequence{\colvec[r]{1 \\ -1 \\ 0},
                           \colvec[r]{0 \\ 0 \\ 1}}
             \qquad
             B_N=\sequence{\colvec[r]{1 \\ 0 \\ 1}}
           \end{equation*}
           the representation with respect to the concatenation is this.
           \begin{equation*}
             \colvec[r]{3 \\ 0 \\ 1}
               =0\cdot\colvec[r]{1 \\ -1 \\ 0}
                -2\cdot\colvec[r]{0 \\ 0 \\ 1}
                +3\cdot\colvec[r]{1 \\ 0 \\ 1}
           \end{equation*}
           and so the projection is this.
           \begin{equation*}
             \proj{\colvec[r]{3 \\ 0 \\ 1}}{M,N}=\colvec[r]{0 \\ 0 \\ -2}
           \end{equation*}
       \end{exparts}
     \end{answer}
  \recommended \item 
     Find \( M^\perp \).
     \begin{exparts*}
       \partsitem \( M=\set{\colvec{x \\ y}\suchthat x+y=0 } \)
       \partsitem \( M=\set{\colvec{x \\ y}\suchthat -2x+3y=0 } \)
       \partsitem \( M=\set{\colvec{x \\ y}\suchthat x-y=0 } \)
       \partsitem \( M=\set{\zero\,} \)
       \partsitem \( M=\set{\colvec{x \\ y}\suchthat x=0 } \)
       \partsitem \( M=\set{\colvec{x \\ y \\ z}\suchthat -x+3y+z=0 } \)
       \partsitem \( M=\set{\colvec{x \\ y \\ z}\suchthat 
                          \text{$x=0$ and $y+z=0$} } \)
     \end{exparts*}
     \begin{answer}
       As in \nearbyexample{ex:OrthoCompOne}, we can simplify the calculation
       by just finding the space of vectors perpendicular to all the the
       vectors in  $M$'s basis.
       \begin{exparts}
         \partsitem Parametrizing to get
           \begin{equation*}
             M=\set{c\cdot\colvec[r]{-1 \\ 1}\suchthat c\in\Re}
           \end{equation*}
           gives that 
           \begin{equation*}
             M^\perp
              \set{\colvec{u \\ v}
                   \suchthat 
                      0=\colvec{u \\ v}\dotprod\colvec[r]{-1 \\ 1}}
              =\set{\colvec{u \\ v}\suchthat 0=-u+v}
           \end{equation*}
           Parametrizing the one-equation linear system gives this 
           description.
           \begin{equation*}
             M^\perp=\set{k\cdot\colvec[r]{1 \\ 1}\suchthat k\in\Re}
           \end{equation*}
         \partsitem As in the answer to the prior part, we can describe $M$ as
           a span
           \begin{equation*}
             M=\set{c\cdot\colvec[r]{3/2 \\ 1}\suchthat c\in\Re}
             \qquad
             B_M=\sequence{\colvec[r]{3/2 \\ 1}}
           \end{equation*}
           and then $M^\perp$ is the set of vectors perpendicular to the 
           one vector in this basis.
           \begin{equation*}
             M^\perp
             =\set{\colvec{u \\ v}\suchthat (3/2)\cdot u+1\cdot v=0}
             =\set{k\cdot\colvec[r]{-2/3 \\ 1}\suchthat k\in\Re}
           \end{equation*}
         \partsitem Parametrizing the linear requirement in the description
            of $M$ gives this basis.
            \begin{equation*}
              M=\set{c\cdot\colvec[r]{1 \\ 1}\suchthat c\in\Re}
              \qquad
              B_M=\sequence{\colvec[r]{1 \\ 1}}
            \end{equation*}
            Now, $M^\perp$ is the set of vectors perpendicular to (the one
            vector in) $B_M$.
            \begin{equation*}
              M^\perp
              =\set{\colvec{u \\ v}\suchthat u+v=0}
              =\set{k\cdot\colvec[r]{-1 \\ 1}\suchthat k\in\Re}
            \end{equation*}
            (By the way, this answer checks with the first item in this 
            question.)
          \partsitem Every vector in the space is perpendicular to the zero
            vector so $M^\perp=\Re^n$.
          \partsitem The appropriate description and basis for $M$ are routine.
            \begin{equation*}
              M=\set{y\cdot\colvec[r]{0 \\ 1}\suchthat y\in\Re}
              \qquad
              B_M=\sequence{\colvec[r]{0 \\ 1}}
            \end{equation*}
            Then
            \begin{equation*}
              M^\perp
              =\set{\colvec{u \\ v}\suchthat 0\cdot u+1\cdot v=0}
              =\set{k\cdot\colvec[r]{1 \\ 0}\suchthat k\in\Re}
            \end{equation*}
            and so $(\text{$y$-axis})^\perp=\text{$x$-axis}$.
          \partsitem The description of $M$ is easy to find by parametrizing.
            \begin{equation*}
              M=\set{c\cdot\colvec[r]{3 \\ 1 \\ 0}
                     +d\cdot\colvec[r]{1 \\ 0 \\ 1}\suchthat c,d\in\Re}
              \qquad
              B_M=\sequence{\colvec[r]{3 \\ 1 \\ 0},\colvec[r]{1 \\ 0 \\ 1}}
            \end{equation*}
            Finding $M^\perp$ here just requires solving a linear system
            with two equations
            \begin{equation*}
              \begin{linsys}{3}
                3u  &+  &v  &   &   &=  &0  \\
                 u  &   &   &+  &w  &=  &0
              \end{linsys}
              \grstep{-(1/3)\rho_1+\rho_2}
              \begin{linsys}{3}
                3u  &+  &v         &   &   &=  &0  \\
                    &   &-(1/3)v   &+  &w  &=  &0
              \end{linsys}
            \end{equation*}
            and parametrizing.
            \begin{equation*}
              M^\perp
              =\set{k\cdot\colvec[r]{-1 \\ 3 \\ 1}\suchthat k\in\Re}
            \end{equation*}
          \partsitem Here, $M$ is one-dimensional
            \begin{equation*}
              M=\set{c\cdot\colvec[r]{0 \\ -1 \\ 1}\suchthat c\in\Re}
              \qquad
              B_M=\sequence{\colvec[r]{0 \\ -1 \\ 1}}
            \end{equation*}
            and as a result, $M^\perp$ is two-dimensional.
            \begin{equation*}
              M^\perp
              =\set{\colvec{u \\ v \\ w}
                     \suchthat 0\cdot u-1\cdot v+1\cdot w=0}
              =\set{j\cdot\colvec[r]{1 \\ 0 \\ 0}
                    +k\cdot\colvec[r]{0 \\ 1 \\ 1}\suchthat j,k\in\Re}
            \end{equation*}
       \end{exparts}
     \end{answer}
  \recommended \item Find the orthogonal projection of the vector into 
    the subspace.
    \begin{equation*}
      \colvec{1 \\ 2 \\ 0} 
      \qquad
      S=\spanof{\,\set{\colvec{0 \\ 2 \\ 0},\colvec{1 \\ -1 \\ 1}}\,}
    \end{equation*}
    \begin{answer}
      Where
      \begin{equation*}
        A=
        \begin{mat}
          0 &1 \\
          2 &-1 \\
          0 &1
        \end{mat}
      \end{equation*}
      straightforward, although tedious, calculation gives this.
      % sage: A = matrix(QQ, [[0,1], [2,-1], [0,1]])
      % sage: A*(A.transpose()*A).inverse()*A.transpose()
      % [1/2    0  1/2]
      % [  0    1    0]
      % [1/2    0  1/2]
      \begin{equation*}
        A\bigl(\trans{A}A\bigr)^{-1}\trans{A}=
        \begin{mat}
          1/2 &0 &1/2 \\
           0  &1 &0    \\
          1/2 &0 &1/2
        \end{mat}
      \end{equation*}
      When applied to the vector we get the projection.
      \begin{equation*}
        \begin{mat}
          1/2 &0 &1/2 \\
           0  &1 &0    \\
          1/2 &0 &1/2
        \end{mat}
        \colvec{1 \\ 2 \\ 0}
        =
        \colvec{1/2 \\ 2 \\  1/2}        
      \end{equation*}
      % sage: v = vector(QQ, [1, 2, 0])
      % sage: v
      % (1, 2, 0)
      % sage: A*(A.transpose()*A).inverse()*A.transpose()*v
      % (1/2, 2, 1/2)
    \end{answer}
  \recommended \item With the same subspace as in the prior problem, find the 
    orthogonal projection of this vector.
    \begin{equation*}
      \colvec{1 \\ 2 \\ -1}
    \end{equation*}
    \begin{answer}
      Using the matrix calculation from the prior answer, this 
      \begin{equation*}
        \begin{mat}
          1/2 &0 &-1/2 \\
           0  &1 &0    \\
         -1/2 &0 &1/2
        \end{mat}
        \colvec{1 \\ 2 \\ -1}
        =
        \colvec{1 \\ 2 \\ -1}        
      \end{equation*}
      shows the vector is in the subspace~$S$.
    \end{answer}
  \recommended \item \label{exer:ProjMinimizesDistSubspace}
    Let $\vec{p}$ be the orthogonal projection of 
    $\vec{v}\in\Re^n$ onto a subspace~$S$.
    Show that $\vec{p}$ is the point in the subspace closest to~$\vec{v}$.
    \begin{answer}
      Suppose $\vec{w}\in S$.
      Because this is orthogonal projection, 
      the two vectors $\vec{v}-\vec{p}$ and $\vec{p}-\vec{w}$ are at 
      a right angle.
      The Triangle Inequality applies and the hypotenuse $\vec{v}-\vec{w}$
      is therefore at least as long as $\vec{v}-\vec{p}$. 
    \end{answer}
  \item 
    This subsection shows how to project orthogonally in two ways,
    the method of \nearbyexample{ex:OrthProjOne} 
    and~\ref{ex:ProjIntoMAlongNGener}, and the method of 
    \nearbytheorem{th:OrthProjMat}.
    To compare them, 
    consider the plane $P$ specified by $3x+2y-z=0$ in $\Re^3$.
    \begin{exparts}
      \partsitem Find a basis for $P$.
      \partsitem Find $P^\perp$ and a basis for $P^\perp$. 
      \partsitem Represent this vector with respect to the concatenation
        of the two bases from the prior item.
        \begin{equation*}
          \vec{v}=\colvec[r]{1 \\ 1 \\ 2}
        \end{equation*}
      \partsitem Find the orthogonal projection of $\vec{v}$ into $P$
        by keeping only the $P$ part from the prior item.
      \partsitem Check that against the result from applying
        \nearbytheorem{th:OrthProjMat}.
    \end{exparts}
    \begin{answer}
      \begin{exparts}
        \partsitem Parametrizing the equation leads to this basis for
          $P$.
          \begin{equation*}
            B_P=\sequence{\colvec[r]{1 \\ 0 \\ 3},
                          \colvec[r]{0 \\ 1 \\ 2} }
          \end{equation*}
        \partsitem Because $\Re^3$ is three-dimensional 
          and $P$ is two-dimensional, the complement $P^\perp$ must be
          a line.
          Anyway, the calculation as in \nearbyexample{ex:OrthoCompOne}
          \begin{equation*}
            P^\perp=\set{\colvec{x \\ y \\ z}
                         \suchthat
                         \begin{mat}[r]
                           1  &0 &3  \\
                           0  &1 &2
                         \end{mat}
                         \colvec{x \\ y \\ z}
                         =\colvec[r]{0 \\ 0}\;}
          \end{equation*}
          gives this basis for $P^\perp$.
          \begin{equation*}
            B_{P^\perp}=\sequence{\colvec[r]{3 \\ 2 \\ -1} }
          \end{equation*}
        \partsitem $ \displaystyle       
             \colvec[r]{1 \\ 1 \\ 2}=
             (5/14)\cdot\colvec[r]{1 \\ 0 \\ 3}+
             (8/14)\cdot\colvec[r]{0 \\ 1 \\ 2}+
             (3/14)\cdot\colvec[r]{3 \\ 2 \\ -1}$
        \partsitem $\proj{\colvec[r]{1 \\ 1 \\ 2}}{P}
          =\colvec[r]{5/14 \\ 8/14 \\ 31/14}$
        \partsitem The matrix of the projection
          \begin{multline*}
            \begin{mat}[r]
              1  &0 \\
              0  &1 \\
              3  &2
            \end{mat}
            \bigl(
              \begin{mat}[r]
                1  &0  &3 \\
                0  &1  &2 
              \end{mat}
              \begin{mat}[r]
                1  &0 \\
                0  &1 \\
                3  &2
              \end{mat}
            \bigr)^{-1}
            \begin{mat}[r]
              1  &0  &3 \\
              0  &1  &2 
            \end{mat}                               \\
            \begin{aligned}
            &=
            \begin{mat}[r]
              1  &0 \\
              0  &1 \\
              3  &2
            \end{mat}
              \begin{mat}[r]
                10  &6  \\
                6   &5
              \end{mat}^{-1}
            \begin{mat}[r]
              1  &0  &3 \\
              0  &1  &2 
            \end{mat}                                     \\
            &=
            \frac{1}{14}
            \begin{mat}[r]
              5  &-6 &3  \\
             -6  &10 &2  \\
              3  &2  &13
            \end{mat}
            \end{aligned}
          \end{multline*}
          when applied to the vector, yields the expected result.
          \begin{equation*}
            \frac{1}{14}
            \begin{mat}[r]
              5  &-6 &3  \\
             -6  &10 &2  \\
              3  &2  &13
            \end{mat}
            \colvec[r]{1 \\ 1 \\ 2}
            =\colvec[r]{5/14 \\ 8/14 \\ 31/14}
          \end{equation*}
      \end{exparts}
    \end{answer}
  \item 
    We have three ways to find the orthogonal projection of a 
    vector into a line, the \nearbydefinition{def:ProjIntoLine} way from
    the first subsection of this section, the  
    \nearbyexample{ex:OrthProjOne}  and~\ref{ex:ProjIntoMAlongNGener}   
    way of representing 
    the vector with respect to a basis for the space and then keeping the
    $M$~part, and the way of \nearbytheorem{th:OrthProjMat}.
    For these cases, do all three ways.
     \begin{exparts}
       \partsitem \( \vec{v}=\colvec[r]{1 \\ -3},\quad
                M=\set{\colvec{x \\ y}\suchthat x+y=0} \)
       \partsitem \( \vec{v}=\colvec[r]{0 \\ 1 \\ 2},\quad
                M=\set{\colvec{x \\ y \\ z}
                        \suchthat \text{$x+z=0$ and $y=0$}} \)
     \end{exparts}
     \begin{answer}
       \begin{exparts}
         \partsitem Parametrizing gives this.
           \begin{equation*}
             M=\set{c\cdot\colvec[r]{-1 \\ 1}\suchthat c\in\Re}
           \end{equation*}
           
           For the first way, we take the vector spanning the line $M$ to be
           \begin{equation*}
             \vec{s}=\colvec[r]{-1 \\ 1}
           \end{equation*}
           and the \nearbydefinition{def:ProjIntoLine} formula gives this.
           \begin{equation*}
             \proj{\colvec[r]{1 \\ -3}}{\spanof{\vec{s}\,}}
              =\frac{\colvec[r]{1 \\ -3}\dotprod\colvec[r]{-1 \\ 1}}{
                     \colvec[r]{-1 \\ 1}\dotprod\colvec[r]{-1 \\ 1}}
                \cdot\colvec[r]{-1 \\ 1}
              =\frac{-4}{2}
                \cdot\colvec[r]{-1 \\ 1}
              =\colvec[r]{2 \\ -2}
           \end{equation*}

           For the second way, we fix
           \begin{equation*}
             B_M=\sequence{\colvec[r]{-1 \\ 1}}
           \end{equation*}
           and so
           (as in \nearbyexample{ex:OrthoCompOne} and~\ref{ex:OrthoCompTwo}, 
           we can just find the vectors perpendicular to all of the members 
           of the basis)
           \begin{equation*}
             M^\perp
             =\set{\colvec{u \\ v}\suchthat -1\cdot u+1\cdot v=0}
             =\set{k\cdot\colvec[r]{1 \\ 1}\suchthat k\in\Re}
             \qquad
             B_{M^\perp}=\sequence{\colvec[r]{1 \\ 1}}
           \end{equation*}
           and representing the vector with respect to the concatenation
           gives this.
           \begin{equation*}
             \colvec[r]{1 \\ -3}=-2\cdot\colvec[r]{-1 \\ 1}-1\cdot\colvec[r]{1 \\ 1}
           \end{equation*}
           Keeping the $M$~part yields the answer.
           \begin{equation*}
             \proj{\colvec[r]{1 \\ -3}}{M,M^\perp}=\colvec[r]{2 \\ -2}
           \end{equation*}

           The third part is also a simple calculation
           (there is a $\nbyn{1}$ matrix in the middle,
           and the inverse of it is also $\nbyn{1}$)
           \begin{multline*}
             A\left(\trans{A}A\right)^{-1}\trans{A}           \\
             \begin{aligned}
             &=
             \begin{mat}[r]
               -1 \\ 1
             \end{mat}
             \left(
               \begin{mat}[r]
                 -1  &1
               \end{mat}
               \begin{mat}[r]
                 -1  \\
                  1
               \end{mat}
             \right)^{-1}
             \begin{mat}[r]
               -1  &1
             \end{mat}
             =
             \begin{mat}[r]
               -1 \\ 1
             \end{mat}
               \begin{mat}[r]
                 2
               \end{mat}^{-1}
             \begin{mat}[r]
               -1  &1
             \end{mat}               \\
             &=
             \begin{mat}[r]
               -1 \\ 1
             \end{mat}
               \begin{mat}[r]
                 1/2
               \end{mat}
             \begin{mat}[r]
               -1  &1
             \end{mat}               
             =
             \begin{mat}[r]
               -1 \\ 1
             \end{mat}
             \begin{mat}[r]
               -1/2  &1/2
             \end{mat}               
             =
             \begin{mat}[r]
               1/2  &-1/2  \\
              -1/2  &1/2  
             \end{mat}
             \end{aligned}
           \end{multline*}
           which of course gives the same answer.
           \begin{equation*}
             \proj{\colvec[r]{1 \\ -3}}{M}
             =
             \begin{mat}[r]
               1/2  &-1/2  \\
              -1/2  &1/2  
             \end{mat}
             \colvec[r]{1 \\ -3}
             =\colvec[r]{2 \\ -2}
           \end{equation*}
         \partsitem Parametrization gives this.
           \begin{equation*}
             M=\set{c\cdot\colvec[r]{-1 \\ 0 \\ 1}\suchthat c\in\Re}
           \end{equation*}
           With that, the formula for the first way gives this.
           \begin{equation*}
             \frac{\colvec[r]{0 \\ 1 \\ 2}\dotprod\colvec[r]{-1 \\ 0 \\ 1}}{
                   \colvec[r]{-1 \\ 0 \\ 1}\dotprod\colvec[r]{-1 \\ 0 \\ 1}}
               \cdot\colvec[r]{-1 \\ 0 \\ 1}
             =\frac{2}{2}
               \cdot\colvec[r]{-1 \\ 0 \\ 1}
             =\colvec[r]{-1 \\ 0 \\ 1}
           \end{equation*}
           To proceed by the second method we find $M^\perp$,
           \begin{equation*}
             M^\perp
             =\set{\colvec{u \\ v \\ w}\suchthat -u+w=0}
             =\set{j\cdot\colvec[r]{1 \\ 0 \\ 1}
                   +k\cdot\colvec[r]{0 \\ 1 \\ 0}\suchthat j,k\in\Re}
           \end{equation*}
           find the representation of the given vector with respect to the
           concatenation of the bases $B_M$ and $B_{M^\perp}$
           \begin{equation*}
             \colvec[r]{0 \\ 1 \\ 2}
              =1\cdot\colvec[r]{-1 \\ 0 \\ 1}
               +1\cdot\colvec[r]{1 \\ 0 \\ 1}
               +1\cdot\colvec[r]{0 \\ 1 \\ 0}
           \end{equation*}
           and retain only the $M$~part.
           \begin{equation*}
             \proj{\colvec[r]{0 \\ 1 \\ 2}}{M}=1\cdot\colvec[r]{-1 \\ 0 \\ 1}
                       =\colvec[r]{-1 \\ 0 \\ 1}
           \end{equation*}
           Finally, for the third method, the matrix calculation
           \begin{multline*}
             A\left(\trans{A}A\right)^{-1}\trans{A}                 \\
             \begin{aligned}
             &=
             \begin{mat}[r]
               -1 \\ 0  \\  1
             \end{mat}
             \bigl(
               \begin{mat}[r]
                 -1  &0  &1
               \end{mat}
               \begin{mat}[r]
                 -1  \\
                  0  \\
                  1
               \end{mat}
             \bigr)^{-1}
             \begin{mat}[r]
               -1  &0  &1
             \end{mat}
             =
             \begin{mat}[r]
               -1 \\ 0  \\  1
             \end{mat}
               \begin{mat}[r]
                 2
               \end{mat}^{-1}
             \begin{mat}[r]
               -1  &0  &1
             \end{mat}              \\
             &=
             \begin{mat}[r]
               -1 \\ 0  \\  1
             \end{mat}
               \begin{mat}[r]
                 1/2
               \end{mat}
             \begin{mat}[r]
               -1  &0  &1
             \end{mat}
             =
             \begin{mat}[r]
               -1 \\ 0  \\  1
             \end{mat}
             \begin{mat}[r]
               -1/2  &0  &1/2
             \end{mat}                                 \\
             &=
             \begin{mat}[r]
               1/2  &0  &-1/2  \\
               0    &0  &0     \\
              -1/2  &0  &1/2
             \end{mat}
             \end{aligned}
           \end{multline*}
           followed by matrix-vector multiplication
           \begin{equation*}
             \proj{\colvec[r]{0 \\ 1 \\ 2}}{M}
             \begin{mat}[r]
               1/2  &0  &-1/2  \\
               0    &0  &0     \\
              -1/2  &0  &1/2
             \end{mat}
             \colvec[r]{0 \\ 1 \\ 2}
             =\colvec[r]{-1 \\ 0 \\ 1}
           \end{equation*}
           gives the answer.
       \end{exparts}
     \end{answer}
  \item 
    Check that the operation of \nearbydefinition{def:ProjIntoMAlongN}
    is well-defined.  
    That is, in \nearbyexample{ex:OrthProjOne} 
    and~\ref{ex:ProjIntoMAlongNGener}, 
    doesn't the answer depend on the choice of bases?
    \begin{answer}
      No, a decomposition of vectors $\vec{v}=\vec{m}+\vec{n}$ into
      $\vec{m}\in M$ and $\vec{n}\in N$ does not depend on the bases
      chosen for the subspaces, as we showed 
      in the Direct Sum subsection.
    \end{answer}
  \item 
    What is the orthogonal projection into the trivial subspace?
    \begin{answer}
      The orthogonal projection of a vector into a subspace is a member of
      that subspace.
      Since a trivial subspace has only one member, $\zero$, the projection of 
      any vector must equal $\zero$.
    \end{answer}
  \item 
    What is the projection of \( \vec{v} \) into \( M \) along \( N \)
    if \( \vec{v}\in M \)?
    \begin{answer}
      The projection into $M$ along
      $N$ of a $\vec{v}\in M$ is $\vec{v}$.
      Decomposing
      $\vec{v}=\vec{m}+\vec{n}$ gives $\vec{m}=\vec{v}$ and $\vec{n}=\zero$,
      and dropping the $N$~part but retaining the $M$~part results in a 
      projection of $\vec{m}=\vec{v}$. 
    \end{answer}
  \item
    Show that if \( M\subseteq\Re^n \) is a subspace with orthonormal basis
    \( \basis{\kappa}{n} \) then
    the orthogonal projection of \( \vec{v} \) into \( M \) is this.
    \begin{equation*}
      (\vec{v}\dotprod\vec{\kappa}_1)\cdot\vec{\kappa}_1+
      \dots+
      (\vec{v}\dotprod\vec{\kappa}_n)\cdot\vec{\kappa}_n
    \end{equation*}
    \begin{answer}
      The proof of \nearbylemma{le:OrthoProjWellDefd} shows that
      each vector $\vec{v}\in\Re^n$ 
      is the sum of its orthogonal projections
      into the lines spanned by the basis vectors.
      \begin{equation*}
        \vec{v}=\proj{\vec{v}\,}{\spanof{\vec{\kappa}_1}}
                 +\dots+\proj{\vec{v}\,}{\spanof{\vec{\kappa}_n}}
               =\frac{\vec{v}\dotprod\vec{\kappa}_1}{
                      \vec{\kappa}_1\dotprod\vec{\kappa}_1}
                   \cdot\vec{\kappa}_1
               +\dots
               +\frac{\vec{v}\dotprod\vec{\kappa}_n}{
                      \vec{\kappa}_n\dotprod\vec{\kappa}_n}
                   \cdot\vec{\kappa}_n
      \end{equation*}
      Since the basis is orthonormal, the bottom of each fraction has 
      $\vec{\kappa}_i\dotprod\vec{\kappa}_i=1$.
    \end{answer}
  \recommended \item
    Prove that the map \( \map{p}{V}{V} \) is the projection into \( M \)
    along \( N \) if and only if the map 
    \( \identity-p \) is the projection into \( N \) along \( M \).
    (Recall the definition of the difference of two 
    maps:~$(\identity-p)\,(\vec{v})=\identity(\vec{v})-p(\vec{v})
     =\vec{v}-p(\vec{v})$.)
     \begin{answer}
       If $V=M\directsum N$ then every vector decomposes uniquely as
       $\vec{v}=\vec{m}+\vec{n}$.
       For all $\vec{v}$ the map $p$ gives $p(\vec{v})=\vec{m}$ if and only
       if $\vec{v}-p(\vec{v})=\vec{n}$, as required.
     \end{answer}
  \item \label{exer:PerpOnBasisImplPerp} 
    Show that if a vector is perpendicular to every vector in a set then
    it is perpendicular to every vector in the span of that set.
    \begin{answer}
      Let $\vec{v}$ be perpendicular to every $\vec{w}\in S$.
      Then 
      $\vec{v}\dotprod(c_1\vec{w}_1+\dots+c_n\vec{w}_n)
       =\vec{v}\dotprod(c_1\vec{w}_1)
          +\dots+\vec{v}\dotprod (c_n\dotprod\vec{w}_n)
       =c_1(\vec{v}\dotprod\vec{w}_1)
          +\dots+c_n(\vec{v}\dotprod\vec{w}_n)
       =c_1\cdot 0+\dots+c_n\cdot 0=0$.
    \end{answer}
  \item
    True or false:~the intersection of a subspace and its orthogonal
    complement is trivial.
    \begin{answer}
      True; the only vector orthogonal to itself is the zero vector.
    \end{answer}
  \item 
    Show that the dimensions of orthogonal complements add to the 
    dimension of the entire space.
    \begin{answer}
      This is immediate from the statement in 
      \nearbylemma{le:OrthoProjWellDefd} that the space is the direct sum 
      of the two.
    \end{answer}
  \item
    Suppose that \( \vec{v}_1,\vec{v}_2\in\Re^n \) are such that for all
    complements \( M,N\subseteq\Re^n \), the projections of \( \vec{v}_1 \)
    and \( \vec{v}_2 \) into \( M \) along \( N \) are equal.
    Must \( \vec{v}_1 \) equal \( \vec{v}_2 \)?
    (If so, what if we relax the condition to:~all
     orthogonal projections of the two are equal?)
    \begin{answer}
      The two must be equal, even only under the seemingly weaker
      condition that they yield the same result on all orthogonal projections.
      Consider the subspace $M$ spanned by the set 
      $\set{\vec{v}_1,\vec{v}_2}$.
      Since each is in $M$, the orthogonal projection of $\vec{v}_1$ into 
      $M$ is $\vec{v}_1$ and the orthogonal projection of $\vec{v}_2$ into
      $M$ is $\vec{v}_2$.
      For their projections into $M$ to be equal, they must be equal.
    \end{answer}
  \recommended \item \label{exer:AlgOfPerps}
    Let \( M,N \) be subspaces of \( \Re^n \).
    The perp operator acts on subspaces; we can ask how it interacts
    with other such operations.
    \begin{exparts}
      \partsitem Show that two perps cancel: \( (M^\perp)^\perp=M \).
      \partsitem %Schaums p331 #114
        Prove that \( M\subseteq N \) implies
           that \( N^\perp\subseteq M^\perp \).
      \partsitem Show that \( (M+N)^\perp=M^\perp\intersection N^\perp \).
      %\partsitem Show that \( (M\intersection N)^\perp=M^\perp +N^\perp \).
    \end{exparts}
    \begin{answer}
      \begin{exparts}
        \partsitem 
          We will show that the sets are mutually inclusive,
          $M\subseteq (M^\perp)^\perp$ and $(M^\perp)^\perp \subseteq M$.
          For the first, 
          if $\vec{m}\in M$ then by the definition of the perp operation,
          $\vec{m}$ is perpendicular to every $\vec{v}\in M^\perp$,
          and therefore (again by the definition of the perp operation)
          $\vec{m}\in (M^\perp)^\perp$.
          For the other direction, consider $\vec{v}\in (M^\perp)^\perp$.
          \nearbylemma{le:OrthoProjWellDefd}'s proof shows that  
          $\Re^n=M\directsum M^\perp$ and that
          we can give an orthogonal basis for the space 
          $\sequence{\vec{\kappa}_1,\dots,\vec{\kappa}_k,
                        \vec{\kappa}_{k+1},\dots,\vec{\kappa}_n}$
          such that the first half
          $\sequence{\vec{\kappa}_1,\dots,\vec{\kappa}_k}$ is a basis for
          $M$ and the second half is a basis for $M^\perp$.
          The proof also checks that
          each vector in the space is the sum of its orthogonal projections 
          into the lines spanned by these basis vectors.
          \begin{equation*}
             \vec{v}=\proj{\vec{v}\,}{\spanof{\vec{\kappa}_1}}
                          +\dots+\proj{\vec{v}\,}{\spanof{\vec{\kappa}_n}}
          \end{equation*}
          Because $\vec{v}\in (M^\perp)^\perp$, it is perpendicular to 
          every vector in $M^\perp$, and so the projections in 
          the second half are all zero. 
          Thus $\vec{v}=\proj{\vec{v}\,}{\spanof{\vec{\kappa}_1}}
                          +\dots+\proj{\vec{v}\,}{\spanof{\vec{\kappa}_k}}$,
          which is a linear combination of vectors from $M$, and 
          so $\vec{v}\in M$.
          (\textit{Remark}.
           Here is a slicker way to do the second half:~write the space
           both as $M\directsum M^\perp$ and as 
           $M^\perp\directsum (M^\perp)^\perp$.
           Because the first half showed that $M\subseteq (M^\perp)^\perp$
           and the prior sentence shows that the dimension of the two
           subspaces $M$ and $(M^\perp)^\perp$ are equal, we can conclude
           that $M$ equals $(M^\perp)^\perp$.)
        \partsitem Because $M\subseteq N$, any $\vec{v}$ that is perpendicular
           to every vector in $N$ is also perpendicular to every vector in 
           $M$.
           But that sentence simply says that $N^\perp\subseteq M^\perp$.
        \partsitem We will again show that the sets are equal by mutual 
           inclusion.
           The first direction is easy; any $\vec{v}$ perpendicular to
           every vector in 
           $M+N=\set{\vec{m}+\vec{n}\suchthat \vec{m}\in M,\, \vec{n}\in N}$
           is perpendicular to every vector of the form $\vec{m}+\zero$
           (that is, every vector in $M$) and every vector of the form
           $\zero+\vec{n}$ (every vector in $N$), and so 
           $(M+N)^\perp\subseteq M^\perp\intersection N^\perp$. 
           The second direction is also routine; any vector 
           $\vec{v}\in M^\perp\intersection N^\perp$ 
           is perpendicular to any vector of the form $c\vec{m}+d\vec{n}$
           because 
           $\vec{v}\dotprod (c\vec{m}+d\vec{n})
             =c\cdot(\vec{v}\dotprod\vec{m})+d\cdot(\vec{v}\dotprod\vec{n})
             =c\cdot 0+d\cdot 0
             =0$.
      \end{exparts}
    \end{answer}
  \recommended \item 
    The material in this subsection allows us to express a geometric 
    relationship that we have not yet seen  between the range space and the 
    null space of a linear map.
    \begin{exparts}
      \partsitem Represent $\map{f}{\Re^3}{\Re}$ given by 
        \begin{equation*}
          \colvec{v_1 \\ v_2 \\ v_3}
          \mapsto
          1v_1+2v_2+ 3v_3
        \end{equation*}
        with respect to the standard bases and show that
        \begin{equation*}
          \colvec[r]{1 \\ 2 \\ 3}
        \end{equation*}
        is a member of the perp of the null space.
        Prove that  $\nullspace{f}^\perp$ is equal to the span of this 
        vector.
      \partsitem Generalize that to apply to any $\map{f}{\Re^n}{\Re}$.
      \partsitem Represent $\map{f}{\Re^3}{\Re^2}$ 
        \begin{equation*}
          \colvec{v_1 \\ v_2 \\ v_3}
          \mapsto
          \colvec{1v_1+2v_2+ 3v_3 \\
                  4v_1+5v_2+ 6v_3}   
        \end{equation*}
        with respect to the standard bases and show that
        \begin{equation*}
          \colvec[r]{1 \\ 2 \\ 3},
          \;\colvec[r]{4 \\ 5 \\ 6}
        \end{equation*}
        are both members of the perp of the null space.
        Prove that $\nullspace{f}^\perp$ is the span of these two.
        (\textit{Hint}.  
         See the third item of \nearbyexercise{exer:AlgOfPerps}.)
      \partsitem Generalize that to apply to any $\map{f}{\Re^n}{\Re^m}$.
    \end{exparts}
    In \cite{Strang93}
    this is called the 
    \definend{Fundamental Theorem of Linear Algebra}%
    \index{Fundamental Theorem!of Linear Algebra}
    \begin{answer}
      \begin{exparts}
        \partsitem The representation of
          \begin{equation*}
            \colvec{v_1 \\ v_2 \\ v_3}
            \mapsunder{f}
            1v_1+2v_2+ 3v_3    
          \end{equation*}
          is this.
          \begin{equation*}
            \rep{f}{\stdbasis_3,\stdbasis_1}
            =\begin{mat}[r]
               1  &2  &3
             \end{mat}
          \end{equation*}
          By the definition of $f$
          \begin{equation*}
            \nullspace{f}
            =\set{\colvec{v_1 \\ v_2 \\ v_3}
                  \suchthat 1v_1+2v_2+3v_3=0}
            =\set{\colvec{v_1 \\ v_2 \\ v_3}
                  \suchthat 
                   \colvec[r]{1 \\ 2 \\ 3}\dotprod\colvec{v_1 \\ v_2 \\ v_3}=0}
          \end{equation*}
          and this second description exactly says this.
          \begin{equation*}
            \nullspace{f}^\perp
            =\spanof{\set{\colvec[r]{1 \\ 2 \\ 3}}}
          \end{equation*}
        \partsitem The generalization is that for any $\map{f}{\Re^n}{\Re}$
           there is a vector $\vec{h}$ so that
           \begin{equation*}
             \colvec{v_1 \\ \vdots \\ v_n}
             \mapsunder{f}
             h_1v_1+\dots+h_nv_n
           \end{equation*}
           and $\vec{h}\in\nullspace{f}^\perp$.
           We can prove this by, as in the prior item, 
           representing $f$ with
           respect to the standard bases and taking $\vec{h}$ to be the
           column vector gotten by transposing the one row of that 
           matrix representation. 
         \partsitem Of course, 
           \begin{equation*}
             \rep{f}{\stdbasis_3,\stdbasis_2}
             =
             \begin{mat}[r]
               1  &2  &3  \\
               4  &5  &6
             \end{mat}
           \end{equation*}
           and so the null space is this set.
           \begin{equation*}
             \nullspace{f}
             \set{\colvec{v_1 \\ v_2 \\ v_3}
                  \suchthat
                   \;\begin{mat}[r]
                     1  &2  &3  \\
                     4  &5  &6
                   \end{mat}
                   \colvec{v_1 \\ v_2 \\ v_3}
                   =\colvec[r]{0  \\ 0}\;}
           \end{equation*}
           That description makes clear that
           \begin{equation*}
             \colvec[r]{1 \\ 2 \\ 3},
              \,\colvec[r]{4 \\ 5 \\ 6}\in\nullspace{f}^\perp
           \end{equation*}
           and since $\nullspace{f}^\perp$ is a subspace of $\Re^n$,
           the span of the two vectors is a subspace of the perp of 
           the null space.
           To see that this containment is an equality, take
           \begin{equation*}
             M=\spanof{\set{\colvec[r]{1 \\ 2 \\ 3}}}
             \qquad
             N=\spanof{\set{\colvec[r]{4 \\ 5 \\ 6}}}
           \end{equation*}
           in the third item of \nearbyexercise{exer:AlgOfPerps}, as suggested
           in the hint.
        \partsitem As above, generalizing from the specific case is easy:~for 
           any $\map{f}{\Re^n}{\Re^m}$
           the matrix $H$ representing the map with respect to the standard
           bases describes the action
           \begin{equation*}
             \colvec{v_1 \\ \vdots \\ v_n}
             \mapsunder{f}
             \colvec{h_{1,1}v_1+h_{1,2}v_2+\dots+h_{1,n}v_n \\
                     \vdots                                 \\
                     h_{m,1}v_1+h_{m,2}v_2+\dots+h_{m,n}v_n}
           \end{equation*}
           and the description of the null space gives that
           on transposing the $m$~rows of $H$
           \begin{equation*}
             \vec{h}_1=\colvec{h_{1,1} \\ h_{1,2} \\ \vdots \\ h_{1,n}},
              \dots
             \vec{h}_m=\colvec{h_{m,1} \\ h_{m,2} \\ \vdots \\ h_{m,n}}
           \end{equation*}
           we have $\nullspace{f}^\perp
                    =\spanof{\set{\vec{h}_1,\dots,\vec{h}_m}}$.
           (\cite{Strang93} describes this space as the 
           transpose of the row space of $H$.)
      \end{exparts}
    \end{answer}
  \item
    Define a \definend{projection\/}\index{projection} 
    to be a linear transformation
    \( \map{t}{V}{V} \) with the property that 
    repeating the projection does nothing more than does the projection
    alone:~\( (\composed{t}{t})\,(\vec{v})=t(\vec{v}) \) 
    for all \( \vec{v}\in V \).
    \begin{exparts}
      \partsitem Show that orthogonal projection into a line has that property.
      \partsitem Show that projection along a subspace has that property.
      \partsitem Show that for any such $t$
        there is a basis \( B=\basis{\beta}{n} \) for \( V \) such that
        \begin{equation*}
          t(\vec{\beta}_i)=
          \begin{cases}
                  \vec{\beta}_i  &i=1,2,\dots,\,r  \\
                  \zero          &i=r+1,r+2,\dots,\,n
                \end{cases}
        \end{equation*}
        where \( r \) is the rank of \( t \).
      \item Conclude that every projection is a projection along a subspace.
      \item Also conclude that every projection has a representation
        \begin{equation*}
          \rep{t}{B,B}=
          \begin{pmat}{c|c}
            I   &Z  \\ \hline
            Z   &Z
          \end{pmat}
        \end{equation*}
        in block partial-identity form.
    \end{exparts}
    \begin{answer}
      \begin{exparts}
        \partsitem First note that if a vector $\vec{v}$ is already in the 
          line then the orthogonal projection gives $\vec{v}$ itself.
          One way to verify this is to apply the 
          formula for projection into the line spanned by a vector $\vec{s}$,
          namely $(\vec{v}\dotprod\vec{s}/\vec{s}\dotprod\vec{s})\cdot\vec{s}$.
          Taking the line as
          $\set{k\cdot\vec{v}\suchthat k\in\Re}$
          (the $\vec{v}=\zero$ case is separate but easy) gives
          $(\vec{v}\dotprod\vec{v}/\vec{v}\dotprod\vec{v})\cdot\vec{v}$,
          which simplifies to $\vec{v}$, as required.
 
          Now, that answers the question because after once projecting into
          the line, the result $\proj{\vec{v}}{\ell}$ is in that line.
          The prior paragraph says that projecting into the same line again 
          will have no effect.  
        \partsitem The argument here is similar to the one in the prior item.
          With $V=M\directsum N$, the projection of $\vec{v}=\vec{m}+\vec{n}$
          is $\proj{\vec{v}\,}{M,N}=\vec{m}$.
          Now repeating the projection will give $\proj{\vec{m}}{M,N}=\vec{m}$,
          as required, because the decomposition of a member of $M$ into the
          sum of a member of $M$ and a member of $N$ is 
          $\vec{m}=\vec{m}+\zero$.
          Thus, projecting twice into $M$ along $N$ has the same effect as 
          projecting once.
        \partsitem As suggested by the prior items, the condition
          gives that $t$ leaves vectors in the range space unchanged,
          and hints that we should take 
          $\vec{\beta}_1$, \ldots, $\vec{\beta}_r$
          to be basis vectors for the range, that is, that we should take
          the range space of $t$ for $M$ (so that $\dim(M)=r$).
          As for the complement, we write $N$ for the null space of $t$
          and we will show that $V=M\directsum N$.

          To show this, we can show that their intersection
          is trivial $M\intersection N=\set{\zero}$ and that they sum to 
          the entire space $M+N=V$.
          For the first, if a vector $\vec{m}$ is in the range space
          then there is a $\vec{v}\in V$ with $t(\vec{v})=\vec{m}$,
          and the condition on $t$ gives that 
          $t(\vec{m})=(\composed{t}{t})\,(\vec{v})=t(\vec{v})=\vec{m}$, while
          if that same vector is also in the null space then $t(\vec{m})=\zero$
          and so the intersection of the range space and null space is trivial.
          For the second, to write an arbitrary $\vec{v}$ as the sum of
          a vector from the range space and a vector from the null space,
          the fact that the condition $t(\vec{v})=t(t(\vec{v}))$ can be 
          rewritten as $t(\vec{v}-t(\vec{v}))=\zero$ suggests taking
          $\vec{v}=t(\vec{v})+(\vec{v}-t(\vec{v}))$.
  
          To finish we taking a basis 
          $B=\sequence{\vec{\beta}_1,\ldots,\vec{\beta}_n}$ for $V$ where
          $\sequence{\vec{\beta}_1,\ldots,\vec{\beta}_r}$ is a basis for 
          the range space $M$ and 
          $\sequence{\vec{\beta}_{r+1},\ldots,\vec{\beta}_n}$ is a 
          basis for the null space $N$.
        \partsitem Every projection (as defined in this exercise) is 
          a projection into its range space and along its null space.
        \partsitem This also follows immediately from the third item.
      \end{exparts}
    \end{answer}
%  \item 
%    Show that projection into a plane followed by projection from the plane
%    to a line in the plane equals the single operation of projection into the
%    line.
%  \item 
%    Assume that \( V \) decomposes as \( M_1\directsum N_1 \) and also as
%    \( M_2\directsum N_2 \).
%    Let the function \( p_1 \) 
%    be the  projection into \( M_1 \) along \( N_1 \) and let
%    \( p_2 \) be the projection into \( M_2 \) along \( N_2 \).
%    Prove that if \( \composed{p_1}{p_2} \) and \( \composed{p_2}{p_1} \) are
%    both the zero map then \( p_1+p_2 \) projects into \( M_1+M_2 \) along
%    \( N_1\intersection N_2 \).
%  \item 
%    Why does the definition of projection into \( M \) along \( N \)
%    require that the subspaces be complements?
%    Can we relax that requirement and just have
%    \( N \intersection M=\set{\zero} \) (but not have the span of
%    \( N\union M \) be the whole space)?
%    Describe the projection of \( \vec{v} \) into \( M \) along \( N \)
%    if \( M=N \).
%    What if \( M\intersection N \) is nontrivial?
%    What if it is trivial?
%  \item 
%    \begin{exparts}
%      \partsitem Show that if the columns of a matrix \( A \) form a linearly
%        independent set then \( \trans{A}A \) is invertible.
%      \partsitem Show that if the columns of \( A \) are mutually 
%        orthogonal and
%        nonzero then \( \trans{A}A \) is the identity matrix.
%    \end{exparts}
  \item 
    A square matrix is \definend{symmetric\/}%\index{symmetric!matrix}%
    \index{matrix!symmetric} 
    if each \( i,j \) entry equals the
    \( j,i \) entry (i.e., if the matrix equals its transpose).
    Show that the projection matrix \( A(\trans{A}A)^{-1}\trans{A} \) 
    is symmetric. 
    \cite{Strang}
    \textit{Hint}.  Find properties of transposes by looking in the index
      under `transpose'.
    \begin{answer}
      For any matrix $M$ we have that $\trans{(M^{-1})}=(\trans{M})^{-1}$,
      and for any two matrices $M$, $N$ we have that 
      $\trans{MN}=\trans{N}\trans{M}$ (provided, of course, 
      that the inverse and product are defined).
      Applying these two gives that the matrix equals its transpose.
      \begin{multline*}
         \trans{\bigl(A(\trans{A}A)^{-1}\trans{A}\bigr)}
         =(\trans{\trans{A}})
           (\trans{\bigl((\trans{A}A)^{-1}\bigr)})
           (\trans{A})                               \\
         =(\trans{\trans{A}})
           ( \bigl(\trans{(\trans{A}A)}\bigr)^{-1} )
           (\trans{A})                    
         =A
           (\trans{A}\trans{\trans{A}})^{-1}
           \trans{A}
         =A
           (\trans{A}A)^{-1}
           \trans{A}
      \end{multline*}
    \end{answer}
\end{exercises}
