% \documentclass[9pt,t]{beamer}
\usefonttheme{professionalfonts}
\usefonttheme{serif}
\PassOptionsToPackage{pdfpagemode=FullScreen}{hyperref}
\PassOptionsToPackage{usenames,dvipsnames}{color}
% \DeclareGraphicsRule{*}{mps}{*}{}
\usepackage{linalgjh}
\usepackage{present}
\usepackage{directories}  % Define \detdir, detmpdir, etc.
\usepackage{xr}\externaldocument{\detdir det1} % read refs from .aux file
\usepackage{xr}\externaldocument{\mapdir map4} % read refs from .aux file
\usepackage{catchfilebetweentags}
\usepackage{etoolbox} % from http://tex.stackexchange.com/questions/40699/input-only-part-of-a-file-using-catchfilebetweentags-package
\makeatletter
\patchcmd{\CatchFBT@Fin@l}{\endlinechar\m@ne}{}
  {}{\typeout{Unsuccessful patch!}}
\makeatother

\mode<presentation>
{
  \usetheme{boxes}
  \setbeamercovered{invisible}
  \setbeamertemplate{navigation symbols}{} 
}
\addheadbox{filler}{\ }  % create extra space at top of slide 
\hypersetup{colorlinks=true,linkcolor=blue} 

\usepackage{tikz}
\usetikzlibrary{backgrounds}\usetikzlibrary{matrix}
\tikzset{node style ge/.style={circle}}

\title[Determinants] % (optional, use only with long paper titles)
{Four.I Definition of Determinant}

\author{\textit{Linear Algebra} \\ {\small Jim Hef{}feron}}
\institute{
  \texttt{http://joshua.smcvt.edu/linearalgebra}
}
\date{}


\subject{Determinants}
% This is only inserted into the PDF information catalog. Can be left
% out. 

\begin{document}
\begin{frame}
  \titlepage
\end{frame}

% =============================================
% \begin{frame}{Reduced Echelon Form} 
% \end{frame}



% ..... Four.I.1,2 .....
\section{Properties of Determinants}
%..........
\begin{frame}{Nonsingular matrices}
\noindent For any matrix, whether or not it is nonsingular is a key question.
Recall that an \( \nbyn{n} \) matrix \( T \) is nonsingular if and only if
each of these holds:%
\ExecuteMetaData[\detdir det1.tex]{EquivalentOfNonsingular}
This chapter develops a formula to determine whether a
matrix is nonsingular.
\end{frame}




\begin{frame}
\ExecuteMetaData[\detdir det1.tex]{DeterminantIntro}  
\end{frame}




\begin{frame}{The plan}
The prior slide gives a formula for the $\nbyn{1}$, $\nbyn{2}$, and~$\nbyn{3}$
determinants. 
But while those three formulas are a help, they don't make clear what
should be the general formula, for the $\nbyn{n}$~case.

So we will proceed by stating some conditions that a
determinant function must satisfy. 
The conditions extrapolate from the $\nbyn{1}$, $\nbyn{2}$, 
and~$\nbyn{3}$ cases; see the book's discussion.
They will let us compute the determinant of a square matrix via Gauss's Method, 
which we know to be fast and easy.

\pause
However, defining the determinant function by giving a list of conditions 
has a downside.
We must verify this function is well-defined,
that there is at least one function satisfying those conditions and also that
there is only one such.
To verify that, after giving the algorithm we will develop a formula,
the permutation expansion,
satisfying the conditions. 
\end{frame}




\begin{frame}{Definition of determinant}
\df[def:Det]
\ExecuteMetaData[\detdir det1.tex]{df:Det}

\pause 
\re[rem:SwapRowsRedun] % \hspace*{-1em} 
\ExecuteMetaData[\detdir det1.tex]{re:SwapRowsRedun}
\end{frame}




\begin{frame}{Consequences of the definition}
\lm[le:IdenRowsDetZero]
\ExecuteMetaData[\detdir det1.tex]{lm:IdenRowsDetZero}

\pause 
\pf 
\ExecuteMetaData[\detdir det1.tex]{pf:IdenRowsDetZero0}

\pause
\ExecuteMetaData[\detdir det1.tex]{pf:IdenRowsDetZero1}
\end{frame}
\begin{frame}
\ExecuteMetaData[\detdir det1.tex]{pf:IdenRowsDetZero2}

\pause
\ExecuteMetaData[\detdir det1.tex]{pf:IdenRowsDetZero3}  
\end{frame}
\begin{frame}
\ExecuteMetaData[\detdir det1.tex]{pf:IdenRowsDetZero4}
\qed
\end{frame}




\begin{frame}
% (As remarked above, it is not evident from the definition that there is
% a function satisfying the conditions, or that there is only one such function.
% We will show later that the determinant exists and is unique.
% For the moment we proceed without that assurance.)

% \pause
The conditions allow us to
compute the determinant of a matrix using Gauss's Method

\ex  On this matrix we can perform the Gauss's Method steps of
$-2\rho_1+\rho_2$ and $-3\rho_1+\rho_3$, followed by
$-(5/3)\rho_2+\rho_3$.
Condition~(1) says that these row combination operations
leave the determinant unchanged.
\begin{equation*}
  \begin{vmat}
    1  &3  &-2 \\
    2  &0  &4  \\
    3  &-1 &5
  \end{vmat}
  =
  \begin{vmat}
    1  &3   &-2 \\
    0  &-6  &-8  \\
    0  &-10 &-11
  \end{vmat}
  =
  \begin{vmat}
    1  &3   &-2 \\
    0  &-6  &-8  \\
    0  &0   &-7/3
  \end{vmat}
\end{equation*}
\pause
The determinant is the product down the
diagonal: 
$1\cdot(-6)\cdot(-7/3)=14$.
\end{frame}

\begin{frame}
\ex
To see what happens with a row swap,
consider this matrix. 
The swap changes the determinant's sign.
\begin{equation*}
  \begin{vmat}
    0  &3  &1 \\
    1  &2  &0 \\
    1  &5  &2
  \end{vmat}
  =
  -\begin{vmat}
    1  &2  &0 \\
    0  &3  &1 \\
    1  &5  &2
  \end{vmat}
\end{equation*}
Finish by performing $-\rho_1+\rho_3$
followed by $-\rho_2+\rho_3$
\begin{equation*}
  =-\begin{vmat}
    1  &2  &0 \\
    0  &3  &1 \\
    0  &3  &2
  \end{vmat}
  =-\begin{vmat}
    1  &2  &0 \\
    0  &3  &1 \\
    0  &0  &1
  \end{vmat}
\end{equation*}
and then multiplying down the diagonal.
The determinant of the original matrix is $-3$.
\end{frame}


\begin{frame}
\ex
Finally, to illustrate condition~(3) contrast these two.
\begin{equation*}
  \begin{vmat}
    5 &10 \\
    3 &4
  \end{vmat}
  \qquad
  \begin{vmat}
    1 &2 \\
    3 &4
  \end{vmat}
\end{equation*}
Condition~(3) gives
\begin{equation*}
  \begin{vmat}
    5 &10 \\
    3 &4
  \end{vmat}
  =5\cdot
  \begin{vmat}
    1 &2 \\
    3 &4
  \end{vmat}
\end{equation*}
and by performing $-3\rho_1+\rho_2$
and multiplying down the diagonal we get this.
\begin{equation*}
  =5\cdot
  \begin{vmat}
    1 &2 \\
    3 &4
  \end{vmat}
  =5\cdot
  \begin{vmat}
    1 &2 \\
    0 &-2
  \end{vmat}
  =5\cdot(-2)
  =-10
\end{equation*}
Thus here is the contrast.
\begin{equation*}
  \begin{vmat}
    5 &10 \\
    3 &4
  \end{vmat}
  =-10
  \qquad
  \begin{vmat}
    1 &2 \\
    3 &4
  \end{vmat}
  =-2
\end{equation*}
\end{frame}



\begin{frame}
  \frametitle{Warning}
The formula for the determinant of a $\nbyn{2}$ matrix has something to
do with multiplying along the diagonals.
\begin{equation*}
  \begin{vmat}
    a &b \\
    c &d
  \end{vmat}
  =ad-bc
\end{equation*}
Sometimes people have learned a mnemonic for the $\nbyn{3}$ formula
that also has to do with multplying diagonals.   
\begin{equation*}
  \begin{vmat}
    a &b &c \\
    d &e &f \\
    g &h &i 
  \end{vmat}
  ={\color{green}aei+bfg+cdh}
   {\color{red}-gec-hfa-idb}
  \qquad
  \begin{tikzpicture}[baseline=(A.center), scale=1]
  \path[draw, line width=4pt, color=green, opacity=0.5] (-1.35,.6)--(.3,-.6);
  \path[draw, line width=4pt, color=green, opacity=0.5] (-.75,.6)--(.8,-.6);
  \path[draw, line width=4pt, color=green, opacity=0.5] (-.25,.6)--(1.3,-.6);
  \path[draw, line width=4pt, color=red, opacity=0.5] (-1.3,-.6)--(.3,.6);
  \path[draw, line width=4pt, color=red, opacity=0.5] (-.75,-.6)--(.8,.6);
  \path[draw, line width=4pt, color=red, opacity=0.5] (-.25,-.6)--(1.3,.6);
  \node (A) at (0,0) {$\begin{pmat}{ccc|cc}
    a &b &c &a &b \\
    d &e &f &d &e \\
    g &h &i &g &h
  \end{pmat}$};
\end{tikzpicture}
\end{equation*}
Don't try to extend to $\nbyn{4}$ or larger sizes.
Instead, for larger matrices use Gauss's Method.
\end{frame}



\begin{frame}{Existence and Uniqueness}
As described before the definition of determinant,
defining the function by giving conditions that it must satisfy
has a downside.
We must verify that there is a 
function satisfying those conditions, and that there is only one such function.

As to existence,
how could there fail to be such a function, when we have been 
computing its outputs?
\ExecuteMetaData[\detdir det1.tex]{DifferentGaussMethodReductions}
But just because we get consistent results in this case
does not mean that all determinant computations
give the same value.
\end{frame}
\begin{frame}
Below we will give an alternative way to compute
the determinant, a formula.
It will make plain that the determinant is a function, 
that it returns well-defined outputs.
Computing a determinant with this formula  
is not practical since it is very slow
but it is nonetheless invaluable for the theory.

\pause
But before that we will show uniqueness, 
that if a function satisfying the conditions
exists then there is only one.

\lm[lm:DetFcnIsUnique]
\ExecuteMetaData[\detdir det1.tex]{lm:DetFcnIsUnique}

\pf 
\ExecuteMetaData[\detdir det1.tex]{pf:DetFcnIsUnique}
\qed
\end{frame}




% ..... Four.I.3 .....
\section{The Permutation Expansion}

\begin{frame}{The determinant function is not linear}
\ex
The determinant does not in general satisfy that 
$\det(k\cdot T)=k\cdot\det(T)$.
The second matrix here is twice the first
but the determinant does not double.
\begin{equation*}
  \begin{vmat}
    3  &-3  &9 \\
    1  &-1   &7 \\
    2  &4   &0
  \end{vmat}
  =-72
  \qquad
  \begin{vmat}
    6  &-6  &18 \\
    2  &-2   &14 \\
    4  &8   &0
  \end{vmat}
  =-576
\end{equation*}
Condition~(3) % of \nearbydefinition{def:Det}%
has the determinant scale one row at a time.   
\begin{align*}
  \begin{vmat}
    6  &-6  &18 \\
    2  &-2   &14 \\
    4  &8   &0
  \end{vmat}
  &=2\cdot
  \begin{vmat}
    3  &-3  &9 \\
    2  &-2   &14 \\
    4  &8   &0
  \end{vmat}           \\
  &=4\cdot
  \begin{vmat}
    3  &-3  &9 \\
    1  &-1   &7 \\
    4  &8   &0
  \end{vmat}           \\
  &=8\cdot
  \begin{vmat}
    3  &-3  &9 \\
    1  &-1   &7 \\
    2  &4   &0
  \end{vmat}         
\end{align*}
\end{frame}
\begin{frame}
So determinants are not linear:~with scalar multiplication,
the scalars come out one row at a time.
What happens with addition?

Consider condition~(3) applied to the case   
% \begin{equation*}
%   \det (\vec{\rho}_1,\dots,k\vec{\rho}_i,\dots,\vec{\rho}_n)
%        = k\cdot \det (\vec{\rho}_1,\dots,\vec{\rho}_i,\dots,\vec{\rho}_n)
% \end{equation*}
$k=2$.
\begin{equation*}
  \det (\vec{\rho}_1,\dots,2\vec{\rho}_i,\dots,\vec{\rho}_n)
       = 2\cdot \det (\vec{\rho}_1,\dots,\vec{\rho}_i,\dots,\vec{\rho}_n)
\end{equation*}
Rewrite it as addition. 
\begin{multline*}
  \det (\vec{\rho}_1,\dots,\vec{\rho}_i+\vec{\rho}_i,\dots,\vec{\rho}_n) \\
       = \det (\vec{\rho}_1,\dots,\vec{\rho}_i,\dots,\vec{\rho}_n)
         +\det (\vec{\rho}_1,\dots,\vec{\rho}_i,\dots,\vec{\rho}_n)
\end{multline*}
Of course this extends to $k=3$, etc.

This hints that besides bringing out scalars one row at a time,
determinants also break along a plus sign one row at a time.
\end{frame}


%..........
\begin{frame}{Multilinear}
\df[def:multilinear]
\ExecuteMetaData[\detdir det1.tex]{df:Multilinear}


\lm[lem:DetsMultilinear]
\ExecuteMetaData[\detdir det1.tex]{lm:DetsMultilinear}

\pause
\pf
\ExecuteMetaData[\detdir det1.tex]{pf:DetsMultilinear0}

\pause
\ExecuteMetaData[\detdir det1.tex]{pf:DetsMultilinear1}
\end{frame}
\begin{frame}
\ExecuteMetaData[\detdir det1.tex]{pf:DetsMultilinear2}
\end{frame}
\begin{frame}
\ExecuteMetaData[\detdir det1.tex]{pf:DetsMultilinear3}
\qed

\medskip
\noindent (\textit{Remark}. 
Some authors use multilinearity to define the determinant in place of our
four conditions that lead to Gauss's Method.)
\end{frame}



\begin{frame}
Multilinearity breaks a determinant into a sum of 
simple ones.

\ex
We can expand this determinant
\begin{equation*}
  \begin{vmat}
    1 &2 \\
    3 &4
  \end{vmat}
\end{equation*}
along the first row
\begin{equation*}
  =\begin{vmat}
    1 &0 \\
    3 &4
  \end{vmat}
  +\begin{vmat}
    0 &2 \\
    3 &4
  \end{vmat}
\end{equation*}
and then expand both of those on second row.
\begin{equation*}
  =\begin{vmat}
    1 &0 \\
    3 &0
  \end{vmat}
  +\begin{vmat}
    1 &0 \\
    0 &4
  \end{vmat}
  +\begin{vmat}
    0 &2 \\
    3 &0
  \end{vmat}
  +\begin{vmat}
    0 &2 \\
    0 &4
  \end{vmat}
\end{equation*}
Each matrix is simple in that  
of its rows are all zeroes except
for a single entry from the starting matrix.

\pause
Of these four, the first and last are~$0$ 
because the matrices are
nonsingular, since they have a second row that is a multiple of 
the first.
We are left with two determinants, where 
in each the matrix is all zeros except for 
one entry from the starting matrix in each row and each column.
We'll show our strategy for evaluating these determinants below.
\end{frame}




\begin{frame}
\ex
Similarly we can use multilinearity to expand this determinant. 
\begin{align*}
  \begin{vmat}
    1 &2 &3 \\
    4 &5 &6 \\
    7 &8 &9
  \end{vmat}
  &=\begin{vmat}
    1 &0 &0 \\
    4 &0 &0 \\
    7 &0 &0
  \end{vmat}
  +\begin{vmat}
    1 &0 &0 \\
    4 &0 &0 \\
    0 &8 &0
  \end{vmat}
  +\begin{vmat}
    1 &0 &0 \\
    4 &0 &0 \\
    0 &0 &9
  \end{vmat}  \\
  &\quad+\begin{vmat}
    1 &0 &0 \\
    0 &5 &0 \\
    7 &0 &0
  \end{vmat}
  +\begin{vmat}
    1 &0 &0 \\
    0 &5 &0 \\
    0 &8 &0
  \end{vmat}
  +\begin{vmat}
    1 &0 &0 \\
    0 &5 &0 \\
    0 &0 &9
  \end{vmat}           \\
  &\quad+\begin{vmat}
    1 &0 &0 \\
    0 &0 &6 \\
    7 &0 &0
  \end{vmat}
  +\begin{vmat}
    1 &0 &0 \\
    0 &0 &6 \\
    0 &8 &0
  \end{vmat}
  +\begin{vmat}
    1 &0 &0 \\
    0 &0 &6 \\
    0 &0 &9
  \end{vmat}       \\
  &\quad+\begin{vmat}
    0 &2 &0 \\
    4 &0 &0 \\
    7 &0 &0
  \end{vmat}
  +\qquad\cdots\qquad+
  \begin{vmat}
    0 &0 &3 \\
    0 &0 &6 \\
    0 &0 &9
  \end{vmat}
\end{align*}
In each of the $3^3=27$~determinants on the right the matrix is all zeros but
for a single entry from the starting matrix in each row. 
\end{frame}

\begin{frame}
\begin{equation*}
  \begin{vmat}
    1 &2 &3 \\
    4 &5 &6 \\
    7 &8 &9
  \end{vmat}
  =\begin{vmat}
    1 &0 &0 \\
    4 &0 &0 \\
    7 &0 &0
  \end{vmat}
  +\begin{vmat}
    1 &0 &0 \\
    4 &0 &0 \\
    0 &8 &0
  \end{vmat}
  +\cdots+
  \begin{vmat}
    0 &0 &3 \\
    0 &0 &6 \\
    0 &8 &0
  \end{vmat}
  +\begin{vmat}
    0 &0 &3 \\
    0 &0 &6 \\
    0 &0 &9
  \end{vmat}
\end{equation*}
For any of these, if two of the matrix rows have their original 
matrix entries in the same column then the determinant is~$0$
since then one matrix row is a multiple of the other.

\pause
We've reduced to a sum of determinants, 
where each matrix is all $0$'s but
for a single entry from the original in each row and column.
There are $3\cdot 2\cdot 1=6$ of these. 
\end{frame}
\begin{frame}
\vspace*{-3ex}
\begin{align*}
  \begin{vmat}
    1 &2 &3 \\
    4 &5 &6 \\
    7 &8 &9
  \end{vmat}
  &=\begin{vmat}
    1 &0 &0 \\
    0 &5 &0 \\
    0 &0 &9
  \end{vmat}
  +\begin{vmat}
    1 &0 &0 \\
    0 &0 &6 \\
    0 &8 &0
  \end{vmat}            \\
  &\quad+\begin{vmat}
    0 &2 &0 \\
    4 &0 &0 \\
    0 &0 &9
  \end{vmat}
  +\begin{vmat}
    0 &2 &0 \\
    0 &0 &6 \\
    7 &0 &0
  \end{vmat}         \\
  &\quad+\begin{vmat}
    0 &0 &3 \\
    4 &0 &0 \\
    0 &8 &0
  \end{vmat}
  +\begin{vmat}
    0 &0 &3 \\
    0 &5 &0 \\
    7 &0 &0
  \end{vmat}            \\
  &=45\cdot\begin{vmat}
    1 &0 &0 \\
    0 &1 &0 \\
    0 &0 &1
  \end{vmat}
  +48\cdot\begin{vmat}
    1 &0 &0 \\
    0 &0 &1 \\
    0 &1 &0
  \end{vmat}            \\
  &\quad+72\cdot\begin{vmat}
    0 &1 &0 \\
    1 &0 &0 \\
    0 &0 &1
  \end{vmat}
  +84\cdot\begin{vmat}
    0 &1 &0 \\
    0 &0 &1 \\
    1 &0 &0
  \end{vmat}         \\
  &\quad+96\cdot\begin{vmat}
    0 &0 &1 \\
    1 &0 &0 \\
    0 &1 &0
  \end{vmat}
  +105\cdot\begin{vmat}
    0 &0 &1 \\
    0 &1 &0 \\
    1 &0 &0
  \end{vmat}
\end{align*}
\end{frame}
\begin{frame}
\noindent
After bringing out each entry from the original matrix, we are left with 
matrices that are all $0$'s except for a single~$1$ in each row and column.
\end{frame}



\begin{frame}{Permutation matrices}
\ExecuteMetaData[\detdir det1.tex]{NotationForPermutationMatrices}

\pause
\df[df:permutation]
\ExecuteMetaData[\detdir det1.tex]{df:permutation}

So, in a permutation each number 
$1$, \ldots, $n$ is the output associated with one and only one input.
We sometimes denote a permutation as the sequence 
$\phi=\sequence{\phi(1),\phi(2),\ldots,\phi(n)}$.

\ex[ex:AllTwoThreePerms]
These are the $2$-permutations.
\begin{center}
  \begin{tabular}{rcc}
    $\phi_1$:  &$1\mapsto 1$  &$2\mapsto 2$  \\
    $\phi_2$:  &$1\mapsto 2$  &$2\mapsto 1$      
  \end{tabular}
\end{center}
The sequence notation is shorter:
\( \phi_1=\sequence{1,2} \) and \( \phi_2=\sequence{2,1} \).
% \ExecuteMetaData[\detdir det1.tex]{ex:AllTwoPerms}
\ex[ex:AllThreePerms]
\ExecuteMetaData[\detdir det1.tex]{ex:AllThreePerms}
\end{frame}
\begin{frame}
\ExecuteMetaData[\detdir det1.tex]{AssociatePermutationWithPermutationMatrix}

\ex
Associated with the $4$-permutation $\phi=\sequence{2,4,3,1}$ is the
matrix whose rows are the matching $\iota$'s.
\begin{equation*}
  P_{\phi}
  =
  \begin{mat}
    \iota_2 \\
    \iota_4 \\
    \iota_3 \\
    \iota_1
  \end{mat}
  =
  \begin{mat}
    0 &1 &0 &0 \\
    0 &0 &0 &1 \\
    0 &0 &1 &0 \\
    1 &0 &0 &0
  \end{mat}
\end{equation*}
\end{frame}




\begin{frame}{Permutation expansion}
\df[df:PermutationExpansion]
\ExecuteMetaData[\detdir det1.tex]{df:PermutationExpansion}

\pause
\medskip
\ExecuteMetaData[\detdir det1.tex]{SummationForPermutationExpansion}
\end{frame}
\begin{frame}
\ex
Recall that there are two $2$-permutations
\( \phi_1=\sequence{1,2} \) and \( \phi_2=\sequence{2,1} \).
% So for the $\nbyn{2}$ case, the sum over all permutations has two terms.
These are the associated permutation matrices
\begin{equation*}
  P_{\phi_1}=
  \begin{mat}
    1 &0 \\
    0 &1
  \end{mat}
  \qquad
  P_{\phi_2}=
  \begin{mat}
    0 &1 \\
    1 &0
  \end{mat}
\end{equation*}
\pause
giving this expansion.
\begin{align*}
  \begin{vmat}
    t_{1,1}  &t_{1,2} \\
    t_{2,1}  &t_{2,2}
  \end{vmat}
  &=
  t_{1,1}t_{2,2}\cdot
  \begin{vmat}
    1  &0 \\
    0  &1
  \end{vmat}               
  +
  t_{1,2}t_{2,1}\cdot
  \begin{vmat}
    0  &1 \\
    1  &0
  \end{vmat}               \\
  &=
  t_{1,1}t_{2,2}\cdot 1
  +
  t_{1,2}t_{2,1}\cdot (-1)
\end{align*}
(From prior work we know that the determinant $\deter{P_{\phi_2}}$ 
equals~$-1$ because we can bring that to 
the identity matrix with one row swap.)
\pause
Renaming the matrix entries gives the familiar $\nbyn{2}$ formula.
\begin{equation*}
  \begin{vmat}
    a  &b  \\
    c  &d
  \end{vmat}
  =ad-bc
\end{equation*}
\end{frame}




\begin{frame}
The only thing remaining in our process of finding a formula for 
the determinant (not involving Gauss's Method) is to give a formula for
the determinant of such matrices.
We do that in the next subsection.

\pause\bigskip
The next subsection is optional 
so for those not going on, we state its results here.
We will follow up on the second result.

\th[th:DetsExist]
\ExecuteMetaData[\detdir det1.tex]{th:DetsExist}

\th[th:DeterminantOfAMatrixEqualsDeterminantOfTranspose]
\ExecuteMetaData[\detdir det1.tex]{th:DeterminantOfAMatrixEqualsDeterminantOfTranspose}
\end{frame}
\begin{frame}{Column properties from row properties}
The fact that the determinant of the transpose equals the determinant
of the matrix allows us to transfer results: statements about rows 
become statements about columns.

For instance, the first condition in the definition of determinant is that a
determinant is unchanged by a row combination:~where 
$A$ is square and  
$A\!\!\raisebox{-.5ex}{\smash{\grstep{k\rho_i+\rho_j}}}\!\! \hat{A}$
(with~$i\neq j$)
then $\det(A)=\det{\hat{A}}$.
To see that the same holds for columns, observe that 
we can do a column combination
by transposing, doing a row combination,
and then transposing back.
These three don't change the determinant.

\pause
\co[cor:ColSwapChgSign]
\ExecuteMetaData[\detdir det1.tex]{co:ColSwapChgSign}
\pause
\pf
\ExecuteMetaData[\detdir det1.tex]{pf:ColSwapChgSign}
\qed
\end{frame}




% ..... Four.I.4 .....
\section{Determinants Exist (optional)}
%..........
\begin{frame}{Inversion}
\df[df:Inversion]
\ExecuteMetaData[\detdir det1.tex]{df:Inversion}
\pause
\ex The permutation $\phi=\sequence{3,2,1}$ has three inversions:
$3$ is before $2$, $3$ is before $1$, and $2$ is before $1$.
\begin{center} \small
  \begin{tabular}{r|cc}
    \multicolumn{1}{r|}{\makebox[0em][r]{before}} 
        &$1$  &$2$  \\ \cline{1-2}
    $2$ &\multicolumn{1}{c|}{$\times$} \\ \cline{3-3}
    $3$ &$\times$ &\multicolumn{1}{c|}{$\times$} \\ \cline{2-3}
  \end{tabular}
\end{center}
\end{frame}
\begin{frame}
\ex
This matrix
\begin{equation*}
  P_{\phi}=
  \begin{mat}
    0 &1 &0 &0 \\
    1 &0 &0 &0 \\
    0 &0 &0 &1 \\
    0 &0 &1 &0
  \end{mat}
  =
  \begin{mat}
    \iota_2 \\
    \iota_1 \\
    \iota_4 \\
    \iota_3 
  \end{mat}
\end{equation*}
associated with the permutation $\phi=\sequence{2,1,4,3}$
has two inversions:~row~$2$ 
comes before row~$1$, and row~$4$ is before row~$3$.
\begin{center} \small
  \begin{tabular}{r|ccc}
    \multicolumn{1}{r|}{\makebox[0em][r]{before}} 
        &$1$  &$2$ &$3$ \\ \cline{1-2}
    $2$ &\multicolumn{1}{c|}{$\times$} \\ \cline{3-3}
    $3$ & &\multicolumn{1}{c|}{\ }  \\ \cline{4-4}
    $4$ & & &\multicolumn{1}{c|}{$\times$} \\ \cline{2-4}
  \end{tabular}
\end{center}
\end{frame}




%..........
\begin{frame}
\lm[le:SwapsChangeSgn]
\ExecuteMetaData[\detdir det1.tex]{lm:SwapsChangeSgn}

\pause
\iftoggle{showallproofs}{\pf
  \ExecuteMetaData[\detdir det1.tex]{pf:SwapsChangeSgn0}
}{%

  \medskip
  The book contains the proof; 
  we will illustrate.
  There are two cases, that the rows are adjacent and that they aren't.
  \ex Swapping adjacent rows
  \begin{equation*}
    \begin{mat}
      0 &0 &0 &1 \\
      1 &0 &0 &0 \\
      0 &0 &1 &0 \\
      0 &1 &0 &0 
    \end{mat}
    \grstep{\rho_2\leftrightarrow\rho_3}
    \begin{mat}
      0 &0 &0 &1 \\
      0 &0 &1 &0 \\
      1 &0 &0 &0 \\
      0 &1 &0 &0 
    \end{mat}
  \end{equation*}
  changes the number of inversions by one.
  \begin{center} \small
    \hspace*{1.5em}  % nudge it to the right
    \begin{tabular}{r|ccc}
      \multicolumn{1}{r|}{\makebox[0em][r]{before}} 
          &$1$  &$2$ &$3$ \\ \cline{1-2}
      $2$ &\multicolumn{1}{c|}{\ } \\ \cline{3-3}
      $3$ & &\multicolumn{1}{c|}{$\times$}  \\ \cline{4-4}
      $4$ &$\times$ &$\times$ &\multicolumn{1}{c|}{$\times$} \\ \cline{2-4}
    \end{tabular}
    \hspace*{4em}
    \begin{tabular}{r|ccc}
      \multicolumn{1}{r|}{\makebox[0em][r]{before}} 
          &$1$  &$2$ &$3$ \\ \cline{1-2}
      $2$ &\multicolumn{1}{c|}{} \\ \cline{3-3}
      $3$ &$\times$ &\multicolumn{1}{c|}{$\times$}  \\ \cline{4-4}
      $4$ &$\times$ &$\times$ &\multicolumn{1}{c|}{$\times$} \\ \cline{2-4}
    \end{tabular}
  \end{center}
  Except for the $\iota_1$ and $\iota_3$ involved in the swap, 
  all of the inter-row relationships are unchanged.
}
\end{frame}
\iftoggle{showallproofs}{
  \begin{frame}
  \ExecuteMetaData[\detdir det1.tex]{pf:SwapsChangeSgn1}
  \end{frame}
  \begin{frame}
  \noindent\ExecuteMetaData[\detdir det1.tex]{pf:SwapsChangeSgn2}
  \qed
  \end{frame}
}{%
  \begin{frame}
  \ex To swap non-adjacent rows
  \begin{equation*}
    \begin{mat}
      0 &0 &0 &1 \\
      1 &0 &0 &0 \\
      0 &0 &1 &0 \\
      0 &1 &0 &0 
    \end{mat}
    \grstep{\rho_1\leftrightarrow\rho_4}
    \begin{mat}
      0 &1 &0 &0 \\
      1 &0 &0 &0 \\
      0 &0 &1 &0 \\
      0 &0 &0 &1 
    \end{mat}
  \end{equation*}
  do a sequence of swaps of adjacent rows. 
  \begin{align*}
    \begin{mat}
      0 &0 &0 &1 \\
      1 &0 &0 &0 \\
      0 &0 &1 &0 \\
      0 &1 &0 &0 
    \end{mat} 
    &\grstep{\rho_1\leftrightarrow\rho_2}
    \grstep{\rho_2\leftrightarrow\rho_3}
    \grstep{\rho_3\leftrightarrow\rho_4}
    \begin{mat}
      1 &0 &0 &0 \\
      0 &0 &1 &0 \\
      0 &1 &0 &0 \\
      0 &0 &0 &1 
    \end{mat}                                        \\
    &\grstep{\rho_3\leftrightarrow\rho_2}
    \grstep{\rho_2\leftrightarrow\rho_1}
    \begin{mat}
      0 &1 &0 &0 \\
      1 &0 &0 &0 \\
      0 &0 &1 &0 \\
      0 &0 &0 &1 
    \end{mat}
  \end{align*}
  That's an odd number of swaps, five of them,
  each of which flips the parity of the number of
  inversions. 
  Altogether the parity changes, from even to odd. 
  \end{frame}
}



%..........
\begin{frame}{Signum}
\df[df:Signum]
\ExecuteMetaData[\detdir det1.tex]{df:Signum}

\pause
\ex The permutation
$\phi=\sequence{3,2,1}$
associated with this matrix 
\begin{equation*}
  P_{\phi}=
  \begin{mat}
    0 &0 &1 \\
    0 &1 &0 \\
    1 &0 &0
  \end{mat}
\end{equation*}
has an odd number of inversions, three.
\begin{center} \small
  \begin{tabular}{r|cc}
    \multicolumn{1}{r|}{\makebox[0em][r]{before}} 
        &$1$  &$2$  \\ \cline{1-2}
    $2$ &\multicolumn{1}{c|}{$\times$} \\ \cline{3-3}
    $3$ &$\times$ &\multicolumn{1}{c|}{$\times$} \\ \cline{2-3}
  \end{tabular}
\end{center}
The signum is $\sgn(\phi)=-1$.

\pause
\ex
The permutation $\phi=\sequence{3,2,4,1}$
has four inversions: $3$ is before $2$ and~$1$,
$2$ is before~$1$, and $4$ is before~$1$.
So $\sgn(\phi)=+1$.
\end{frame}




%..........
\begin{frame}
\co[cor:ParityInversEqParitySwaps]
\ExecuteMetaData[\detdir det1.tex]{co:ParityInversEqParitySwaps}

\pause
\pf
\ExecuteMetaData[\detdir det1.tex]{pf:ParityInversEqParitySwaps}
\qed
\end{frame}




%..........
\begin{frame}{Plan accomplished}
We are in the process of showing that
a function exists that satisfies the four conditions in the definition
of determinant.
We must show that for each input square matrix there is a well-defined 
output value~\Dash Gauss's Method can be done in more than one way so 
it isn't obvious that by keeping track of signs and multiplying down the
diagonal we always get the same output.
Consequently we have turned to getting an alternate formula 
that obviously gives only one output.

\pause
\ExecuteMetaData[\detdir det1.tex]{DefiningDFunction}
\end{frame}
\begin{frame}
\lm[lm:DeterminantsExist]
\ExecuteMetaData[\detdir det1.tex]{lm:DeterminantsExist}

\iftoggle{showallproofs}{
  \pf
  \ExecuteMetaData[\detdir det1.tex]{pf:DeterminantsExist0}

  \pause
  \ExecuteMetaData[\detdir det1.tex]{pf:DeterminantsExist1}
  % \end{frame}
  % \begin{frame}

  \pause
  \ExecuteMetaData[\detdir det1.tex]{pf:DeterminantsExist2}
}{%

  \pause  
  \smallskip
  The book has the proof, which verifies that the definition of 
  determinant's four conditions
  are satisfied by the function~$d$.
  We shall here instead go through an illustrative example.
}
\end{frame}
\iftoggle{showallproofs}{
  \begin{frame}
  \ExecuteMetaData[\detdir det1.tex]{pf:DeterminantsExist3}
  \end{frame}
  \begin{frame}
  \noindent\ExecuteMetaData[\detdir det1.tex]{pf:DeterminantsExist4}
  \end{frame}
  \begin{frame}
  \ExecuteMetaData[\detdir det1.tex]{pf:DeterminantsExist5}
  \end{frame}
  \begin{frame}
  \ExecuteMetaData[\detdir det1.tex]{pf:DeterminantsExist6}
  \end{frame}
  \begin{frame}
  \ExecuteMetaData[\detdir det1.tex]{pf:DeterminantsExist7}% 
  \qed
  \end{frame}
}{}

\iftoggle{showallproofs}{}{%
  \begin{frame}
  \ex
  This is the $d$~expansion formula for a $\nbyn{3}$~matrix.
  (To save space on the slides, 
  in this example we write $s(\cdots\hspace*{-1\mu})$ 
  in place of 
  $\sgn(\cdots\hspace*{-1\mu})$.) 
  \begin{align*}
    d(\begin{mat}
      a &b &c \\
      e &f &g \\
      h &i &j
    \end{mat})
    &=afj\cdot s(\begin{mat}
      1 &0 &0 \\
      0 &1 &0 \\
      0 &0 &1
    \end{mat})
    +agi\cdot s(\begin{mat}
      1 &0 &0 \\
      0 &0 &1 \\
      0 &1 &0
    \end{mat})            \\
    &\quad+bej\cdot s(\begin{mat}
      0 &1 &0 \\
      1 &0 &0 \\
      0 &0 &1
    \end{mat})
    +cei\cdot s(\begin{mat}
      0 &1 &0 \\
      0 &0 &1 \\
      1 &0 &0
    \end{mat})         \\
    &\quad+cfh\cdot s(\begin{mat}
      0 &0 &1 \\
      1 &0 &0 \\
      0 &1 &0
    \end{mat})
    +ceg\cdot s(\begin{mat}
      0 &0 &1 \\
      0 &1 &0 \\
      1 &0 &0
    \end{mat})
    \tag{$*$}
  \end{align*}
  We illustrate the proof's verification of the four conditions 
  using this generic matrix.
  \begin{equation*}
    M=\begin{mat}
      1 &2 &3 \\
      4 &5 &6 \\
      7 &8 &9
    \end{mat}
  \end{equation*}
  \end{frame}
  \begin{frame}
  Condition~(4) is easy.
  In terms of equation~($*$), in the identity matrix
  all entries are~$0$
  other than $a$, $f$, and $j$, which are~$1$.
  So equation~($*$) has only one non-zero term, and the sum is~$1$.

  \pause
  For condition~(3) consider the effect of $k\rho_2$ on the generic example
  matrix.
  Here is the expansion, following~($*$).
  \begin{align*}
    d(\begin{mat}
      1  &2  &3 \\
      4k &5k &6k \\
      7  &8  &9
    \end{mat})                                
    &=1(5k)9\cdot
    s(\begin{mat}
      1 &0 &0 \\
      0 &1 &0 \\ 
      0 &0 &1
    \end{mat})
    +1(6k)8\cdot
    s(\begin{mat}
      1 &0 &0 \\
      0 &0 &1 \\
      0 &1 &0
    \end{mat})            \\
    &\quad+2(4k)9\cdot 
    s(\begin{mat}
      0 &1 &0 \\
      1 &0 &0 \\
      0 &0 &1
    \end{mat})
     +2(6k)7\cdot 
    s(\begin{mat}
      0 &1 &0 \\
      0 &0 &1 \\
      1 &0 &0
    \end{mat})         \\
    &\quad+3(4k)8\cdot
    s(\begin{mat}
      0 &0 &1 \\
      1 &0 &0 \\
      0 &1 &0
    \end{mat})
    +3(5k)7\cdot
    s(\begin{mat}
      0 &0 &1 \\
      0 &1 &0 \\
      1 &0 &0
    \end{mat})
  \end{align*}
  The $k$ factors out and the result is $k$ times the $d$~value of the
  original generic matrix.
  \end{frame}
  \begin{frame}  \vspace*{-1ex}
  Next, condition~(2).  Contrast the $d$~expansion of these two.
  \begin{equation*}
    \begin{mat}
      1 &2 &3 \\ 
      4 &5 &6 \\
      7 &8 &9 
    \end{mat}
    \grstep{\rho_2\leftrightarrow\rho_3}    
       \begin{mat}
      1 &2 &3 \\
      7 &8 &9 \\
      4 &5 &6
    \end{mat}                           
  \end{equation*}
  The first expansion is
  \begin{align*}
    &1(5)9\cdot s(\begin{mat}
      1 &0 &0 \\
      0 &1 &0 \\
      0 &0 &1
    \end{mat})
    +1(6)8\cdot s(\begin{mat}
      1 &0 &0 \\
      0 &0 &1 \\
      0 &1 &0
    \end{mat})           
    +2(4)9\cdot s(\begin{mat}
      0 &1 &0 \\
      1 &0 &0 \\
      0 &0 &1
    \end{mat})                          \\
    &\quad+2(6)7\cdot s(\begin{mat}
      0 &1 &0 \\
      0 &0 &1 \\
      1 &0 &0
    \end{mat})         
    +3(4)8\cdot s(\begin{mat}
      0 &0 &1 \\
      1 &0 &0 \\
      0 &1 &0
    \end{mat})
    +3(5)7\cdot s(\begin{mat}
      0 &0 &1 \\
      0 &1 &0 \\
      1 &0 &0
    \end{mat})           
  \end{align*}
  while the second is this.
  \begin{align*}
    &1(8)6\cdot s(\begin{mat}
      1 &0 &0 \\
      0 &1 &0 \\
      0 &0 &1
    \end{mat})
    +1(9)5\cdot s(\begin{mat}
      1 &0 &0 \\
      0 &0 &1 \\
      0 &1 &0
    \end{mat})           
    +2(7)6\cdot s(\begin{mat}
      0 &1 &0 \\
      1 &0 &0 \\
      0 &0 &1
    \end{mat})                          \\
    &\quad+2(9)4\cdot s(\begin{mat}
      0 &1 &0 \\
      0 &0 &1 \\
      1 &0 &0
    \end{mat})         
    +3(7)5\cdot s(\begin{mat}
      0 &0 &1 \\
      1 &0 &0 \\
      0 &1 &0
    \end{mat})
    +3(8)4\cdot s(\begin{mat}
      0 &0 &1 \\
      0 &1 &0 \\
      1 &0 &0
    \end{mat})  
  \end{align*}
  \end{frame}
  \begin{frame} 
  The two match term-by-term.
  For instance, both have terms involving $1\cdot 5\cdot 9$ and the 
  associated matrices differ by a row swap.
  The signum function causes them to differ in
  sign.  
  The expansions are negatives of each other.

  \pause
  Last, condition~(1).  
  By equation~($*$) the second matrix here
  \begin{equation*}
    \begin{mat}
      1    &2    &3 \\
      4    &5    &6 \\
      7    &8    &9 
    \end{mat}
    \grstep{10\rho_1+\rho_2}
    \begin{mat}
      1    &2    &3 \\
      10+4 &20+5 &30+6 \\
      7    &8    &9 
    \end{mat}
  \end{equation*}
  has this $d$ expansion.
  \begin{align*}
    &1(20+5)9\cdot s(\begin{mat}
      1 &0 &0 \\
      0 &1 &0 \\
      0 &0 &1
    \end{mat})
    +1(30+6)8\cdot s(\begin{mat}
      1 &0 &0 \\
      0 &0 &1 \\
      0 &1 &0
    \end{mat})                           \\           
    &\quad+2(10+4)9\cdot s(\begin{mat}
      0 &1 &0 \\
      1 &0 &0 \\
      0 &0 &1
    \end{mat})                          
    +2(30+6)7\cdot s(\begin{mat}
      0 &1 &0 \\
      0 &0 &1 \\
      1 &0 &0
    \end{mat})                          \\
    &\quad+3(10+4)8\cdot s(\begin{mat}
      0 &0 &1 \\
      1 &0 &0 \\
      0 &1 &0
    \end{mat})
    +3(20+5)7\cdot s(\begin{mat}
      0 &0 &1 \\
      0 &1 &0 \\
      1 &0 &0
    \end{mat})           
  \end{align*}
  \end{frame}
  \begin{frame}
  \noindent   Break along the plus signs to get two sums.
  One is the $d$~expansion of the original generic matrix.
  \begin{align*}
    &1(5)9\cdot s(\begin{mat}
      1 &0 &0 \\
      0 &1 &0 \\
      0 &0 &1
    \end{mat})
    +1(6)8\cdot s(\begin{mat}
      1 &0 &0 \\
      0 &0 &1 \\
      0 &1 &0
    \end{mat})                             
    +2(4)9\cdot s(\begin{mat}
      0 &1 &0 \\
      1 &0 &0 \\
      0 &0 &1
    \end{mat})                          \\
    &\quad+2(6)7\cdot s(\begin{mat}
      0 &1 &0 \\
      0 &0 &1 \\
      1 &0 &0
    \end{mat})                          
    +3(4)8\cdot s(\begin{mat}
      0 &0 &1 \\
      1 &0 &0 \\
      0 &1 &0
    \end{mat})
    +3(5)7\cdot s(\begin{mat}
      0 &0 &1 \\
      0 &1 &0 \\
      1 &0 &0
    \end{mat})           
    \\          
    &=d(\begin{mat}
      1    &2    &3 \\
      4    &5    &6 \\
      7    &8    &9 
    \end{mat})
  \end{align*}
  The other is this.
  \begin{align*}
    &1(20)9\cdot s(\begin{mat}
      1 &0 &0 \\
      0 &1 &0 \\
      0 &0 &1
    \end{mat})
    +1(30)8\cdot s(\begin{mat}
      1 &0 &0 \\
      0 &0 &1 \\
      0 &1 &0
    \end{mat})                                      
    +2(10)9\cdot s(\begin{mat}
      0 &1 &0 \\
      1 &0 &0 \\
      0 &0 &1
    \end{mat})                          \\                          
    &+2(30)7\cdot s(\begin{mat}
      0 &1 &0 \\
      0 &0 &1 \\
      1 &0 &0
    \end{mat})                         
    +3(10)8\cdot s(\begin{mat}
      0 &0 &1 \\
      1 &0 &0 \\
      0 &1 &0
    \end{mat})
    +3(20)7\cdot s(\begin{mat}
      0 &0 &1 \\
      0 &1 &0 \\
      1 &0 &0
    \end{mat})                    
    % \\          
    % &=10\cdot d(\begin{mat}
    %   1    &2    &3 \\
    %   1    &2    &3 \\
    %   7    &8    &9 
    % \end{mat})
  \end{align*}
  \end{frame}
  \begin{frame}
  \noindent Factor out~$10$. 
  What's left, by equation~($*$), is this $d$ expansion.
  \begin{align*}
    &1(2)9\cdot s(\begin{mat}
      1 &0 &0 \\
      0 &1 &0 \\
      0 &0 &1
    \end{mat})
    +1(3)8\cdot s(\begin{mat}
      1 &0 &0 \\
      0 &0 &1 \\
      0 &1 &0
    \end{mat})                                      
    +2(1)9\cdot s(\begin{mat}
      0 &1 &0 \\
      1 &0 &0 \\
      0 &0 &1
    \end{mat})                          \\                          
    &+2(3)7\cdot s(\begin{mat}
      0 &1 &0 \\
      0 &0 &1 \\
      1 &0 &0
    \end{mat})                         
    +3(1)8\cdot s(\begin{mat}
      0 &0 &1 \\
      1 &0 &0 \\
      0 &1 &0
    \end{mat})
    +3(2)7\cdot s(\begin{mat}
      0 &0 &1 \\
      0 &1 &0 \\
      1 &0 &0
    \end{mat})                    \\          
    &=d(\begin{mat}
      1    &2    &3 \\
      1    &2    &3 \\
      7    &8    &9 
    \end{mat})
  \end{align*}
  This matrix has two identical rows so its $d$~expansion gives~$0$
  (condition~(2) shows this, since swapping the two rows changes the sign but
  doesn't change the matrix).
  Hence the operation $10\rho_1+\rho_2$ does not change the value of
  the function~$d$.
  \end{frame}
} % end of iftoggle

%..........
\begin{frame}{The determinant of the transpose}
\th[th:DetMatrixEqualsDetTrans]
\ExecuteMetaData[\detdir det1.tex]{th:DetMatrixEqualsDetTrans}

\pf
\ExecuteMetaData[\detdir det1.tex]{pf:DetMatrixEqualsDetTrans0}
\end{frame}
\begin{frame}
\ExecuteMetaData[\detdir det1.tex]{pf:DetMatrixEqualsDetTrans1} 
\end{frame}
\begin{frame}
\ExecuteMetaData[\detdir det1.tex]{pf:DetMatrixEqualsDetTrans2}%
\qed

\pause
\ex
We know the formula for $\nbyn{2}$~matrices.
\begin{equation*}
  |A|=
  \begin{vmat}
    a  &b  \\
    c  &d
  \end{vmat}
  =ad-bc
  \qquad
  |\trans{A}|=
  \begin{vmat}
    a  &c  \\
    b  &d
  \end{vmat}
  =ad-cb
\end{equation*}
\end{frame}



% % ..... Four.I.2 .....
% \section{}
% %..........
% \begin{frame}
% \end{frame}




%...........................
% \begin{frame}
% \ExecuteMetaData[../gr3.tex]{GaussJordanReduction}
% \df[def:RedEchForm]
% 
% \end{frame}
\end{document}
