% \documentclass[9pt,t]{beamer}
\usefonttheme{professionalfonts}
\usefonttheme{serif}
\PassOptionsToPackage{pdfpagemode=FullScreen}{hyperref}
\PassOptionsToPackage{usenames,dvipsnames}{color}
% \DeclareGraphicsRule{*}{mps}{*}{}
\usepackage{linalgjh}
\usepackage{present}
\usepackage{directories}  % Defines \vsdir, \vsmpdir
\usepackage{xr}\externaldocument{\vsdir vs2} % read refs from .aux file
\usepackage{xr}\externaldocument{\grdir gr3} % read refs from .aux file
\usepackage{catchfilebetweentags}
\usepackage{etoolbox} % from http://tex.stackexchange.com/questions/40699/input-only-part-of-a-file-using-catchfilebetweentags-package
\makeatletter
\patchcmd{\CatchFBT@Fin@l}{\endlinechar\m@ne}{}
  {}{\typeout{Unsuccessful patch!}}
\makeatother

\mode<presentation>
{
  \usetheme{boxes}
  \setbeamercovered{invisible}
  \setbeamertemplate{navigation symbols}{} 
}
\addheadbox{filler}{\ }  % create extra space at top of slide 
\hypersetup{colorlinks=true,linkcolor=blue} 

\title[Linear Independence] % (optional, use only with long paper titles)
{Two.II Linear Independence}

\author{\textit{Linear Algebra}, edition four \\ {\small Jim Hef{}feron}}
\institute{
  \texttt{http://joshua.smcvt.edu/linearalgebra}
}
\date{}


\subject{Linear Independence}
% This is only inserted into the PDF information catalog. Can be left
% out. 

\begin{document}
\begin{frame}
  \titlepage
\end{frame}

% =============================================
% \begin{frame}{Reduced Echelon Form} 
% \end{frame}



% ..... Two.I.1 .....
\section{Definition and examples}
%..........
\begin{frame}{Linear independence}
\df[def:LinInd]
In any vector space, a set of vectors is
\definend{linearly independent}
if none of its elements is a linear combination of the 
others from the set.
Otherwise the set is \definend{linearly dependent}.
% Makes footnote: \ExecuteMetaData[\vsdir vs2.tex]{df:LinInd}

\pause\medskip
\ExecuteMetaData[\vsdir vs2.tex]{LinInd}
\end{frame}



%..........
\begin{frame}
\lm[le:LDIffANonTrivLinRel]
\ExecuteMetaData[\vsdir vs2.tex]{lm:LDIffANonTrivLinRel}

\pause
\pf
\ExecuteMetaData[\vsdir vs2.tex]{pf:LDIffANonTrivLinRel}
\qed

\bigskip
So to decide if a list of vectors 
$\vec{s}_0, \ldots,\vec{s}_n$
is linearly independent,
set up the equation
$\vec{0}=c_0\vec{s}_0+\cdots+c_n\vec{s}_n$,
and calculate whether it has any solutions, other than
the trivial one where all coefficents are zero.
\end{frame}



%..........
\begin{frame}
\ex 
This set of vectors in the plane $\Re^2$ is linearly independent.
\begin{equation*}
  \set{\colvec{1 \\ 0},
       \colvec{0 \\ 1}}
\end{equation*}
The only solution to this equation
\begin{equation*}
  c_1\colvec{1 \\ 0}+c_2\colvec{0 \\ 1}=\colvec{0 \\ 0}
\end{equation*}
is trivial $c_1=0$, $c_2=0$.
\pause
\ex In the vector space of cubic polynomials 
$\polyspace_3=\set{a_0+a_1x+a_2x^2+a_3x^3\suchthat a_i\in\Re}$ the set
$\set{1-x,1+x}$ is linearly independent.
Setting up the equation
$c_0(1-x)+c_1(1+x)=0$ and considering the constant term and linear term,
leads to this system
\begin{equation*}
  \begin{linsys}{2}
    c_0  &+ &c_1 &=  &0 \\
    -c_0 &+ &c_1 &=  &0 
  \end{linsys}
\end{equation*}
which has only the trivial solution.
\end{frame}



%..........
\begin{frame}
\ex
The nonzero rows of this matrix form a linearly independent set.
\begin{equation*}
  \begin{mat}[r]
    2 &0  &1   &-1  \\
    0 &1  &-3  &1/2  \\
    0 &0  &0   &5    \\
    0 &0  &0   &0
  \end{mat}
\end{equation*}
We showed in Lemma~One.III.\ref{le:EchFormNoLinCombo} that in any
echelon form matrix the nonzero
rows form a linearly independent set. 

\pause
\ex
This subset of $\Re^3$ is linearly dependent.
\begin{equation*}
  \set{\colvec[r]{1  \\ 1 \\ 3}, 
       \colvec[r]{-1 \\ 1 \\ 0},
       \colvec[r]{1  \\ 3 \\ 6}}
\end{equation*}
One way to see that is to spot that the third vector is twice the first plus 
the second.
Another way is to solve the linear system
\begin{equation*}
  \begin{linsys}{3}
    c_1  &-  &c_2  &+  &c_3    &=  &0  \\
    c_1  &+  &c_2  &+  &3c_3   &=  &0  \\
    3c_1 &   &     &+  &6c_3   &=  &0
  \end{linsys}
\end{equation*}
and note that it has more than just the solution $c_1=c_2=c_3=0$.
\end{frame}






%..........
\begin{frame}
\lm[lm:AddVecSpanNoGrowIffVecInSpan]
\ExecuteMetaData[\vsdir vs2.tex]{lm:AddVecSpanNoGrowIffVecInSpan}

\pause
\iftoggle{showallproofs}{
  \pf
  \ExecuteMetaData[\vsdir vs2.tex]{pf:AddVecSpanNoGrowIffVecInSpan0}
  
  \pause
  \ExecuteMetaData[\vsdir vs2.tex]{pf:AddVecSpanNoGrowIffVecInSpan1}
  \qed
}{
  \bigskip
  \ex The book has the proof; here is an illustration.
  The span of this set is the $xy$-plane.
  \begin{equation*}
    P=\set{\colvec{2 \\ 0 \\ 0},
           \colvec{0 \\ 2 \\ 0}}\subset\Re^3
  \end{equation*}
  If we expand the set by adding a vector $\set{\vec{p}_1,\vec{p}_2,\vec{q}}$ 
  then there are two possibilities. 
  \begin{equation*}
    P_0=\set{\colvec{2 \\ 0 \\ 0},
                 \colvec{0 \\ 2 \\ 0},
                 \colvec{3 \\ 2 \\ 0}}
    \qquad
    P_1=\set{\colvec{2 \\ 0 \\ 0},
                 \colvec{0 \\ 2 \\ 0},
                 \colvec{0 \\ 0 \\ -1}}
  \end{equation*}
  If the new vector is already in the starting span $\vec{q}\in\spanof{P}$
  then the span is unchanged $\spanof{P_0}=\spanof{P}$.
  But if the new vector is outside the starting span $\vec{q}\notin\spanof{P}$
  then the span grows $\spanof{P_1}\supsetneq\spanof{P}$.
}
\end{frame}

\begin{frame}
\co[cor:OmitVecNotShrinkSpanIffVecIsDep]
\ExecuteMetaData[\vsdir vs2.tex]{cor:OmitVecNotShrinkSpanIffVecIsDep}

\ex
These two subsets of $\R^3$ have the same span
\begin{equation*}
  \set{\colvec{1 \\ 2 \\ 3}, \colvec{4 \\ 5 \\ 6}, \colvec{7 \\ 8 \\ 9}}
  \qquad
  \set{\colvec{1 \\ 2 \\ 3}, \colvec{4 \\ 5 \\ 6}}
\end{equation*}
because in the first set $\vec{v}_3=2\vec{v}_2-\vec{v}_1$.

\pause
\co[cor:LiSetMinSpanSet]
\ExecuteMetaData[\vsdir vs2.tex]{cor:LiSetMinSpanSet}
\iftoggle{showallproofs}{
  \pf
  \ExecuteMetaData[\vsdir vs2.tex]{pf:LiSetMinSpanSet}\qed
}{

  \ex
  This is a linearly independent subset of $\polyspace_3$
  \begin{equation*}
    S=\set{1+x,1-x,x^2}
  \end{equation*}
  Removal of any element, such as if we remove $1-x$ to get 
  $\hat{S}=\set{1+x,x^2}$, will make the span smaller:
  $\spanof{\hat{S}}\subsetneq\spanof{S}$.
}
\end{frame}


%..........
\begin{frame}
\lm[lm:AddVecLiSetIsLiIffVecNotInSpan]
\ExecuteMetaData[\vsdir vs2.tex]{lm:AddVecLiSetIsLiIffVecNotInSpan}

\iftoggle{showallproofs}{
\pause
\pf
\ExecuteMetaData[\vsdir vs2.tex]{pf:AddVecLiSetIsLiIffVecNotInSpan0}

\ExecuteMetaData[\vsdir vs2.tex]{pf:AddVecLiSetIsLiIffVecNotInSpan1}
\qed}{
\ex The book has the proof; here is an illustration.
Consider this linearly independent subset of $\polyspace_2$.
\begin{equation*}
  S=\set{1-x,1+x}
\end{equation*}
Its span $\spanof{S}$ is the set of
linear polynomials $\set{a+bx\suchthat a,b\in\Re}$.
(To check: consider $a+bx=r_1(1-x)+r_2(1+x)$, which gives a linear system
with equations
$r_1+r_2=a$ and 
$-r_1+r_2=b$, having
the solution 
$r_2=(1/2)a+(1/2)b$ and 
$r_1=(1/2)a-(1/2)b$.)  

Here are two supersets.
\begin{equation*}
  S_1=S\cup\set{2+2x}
  \qquad
  S_2=S\union\set{2+x^2}
\end{equation*}

\pause
On the left, adding a linear polynomial just adds ``repeat information''
so $\spanof{S_1}=\spanof{S}$ and $S_1$ is linearly dependent.

The right, with ``new information,'' enlarges the span 
$\spanof{S_2}=\polyspace_2\supsetneq \spanof{S}$
and the new set $S_2$ is also linearly independent.
(To check this, use  
$a+bx+cx^2=r_1(1-x)+r_2(1+x)+r_3(2+x^2)$ to get a linear system
with solution $r_3=c$, $r_2=(1/2)a+(1/2)b$ and $r_1=(1/2)a-(1/2)b-c$.)
}
\end{frame}




%..........
\begin{frame}
\co[th:AlwaysAnLDSubset]
\ExecuteMetaData[\vsdir vs2.tex]{th:AlwaysAnLDSubset}

\iftoggle{showallproofs}{
  \pause
  \pf
  \ExecuteMetaData[\vsdir vs2.tex]{pf:AlwaysAnLDSubset0}

  \ExecuteMetaData[\vsdir vs2.tex]{pf:AlwaysAnLDSubset1}

  \ExecuteMetaData[\vsdir vs2.tex]{pf:AlwaysAnLDSubset2}
  \qed
}{
  \medskip\par
  The book has a proof.  
  Instead, consider the example on the next slide.
}
\end{frame}
\begin{frame}
\ex
Consider this subset of $\Re^2$. 
\begin{equation*}
  S=\set{\vec{s}_1,\vec{s}_2,\vec{s}_3,\vec{s}_4,\vec{s}_5}
   =\set{\colvec{2 \\ 2},
       \colvec{3 \\ 3},
       \colvec{1 \\ 4},
       \colvec{0 \\ -1},
       \colvec{1 \\ -1} }
\end{equation*}
The linear relationship
\begin{equation*}
       r_1\colvec{2 \\ 2}
       +r_2\colvec{3 \\ 3}
       +r_3\colvec{1 \\ 4}
       +r_4\colvec{0 \\ -1}
       +r_5\colvec{1 \\ -1}
       =\colvec{0 \\ 0}
  % \tag{*}
\end{equation*}
gives a system of equations.
\begin{equation*}
   \begin{linsys}{5}
      2r_1 &+ &3r_2 &+ &r_3  &  &    &+ &r_5 &=  &0 \\
      2r_1 &+ &3r_2 &+ &4r_3 &- &r_4 &- &r_5 &=  &0 
    \end{linsys}       %                                     \\                 
    \grstep{-\rho_1+\rho_2}
    \begin{linsys}{5}
      2r_1 &+ &3r_2 &+ &r_3   &  &    &+ &r_5  &=  &0 \\
           &  &     &+ &3r_3 &- &r_4  &- &2r_5 &=  &0 
    \end{linsys}
\end{equation*}
\end{frame}
\begin{frame}
\noindent
Parametrize by expressing the leading variables
$r_1$ and $r_3$ in terms of the free variables.
\begin{equation*}
  \set{\colvec{r_1 \\ r_2 \\ r_3 \\ r_4 \\ r_5}
       =\colvec{-3/2 \\ 1 \\ 0 \\ 0 \\ 0}r_2
        +\colvec{-1/6 \\ 0 \\ 1/3 \\ 1 \\ 0}r_4
        +\colvec{-5/6 \\ 0 \\ 2/3 \\ 0 \\ 1}r_5
       \suchthat r_2,r_4,r_5\in\Re}
\end{equation*}
\pause
Set $r_5=1$ and $r_2=r_4=0$ to get
$r_1=-5/6$ and $r_3=2/3$,
\begin{equation*}
       -\frac{5}{6}\cdot\colvec{2 \\ 2}
       +0\cdot\colvec{3 \\ 3}
       +\frac{2}{3}\cdot\colvec{1 \\ 4}
       +0\cdot\colvec{0 \\ -1}
       +1\cdot\colvec{1 \\ -1}
       =\colvec{0 \\ 0}                          
\end{equation*}
showing that  
$\vec{s}_5$ is in the span of the set $\set{\vec{s}_1,\vec{s}_3}$.

Similarly, setting $r_4=1$ and the other parameters to~$0$ shows
$\vec{s}_4$ is in the span of the set $\set{\vec{s}_1,\vec{s}_3}$.
Also, setting $r_2=1$ and the other parameters to~$0$ shows
$\vec{s}_2$ is in the span of the same set.

So without shrinking the span 
we can omit the vectors $\vec{s}_2$, $\vec{s}_4$, $\vec{s}_5$
associated with the free variables.  
The set of vectors associated with the leading variables,
$\set{\vec{s}_1,\vec{s}_3}$, is linearly independent
and so we cannot omit any members without shrinking the span.
% (In ($*$) note that $\vec{s}_2$ is linearly dependent on $\vec{s}_1$
% and $r_2$ did not end as a leading variable.)
\end{frame}



%..........
\begin{frame}
\co[cor:LDMeansLC]
\ExecuteMetaData[\vsdir vs2.tex]{co:LDMeansLC}

\pause
\pf
\ExecuteMetaData[\vsdir vs2.tex]{pf:LDMeansLC}
\qed
\end{frame}




%..................
\begin{frame}{Linear independence and subset}
\lm[le:SubsetPreserveLI]
\ExecuteMetaData[\vsdir vs2.tex]{lm:SubsetPreserveLI}

\pf
\ExecuteMetaData[\vsdir vs2.tex]{pf:SubsetPreserveLI}
\qed

\pause
\medskip
This table summarizes the cases.
\ExecuteMetaData[\vsdir vs2.tex]{IndependenceAndSubsetTable}
\medskip

An example of the lower left is that the set $S$ of all vectors in the
space $\Re^2$ is linearly dependent but the subset $\hat{S}$ consisting of only the 
unit vector on the $x$-axis is independent.
By interchanging $\hat{S}$ with $S$ that's also an example of the upper right.
\end{frame}



%...........................
% \begin{frame}
% \ExecuteMetaData[../gr3.tex]{GaussJordanReduction}
% \df[def:RedEchForm]
% 
% \end{frame}
\end{document}
