\documentclass[11pt]{article}
\usepackage[T1]{fontenc}
\usepackage[tracking]{microtype}

\usepackage[sc,osf]{mathpazo}   % With old-style figures and real smallcaps.
\linespread{1.025}              % Palatino leads a little more leading

% Euler for math and numbers
\usepackage[euler-digits,small]{eulervm}
\AtBeginDocument{\renewcommand{\hbar}{\hslash}}

\usepackage{hyperref}
\hypersetup{
    bookmarks=false,         % show bookmarks bar?
    unicode=true,          % non-Latin characters in Acrobat’s bookmarks
    pdftoolbar=true,        % show Acrobat’s toolbar?
    pdfmenubar=true,        % show Acrobat’s menu?
    pdffitwindow=false,     % window fit to page when opened
    pdfstartview={FitH},    % fits the width of the page to the window
    pdftitle={Slides for Linear Algebra by Hefferon},    % title
    pdfauthor={Jim Hefferon},     % author
    pdfsubject={Slides for Linear Algebra by Hefferon}, % subject of the document
    pdfcreator={Jim Hefferon},   % creator of the document
    pdfproducer={pdflatex}, % producer of the document
    pdfkeywords={Linear Algebra, Hefferon, Free textbook, slides}, % list of keywords
    pdfnewwindow=true,      % links in new PDF window
    colorlinks=true,       % false: boxed links; true: colored links
    linkcolor=blue,          % color of internal links (change box color with linkbordercolor)
    citecolor=blue,        % color of links to bibliography
    filecolor=blue,      % color of file links
    urlcolor=blue           % color of external links
}

\title{Beamer slides \\
       for \\
       \textit{Linear Algebra}}
\author{Jim Hef{}feron}
\date{2020-April-08}

\pagestyle{empty}
\begin{document}
\maketitle\thispagestyle{empty}
If you teach this class using a beamer, a video projector, then
you may find useful the slides that I use for that.

\begin{center}\small
\begin{tabular}{l|l}
  \multicolumn{1}{c}{\textit{File}}
     &\multicolumn{1}{c}{\textit{Description}}  \\
  \hline
  \href{file:one/one_i.pdf}{one\_i.pdf}  &Chapter One, Section I  \\
  \href{file:one/one_ii.pdf}{one\_ii.pdf}  &\quad Section II  \\
  \href{file:one/one_iii.pdf}{one\_iii.pdf}  &\quad Section III  \\
  \hline
  \href{file:two/two_i.pdf}{two\_i.pdf}  &Chapter Two, Section I  \\
  \href{file:two/two_ii.pdf}{two\_ii.pdf}  &\quad Section II  \\
  \href{file:two/two_iii.pdf}{two\_iii.pdf}  &\quad Section III  \\
  \hline
  \href{file:three/three_i.pdf}{three\_i.pdf}  &Chapter Three, Section I  \\
  \href{file:three/three_ii.pdf}{three\_ii.pdf}  &\quad Section II  \\
  \href{file:three/three_ii_a.pdf}{three\_ii\_a.pdf}  &\quad Geometry of linear maps  \\
  \href{file:three/three_iii.pdf}{three\_iii.pdf}  &\quad Section III  \\
  \href{file:three/three_iv.pdf}{three\_iv.pdf}  &\quad Section IV  \\
  \href{file:three/three_v.pdf}{three\_v.pdf}  &\quad Section V  \\
  \href{file:three/three_vi.pdf}{three\_vi.pdf}  &\quad Section VI  \\
  \hline
  \href{file:four/four_i.pdf}{four\_i.pdf}  &Chapter Four, Section I  \\
  \href{file:four/four_ii.pdf}{four\_ii.pdf}  &\quad Section II  \\
  \href{file:four/four_iii.pdf}{four\_iii.pdf}  &\quad Section III  \\
  \hline
  \href{file:five/five_i.pdf}{five\_i.pdf}  &\quad Chapter Five, Section I  \\
  \href{file:five/five_ii.pdf}{five\_ii.pdf}  &\quad Section II  \\
  \href{file:five/five_ii_a.pdf}{five\_ii\_a.pdf}  &\quad Geometry of eigenvectors
\end{tabular}
\end{center}

Two comments.
First, these are compiled from the same source as the book so
definitions, etc., should agree exactly.
The numbers of theorems, etc., should also agree.
However, here I give different examples, so that
students see twice as many.

Also, each slide deck comes in three varieties.
For instance, in class I use \href{file:one/one_i.pdf}{one\_i.pdf}.
That omits some proofs in favor of examples. 
If you want all of
the proofs then use \href{file:one/one_i_allproofs.pdf}{one\_i\_allproofs.pdf}.
For printing or for making available on 
a learning management system, you want to have just full pages, without
the mid-slide pauses. 
For this use 
\href{file:one/one_i_handout.pdf}{one\_i\_handout.pdf}.
\end{document}
