\documentclass[11pt]{article}
\usepackage[margin=1in]{geometry}
\usepackage{../linalgjh}

\setlength{\parindent}{0em}
\pagestyle{empty}
\begin{document}\thispagestyle{empty}
\makebox[\linewidth]{\textbf{Homework, MA~213}\hspace*{4in}\textbf{Due: 2014-Oct-24}}

\vspace*{3ex}
\textit{You may work with others to figure out how to do questions, 
and you are welcome to look for answers in the book, online, by talking
to someone who had the course before, etc.
However, you must write 
the answers on your own.
You must also show your work (you may, of course, 
quote any result from the book).}

\begin{enumerate}
\item
Verify that each map is a homomorphism.
  \begin{enumerate}
  \item $\map{h}{\polyspace_3}{\Re^2}$ given by
    \begin{equation*}
      ax^2+bx+c\mapsto \colvec{a+b \\ a+c}
    \end{equation*}

This verifies that the map preserves linear combinations.
\begin{align*}
  h(\,d_1(a_1x^2+b_1x+c_1)+d_2(a_2x^2+b_2x+c_2)\,)
   &=h((d_1a_1+d_2a_2)x^2+(d_1b_1+d_2b_2)x+(d_1c_1+d_2c_2))  \\
   &=\colvec{(d_1a_1+d_2a_2)+(d_1b_1+d_2b_2) \\ (d_1a_1+d_2a_2)+(d_1c_1+d_2c_2)} \\
   &=\colvec{d_1a_1+d_1b_1 \\  d_1a_1+d_1c_1}+
       \colvec{d_2a_2+d_2b_2 \\  d_2a_2+d_2c_2}  \\
   &=d_1\colvec{a_1+b_1 \\  a_1+c_1}+
       d_2\colvec{a_2+b_2 \\  a_2+c_2}  \\
  &=d_1\cdot h(a_1x^2+b_1x+c_1)+d_2\cdot h(a_2x^2+b_2x+c_2)
\end{align*}

  \item $\map{f}{\Re^2}{\Re^3}$ given by 
    \begin{equation*}
      \colvec{x \\ y}\mapsto\colvec{0 \\ x-y \\ 3y} 
    \end{equation*}

This verifies that the map perserves linear combinations.
\begin{align*}
  f(\,a_1\colvec{x_1 \\ y_1}+a_2\colvec{x_2 \\ y_2}\,)
    &=f(\,\colvec{a_1x_1+a_2x_2 \\ a_1y_1+a_2y_2}\,)  \\
    &=\colvec{0 \\ (a_1x_1+a_2x_2)-(a_1y_1+a_2y_2) \\ 3(a_1y_1+a_2y_2) }  \\
    &=a_1\colvec{0 \\ x_1-y_1 \\ 3y_1}
       +a_2\colvec{0 \\ x_2-y_2 \\ 3y_2}  \\
    &=a_1f(\colvec{x_1 \\ y_1})+a_2f(\colvec{x_2 \\ y_2})
\end{align*}
  \end{enumerate}





\item For each map in the prior question, describe the range space and find
  the rank of the map.

  \begin{enumerate}
  \item The range of $h$ is all of the codomain~$\Re^2$ because given 
    \begin{equation*}
      \colvec{x \\ y}\in\Re^2
    \end{equation*}
    it is the image under~$h$ of the domain vector $0x^2+bx+c$.
    So the rank of~$h$ is~$2$.

  \item
    The range is the $yz$~plane.  Any 
    \begin{equation*}
      \colvec{0 \\ a \\ b}
    \end{equation*}
    is the image under~$f$ of this domain vector.
    \begin{equation*}
      \colvec{a+b/3  \\ b/3}
    \end{equation*}
    So the rank of the map is~$2$.
  \end{enumerate}


\item Verify that this map is an isomorphism: 
  $\map{h}{\Re^4}{\matspace_{\nbyn{2}}}$ given by
  \begin{equation*}
    \colvec{a \\ b \\ c \\ d}
    \mapsto
    \begin{mat}
      c  &a+d \\
      b  &d
    \end{mat}
  \end{equation*}

We first verify that $h$ is one-to-one.
To do this we will show that $h(\vec{v}_1)=h(\vec{v}_2)$ implies that 
$\vec{v}_1=\vec{v}_2$.
So assume that
\begin{equation*}
    h(\vec{v}_1)
    =
    h(\colvec{a_1 \\ b_1 \\ c_1 \\ d_1})
    =
    h(\colvec{a_2 \\ b_2 \\ c_2 \\ d_2})
    =h(\vec{v}_2)
\end{equation*}
which gives
\begin{equation*}
    \begin{mat}
      c_1  &a_1+d_1 \\
      b_1  &d_1
    \end{mat}
    =
    \begin{mat}
      c_2  &a_2+d_2 \\
      b_2  &d_2
    \end{mat}
\end{equation*}
from which we conclude that 
$c_1=c_2$ (by the upper-left entries),
$b_1=b_2$ (by the lower-left entries),
$d_1=d_2$ (by the lower-right entries),
and with this last we get $a_1=a_2$ (by the upper right).
Therefore $\vec{v}_1=\vec{v}_2$.

Next we will show that the map is onto, that every member of the codomain
$\matspace_{\nbyn{2}}$ is the image of some four-tall member of the domain.
So, given
\begin{equation*}
    \vec{w}=
    \begin{mat}
      m  &n \\
      p  &q
    \end{mat}
    \in\matspace_{\nbyn{2}}
\end{equation*}
observe that it is the image of this domain vector.
\begin{equation*}
  \vec{v}=
  \colvec{n-q \\ p \\ m \\ q}
\end{equation*}

To finish we verify that the map preserves linear combinations.
\begin{align*}
    h(r_1\cdot\colvec{a_1 \\ b_1 \\ c_1 \\ d_1}
      +
      r_2\cdot\colvec{a_2 \\ b_2 \\ c_2 \\ d_2})
    &=h(\colvec{r_1a_1+r_2a_2 \\ r_1b_1+r_2b_2 \\ r_1c_1+r_2c_2 \\ r_1d_1+r_2d_2})  \\
    &=
    \begin{mat}
       r_1c_1+r_2c_2 &(r_1a_1+r_2a_2)+(r_1d_1+r_2d_2)  \\
       r_1b_1+r_2b_2 &r_1d_1+r_2d_2
    \end{mat}                               \\
    &=
    r_1\begin{mat}
       c_1 &a_1+d_1  \\
       b_1 &d_1
    \end{mat}                               
    +
    r_2\begin{mat}
       c_2 &a_2+d_2  \\
       b_2 &d_2
    \end{mat}                               \\
   &=r_1\cdot h(\colvec{a_1 \\ b_1 \\ c_1 \\ d_1})
      +
      r_2\cdot h(\colvec{a_2 \\ b_2 \\ c_2 \\ d_2})
\end{align*}


\end{enumerate}
\end{document}
