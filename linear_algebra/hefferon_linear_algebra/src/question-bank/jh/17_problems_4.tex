% \documentclass[11pt,answers]{examjh}
\documentclass[11pt,noanswers]{examjh}
\usepackage{../../linalgjh}
\examhead{MA 213 Hef{}feron, 2017-Fall}{Due: Tue 2017-Nov-14}

\setlength{\parindent}{0em}
\begin{document}
\makebox[\linewidth]{\textbf{Homework 4, MA~213}}

\vspace*{3ex}
\textit{You may work with others to figure out how to do questions, 
and you are welcome to look for answers in the book, online, by talking
to someone who had the course before, etc.
However, you must write 
the answers on your own.
You must also show your work (you may, of course, 
quote any result from the book).}

\begin{questions}
\question Verify that the map $\map{h}{\Re^m}{\Re^n}$ represented by this matrix
  with respect to the standard bases
  \begin{equation*}
    \begin{mat}
      2  &1  &0  \\
      3  &1  &1  \\
      7  &2  &1
    \end{mat}
  \end{equation*}
  is an isomorphism.
  What is the range space?
  The null space?

\question
  Assume that each matrix represents a map $\map{h}{\Re^m}{\Re^n}$
  with respect to the standard bases.
  In each case, 
  (i)~state $m$ and~$n$
  (ii)~find $\rangespace{h}$ and $\rank(h)$
  (iii)~find $\nullspace{h}$ and $\nullity(h)$,
  and (iv)~state whether the map is onto and whether it is one-to-one.
  \begin{enumerate}
  \item
    $
    \begin{mat}
      2  &1  \\ 
      1  &1
    \end{mat}
    $
  \item
    $
    \begin{mat}
      2  &3  &3  \\
      0  &1  &3  \\ 
     -2  &-1 &2 
    \end{mat}
    $
  \item
    $
    \begin{mat}
      1  &1  \\ 
      2  &1  \\
      3  &1
    \end{mat}
    $
  \end{enumerate}

\item For these 
  \begin{equation*}
    A=
    \begin{mat}
      2  &1  \\
      4  &3
    \end{mat}
    \quad
    B=
    \begin{mat}
      2  &2  &2 \\
      0  &1  &0  \\
      1  &1  &1
    \end{mat}
    \quad
    C=
    \begin{mat}
      2  &1  &1 \\
      1  &1  &2
      \end{mat}
    \quad
    \vec{v}=\colvec{1 \\ 1 \\ -1}  
  \end{equation*}
    \begin{enumerate}
      \item Find, or state ``not defined'': $6A$, $6C$. 
      \item Find, or state ``not defined'': $A\vec{v}$, $C\vec{v}$. 
      \item Find, or state ``not defined'': $A+B$, $B+C$. 
    \end{enumerate}

\end{questions}
\end{document}
