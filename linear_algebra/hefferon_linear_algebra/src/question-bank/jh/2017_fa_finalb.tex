% \documentclass[11pt]{examjh}
\documentclass[11pt,answers]{examjh}
\usepackage{../../linalgjh}
\examhead{MA 213 Hef{}feron, 2017-Fall}{Final Exam, MA 213-B}

\begin{document}
\begin{questions}



\question
Perform the operations, or say ``not defined.''
\begin{parts}
\item
$
\begin{mat}
1 &3 &-1 \\
2 &4 &2 
\end{mat}
\begin{mat}
3 &0 \\
2 &0 \\
1 &-1
\end{mat}
$
\begin{solution}[1in]
$
\begin{mat}
8  &1 \\
16 &-2
\end{mat}
$
\end{solution}

\item
$
  \det(
  \begin{mat}
  3 &3  &-1 \\
  0 &0 &2
  \end{mat}
)
$
\begin{solution}[1in]
The determinant of a non-square matrix is not defined. 
\end{solution}
\end{parts}



\question
Prove that this subset of $\R^3$ is a subspace.
\begin{equation*}
  \set{\colvec{x \\ y \\ z}\suchthat x+2y-3z=0}
\end{equation*}
\begin{solution}[2in]
It is clearly nonempty.
We show it is closed under linear combinations.
Let
\begin{equation*}
\vec{v}_1=\colvec{x_1 \\ y_1 \\ z_1}
\quad
\vec{v}_2=\colvec{x_2 \\ y_2 \\ z_2}
\end{equation*}
be members of the set so that $x_1+2y_1-3z_1=0$ and $x_2+2y_2-3z_2=0$.
Then
\begin{equation*}
  r_1\vec{v}_1+r_2\vec{2}
  =\colvec{r_1x_1+r_2x_2 \\ r_1y_1+r_2y_2 \\ r_1z_1+r_2z_2}
\end{equation*}
is also a member of the set because
$(r_1x_1+r_2x_2)+2(r_1y_1+r_2y_2)-3(r_1z_1+r_2z_2)
=r_1(x_1+2y_2-3z_1)+r_2(x_2+2y_2-3z_2)=r1\cdot 0+r2\cdot 0=0$.
\end{solution}




\question
Consider the map $\map{h}{\polyspace_2}{\Re^2}$ defined by this.
\begin{equation*}
  ax^2+bx+c\mapsto \colvec{a+b \\ a+c}
\end{equation*}
\begin{parts}
\item
Prove that it is a homomorphism.
\begin{solution}[1.5in]
We show that it preserves linear combinations.
Suppose that $\vec{v}_1,\vec{v}_2\in\polyspace_2$.
\begin{align*}
  h(r_1\vec{v}_1+r_2\vec{v}_2)
  &=
  h(r_1(a_1x^2+b_1x+c_1)+r_2(a_2x^2+b_2x+c_2))     \\
  &=
  h((r_1a_1+r_2a_2)x^2+(r_1b_1+r_2b_2)x+(r_1c_1+r_2c_2))    \\
  &=
  \colvec{(r_1a_1+r_2a_2)+(r_1b_1+r_2b_2) \\ (r_1a_1+r_2a_2)+(r_1c_1+r_2c_2)} \\
  &=
  r_1\colvec{a_1+b_1 \\ a_1+c_1}
  +
  r_2\colvec{a_2+b_2 \\ a_2+c_2} \\
  &=r_1h(\vec{v}_1)+r_2h(\vec{v}_2)
\end{align*}
\end{solution}

\item
Represent it with respect to these bases.
\begin{equation*}
B=\sequence{2,2x,2x^2}
\quad
D=\sequence{\colvec{1 \\ 0},\colvec{1 \\ 1}}
\end{equation*}
\begin{solution}[2.25in]
The map has this effect on the the elements of~$B$.
\begin{equation*}
  2\mapsto\colvec{0 \\ 2}
  \quad
  2x\mapsto\colvec{2 \\ 0}
  \quad
  2x^2\mapsto\colvec{2 \\ 2}
\end{equation*}
Representing those with respect to the basis~$D$
\begin{equation*}
  \rep{\colvec{0 \\ 2}}{D}=\colvec{-2 \\ 2}
  \quad
  \rep{\colvec{2 \\ 0}}{D}=\colvec{2 \\ 0}
  \quad
  \rep{\colvec{2 \\ 2}}{D}=\colvec{0 \\ 2}
\end{equation*}
and concatenating gives the answer.
\begin{equation*}
  \rep{h}{B,D}=
  \begin{mat}
  -2 &2 &0 \\
  2  &0 &2
  \end{mat}
\end{equation*}
\end{solution}
\item What is the rank and nullity of $h$?
\begin{solution}[1.5in]
By eye, the matrix has rank~$2$.
Since the dimension of the domain is~$3$, the nullity is~$1$.
\end{solution}
\end{parts}




\question  % Five.II.3.20
For this matrix
\begin{equation*}
\begin{mat}
1 &2  \\
4 &3
\end{mat}
\end{equation*}
\begin{parts}
\item
find the characteristic polynomial and the eigenvalues,
\begin{solution}[2in]
The equation
\begin{equation*}
\begin{vmatrix}
1-x  &2  \\
4  &3-x
\end{vmatrix}=0
\end{equation*}
gives the characteristic equation
$0=(1-x)(3-x)-8=x^2-4x-5$.
The eigenvalues are $\lambda_1=5$ and  $\lambda_2=-1$.
\end{solution}

\item
draw the change of basis diagram,
\begin{solution}[1in]
\begin{equation*}
  \begin{CD}
    V_{\wrt{\stdbasis_2}}                   @>t>T>        V_{\wrt{\stdbasis_2}}       \\
    @V{\scriptstyle\identity} VV              @V{\scriptstyle\identity} VV \\
    V_{\wrt{B}}                   @>t>\hat{T}>        V_{\wrt{B}}
  \end{CD}
\end{equation*}
\end{solution}


\item
find a basis $B$ so that the matrix is diagonal. 
\begin{solution}[2in]
Find the eigenvectors associated with the eigenvalue of $\lambda_1=5$ as here.
\begin{equation*}
\begin{linsys}{2}
  -4x &+ &2y &= &0 \\
   4x &- &2y &= &0 
\end{linsys}
\grstep{\rho_1+\rho_2}   
\begin{linsys}{2}
  -4x &+ &2y &= &0 \\
      &  &0  &= &0
\end{linsys}
\qquad
V_5=\set{\colvec{1/2 \\ 1}y\suchthat y\in\C}
\end{equation*}
Find the eigenvectors associated with the eigenvalue of $\lambda_2=-1$ as here.
\begin{equation*}
\begin{linsys}{2}
   2x &+ &2y &= &0 \\
   4x &+ &4y &= &0 
\end{linsys}
\grstep{-2\rho_1+\rho_2}   
\begin{linsys}{2}
  2x &+ &2y &= &0 \\
     &  &0  &= &0
\end{linsys}
\qquad
V_{-1}=\set{\colvec{-1 \\ 1}y\suchthat y\in\C}
\end{equation*}
This is the basis.
\begin{equation*}
  B=\sequence{\colvec{1/2 \\ 1},
              \colvec{-1 \\ 1}}
\end{equation*}
\end{solution}
\end{parts}



\question
Produce a basis for the space.
\begin{equation*}
S=\set{ax^2+bx+c\in\polyspace_2\suchthat \text{$a-c=0$ and $b+c=0$}}
\end{equation*}
You must check that it is a basis.
\begin{solution}[2in]
Parametrizing the solution set of the system
\begin{equation*}
\begin{linsys}{3}
  a &  &  &- &c &= &0 \\
    &  &b &+ &c &= &0
\end{linsys}
\end{equation*}    
gives
\begin{equation*}
  \set{\colvec{a \\ b \\ c}=\colvec{1 \\ -1 \\ 1}c\suchthat c\in\Re}
\end{equation*}
and so a basis for $S$ has one element, $\sequence{x^2-x+1}$.
Since the element is non-zero, it is clearly linearly independent.
The parametrization shows that it spans the given space~$S$.
\end{solution}





\question
Consider the map $\map{t}{\Re^2}{\Re^2}$ that rotates vectors through an
angle of $\theta$ radians counterclockwise.
Represent it with respect to the standard basis.
\begin{solution}[1.5in]
On the basis elements the effect of the transformation is
\begin{equation*}
\colvec{1 \\ 0}\mapsto\colvec{\cos \pi/6 \\ \sin \pi/6}
\qquad
\colvec{0 \\ 1}\mapsto\colvec{-\sin \pi/6 \\ \cos \pi/6}
\end{equation*}
and with respect to the standard basis vectors represent themselves, 
so the matrix representation is this.
\begin{equation*}
T=\rep{t}{\stdbasis_2,\stdbasis_2}=
\begin{mat}
\cos\pi/6  &-\sin\pi/6 \\
\sin\pi/6   &\cos\pi/6
\end{mat}
\end{equation*}
\end{solution}




\question
Find the determinant of these matrices.
State whether each is singular or nonsingular.
\begin{parts}
\item $
\begin{mat}
3 &2 \\
3 &4
\end{mat}
$
\begin{solution}[0.5in]
$(3)(4)-(2)(3)=6$
\end{solution}

\item
$
\begin{mat}
1 &0 &3  \\
2 &1 &-1 \\
2 &0 &4
\end{mat}
$
\begin{solution}[2in]
Doing the Laplace expansion down the second column gives this.
\begin{equation*}
\begin{vmatrix}
1 &0 &3  \\
2 &1 &-1 \\
2 &0 &4
\end{vmatrix}
=-0
+1\cdot
\begin{vmatrix}
1 &3 \\
2 &4
\end{vmatrix}
-0
=-2
\end{equation*}
\end{solution}
\end{parts}




% \question
% Find the range space and the rank of this linear map $\map{h}{\polyspace_2}{\R^2}$.
% \begin{equation*}
%   ax^2+bx+c\mapsto\colvec{a+b \\ a-b}
% \end{equation*}
% \begin{solution}[2in]
% One way is to represent the map with respect to some bases.
% We take these.
% \begin{equation*}
% B=\sequence{1,x,x^2}
% \qquad
% D=\stdbasis_2
% \end{equation*}
% We have 
% \begin{equation*}
% 1\mapsunder{h}\colvec{0 \\ 0}
% \quad
% x\mapsunder{h}\colvec{1 \\ -1}
% \quad
% x^2\mapsunder{h}\colvec{1 \\ 1}
% \end{equation*}
% and so we have $\rep{h}{B,D}$.
% \begin{equation*}
% \rep{\colvec{0 \\ 0}}{D}=\colvec{0 \\ 0}
% \quad
% \rep{\colvec{1 \\ -1}}{D}=\colvec{1 \\ -1}
% \quad
% \rep{\colvec{1 \\ 1}}{D}=\colvec{1 \\ 1}
% \qquad
% \begin{mat}
% 0 &1  &1 \\
% 0 &-1 &1
% \end{mat}
% \end{equation*}
% By eye we see that Gauss's method on that matrix gives that its rank is two.
% This is also the rank of the map.
% Because the codomain has dimension two, the map is onto and the range space
% is all of $\R^2$.
% \end{solution}




\question % One.I.3.9
Find the solution set of this linear system.
Express it as a span of a minimal set of vectors.
\begin{equation*}
\begin{linsys}{4}
  x &   &  &+  &z  &+  &w  &=  &-1  \\
  2x&-  &y &   &   &+  &w  &=  &3   \\
  x &+  &y &+  &3z &+  &2w  &=  &1
\end{linsys}
\end{equation*}
\begin{solution}[2.5in]
\begin{equation*}
  \begin{linsys}{4}
    x  &   &  &+  &z  &+ &w  &=  &-1  \\
   2x  &-  &y &   &   &+ &w  &=  &3   \\
    x  &+  &y &+  &3z &+ &2w &=  &1   
  \end{linsys}
  \grstep[-\rho_1+\rho_3]{-2\rho_1+\rho_2}
  \begin{linsys}{4}
    x  &   &  &+  &z  &+ &w  &=  &-1  \\
       &   &-y&-  &2z &- &w  &=  &5   \\
       &   &y &+  &2z &+ &w  &=  &2   
   \end{linsys}
\end{equation*}
It has no solutions because the final two equations
conflict.
The solution set is empty.
\end{solution}


\question
Find the inverse of each matrix, or state that none exists.
\begin{parts}
\item % Three.Iv.4.15(e)
$
\begin{mat}
0 &1 &5  \\
0 &-2 &4  \\
2 &3 &-2 
\end{mat}
$
\begin{solution}[2in]
          \begin{multline*}
            \begin{pmat}{rrr|rrr}
              0  &1  &5  &1  &0  &0  \\
              0  &-2 &4  &0  &1  &0  \\ 
              2  &3  &-2 &0  &0  &1
            \end{pmat}
            \grstep{\rho_3\leftrightarrow\rho_1}\;
            \begin{pmat}{rrr|rrr}
              2  &3  &-2 &0  &0  &1  \\
              0  &-2 &4  &0  &1  &0  \\ 
              0  &1  &5  &1  &0  &0  
            \end{pmat}                                            \\
            \begin{aligned}
              &\grstep{(1/2)\rho_2+\rho_3}
              \begin{pmat}{rrr|rrr}
                2  &3  &-2 &0  &0   &1  \\
                0  &-2 &4  &0  &1   &0  \\ 
                0  &0  &7  &1  &1/2 &0  
              \end{pmat}                                             \\
              &\grstep[-(1/2)\rho_2 \\ (1/7)\rho_3]{(1/2)\rho_1}
              \begin{pmat}{rrr|rrr}
                1  &3/2  &-1 &0    &0     &1/2  \\
                0  &1    &-2 &0    &-1/2  &0    \\ 
                0  &0    &1  &1/7  &1/14  &0  
              \end{pmat}                                   \\
              &\grstep[\rho_3+\rho_1]{2\rho_3+\rho_2}
              \begin{pmat}{rrr|rrr}
                1  &3/2  &0  &1/7  &1/14  &1/2  \\
                0  &1    &0  &2/7  &-5/14 &0    \\ 
                0  &0    &1  &1/7  &1/14  &0  
              \end{pmat}                                   \\
              &\grstep{-(3/2)\rho_2+\rho_1}
              \begin{pmat}{rrr|rrr}
                1  &0    &0  &-2/7 &17/28 &1/2  \\
                0  &1    &0  &2/7  &-5/14 &0    \\ 
                0  &0    &1  &1/7  &1/14  &0  
              \end{pmat}
            \end{aligned}
          \end{multline*}
\end{solution}
\item
$
\begin{mat}
2  &6 \\
4  &12
\end{mat}
$
\begin{solution}[1in]
The determinant is zero so the matrix has no inverse.
\end{solution}
\end{parts}



\question
Find and parametrize the solution set.
What is the dimension of the solution set of the associated homogeneous
system?
\begin{equation*}
       \begin{linsys}{3}
         2x  &+ &y  &  &  &=  &3  \\
         3x  &- &2y &+ &z &=  &1  \\
         7x  &  &   &+ &z &=  &7  
         \end{linsys}
\end{equation*}
  \begin{solution}[2.5in]
    \begin{equation*}
      \grstep[-(7/2)\rho_1+\rho_3]{-(3/2)\rho_1+\rho_2}
      \begin{amat}{3}
        2 &1    &0 &3 \\
        0 &-7/2 &1 &-7/2 \\
        0 &-7/2 &1 &-7/2 
      \end{amat}
      \grstep{-\rho_2+\rho_3}
      \begin{amat}{3}
        2 &1    &0 &3 \\
        0 &-7/2 &1 &-7/2 \\
        0 &0    &0 &0 
      \end{amat}
    \end{equation*}
    The parametrization is this.
    \begin{equation*}
      S=\set{\colvec{x \\ y \\ z}=
             \colvec{1 \\ 1 \\ 0}+
             \colvec{-1/7 \\ 2/7 \\ 1}z\suchthat z\in\Re}
     \end{equation*}
     The associated homogeneous system has dimension~$1$.        
  \end{solution}

  
\question
Find the eigenvalues of this matrix.
\begin{equation*}
\begin{mat}
3 &0 \\
8 &-1
\end{mat}
\end{equation*}
\begin{solution}[2in]
Find the eigenvalues via the characteristic equation.
\begin{equation*}
0=\begin{vmatrix}
3-x &0  \\
8   &-1-x
\end{vmatrix}
=(3-x)(-1-x)-(8)(0)
\qquad \lambda_1=3, \; \lambda_2=-1
\end{equation*}
\end{solution}
\end{questions}
\end{document}

