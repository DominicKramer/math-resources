\documentclass[11pt]{article}
\usepackage[margin=1in]{geometry}
\usepackage{../linalgjh}

\setlength{\parindent}{0em}
\pagestyle{empty}
\begin{document}\thispagestyle{empty}
\makebox[\linewidth]{\textbf{Homework, MA~213}\hspace*{4in}\textbf{Due: 2014-Nov-17}}

\vspace*{3ex}
\textit{You may work with others to figure out how to do questions, 
and you are welcome to look for answers in the book, online, by talking
to someone who had the course before, etc.
However, you must write 
the answers on your own.
You must also show your work (you may, of course, 
quote any result from the book).}

\begin{enumerate}
\item
  For each space find the matrix changing a vector representation with 
  respect to $B$ to one with respect to~$D$.
  \begin{enumerate}
  \item $V=\Re^3$, $B=\stdbasis_3$, 
      $D=\sequence{\colvec{1 \\ 2 \\ 3},
                   \colvec{1 \\ 1 \\ 1},
                   \colvec{0 \\ 1 \\ -1}}$

Start by computing the effect of the identity function on each element of the
starting basis~$B$.  Obviously this is the effect.
\begin{equation*}
  \colvec{1 \\ 0 \\ 0}\mapsunder{\identity}\colvec{1 \\ 0 \\ 0}
  \quad
  \colvec{0 \\ 1 \\ 0}\mapsunder{\identity}\colvec{0 \\ 1 \\ 0}
  \quad
  \colvec{0 \\ 0 \\ 1}\mapsunder{\identity}\colvec{0 \\ 0 \\ 1}
\end{equation*}
Now represent the three outputs with respect to the ending basis.
\begin{equation*}
  \rep{\colvec{1 \\ 0 \\ 0}}{D}=\colvec{-2/3 \\ 5/3 \\ -1/3}
  \quad
  \rep{\colvec{0 \\ 1 \\ 0}}{D}=\colvec{1/3 \\ -1/3 \\ 2/3}
  \quad
  \rep{\colvec{0 \\ 0 \\ 1}}{D}=\colvec{1/3 \\ -1/3 \\ -1/3}
\end{equation*}
Concatenate them into a basis.
\begin{equation*}
  \rep{\identity}{B,D}=
  \begin{mat}
    -2/3 &1/3  &1/3  \\
     5/3 &-1/3 &-1/3 \\
    -1/3 &2/3  &-1/3
  \end{mat}
\end{equation*}


  \item $V=\Re^3$,  
      $B=\sequence{\colvec{1 \\ 2 \\ 3},
                   \colvec{1 \\ 1 \\ 1},
                   \colvec{0 \\ 1 \\ -1}}$,
      $D=\stdbasis_3$

One way to find this is to take the inverse of the prior matrix, since it 
converts bases inthe other direction.
Alternatively, we can compute these three
\begin{equation*}
  \rep{\colvec{1 \\ 2 \\ 3}}{\stdbasis_3}
    =\colvec{1 \\ 2 \\ 3}
  \quad
  \rep{\colvec{1 \\ 1 \\ 1}}{\stdbasis_3}
    =\colvec{1 \\ 1 \\ 1}
  \quad
  \rep{\colvec{0 \\ 1 \\ -1}}{\stdbasis_3}
     =\colvec{0 \\ 1 \\ -1}
\end{equation*}
and put them in a matrix.
\begin{equation*}
  \rep{\identity}{B,D}=
  \begin{mat}
    1 &1 &0 \\
    2 &1 &1 \\
    3 &1 &-1
  \end{mat}
\end{equation*}

  \item $V=\polyspace_2$,
    $B=\sequence{x^2, x^2+x, x^2+x+1}$,
    $D=\sequence{2, -x, x^2}$

Representing $\identity(x^2)$, $\identity(x^2+x)$, and~$\identity(x^2+x+1)$
with respect to the ending basis gives this.
\begin{equation*}
  \rep{x^2}{D}=\colvec{0 \\ 0 \\ 1}
  \quad
  \rep{x^2+x}{D}=\colvec{0 \\ -1 \\ 1}
  \quad
  \rep{x^2+x+1}{D}=\colvec{1/2 \\ -1 \\ 1}
\end{equation*}
Put them together.
\begin{equation*}
  \rep{\identity}{B,D}=
  \begin{mat}
    0 &0 &1/2 \\
    0 &-1 &-1 \\
    1 &1  &1
  \end{mat}
\end{equation*}
  \end{enumerate}

\item Find the $P$ and~$Q$ to express $H$ via $PHQ$ as a block partial identity
  matrix.
  \begin{equation*}
    H=
    \begin{mat}
      2 &1  &1  \\
      3 &-1 &0  \\
      1 &3  &2 
    \end{mat}
  \end{equation*}

Gauss's Method gives this.
\begin{equation*}
    \begin{mat}
      2 &1  &1  \\
      3 &-1 &0  \\
      1 &3  &2 
    \end{mat}
    \grstep[-(1/2)\rho_1+\rho_3]{-(3/2)\rho_1+\rho_2}
    \grstep{\rho_2+\rho_3}
    \grstep[-(2/5)\rho_2]{(1/2)\rho_1}
    \begin{mat}
      1 &1/2  &1/2  \\
      0 &1    &3/5  \\
      0 &0    &0 
    \end{mat}
\end{equation*}
Column operations complete the job of reaching the canonical form
for matrix equivalence.
\begin{equation*}
  \grstep{-(3/5)\text{col}_2+\text{col}_3}
  \grstep[-(1/5)\text{col}_1+\text{col}_3]{-(1/2)\text{col}_1+\text{col}_2}
    \begin{mat}
      1 &0    &0  \\
      0 &1 &0  \\
      0 &0  &0 
    \end{mat}
\end{equation*}
Then these are the two matrices.
\begin{gather*}
  P=
  \begin{mat}
    1 &0 &0 \\
    0 &-2/5 &0 \\
    0 &0 &1
   \end{mat}
  \begin{mat}
    1/2 &0 &0 \\
    0 &1 &0 \\
    0 &0 &1
   \end{mat}
  \begin{mat}
    1 &0 &0 \\
    0 &1 &0 \\
    0 &1 &1
   \end{mat}
  \begin{mat}
    1    &0 &0 \\
    0    &1 &0 \\
    -1/2 &0 &1
   \end{mat}
  \begin{mat}
    1    &0 &0 \\
    -3/2 &1 &0 \\
    0    &0 &1
   \end{mat} 
   =
   \begin{mat}
     1/2   &0     &0 \\
     3/5   &-2/5  &0 \\
     -2    &1     &1
   \end{mat}                              \\
  Q=
  \begin{mat}
    1 &0 &0 \\
    0 &1 &-3/5 \\
    0 &0 &1
  \end{mat}
  \begin{mat}
    1 &-1/2 &0 \\
    0 &1    &0 \\
    0 &0    &1
  \end{mat}
  \begin{mat}
    1 &0 &-1/5 \\
    0 &1 &0 \\
    0 &0 &1
  \end{mat}
  =
  \begin{mat}
    1 &1/2 &-1/5 \\
    0 &1   &-3/5 \\
    0 &0   &1
  \end{mat}
\end{gather*}




\item Project the vector to the line.
  \begin{equation*}
    \vec{v}=\colvec{3 \\ 2}
    \qquad
    L=\set{c\colvec{1 \\ -1}\suchthat c\in\Re}
  \end{equation*}

The formula is straightforward.
\begin{equation*}
  \frac{\colvec{3 \\ 2}\dotprod\colvec{1 \\ -1}}{\colvec{1 \\ -1}\dotprod\colvec{1 \\ -1}}\cdot\colvec{1 \\ -1}  
  =\colvec{1/2 \\ -1/2}  
\end{equation*}



\item Express this nonsingular matrix as a product of elementary reduction
  matrices.
  \begin{equation*}
    T=\begin{mat}
      1 &2  &0 \\
      2 &-1 &0 \\
      3 &1 &2
    \end{mat}
  \end{equation*}

The Gauss-Jordan reduction is routine.
\begin{equation*}
    \begin{mat}
      1 &2  &0 \\
      2 &-1 &0 \\
      3 &1 &2
    \end{mat}
  \grstep[-3\rho_1+\rho_3]{-2\rho_1+\rho_2}
  \grstep{-\rho_2+\rho_3}
  \grstep[(1/2)\rho_3]{-(1/5)\rho_2}
  \grstep{-2\rho_2+\rho_1}
  \begin{mat}
    1 &0 &0 \\
    0 &1 &0 \\
    0 &0 &1
  \end{mat}
\end{equation*}
Thus we know elementary reduction matrices $R_1,\ldots,R_6$ such that
$R_6\cdot R_5\cdots R_1\cdot T=I$.
Move the matrices to the other side (paying attention to order; 
you first multiply both sides from the left by $R_6^{-1}$, etc.).
\begin{align*}
  T=
  &\begin{mat}
    1 &0 &0 \\
    -2 &1 &0 \\
    0 &0 &1 
  \end{mat}^{-1}
  \begin{mat}
    1 &0 &0 \\
    0 &1 &0 \\
    -3 &0 &1 
  \end{mat}^{-1}
  \begin{mat}
    1 &0 &0 \\
    0 &1 &0 \\
    0 &-1 &1 
  \end{mat}^{-1}
  \begin{mat}
    1 &0 &0 \\
    0 &-1/5 &0 \\
    0 &0 &1 
  \end{mat}^{-1}
  \begin{mat}
    1 &0 &0 \\
    0 &1 &0 \\
    0 &0 &1/2 
  \end{mat}^{-1}
  \begin{mat}
    1 &-2 &0 \\
    0 &1 &0 \\
    0 &0 &1 
  \end{mat}^{-1}             \\
  &=
  \begin{mat}
    1 &0 &0 \\
    2 &1 &0 \\
    0 &0 &1
  \end{mat}
  \begin{mat}
    1 &0 &0 \\
    0 &1 &0 \\
    3 &0 &1
  \end{mat}
  \begin{mat}
    1 &0 &0 \\
    0 &1 &0 \\
    0 &1 &1
  \end{mat}
  \begin{mat}
    1 &0 &0 \\
    0 &-5 &0 \\
    0 &0 &1
  \end{mat}
  \begin{mat}
    1 &0 &0 \\
    0 &1 &0 \\
    0 &0 &2
  \end{mat}
  \begin{mat}
    1 &2 &0 \\
    0 &1 &0 \\
    0 &0 &1
  \end{mat}
\end{align*}



\end{enumerate}
\end{document}


sage: eqs=[x+y==1, 2*x+y+z==0, 3*x+y-z==0]
sage: solve(eqs, x,y,z)
[[x == (-2/3), y == (5/3), z == (-1/3)]]
sage: eqs=[x+y==0, 2*x+y+z==1, 3*x+y-z==0]
sage: solve(eqs, x,y,z)
[[x == (1/3), y == (-1/3), z == (2/3)]]
sage: eqs=[x+y==0, 2*x+y+z==0, 3*x+y-z==1]
sage: solve(eqs, x,y,z)
[[x == (1/3), y == (-1/3), z == (-1/3)]]


load "../lab/gauss_method.sage"
sage: H=matrix(QQ, [[2,1,1], [3,-1,0], [1,3,2]])
sage: gauss_jordan(H)
[ 2  1  1]
[ 3 -1  0]
[ 1  3  2]
 take -3/2 times row 1 plus row 2
 take -1/2 times row 1 plus row 3
[   2    1    1]
[   0 -5/2 -3/2]
[   0  5/2  3/2]
 take 1 times row 2 plus row 3
[   2    1    1]
[   0 -5/2 -3/2]
[   0    0    0]
 take 1/2 times row 1
 take -2/5 times row 2
[  1 1/2 1/2]
[  0   1 3/5]
[  0   0   0]
 take -1/2 times row 2 plus row 1
[  1   0 1/5]
[  0   1 3/5]
[  0   0   0]
sage: M1=matrix(QQ,[[1,0,0], [0,-2/5,0], [0,0,1]])
sage: M1
[   1    0    0]
[   0 -2/5    0]
[   0    0    1]
sage: M2=matrix(QQ,[[1/2,0,0], [0,1,0], [0,0,1]])
sage: M2
[1/2   0   0]
[  0   1   0]
[  0   0   1]
sage: M3=matrix(QQ,[[1,0,0], [0,1,0], [0,1,1]])
sage: M3
[1 0 0]
[0 1 0]
[0 1 1]
sage: M4=matrix(QQ,[[1,0,0], [0,1,0], [-1/2,0,1]])
sage: M4
[   1    0    0]
[   0    1    0]
[-1/2    0    1]
sage: M5=matrix(QQ,[[1,0,0], [-3/2,1,0], [0,0,1]])
sage: M5
[   1    0    0]
[-3/2    1    0]
[   0    0    1]
sage: M1*M2*M3*M4*M5
[ 1/2    0    0]
[ 3/5 -2/5    0]
[  -2    1    1]
sage: M1=matrix(QQ,[[1,0,0], [0,1,-3/5], [0,0,1]])
sage: M2=matrix(QQ,[[1,1/2,0], [0,1,0], [0,0,1]])
sage: M1
[   1    0    0]
[   0    1 -3/5]
[   0    0    1]
sage: M2
[  1 1/2   0]
[  0   1   0]
[  0   0   1]
sage: M3=matrix(QQ,[[1,0,-1/5], [0,1,0], [0,0,1]])
sage: M3
[   1    0 -1/5]
[   0    1    0]
[   0    0    1]
sage: M1*M2*M3
[   1  1/2 -1/5]
[   0    1 -3/5]
[   0    0    1]
s