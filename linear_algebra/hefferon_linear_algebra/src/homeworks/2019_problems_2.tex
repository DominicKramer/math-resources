\documentclass[answers]{examjh}
\usepackage{../sty/linalgjh}

\renewcommand{\legalesetext}{
  While you are working on these questions
  you may confer with others in the class, 
  or talk to someone who had the course before, etc.
  % or look at the book or in the library or online, etc.
  However, when you write up the answer, you must write on your own.
  Note also: \textit{you must show your work}.}
\examhead{MA~213 homework two}{Due: 2019-Oct-07}

\begin{document}

\begin{questions}
\question Verify that each is a vector space.   
\begin{parts}   
  \item The collection of $\nbyn{2}$ matrices with $0$'s in the
     upper right and lower left entries.     
     \begin{equation*}       
       \set{         
         \begin{mat}           
           a  &0  \\ 
           0  &b          
        \end{mat}         
        \suchthat a,b\in\Re}     
     \end{equation*}
   \begin{solution}
   This is a subset of the vector space of all $\nbyn{2}$ matrices. 
   So
   to verify that it is a subspace we need only check that this set
   is closed under addition and scalar multiplication.
   It is closed under addition because the sum of two diagonal matrices
   is another diagonal matrix.
   \begin{equation*}
      \begin{mat}           
        a_1  &0  \\ 
        0    &b_1          
      \end{mat}
      +
      \begin{mat}           
        a_2  &0  \\ 
        0    &b_2          
      \end{mat}
      =
      \begin{mat}           
        a_1+a_2  &0        \\ 
        0        &b_1+b_2          
      \end{mat}         
   \end{equation*}
   The scalar multiple of a diagonal matrix
   is a diagonal matrix.
   \begin{equation*}
      r\cdot\begin{mat}           
        a  &0  \\ 
        0  &b          
      \end{mat}
      =
      \begin{mat}           
        ra    &0        \\ 
        0     &rb          
      \end{mat}         
   \end{equation*}
   \end{solution}
  \item The collection $S=\set{a_0+a_1x+a_3x^3 \suchthat a_0,a_1,a_3\in\Re}$
      of cubic polynomials with no quadratic term.
    \begin{solution}
      This is a subset of the vector space $\polyspace_3$ so to show it is a 
      subspace we need only
      check closure under addition and scalar multiplication.
      The sum of two members of~$S$ is another member of~$S$.
      \begin{equation*}
        (a_0+a_1x+a_3x^3)
        +(b_0+b_1x+b_3x^3)
        =(a_0+b_0)+(a_1+b_1)x+(a_3+b_3)x^3
      \end{equation*}
      A scalar multiple of such a polynomial is another such polynomial.
      \begin{equation*}
        r\cdot(a_0+a_1x+a_3x^3)
        =(ra_0)+(ra_1)x+(ra_3)x^3
      \end{equation*}
    \end{solution}
\end{parts} 

\question Determine if each set is linearly independent, in the natural vector space.   
  \begin{parts}   
    \item $\set{\colvec{1 \\ 2 \\ 0},
               \colvec{-1 \\ 1 \\ 0}}$
    \begin{solution}
    This set is linearly independent because the second vector is not
    a multiple of the first, by inspection.
    Or, we can set up the equation.
    \begin{equation*}
      c_1\colvec{1 \\ 2 \\ 0}+c_2\colvec{-1 \\ 1 \\ 0}=\colvec{0 \\ 0 \\ 0}
    \end{equation*}
    Here is the solution.
    \begin{equation*}
      \begin{linsys}{2}
        c_1 &- &c_2 &= &0 \\
       2c_1 &+ &c_2 &= &0 \\
            &  &0   &= &0
      \end{linsys}
      \grstep{-2\rho_1+\rho_2}
      \begin{linsys}{2}
        c_1 &- &c_2  &= &0 \\
            &  &3c_2 &= &0 \\
            &  &0    &= &0
      \end{linsys}
    \end{equation*}
    The only solution is the trivial one $c_1=c_2=0$ so the set is
    linearly independent.
    \end{solution}
    \item $\set{\rowvec{1 &3 &1}, \rowvec{-1 &4 &3}, \rowvec{-1 &11 &7}}$
    \begin{solution}
    By eye we can spot that two times the second vector plus the first
    equals the third, so this set is linearly dependent.
    Or, we can set up the system
    \begin{equation*}
      c_1\rowvec{1 &3 &1}+c_2\rowvec{-1 &4 &3}+c_3\rowvec{-1 &11 &7}
      =\rowvec{0 &0 &0}
    \end{equation*}
    and solve.
    \begin{equation*}
      \begin{linsys}{3}
        c_1 &- &c_2  &- &c_3   &= &0 \\
       3c_1 &+ &4c_2 &+ &11c_3 &= &0 \\
        c_1 &+ &3c_2 &+ &7c_3  &= &0
      \end{linsys}
      \grstep[-\rho_1+\rho_3]{-3\rho_1+\rho_2}
      \begin{linsys}{3}
        c_1 &- &c_2  &- &c_3   &= &0 \\
            &  &7c_2 &+ &14c_3 &= &0 \\
            &  &4c_2 &+ &8c_3  &= &0
      \end{linsys}
      \grstep{-(4/7)\rho_2+\rho_3}
      \begin{linsys}{3}
        c_1 &- &c_2  &- &c_3   &= &0 \\
            &  &7c_2 &+ &14c_3 &= &0 \\
            &  &     &  &0     &= &0
      \end{linsys}
    \end{equation*}
    This system has infinitely many solutions, so it has more than just one
    solution, so the original set is linearly dependent.
    \end{solution}
    \item $\set{\begin{mat}
                 5 &4 \\
                 1 &2
               \end{mat},
               \begin{mat}
                 0 &0 \\
                 0 &0
               \end{mat},
               \begin{mat}
                 1 &0 \\
                -1 &4
               \end{mat}
            }$
       \begin{solution}
         Any set containing the zero element is linearly dependent.     
       \end{solution}
   \end{parts} 

\question Is the vector in the span of the set?   
  \begin{equation*}
     \colvec{1 \\ 0 \\ 3}
     \quad
     \set{\colvec{2 \\ 1 \\ -1},          
     \colvec{1 \\ -1 \\ 1}}   
  \end{equation*}
 \begin{solution}
 Set up the equation
 \begin{equation*}
   \colvec{1 \\ 0 \\ 3}=c_1\colvec{2 \\ 1 \\ -1}  +c_2\colvec{1 \\ -1 \\ 1}
 \end{equation*}
 and solve.
 \begin{equation*}
   \begin{linsys}{2}
     2c_1 &+ &c_2  &= &1  \\
      c_1 &- &c_2  &= &0  \\
     -c_1 &+ &c_2  &= &3  \\
   \end{linsys}
   \grstep[(1/2)\rho_1+\rho_3]{-(1/2)\rho_1+\rho_2}
   \begin{linsys}{2}
     2c_1 &+ &c_2        &= &1  \\
          &  &-(3/2)c_2  &= &0  \\
          &  &(3/2)c_2   &= &7/2  \\
   \end{linsys}
   \grstep{\rho_2+\rho_3}
   \begin{linsys}{2}
     2c_1 &+ &c_2       &= &0    \\
          &  &-(3/2)c_2  &= &0    \\
          &  &0          &= &7/2  \\
   \end{linsys}
 \end{equation*}
 Because there is no solution, the vector is not in the span.
 \end{solution}
\end{questions}
\end{document}

