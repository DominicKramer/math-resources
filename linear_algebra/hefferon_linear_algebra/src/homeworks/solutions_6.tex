\documentclass[11pt]{article}
\usepackage[margin=1in]{geometry}
\usepackage{../linalgjh}

\setlength{\parindent}{0em}
\pagestyle{empty}
\begin{document}\thispagestyle{empty}
\makebox[\linewidth]{\textbf{Homework, MA~213}\hspace*{4in}\textbf{Due: 2014-Nov-03}}

\vspace*{3ex}
\textit{You may work with others to figure out how to do questions, 
and you are welcome to look for answers in the book, online, by talking
to someone who had the course before, etc.
However, you must write 
the answers on your own.
You must also show your work (you may, of course, 
quote any result from the book).}

\begin{enumerate}
\item
  Assume that each matrix represents a map $\map{h}{\Re^m}{\Re^n}$
  with respect to the standard bases.
  In each case, 
  (i)~state $m$ and~$n$
  (ii)~find $\rangespace{h}$ and $\rank(h)$
  (iii)~find $\nullspace{h}$ and $\nullity(h)$,
  and (iv)~state whether the map is onto and whether it is one-to-one.
  \begin{enumerate}
  \item
    $
    \begin{mat}
      2  &1  \\ 
      -1 &3
    \end{mat}
    $

    For parts of the answer we will need to solve this system
    \begin{equation*}
      \begin{mat}
        2  &1  \\ 
       -1  &3        
      \end{mat}
      \colvec{x \\ y}
      =
      \colvec{a \\ b}
      \tag{$*$}
    \end{equation*}
    for $x$ and~$y$ so we do that calculation first.
    \begin{equation*}
      \begin{amat}{2}
      2  &1 &a  \\ 
      -1 &3 &b       
      \end{amat}
      \grstep{(1/2)\rho_1+\rho_2}
      \grstep[(2/7)\rho_2]{(1/2)\rho_1}
      \grstep{-(1/2)\rho_2+\rho_1}
      \begin{amat}{2}
        1  &0 &(3/7)a-(1/7)b  \\ 
        0  &1 &(1/7)a+(2/7)b       
      \end{amat}
    \end{equation*}
    (i)~The dimension of the domain space is the number of columns~$m=2$,
    and the dimension of the codomain space is the number of rows~$n=2$.
    (ii)~For all
    \begin{equation*}
      \colvec{a \\ b}\in\Re^2
    \end{equation*}
    in equation~($*$) the system has a solution, by the calculation.
    So the range space is all of the codomain~$\rangespace{h}=\Re^2$.
    The map's rank is the dimension of the range,~$2$
    (iii)~Again by the calculation, setting $a=b=0$ in equation~($*$)
    gives that $x=y=0$.
    The null space is the trivial subspace of the domain.
    \begin{equation*}
      \nullspace{h}=\set{\colvec{0 \\ 0}}
    \end{equation*}
    The nullity is the dimension of that null space,~$0$.
    (iv)~The map is onto because the range space is all of the codomain.
    The map is one-to-one because the null space is trivial. 

  \item
    $
    \begin{mat}
      0  &1  &3  \\ 
      2  &3  &4  \\
     -2  &-1 &2 
    \end{mat}
    $

   The calculation is this.
    \begin{multline*}
      \begin{amat}{3}
        0  &1  &3  &a \\ 
        2  &3  &4  &b \\
       -2  &-1 &2  &c
      \end{amat}
      \grstep{\rho_1\leftrightarrow\rho_2}
      \grstep{\rho_1+\rho_3}
      \grstep{-2\rho_2+\rho_3}                 \\
      \grstep{(1/2)\rho_1}
      \grstep{-(3/2)\rho_2+\rho_1}
      \begin{amat}{3}
        1  &0 &-5/2 &-(3/2)a+(1/2)b  \\ 
        0  &1 &3    &a  \\
        0  &0 &0    &-2a+b+c      
      \end{amat}
    \end{multline*}
    (i)~The dimension of the domain space is the number of columns,~$m=3$,
    and the dimension of the codomain space is the number of rows,~$n=3$.
    (ii)~There are codomain triples
    \begin{equation*}
      \colvec{a \\ b \\ c}\in\Re^3
    \end{equation*}
    for which the system does not have a solution,
    specifically the system only has a solution if $-2a+b+c=0$.
    \begin{equation*}
      \rangespace{h}=\set{\colvec{a \\ b \\ c}\suchthat a=(b+c)/2}
                    =\set{\colvec{1/2 \\ 1 \\ 0}b
                          +\colvec{1/2 \\ 0 \\ 1}c\suchthat b,c\in\Re} 
    \end{equation*}
    The map's rank is the range's dimension,~$2$
    (iii)~Setting $a=b=c=0$ in the calculation
    gives infinitely many solutions.
    Paramatrizing using the free variable~$z$ leads to this description
    of the nullspace.
    \begin{equation*}
      \nullspace{h}=\set{\colvec{x \\ y \\ z}\suchthat 
                       \text{$y=-3z$ and $x=(5/2)z$}}
                   =\set{\colvec{5/2 \\ -3 \\ 1}z\suchthat z\in\Re}
    \end{equation*}
    The nullity is the dimension of that null space,~$1$.
    (iv)~The map is not onto because the range space is not all of the codomain.
    The map is not one-to-one because the null space is not trivial. 



  \item
    $
    \begin{mat}
      1  &1  \\ 
      2  &1  \\
      3  &1
    \end{mat}
    $
  \end{enumerate}

  Here is the calculation.
    \begin{equation*}
      \begin{amat}{2}
        1  &1  &a \\ 
        2  &1  &b \\
        3  &1  &c
      \end{amat}
      \grstep[-3\rho_1+\rho_3]{-2\rho_1+\rho_2}
      \grstep{-2\rho_2+\rho_3}
      \grstep{-\rho_2}                 
      \grstep{-\rho_2+\rho_1}
      \begin{amat}{2}
        1  &0 &-a+b  \\ 
        0  &1 &2a-b  \\
        0  &0 &a-2b+c      
      \end{amat}
    \end{equation*}
  (i)~The domain has dimension $m=2$ while the codomain has dimension~$n=3$.
  (ii)~The range is this subspace of the codomain.
  \begin{equation*}
    \rangespace{h}=\set{\colvec{2b-c \\ b \\ c}\suchthat b,c\in\Re}
        =\set{\colvec{2 \\ 1 \\ 0}b+\colvec{-1 \\ 0 \\ 1}c\suchthat b,c\in\Re}
  \end{equation*}
  The rank is~$2$.
  (iii)~The null space is the trivial subspace of the domain.
  \begin{equation*}
    \nullspace{h}=\set{\colvec{x \\ y}=\colvec{0 \\ 0}}
  \end{equation*}
  The nullity is~$0$.
  (iv)~The map is not onto, and is one-to-one.


\item Verify that the map $\map{h}{\Re^m}{\Re^n}$ represented by this matrix
  with respect to the standard bases
  \begin{equation*}
    \begin{mat}
      2  &1  &0  \\
      3  &1  &1  \\
      7  &2  &1
    \end{mat}
  \end{equation*}
  is an isomorphism. 

  We have shown that, given a $\nbym{n}{m}$ matrix~$H$ and vector spaces 
  $V,W$ of dimension~$m$ and~$n$, if we fix bases $B\subseteq V$ and
  $D\subseteq W$ then the map $\map{h}{V}{W}$ defined by $\rep{h}{B,D}=H$
  is linear (Theorem Three.III.2.2).
  So it only remains to show that the map is onto and one-to-one.
  For that we don't need to augment the matrix with $a$, $b$, 
  and~$c$; this calculation
    \begin{equation*}
      \begin{mat}
        2  &1  &0 \\ 
        3  &1  &1 \\
        7  &2  &1
      \end{mat}
      \grstep[-(7/2)\rho_1+\rho_3]{-(3/2)\rho_1+\rho_2}
      \grstep{-3\rho_2+\rho_3}
      \grstep[-2\rho_2 \\ -(1/2)\rho_3]{(1/2)\rho_1}                 
      \grstep{2\rho_3+\rho_2}
      \grstep{-(1/2)\rho_2+\rho_1}
      \begin{mat}
        1  &0 &0  \\ 
        0  &1 &0  \\
        0  &0 &1      
      \end{mat}
    \end{equation*}
 shows that for each codomain vector there is one and only one associated
 domain vector.

\item For these matrices
  \begin{equation*}
    A=
    \begin{mat}
      2  &1  \\
      4  &3
    \end{mat}
    \quad
    B=
    \begin{mat}
      3  &4  &-2 \\
      0  &0  &0  \\
      1  &-1 &5
    \end{mat}
    \quad
    C=
    \begin{mat}
      2  &1  &1 \\
      1  &1  &2
    \end{mat}
  \end{equation*}
    \begin{enumerate}
      \item Find, or state ``not defined'': $5A$, $6B$, $7C$. 

  \begin{equation*}
    5A=
    \begin{mat}
      10  &5  \\
      20  &15
    \end{mat}
    \quad
    6B=
    \begin{mat}
      18  &24  &-12 \\
      0  &0  &0  \\
      6  &-6 &30
    \end{mat}
    \quad
    C=
    \begin{mat}
      14  &7  &7 \\
      7  &7  &14
    \end{mat}
  \end{equation*}

      \item Find, or state ``not defined'': $A+B$, $B+C$. 
    \end{enumerate}

  Neither is defined.

\end{enumerate}
\end{document}


sage: load "../lab/gauss_method.sage"
sage: M = matrix(QQ, [[2,1], [-1,3]])
sage: gauss_jordan(M)
[ 2  1]
[-1  3]
 take 1/2 times row 1 plus row 2
[  2   1]
[  0 7/2]
 take 1/2 times row 1
 take 2/7 times row 2
[  1 1/2]
[  0   1]
 take -1/2 times row 2 plus row 1
[1 0]
[0 1]
sage: var('x,y,a,b')
(x, y, a, b)
sage: eqs=[2*x+y==a, -x+3*y==b]
sage: solve(eqs, x, y)
[[x == 3/7*a - 1/7*b, y == 1/7*a + 2/7*b]]


sage: M = matrix(QQ, [[0,1,3], [2,3,4], [-2,-1,2]])
sage: load "../lab/gauss_method.sage"
/usr/lib/sagemath/local/lib/python2.7/site-packages/sage/misc/sage_extension.py:371: DeprecationWarning: Use %runfile instead of load.
See http://trac.sagemath.org/12719 for details.
  line = f(line, line_number)
sage: gauss_jordan(M)
[ 0  1  3]
[ 2  3  4]
[-2 -1  2]
 swap row 1 with row 2
[ 2  3  4]
[ 0  1  3]
[-2 -1  2]
 take 1 times row 1 plus row 3
[2 3 4]
[0 1 3]
[0 2 6]
 take -2 times row 2 plus row 3
[2 3 4]
[0 1 3]
[0 0 0]
 take 1/2 times row 1
[  1 3/2   2]
[  0   1   3]
[  0   0   0]
 take -3/2 times row 2 plus row 1
[   1    0 -5/2]
[   0    1    3]
[   0    0    0]


sage: M = matrix(QQ, [[1,1], [2,1], [3,1]])
sage: gauss_jordan(M)
[1 1]
[2 1]
[3 1]
 take -2 times row 1 plus row 2
 take -3 times row 1 plus row 3
[ 1  1]
[ 0 -1]
[ 0 -2]
 take -2 times row 2 plus row 3
[ 1  1]
[ 0 -1]
[ 0  0]
 take -1 times row 2
[1 1]
[0 1]
[0 0]
 take -1 times row 2 plus row 1
[1 0]
[0 1]
[0 0]


M = matrix(QQ, [[2,1,0], [3,1,1], [7,2,1]])
sage: gauss_jordan(M)
[2 1 0]
[3 1 1]
[7 2 1]
 take -3/2 times row 1 plus row 2
 take -7/2 times row 1 plus row 3
[   2    1    0]
[   0 -1/2    1]
[   0 -3/2    1]
 take -3 times row 2 plus row 3
[   2    1    0]
[   0 -1/2    1]
[   0    0   -2]
 take 1/2 times row 1
 take -2 times row 2
 take -1/2 times row 3
[  1 1/2   0]
[  0   1  -2]
[  0   0   1]
 take 2 times row 3 plus row 2
[  1 1/2   0]
[  0   1   0]
[  0   0   1]
 take -1/2 times row 2 plus row 1
[1 0 0]
[0 1 0]
[0 0 1]
