%%%%(c)
%%%%(c)  This file is a portion of the source for the textbook
%%%%(c)
%%%%(c)    A First Course in Linear Algebra
%%%%(c)    Copyright 2004 by Robert A. Beezer
%%%%(c)
%%%%(c)  See the file COPYING.txt for copying conditions
%%%%(c)
%%%%(c)
Suppose $\vect{u}$ and $\vect{v}$ are two vectors in $\complex{m}$.  Define a new operation, called ``subtraction,'' as the new vector denoted $\vect{u}-\vect{v}$ and defined by
%
\begin{align*}
\vectorentry{\vect{u}-\vect{v}}{i}=\vectorentry{\vect{u}}{i}-\vectorentry{\vect{v}}{i}&&1\leq i\leq m
\end{align*}
%
Prove that we can express the subtraction of two vectors in terms of our two basic operations.  More precisely, prove that $\vect{u}-\vect{v}=\vect{u}+(-1)\vect{v}$.  So in a sense, subtraction is not something new and different, but is just a convenience.  Mimic the style of similar proofs in this section.