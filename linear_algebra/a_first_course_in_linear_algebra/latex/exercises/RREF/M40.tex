%%%%(c)
%%%%(c)  This file is a portion of the source for the textbook
%%%%(c)
%%%%(c)    A First Course in Linear Algebra
%%%%(c)    Copyright 2004 by Robert A. Beezer
%%%%(c)
%%%%(c)  See the file COPYING.txt for copying conditions
%%%%(c)
%%%%(c)
Consider the two $3\times 4$ matrices below
%
\begin{align*}
B&=
\begin{bmatrix}
1 & 3 & -2 & 2 \\
-1 & -2 & -1 & -1 \\
-1 & -5 & 8 & -3
\end{bmatrix}
&
C&=
\begin{bmatrix}
1 & 2 & 1 & 2 \\
1 & 1 & 4 & 0 \\
-1 & -1 & -4 & 1
\end{bmatrix}
%
\end{align*}
%
\par
%
(a)\quad Row-reduce each matrix and determine that the reduced row-echelon forms of $B$ and $C$ are identical.  From this argue that $B$ and $C$ are row-equivalent.\par
%
(b)\quad In the proof of \acronymref{theorem}{RREFU}, we begin by arguing that entries of row-equivalent matrices are related by way of certain scalars and sums.  In this example, we would write that entries of $B$ from row $i$ that are in column $j$ are linearly related to the entries of $C$ in column $j$ from all three rows
%
\begin{align*}
\matrixentry{B}{ij}
&=
\delta_{i1}\matrixentry{C}{1j}+
\delta_{i2}\matrixentry{C}{2j}+
\delta_{i3}\matrixentry{C}{3j}
&
1&\leq j\leq 4
\end{align*}
%
For each $1\leq i\leq 3$ find the corresponding three scalars in this relationship.  So your answer will be nine scalars, determined three at a time.