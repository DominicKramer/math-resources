%%%%(c)
%%%%(c)  This file is a portion of the source for the textbook
%%%%(c)
%%%%(c)    A First Course in Linear Algebra
%%%%(c)    Copyright 2004 by Robert A. Beezer
%%%%(c)
%%%%(c)  See the file COPYING.txt for copying conditions
%%%%(c)
%%%%(c)
Suppose that the set $S$ in the hypothesis of \acronymref{theorem}{GSP} is not just linearly independent, but is also orthogonal.  Prove that the set $T$ created by the Gram-Schmidt procedure is equal to $S$.  (Note that we are getting a stronger conclusion than $\spn{T}=\spn{S}$ --- the conclusion is that $T=S$.)  In other words, it is pointless to apply the Gram-Schmidt procedure to a set that is already orthogonal.