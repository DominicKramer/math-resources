%%%%(c)
%%%%(c)  This file is a portion of the source for the textbook
%%%%(c)
%%%%(c)    A First Course in Linear Algebra
%%%%(c)    Copyright 2004 by Robert A. Beezer
%%%%(c)
%%%%(c)  See the file COPYING.txt for copying conditions
%%%%(c)
%%%%(c)
\begin{para}When we have a square matrix of size $n$, $A$, and we multiply it by a vector $\vect{x}$ from $\complex{n}$ to form the matrix-vector product (\acronymref{definition}{MVP}), the result is another vector in $\complex{n}$.  So we can adopt a functional view of this computation --- the act of multiplying by a square matrix is a function that converts one vector ($\vect{x}$) into another one ($A\vect{x}$) of the same size.  For some vectors, this seemingly complicated computation is really no more complicated than scalar multiplication.  The vectors vary according to the choice of $A$, so the question is to determine, for an individual choice of $A$, if there are any such vectors, and if so, which ones.  It happens in a variety of situations that these vectors (and the scalars that go along with them) are of special interest.\end{para}
%
\begin{para}We will be solving polynomial equations in this chapter, which raises the specter of roots that are complex numbers.  This distinct possibility is our main reason for entertaining the complex numbers throughout the course.  You might be moved to revisit \acronymref{section}{CNO} and \acronymref{section}{O}.\end{para}
