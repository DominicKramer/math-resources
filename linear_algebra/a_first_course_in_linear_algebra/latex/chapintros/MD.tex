%%%%(c)
%%%%(c)  This file is a portion of the source for the textbook
%%%%(c)
%%%%(c)    A First Course in Linear Algebra
%%%%(c)    Copyright 2004 by Robert A. Beezer
%%%%(c)
%%%%(c)  See the file COPYING.txt for copying conditions
%%%%(c)
%%%%(c)
%
\begin{para}{\sc\large Inner product is ``reversed'' from prior material, see changelog explanation}\end{para}
%
\begin{para}This chapter is about breaking up a matrix $A$ into pieces that somehow combine to recreate $A$.  Usually the pieces are again matrices, and usually they are then combined via matrix multiplication (\acronymref{definition}{MM}).  In some cases, the decomposition will be valid for any matrix, but often we might need extra conditions on $A$, such as being square (\acronymref{definition}{SQM}), nonsingular (\acronymref{definition}{NM}) or diagonalizable (\acronymref{definition}{DZM}) before we can guarantee the decomposition.  If you are comfortable with topics like decomposing a solution vector into linear combinations (\acronymref{subsection}{LC.VFSS}) or decomposing vector spaces into direct sums (\acronymref{subsection}{PD.DS}), then we will be doing similar things in this chapter.  If not, review these ideas and take another look at \acronymref{technique}{DC} on decompositions.\end{para}
%
\begin{para}We have studied one matrix decomposition already, so we will review that here in this introduction, both as a way of previewing the topic in a familiar setting, but also since it does not deserve another section all of its own.\end{para}
%
\begin{para}A diagonalizable matrix (\acronymref{definition}{DZM}) is defined to be a square matrix $A$ such that there is an invertible matrix $S$ and a diagonal matrix $D$ where $\inverse{S}AS=D$.  We can re-write this as $A=SD\inverse{S}$.   Here we have a decomposition of $A$ into three matrices, $S$, $D$ and $\inverse{S}$, which recombine through matrix multiplication to recreate $A$.  We also know that the diagonal entries of $D$ are the eigenvalues of $A$.  We cannot form this decomposition for just any matrix --- $A$ must be square and we know from \acronymref{theorem}{DC} that a matrix of size $n$ is diagonalizable if and only if there is a basis for $\complex{n}$ composed entirely of eigenvectors of $A$, or by \acronymref{theorem}{DMFE} we know that $A$ is diagonalizable if and only if each eigenvalue of $A$ has a geometric multiplicity equal to its algebraic multiplicity.  Some authors prefer to call this an \define{eigen decomposition} of $A$ rather than a \define{matrix diagonalization}.\end{para}
%
\begin{para}Another decomposition, which is similar in flavor to matrix diagonalization, is orthonormal diagonalization (\acronymref{theorem}{OD}).  Here we require the matrix $A$ to be normal and we get the decomposition $A=UD\adjoint{U}$, where $D$ is a diagonal matrix with the eigenvalues of $A$ on the diagonal, and $U$ is unitary.  The hypothesis that $A$ is normal guarantees the decomposition and we get the extra information that $U$ is unitary.\end{para}
%
\begin{para}Each section of this chapter features a different matrix decomposition, with the exception of \acronymref{section}{PSM}, which presents background information on positive semi-definite matrices required for singular value decompositions, square roots and polar decompositions.\end{para}