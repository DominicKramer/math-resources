Similar to (right) null spaces, a left null space can be computed in Sage with the matrix method \verb?.left_kernel()?.  For a matrix $A$, elements of the (right) null space are vectors $\vect{x}$ such that $A\vect{x}=\zerovector$.  If we interpret a vector placed to the left of a matrix as a 1-row matrix, then the product $\vect{x}A$ can be interpreted, according to our definition of matrix multiplication (\acronymref{definition}{MM)}, as the entries of the vector $\vect{x}$ providing the scalars for a linear combination of the \emph{rows} of the matrix $A$.  So you can view each vector in the left null space naturally as the entries of a matrix with a single row, $Y$, with the property that $YA=\zeromatrix$.  \par
%
Given Sage's preference for row vectors, the simpler name \verb?.kernel()? is a synonym for \verb?.left_kernel()?.  Given your text's preference for column vectors, we will continue to rely on the \verb?.right_kernel()?.  Left kernels in Sage have the same options as right kernels and produce vector spaces as output in the same manner.  Once created, a vector space all by itself (with no history) is neither left or right.  Here is a quick repeat of \acronymref{example}{LNS}.
%
\begin{sageexample}
sage: A = matrix(QQ, [[ 1, -3, 1],
...                   [-2,  1, 1],
...                   [ 1,  5, 1],
...                   [ 9, -4, 0]])
sage: A.left_kernel(basis='pivot')
Vector space of degree 4 and dimension 1 over Rational Field
User basis matrix:
[-2  3 -1  1]
\end{sageexample}
%
%
\begin{sageverbatim}
\end{sageverbatim}
%
