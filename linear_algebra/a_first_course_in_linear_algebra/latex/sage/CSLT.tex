As we mentioned in the last section, experimenting with Sage is a worthwhile complement to other methods of learning mathematics.  We have purposely avoided providing illustrations of deeper results, such as \acronymref{theorem}{ILTB} and  \acronymref{theorem}{SLTB}, which you should now be equipped to investigate yourself.  For completeness, and since composition wil be very important in the next few sections, we will provide an illustration of \acronymref{theorem}{CSLTS}.  Similar to what we did in the previous section, we choose dimensions suggested by \acronymref{theorem}{SLTD}, and then use randomly constructed matrices to form a pair of surjective linear transformations.
%
\begin{sageexample}
sage: U = QQ^4
sage: V = QQ^3
sage: W = QQ^2
sage: A = matrix(QQ, 3, 4, [[ 3, -2,  8, -9],
...                         [-1,  3, -4, -1],
...                         [ 3,  2,  8,  3]])
sage: T = linear_transformation(U, V, A, side='right')
sage: T.is_surjective()
True
sage: B = matrix(QQ, 2, 3, [[ 8, -5, 3],
...                         [-2,  1, 1]])
sage: S = linear_transformation(V, W, B, side='right')
sage: S.is_surjective()
True
sage: C = S*T
sage: C.is_surjective()
True
\end{sageexample}
%
\begin{sageverbatim}
\end{sageverbatim}
%

