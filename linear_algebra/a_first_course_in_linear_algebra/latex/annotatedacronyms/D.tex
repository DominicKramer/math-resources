%%%%(c)
%%%%(c)  This file is a portion of the source for the textbook
%%%%(c)
%%%%(c)    A First Course in Linear Algebra
%%%%(c)    Copyright 2004 by Robert A. Beezer
%%%%(c)
%%%%(c)  See the file COPYING.txt for copying conditions
%%%%(c)
%%%%(c)
%%%%%%%%%%%
%%
%%  Annotated Acronyms D
%%  Determinants
%%
%%%%%%%%%%%
%
\annoacro{theorem}{EMDRO}{%
The main purpose of elementary matrices is to provide a more formal foundation for row operations.  With this theorem we can convert the notion of ``doing a row operation'' into the slightly more precise, and tractable, operation of matrix multiplication by an elementary matrix.  The other big results in this chapter are made possible by this connection and our previous understanding of the behavior of matrix multiplication (such as results in \acronymref{section}{MM}).
}
%
\annoacro{theorem}{DER}{%
We define the determinant by expansion about the first row and then prove you can expand about any row (and with \acronymref{theorem}{DEC}, about any column).  Amazing.  If the determinant seems contrived, these results might begin to convince you that maybe something interesting is going on.
}
%
\annoacro{theorem}{DRMM}{%
\acronymref{theorem}{EMDRO} connects elementary matrices with matrix multiplication.  Now we connect determinants with matrix multiplication.  If you thought the definition of matrix multiplication (as exemplified by \acronymref{theorem}{EMP}) was as outlandish as the definition of the determinant, then no more.  They seem to play together quite nicely.
}
%
\annoacro{theorem}{SMZD}{%
This theorem provides a simple test for nonsingularity, even though it is stated and titled as a theorem about singularity.  It'll be helpful, especially in concert with \acronymref{theorem}{DRMM}, in establishing upcoming results about nonsingular matrices or creating alternative proofs of earlier results.  You might even use this theorem as an indicator of how often a matrix is singular.  Create a square matrix at random --- what are the odds it is singular?  This theorem says the determinant has to be zero, which we might suspect is a rare occurrence.  Of course, we have to be a lot more careful about words like ``random,'' ``odds,'' and ``rare'' if we want precise answers to this question.
}
%
% End D.tex annotated acronyms
