%%%%(c)
%%%%(c)  This file is a portion of the source for the textbook
%%%%(c)
%%%%(c)    A First Course in Linear Algebra
%%%%(c)    Copyright 2004 by Robert A. Beezer
%%%%(c)
%%%%(c)  See the file COPYING.txt for copying conditions
%%%%(c)
%%%%(c)
\documentclass[10pt,letterpaper]{book}

%%%%%%%%%%
%%
%%  Packages
%%
%%%%%%%%%%

% ifthen
%   Various options are controlled by
%   the  ifthen  package
%
\usepackage{ifthen}

% amsmath, amssymb  packages
%   amsmath has lots of support for matrices and displayed equations
%   so we use it heavily, see amsldoc.pdf for documentation
%   see short-math-guide.pdf for guidance
%   amssymb gets blackboardbold, proof symbols, etc.
%
\usepackage{amsmath,amssymb}

% fcla-developer, fcla-math, archetypes
%   custom stuff just for this book
%   First controls formatting and references
%   Second is for mathematical structures.
%   Third enforces common look on archetype layouts
%
\usepackage{styles/fcla-math}

%% Hyperref package makes URL's in notices live in PDFs
\usepackage[pdftex]{hyperref}
\hypersetup{breaklinks=true,colorlinks=false}
\hypersetup{pdftitle=A First Course in Linear Algebra - Flashcard Supplement}
\hypersetup{pdfauthor=Robert A. Beezer}
\hypersetup{pdfkeywords=free linear algebra textbook flashcard}

% Layout parameters
%
\newlength{\cardheight}\setlength{\cardheight}{4in}
\newlength{\cardwidth}\setlength{\cardwidth}{6in}
\newlength{\cardmargin}\setlength{\cardmargin}{0.1in}
\newlength{\cardborder}\setlength{\cardborder}{2.5pt}
\newlength{\cardsep}\setlength{\cardsep}{1.0in}

% LaTeX box styles
%
\setlength{\fboxsep}{\cardmargin}
\setlength{\fboxrule}{\cardborder}

% LaTeX printable area, plus some vertical slop
%
\setlength{\textwidth}{\cardwidth}
\setlength{\textheight}{2\cardheight}
\addtolength{\textheight}{\cardsep}
\addtolength{\textheight}{10pt}

% Content region size
%
\newlength{\contentwidth}
\setlength{\contentwidth}{\cardwidth}
\addtolength{\contentwidth}{-2\cardmargin}
\addtolength{\contentwidth}{-2\cardborder}
\newlength{\contentheight}
\setlength{\contentheight}{\cardheight}
\addtolength{\contentheight}{-2\cardmargin}
\addtolength{\contentheight}{-2\cardborder}

% LaTeX page style
\pagestyle{empty}
\setlength{\parindent}{0in}

% PDF Page size
% Run a 0.25 inch margin around all sizes
%
\setlength{\pdfpagewidth}{\textwidth}
\addtolength{\pdfpagewidth}{0.50in}
\setlength{\evensidemargin}{-0.75in}
\setlength{\oddsidemargin}{-0.75in}
\setlength{\pdfpageheight}{\textheight}
\addtolength{\pdfpageheight}{0.50in}
\setlength{\topmargin}{-0.75in}
\setlength{\headheight}{0.00in}
\setlength{\headsep}{0.00in}

%  Number cards sequentially
%
\newcounter{cardnumber}

%  Read in standard contents list from fcla.tex
%  Ignore \chap, \appdx, simplify \sect
\newcommand{\sect}[2]{\input{flash/#1.tex}}
\newcommand{\chap}[2]{\relax}
\newcommand{\appdx}[2]{\relax}

%  Some definitions reference other stuff, this just makes OK-looking text, not a link
%  Notation info is just discarded
%  Likewise for \property in vector space axioms
%  \denote  occurs in some definitions, totally ignore it
%
\newcommand{\acronymref}[2]{\MakeUppercase#1 #2}
\newcommand{\property}[3]{{\bf#1\quad#2}}
\newcommand{\denote}[4]{\relax}

% Redefine  theorem  and  definition  environments, basically simpler
% Arguments:   #1 = acronym, #2 = title, #3 = index (ignored)
\newenvironment{definition}[3]
{{\bf Definition #1\quad#2}\hfill{\bf\arabic{cardnumber}}\par\vspace{\stretch{1}}}{}
\newenvironment{theorem}[3]
{{\bf Theorem #1\quad#2}\hfill{\bf\arabic{cardnumber}}\par\vspace{\stretch{1}}}{}


%  \flashcard macro
%
%  Wrapper for theorems and definitions added by sed script
%  Applies frame box around a parbox of "content" size
%  Topline is Type, Acronym, Title, Series Number
%  Base has copyright notice
%  There is a spacer per card
%  
\newcommand{\flashcard}[1]{%
\stepcounter{cardnumber}%
\fbox{\parbox[t][\contentheight][s]{\contentwidth}
{#1\par%
\vspace{\stretch{10}}%  Push down to bottom edge
\ \hspace{\stretch{1}}%  Push out right
\copyright 2005, 2006\quad Robert A.\ Beezer}}%
\vspace{\cardsep}      % Spacing between cards
}

% Customization
%   Copies of this file are processed by scripts
%   Lines here get replaced by sed under program control
\newcommand{\FCLAversion}{DRAFT (\today)}
%% version substitution %%

\begin{document}
%
\begin{titlepage}
\vspace*{\stretch{2}}
\begin{center}
{\fontsize{24}{24}\selectfont  Flash Cards}\\[12pt]
{\large to accompany}
%%%%(c)
%%%%(c)  This file is a portion of the source for the textbook
%%%%(c)
%%%%(c)    A First Course in Linear Algebra
%%%%(c)    Copyright 2004 by Robert A. Beezer
%%%%(c)
%%%%(c)  See the file COPYING.txt for copying conditions
%%%%(c)
%%%%(c)
\begin{center}
\vspace*{\stretch{2}}
{\fontsize{24}{24}\selectfont A First Course in Linear Algebra}
\par
\vspace*{\stretch{2}}
%
{\large by\\Robert A.\ Beezer\\Department of Mathematics and Computer Science\\University of Puget Sound}\par
%
\vspace*{\stretch{1}}
%
%% Hand-edit below for POD versions
%
Version \FCLAversion\par
%
%% Waldron Edition\\
%% Version 2.00 Content\\
%% \today
%
\vspace*{\stretch{1}}
\end{center}

\end{center}
\end{titlepage}
\vspace*{\stretch{2}}
%%%%(c)
%%%%(c)  This file is a portion of the source for the textbook
%%%%(c)
%%%%(c)    A First Course in Linear Algebra
%%%%(c)    Copyright 2004 by Robert A. Beezer
%%%%(c)
%%%%(c)  See the file COPYING.txt for copying conditions
%%%%(c)
%%%%(c)
%
%  URL for GFDL rather than appendix pointer is real difference
%  Requires hyperref package for  \url{}
%  Otherwise consistent with  notice.tex  for main volumes
%
\noindent
\copyright\ 2004  Robert A. Beezer.\\[12pt]
Permission is granted to copy, distribute and/or modify this document under the terms of the GNU Free Documentation License, Version 1.2 or any later version published by the Free Software Foundation; with no Invariant Sections, no Front-Cover Texts, and no Back-Cover Texts.  A copy of the GNU Free Documentation License can be found at \url{http://www.gnu.org/copyleft/fdl.html} and is incorporated here by this reference.\\[12pt]
The most recent version of this work can always be found at \url{http://linear.ups.edu}.\\[24pt]



\vspace*{\stretch{1}}
\clearpage
%
%%%%%%%%%%%%%%%%%%%
%%
%%  Appendix P Preliminaries
%%
%%%%%%%%%%%%%%%%%%%
%%%%%%%%%%%%%%%%%%%
%
\appdx{P}{Preliminaries}
\sect{CNO}{Complex Number Operations}
\sect{SET}{Sets}
%
%%%%%%%%%%%%%%%%%%%
%%%%%%%%%%%%%%%%%%%
%%
%%  Chapter SLE  Systems of Linear Equations
%%
%%%%%%%%%%%%%%%%%%%
%%%%%%%%%%%%%%%%%%%
%
\chap{SLE}{Systems of Linear Equations}
%
\sect{WILA}{What is Linear Algebra?}
\sect{SSLE}{Solving Systems of Linear Equations}
\sect{RREF}{Reduced Row-Echelon Form}
\sect{TSS}{Types of Solution Sets}
\sect{HSE}{Homogeneous Systems of Equations}
\sect{NM}{Nonsingular Matrices}
%
%%%%%%%%%%%%%%%%%%%
%%%%%%%%%%%%%%%%%%%
%%
%%  Chapter V  Vectors
%%
%%%%%%%%%%%%%%%%%%%
%%%%%%%%%%%%%%%%%%%
%
\chap{V}{Vectors}
%
\sect{VO}{Vector Operations}
\sect{LC}{Linear Combinations}
\sect{SS}{Spanning Sets}
\sect{LI}{Linear Independence}
\sect{LDS}{Linear Dependence and Spans}
\sect{O}{Orthogonality}
%
%%%%%%%%%%%%%%%%%%%
%%%%%%%%%%%%%%%%%%%
%%
%%  Chapter M Matrices
%%
%%%%%%%%%%%%%%%%%%%
%%%%%%%%%%%%%%%%%%%
%
\chap{M}{Matrices}
%
\sect{MO}{Matrix Operations}
\sect{MM}{Matrix Multiplication}
\sect{MISLE}{Matrix Inverses and Systems of Linear Equations}
\sect{MINM}{Matrix Inverses and Nonsingular Matrices}
\sect{CRS}{Column and Row Spaces}
\sect{FS}{Four Subsets}
%
%%%%%%%%%%%%%%%%%%%
%%%%%%%%%%%%%%%%%%%
%%
%%  Chapter VS Vector Spaces
%%
%%%%%%%%%%%%%%%%%%%
%%%%%%%%%%%%%%%%%%%
%
\chap{VS}{Vector Spaces}
%
\sect{VS}{Vector Spaces}
\sect{S}{Subspaces}
\sect{LISS}{Linear Independence and Spanning Sets}
\sect{B}{Bases}
\sect{D}{Dimension}
\sect{PD}{Properties of Dimension}
%
%%%%%%%%%%%%%%%%%%%
%%%%%%%%%%%%%%%%%%%
%%
%%  Chapter D Determinants
%%
%%%%%%%%%%%%%%%%%%%
%%%%%%%%%%%%%%%%%%%
%
\chap{D}{Determinants}
%
\sect{DM}{Determinant of a Matrix}
\sect{PDM}{Properties of Determinants of Matrices}
%
%%%%%%%%%%%%%%%%%%%
%%%%%%%%%%%%%%%%%%%
%%
%%  Chapter E  Eigenvalues
%%
%%%%%%%%%%%%%%%%%%%
%%%%%%%%%%%%%%%%%%%
%
\chap{E}{Eigenvalues}
%
\sect{EE}{Eigenvalues and Eigenvectors}
\sect{PEE}{Properties of Eigenvalues and Eigenvectors}
\sect{SD}{Similarity and Diagonalization}
%
%%%%%%%%%%%%%%%%%%%
%%%%%%%%%%%%%%%%%%%
%%
%%  Chapter LT Linear Transformations
%%
%%%%%%%%%%%%%%%%%%%
%%%%%%%%%%%%%%%%%%%
%
\chap{LT}{Linear Transformations}
%
\sect{LT}{Linear Transformations}
\sect{ILT}{Injective Linear Transformations}
\sect{SLT}{Surjective Linear Transformations}
\sect{IVLT}{Invertible Linear Transformations}
%
%%%%%%%%%%%%%%%%%%%
%%%%%%%%%%%%%%%%%%%
%%
%%  Chapter R Representations
%%
%%%%%%%%%%%%%%%%%%%
%%%%%%%%%%%%%%%%%%%
%
\chap{R}{Representations}
%
\sect{VR}{Vector Representations}
\sect{MR}{Matrix Representations}
\sect{CB}{Change of Basis}
\sect{NLT}{Nilpotent Linear Transformations}
\sect{IS}{Invariant Subspaces}
\sect{JCF}{Jordan Canonical Form}
%
%
\end{document}

