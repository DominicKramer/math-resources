%%%%(c)
%%%%(c)  This file is a portion of the source for the textbook
%%%%(c)
%%%%(c)    A First Course in Linear Algebra
%%%%(c)    Copyright 2004 by Robert A. Beezer
%%%%(c)
%%%%(c)  See the file COPYING.txt for copying conditions
%%%%(c)
%%%%(c)
(a)\quad We will use the three criteria of \acronymref{theorem}{TSS}.  The zero vector of $M_{22}$ is the zero matrix, $\zeromatrix$ (\acronymref{definition}{ZM}), which is a symmetric matrix.  So $S_{22}$ is not empty, since $\zeromatrix\in S_{22}$.\par
%
Suppose that $A$ and $B$ are two matrices in $S_{22}$.  Then we know that $\transpose{A}=A$ and $\transpose{B}=B$.  We want to know if $A+B\in S_{22}$, so test $A+B$ for membership,
%
\begin{align*}
\transpose{\left(A+B\right)}&=\transpose{A}+\transpose{B}&&\text{\acronymref{theorem}{TMA}}\\
&=A+B&&A,\,B\in S_{22}
\end{align*}
%
So $A+B$ is symmetric and qualifies for membership in $S_{22}$.\par
%
Suppose that $A\in S_{22}$ and $\alpha\in\complex{\null}$.  Is $\alpha A\in S_{22}$?  We know that $\transpose{A}=A$.  Now check that,
%
\begin{align*}
\transpose{\alpha A}&=\alpha\transpose{A}&&\text{\acronymref{theorem}{TMSM}}\\
&=\alpha A&&A\in S_{22}
\end{align*}
%
So $\alpha A$ is also symmetric and qualifies for membership in $S_{22}$.\par
%
With the three criteria of \acronymref{theorem}{TSS} fulfilled, we see that $S_{22}$ is a subspace of $M_{22}$.\par
%
(b)\quad An arbitrary matrix from $S_{22}$ can be written as 
$\begin{bmatrix}
a&b\\b&d
\end{bmatrix}$.
We can express this matrix as
%
\begin{align*}
%
\begin{bmatrix}
a&b\\b&d
\end{bmatrix}
&=
\begin{bmatrix}
a&0\\0&0
\end{bmatrix}+
%
\begin{bmatrix}
0&b\\b&0
\end{bmatrix}+
%
\begin{bmatrix}
0&0\\0&d
\end{bmatrix}\\
%
&=
a
\begin{bmatrix}
1&0\\0&0
\end{bmatrix}+
b
\begin{bmatrix}
0&1\\1&0
\end{bmatrix}+
d
\begin{bmatrix}
0&0\\0&1
\end{bmatrix}
%
\end{align*}
%
this equation says that the set
%
\begin{equation*}
T=\set{
\begin{bmatrix}
1&0\\0&0
\end{bmatrix},\,
\begin{bmatrix}
0&1\\1&0
\end{bmatrix},\,
\begin{bmatrix}
0&0\\0&1
\end{bmatrix}
}
%
\end{equation*}
%
spans $S_{22}$.  Is it also linearly independent?\par
%
Write a relation of linear dependence on $S$,
%
\begin{align*}
\zeromatrix&=
a_1
\begin{bmatrix}
1&0\\0&0
\end{bmatrix}+
a_2
\begin{bmatrix}
0&1\\1&0
\end{bmatrix}+
a_3
\begin{bmatrix}
0&0\\0&1
\end{bmatrix}\\
%
\begin{bmatrix}
0&0\\0&0
\end{bmatrix}
&=
\begin{bmatrix}
a_1&a_2\\a_2&a_3
\end{bmatrix}
\end{align*}
%
The equality of these two matrices (\acronymref{definition}{ME}) tells us that $a_1=a_2=a_3=0$, and the only relation of linear dependence on $T$ is trivial.  So $T$ is linearly independent, and hence is a basis of $S_{22}$.\par
%
(c)\quad The basis $T$ found in part (b) has size 3.  So by \acronymref{definition}{D}, $\dimension{S_{22}}=3$.
