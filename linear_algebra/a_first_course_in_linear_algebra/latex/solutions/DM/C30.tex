%%%%(c)
%%%%(c)  This file is a portion of the source for the textbook
%%%%(c)
%%%%(c)    A First Course in Linear Algebra
%%%%(c)    Copyright 2004 by Robert A. Beezer
%%%%(c)
%%%%(c)  See the file COPYING.txt for copying conditions
%%%%(c)
%%%%(c)
In order to exploit the zeros, let's expand along row 3.  We then have 
\begin{align*}
\begin{vmatrix}
2 & 3 & 0 & 2 & 1\\
0 & 1 & 1 & 1& 2\\
0 & 0 & 1 & 2 & 0\\ 
1 & 0 & 3 & 1 & 1\\ 
2 & 1 & 1 & 2 & 1
\end{vmatrix}
&= (-1)^6 
\begin{vmatrix}
2 & 1 & 0 & 1 \\ 
2 & 1 & -1 & 1\\ 
1 & 0 & 1 & 1\\
2 & 1 & 2 & 1
\end{vmatrix}
+ (-1)^7 \cdot 
2\begin{vmatrix}
 2 & 1 & 1 & 1\\ 
2 & 1 & 2 & 1\\
1 & 0 & 3 & 1\\
2 & 1 & 1 & 1
\end{vmatrix}\\
\intertext{Notice that the second matrix here is singular since two rows are identical and thus it cannot row-reduce to an identity matrix. We now have}
&=
\begin{vmatrix}
2 & 1 & 0 & 1 \\ 
2 & 1 & -1 & 1\\ 
1 & 0 & 1 & 1\\
2 & 1 & 2 & 1
\end{vmatrix}
+ 0\\
\intertext{and now we expand on the first row of the first matrix:}
&= 2 
\begin{vmatrix} 
1 & -1 & 1\\
0 & 1 & 1\\
1 & 2 & 1 
\end{vmatrix} - 
\begin{vmatrix} 
2 & -1 & 1\\
1 & 1 & 1\\ 
2 & 2 & 1 
\end{vmatrix} + 0 - 
\begin{vmatrix}
2 & 1 & -1 \\ 
1 & 0 & 1\\ 
2 & 1 & 2 
\end{vmatrix}\\
%
&= 2(-3) - (-3) - (-3) = 0
\end{align*}