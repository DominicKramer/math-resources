%%%%(c)
%%%%(c)  This file is a portion of the source for the textbook
%%%%(c)
%%%%(c)    A First Course in Linear Algebra
%%%%(c)    Copyright 2004 by Robert A. Beezer
%%%%(c)
%%%%(c)  See the file COPYING.txt for copying conditions
%%%%(c)
%%%%(c)
A calculator will provide the eigenvalues $\lambda=2,\,2,\,1,\,0$, so we can reconstruct the characteristic polynomial as
%
\begin{equation*}
\charpoly{A}{x}=(x-2)^2(x-1)x
\end{equation*}
%
so the algebraic multiplicities of the eigenvalues are
%
\begin{align*}
\algmult{A}{2}&=2&
\algmult{A}{1}&=1&
\algmult{A}{0}&=1
\end{align*}
%
Now compute eigenspaces by hand, obtaining null spaces for each of the three eigenvalues by constructing the correct singular matrix (\acronymref{theorem}{EMNS}),
%
\begin{align*}
A-2I_4&=
\begin{bmatrix}
 -1 & 9 & 9 & 24 \\
 -3 & -29 & -29 & -68 \\
 1 & 11 & 11 & 26 \\
 1 & 7 & 7 & 16
\end{bmatrix}
\rref
\begin{bmatrix}
 1 & 0 & 0 & -\frac{3}{2} \\
 0 & 1 & 1 & \frac{5}{2} \\
 0 & 0 & 0 & 0 \\
 0 & 0 & 0 & 0
\end{bmatrix}\\
%
\eigenspace{A}{2}&=\nsp{A-2I_4}
=\spn{\set{\colvector{\frac{3}{2}\\-\frac{5}{2}\\0\\1},\,\colvector{0\\-1\\1\\0}}}
=\spn{\set{\colvector{3\\-5\\0\\2},\,\colvector{0\\-1\\1\\0}}}\\
%
A-1I_4&=
\begin{bmatrix}
0 & 9 & 9 & 24 \\
 -3 & -28 & -29 & -68 \\
 1 & 11 & 12 & 26 \\
 1 & 7 & 7 & 17
\end{bmatrix}
\rref
\begin{bmatrix}
 1 & 0 & 0 & -\frac{5}{3} \\
 0 & 1 & 0 & \frac{13}{3} \\
 0 & 0 & 1 & -\frac{5}{3} \\
 0 & 0 & 0 & 0
\end{bmatrix}\\
%
\eigenspace{A}{1}&=\nsp{A-I_4}
=\spn{\set{\colvector{\frac{5}{3}\\-\frac{13}{3}\\\frac{5}{3}\\1}}}
=\spn{\set{\colvector{5\\-13\\5\\3}}}\\
%
A-0I_4&=
\begin{bmatrix}
 1 & 9 & 9 & 24 \\
 -3 & -27 & -29 & -68 \\
 1 & 11 & 13 & 26 \\
 1 & 7 & 7 & 18
\end{bmatrix}
\rref
\begin{bmatrix}
 1 & 0 & 0 & -3 \\
 0 & 1 & 0 & 5 \\
 0 & 0 & 1 & -2 \\
 0 & 0 & 0 & 0
\end{bmatrix}\\
%
\eigenspace{A}{0}&=\nsp{A-I_4}=\spn{\set{\colvector{3\\-5\\2\\1}}}
%
\end{align*}
%
From this we can compute the dimensions of the eigenspaces to obtain the geometric multiplicities,
%
\begin{align*}
\geomult{A}{2}&=2&
\geomult{A}{1}&=1&
\geomult{A}{0}&=1
\end{align*}
%
For each eigenvalue, the algebraic and geometric multiplicities are equal and so by \acronymref{theorem}{DMFE} we now know that $A$ is diagonalizable.  The construction in \acronymref{theorem}{DC} suggests we form a matrix whose columns are eigenvectors of $A$
%
\begin{equation*}
S=
\begin{bmatrix}
 3 & 0 & 5 & 3 \\
 -5 & -1 & -13 & -5 \\
 0 & 1 & 5 & 2 \\
 2 & 0 & 3 & 1
\end{bmatrix}
\end{equation*}
%
Since $\detname{S}=-1\neq 0$, we know that $S$ is nonsingular (\acronymref{theorem}{SMZD}), so the columns of $S$ are a set of 4 linearly independent eigenvectors of $A$.  By the proof of \acronymref{theorem}{SMZD} we know
%
\begin{equation*}
\similar{A}{S}=
\begin{bmatrix}
 2 & 0 & 0 & 0 \\
 0 & 2 & 0 & 0 \\
 0 & 0 & 1 & 0 \\
 0 & 0 & 0 & 0
\end{bmatrix}
\end{equation*}
%
a diagonal matrix with the eigenvalues of $A$ along the diagonal, in the same order as the associated eigenvectors appear as columns of $S$.