%%%%(c)
%%%%(c)  This file is a portion of the source for the textbook
%%%%(c)
%%%%(c)    A First Course in Linear Algebra
%%%%(c)    Copyright 2004 by Robert A. Beezer
%%%%(c)
%%%%(c)  See the file COPYING.txt for copying conditions
%%%%(c)
%%%%(c)
Take 
%
\begin{align*}
A&=
\begin{bmatrix}
1&0\\0&0
\end{bmatrix}
&
B&=
\begin{bmatrix}
0&0\\1&0
\end{bmatrix}
\end{align*}
%
Then $A$ is already in reduced row-echelon form, and by swapping rows, $B$ row-reduces to $A$.  So the product of the row-echelon forms of $A$ is $AA=A\neq\zeromatrix$.  However, the product $AB$ is the $2\times 2$ zero matrix, which is in reduced-echelon form, and not equal to $AA$.  When you get there, \acronymref{theorem}{PEEF} or \acronymref{theorem}{EMDRO} might shed some light on why we would not expect this statement to be true in general.\par
%
If $A$ and $B$ are nonsingular, then $AB$ is nonsingular (\acronymref{theorem}{NPNT}), and all three matrices $A$, $B$ and $AB$ row-reduce to the identity matrix (\acronymref{theorem}{NMRRI}).  By \acronymref{theorem}{MMIM}, the desired relationship is true.