%%%%(c)
%%%%(c)  This file is a portion of the source for the textbook
%%%%(c)
%%%%(c)    A First Course in Linear Algebra
%%%%(c)    Copyright 2004 by Robert A. Beezer
%%%%(c)
%%%%(c)  See the file COPYING.txt for copying conditions
%%%%(c)
%%%%(c)
Choose bases $B$ and $C$ for the matrix representation,
\begin{align*}
B&=\set{1,\,x,\,x^2,\,x^3}
&
C&=\set{
\begin{bmatrix}1 & 0 \\ 0 & 0\end{bmatrix},\,
\begin{bmatrix}0 & 1 \\ 0 & 0\end{bmatrix},\,
\begin{bmatrix}0 & 0 \\ 1 & 0\end{bmatrix},\,
\begin{bmatrix}0 & 0 \\ 0 & 1\end{bmatrix}
}\\
%
\end{align*}
%
Input to $T$ the vectors of the basis $B$ and coordinatize the outputs relative to $C$,
%
\begin{align*}
\vectrep{C}{\lt{T}{1}}&=
\vectrep{C}{\begin{bmatrix}-1 & 4 \\ 1 & 1\end{bmatrix}}
=
\vectrep{C}{
(-1)\begin{bmatrix}1 & 0 \\ 0 & 0\end{bmatrix}+
4\begin{bmatrix}0 & 1 \\ 0 & 0\end{bmatrix}+
1\begin{bmatrix}0 & 0 \\ 1 & 0\end{bmatrix}+
1\begin{bmatrix}0 & 0 \\ 0 & 1\end{bmatrix}
}
=
\colvector{-1\\4\\1\\1}\\
%
\vectrep{C}{\lt{T}{x}}&=
\vectrep{C}{\begin{bmatrix}4 & -1 \\ 5 & 0\end{bmatrix}}
=
\vectrep{C}{
4\begin{bmatrix}1 & 0 \\ 0 & 0\end{bmatrix}+
(-1)\begin{bmatrix}0 & 1 \\ 0 & 0\end{bmatrix}+
5\begin{bmatrix}0 & 0 \\ 1 & 0\end{bmatrix}+
0\begin{bmatrix}0 & 0 \\ 0 & 1\end{bmatrix}
}
=
\colvector{4\\-1\\5\\0}\\
%
\vectrep{C}{\lt{T}{x^2}}&=
\vectrep{C}{\begin{bmatrix}1 & 6 \\ -2 & 2\end{bmatrix}}
=
\vectrep{C}{
1\begin{bmatrix}1 & 0 \\ 0 & 0\end{bmatrix}+
6\begin{bmatrix}0 & 1 \\ 0 & 0\end{bmatrix}+
(-2)\begin{bmatrix}0 & 0 \\ 1 & 0\end{bmatrix}+
2\begin{bmatrix}0 & 0 \\ 0 & 1\end{bmatrix}
}
=
\colvector{1\\6\\-2\\2}\\
%
\vectrep{C}{\lt{T}{x^3}}&=
\vectrep{C}{\begin{bmatrix}2 & -1 \\ 2 & 5\end{bmatrix}}
=
\vectrep{C}{
2\begin{bmatrix}1 & 0 \\ 0 & 0\end{bmatrix}+
(-1)\begin{bmatrix}0 & 1 \\ 0 & 0\end{bmatrix}+
2\begin{bmatrix}0 & 0 \\ 1 & 0\end{bmatrix}+
5\begin{bmatrix}0 & 0 \\ 0 & 1\end{bmatrix}
}
=
\colvector{2\\-1\\2\\5}
\end{align*}
%
Applying \acronymref{definition}{MR} we have the matrix representation
%
\begin{equation*}
\matrixrep{T}{B}{C}=
\begin{bmatrix}
-1 & 4 & 1 & 2 \\
 4 & -1 & 6 & -1 \\
 1 & 5 & -2 & 2 \\
 1 & 0 & 2 & 5
\end{bmatrix}
\end{equation*}
%
Properties of this matrix representation will translate to properties of the linear transformation The matrix representation is nonsingular since it row-reduces to the identity matrix (\acronymref{theorem}{NMRRI}) and therefore has a column space equal to $\complex{4}$ (\acronymref{theorem}{CNMB}).  The column space of the matrix representation is isomorphic to the range of the linear transformation (\acronymref{theorem}{RCSI}).  So the range of $T$ has dimension 4, equal to the dimension of the codomain $M_{22}$.  By \acronymref{theorem}{ROSLT}, $T$ is surjective.