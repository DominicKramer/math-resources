%%%%(c)
%%%%(c)  This file is a portion of the source for the textbook
%%%%(c)
%%%%(c)    A First Course in Linear Algebra
%%%%(c)    Copyright 2004 by Robert A. Beezer
%%%%(c)
%%%%(c)  See the file COPYING.txt for copying conditions
%%%%(c)
%%%%(c)
We row-reduce the augmented matrix of the system of equations,
%
\begin{align*}
\begin{bmatrix}
 1 & 2 & -4 & -1 & 0 & 32 \\
 1 & 3 & -7 & 0 & -1 & 33 \\
 1 & 0 & 2 & -2 & 3 & 22
\end{bmatrix}
&\rref
\begin{bmatrix}
 \leading{1} & 0 & 2 & 0 & 5 & 6 \\
 0 & \leading{1} & -3 & 0 & -2 & 9 \\
 0 & 0 & 0 & \leading{1} & 1 & -8
\end{bmatrix}
\end{align*}
%
In the spirit of \acronymref{example}{SAA}, we can express the infinitely many solutions of this system compactly with set notation.  The key is to express certain variables in terms of others.  More specifically, each pivot column number is the index of a variable that can be written in terms of the variables whose indices are non-pivot columns.  Or saying the same thing: for each $i$ in $D$, we can find an expression for $x_i$ in terms of the variables without their index in $D$.  Here $D=\set{1,\,2,\,4}$, so
%
\begin{align*}
x_1+2x_3+5x_5=6
\quad&\rightarrow\quad x_1=6-2x_3-5x_5\\
%
x_2-3x_3-2x_5=9
\quad&\rightarrow\quad x_2=9+3x_3+2x_5\\
%
x_4+x_5=-8\quad&\rightarrow\quad x_4=-8-x_5
%
\end{align*}
%
As a set, we write the solutions precisely as
%
\begin{align*}
S&=\setparts{\colvector{6-2x_3-5x_5\\9+3x_3+2x_5\\x_3\\-8-x_5\\x_5}}{x_3,\,x_5\in\complexes}
\end{align*}
%
