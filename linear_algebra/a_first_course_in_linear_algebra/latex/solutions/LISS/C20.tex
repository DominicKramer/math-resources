%%%%(c)
%%%%(c)  This file is a portion of the source for the textbook
%%%%(c)
%%%%(c)    A First Course in Linear Algebra
%%%%(c)    Copyright 2004 by Robert A. Beezer
%%%%(c)
%%%%(c)  See the file COPYING.txt for copying conditions
%%%%(c)
%%%%(c)
Begin with a relation of linear dependence on the vectors in $S$ and massage it according to the definitions of vector addition and scalar multiplication in $M_{22}$,
%
\begin{align*}
\zeromatrix&=
a_1
\begin{bmatrix}
2&-1\\1&3
\end{bmatrix}+
a_2
\begin{bmatrix}
0&4\\-1&2
\end{bmatrix}+
a_3
\begin{bmatrix}
4&2\\1&3
\end{bmatrix}\\
%
\begin{bmatrix}
0&0\\0&0
\end{bmatrix}
&=
\begin{bmatrix}
2a_1+4a_3&
-a_1+4a_2+2a_3\\
a_1-a_2+a_3&
3a_1+2a_2+3a_3
\end{bmatrix}
%
\end{align*}
%
By our definition of matrix equality (\acronymref{definition}{ME}) we arrive at a homogeneous system of linear equations,
%
\begin{align*}
2a_1+4a_3&=0\\
-a_1+4a_2+2a_3&=0\\
a_1-a_2+a_3&=0\\
3a_1+2a_2+3a_3&=0
\end{align*}
%
The coefficient matrix of this system row-reduces to the matrix,
%
\begin{equation*}
\begin{bmatrix}
\leading{1} & 0 & 0\\
0 & \leading{1} & 0\\
0 & 0 & \leading{1}\\
0 & 0 & 0
\end{bmatrix}
\end{equation*}
%
and from this we conclude that the only solution is $a_1=a_2=a_3=0$.  Since the relation of linear dependence (\acronymref{definition}{RLD}) is trivial, the set $S$ is linearly independent (\acronymref{definition}{LI}).