%%%%(c)
%%%%(c)  This file is a portion of the source for the textbook
%%%%(c)
%%%%(c)    A First Course in Linear Algebra
%%%%(c)    Copyright 2004 by Robert A. Beezer
%%%%(c)
%%%%(c)  See the file COPYING.txt for copying conditions
%%%%(c)
%%%%(c)
We begin with a relation of linear dependence using unknown scalars $a$ and $b$.  We wish to know if these scalars {\em must} both be zero.  Recall that the zero vector in $C$ is $(-1,\,-1)$ and that the definitions of vector addition and scalar multiplication are not what we might expect.
%
\begin{align*}
\zerovector
&=(-1,\,-1)\\
&=a(0,\,2) +b(2,\,8)&&\text{\acronymref{definition}{RLD}}\\
&=(0a+a-1,\,2a+a-1) + (2b+b-1,\,8b+b-1)&&\text{Scalar mult., \acronymref{example}{CVS}}\\
&=(a-1,\,3a-1) + (3b-1,\,9b-1)\\
&=(a-1+3b-1+1,\,3a-1+9b-1+1)&&\text{Vector addition, \acronymref{example}{CVS}}\\
&=(a+3b-1,\,3a+9b-1)\\
\end{align*}
%
From this we obtain two equalities, which can be converted to a homogeneous system of equations,
%
\begin{align*}
-1&=a+3b-1&a+3b&=0\\
-1&=3a+9b-1&3a+9b&=0
\end{align*}
%
This homogeneous system has a singular coefficient matrix (\acronymref{theorem}{SMZD}), and so has more than just the trivial solution (\acronymref{definition}{NM}).  Any nontrivial solution will give us a nontrivial relation of linear dependence on $S$.  So $S$ is linearly dependent (\acronymref{definition}{LI}).