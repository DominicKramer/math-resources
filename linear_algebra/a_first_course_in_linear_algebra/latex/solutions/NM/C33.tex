%%%%(c)
%%%%(c)  This file is a portion of the source for the textbook
%%%%(c)
%%%%(c)    A First Course in Linear Algebra
%%%%(c)    Copyright 2004 by Robert A. Beezer
%%%%(c)
%%%%(c)  See the file COPYING.txt for copying conditions
%%%%(c)
%%%%(c)
\acronymref{theorem}{NMRRI} tells us we can answer this question by simply row-reducing the matrix.  Doing this we obtain,
%
\begin{equation*}
\begin{bmatrix}
 \leading{1} & 0 & 0 & 0 \\
 0 & \leading{1} & 0 & 0 \\
 0 & 0 & \leading{1} & 0 \\
 0 & 0 & 0 & \leading{1}
\end{bmatrix}
\end{equation*}
%
Since the reduced row-echelon form of the matrix is the $4\times 4$ identity matrix $I_4$, we know that $B$ is nonsingular.
