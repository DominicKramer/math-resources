%%%%(c)
%%%%(c)  This file is a portion of the source for the textbook
%%%%(c)
%%%%(c)    A First Course in Linear Algebra
%%%%(c)    Copyright 2004 by Robert A. Beezer
%%%%(c)
%%%%(c)  See the file COPYING.txt for copying conditions
%%%%(c)
%%%%(c)
Let $n$ denote the size of the square matrix $A$.  By \acronymref{theorem}{NMRRI} the hypothesis that $A$ is singular implies that $B$ is not the identity matrix $I_n$.  If $B$ has $n$ pivot columns, then it would have to be $I_n$, so $B$ must have fewer than $n$ pivot columns.  But the number of nonzero rows in $B$ ($r$) is equal to the number of pivot columns as well.  So the $n$ rows of $B$ have fewer than $n$ nonzero rows, and $B$ must contain at least one zero row.  By \acronymref{definition}{RREF}, this row must be at the bottom of $B$.\par
%
A proof can also be formulated by first forming the contrapositive of the statement (\acronymref{technique}{CP}) and proving this statement.
