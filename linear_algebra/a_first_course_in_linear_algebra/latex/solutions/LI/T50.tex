%%%%(c)
%%%%(c)  This file is a portion of the source for the textbook
%%%%(c)
%%%%(c)    A First Course in Linear Algebra
%%%%(c)    Copyright 2004 by Robert A. Beezer
%%%%(c)
%%%%(c)  See the file COPYING.txt for copying conditions
%%%%(c)
%%%%(c)
Let $A=\matrixcolumns{A}{n}$.  $\linearsystem{A}{\vect{b}}$ is consistent, so we know the system has at least one solution (\acronymref{definition}{CS}).  We would like to show that there are no more than one solution to the system.  Employing \acronymref{technique}{U}, suppose that $\vect{x}$ and $\vect{y}$ are two solution vectors for $\linearsystem{A}{\vect{b}}$.  By \acronymref{theorem}{SLSLC} we know we can write,
%
\begin{align*}
\vect{b}&=
\vectorentry{\vect{x}}{1}A_1+
\vectorentry{\vect{x}}{2}A_2+
\vectorentry{\vect{x}}{3}A_3+
\cdots+
\vectorentry{\vect{x}}{n}A_n\\
%
\vect{b}&=
\vectorentry{\vect{y}}{1}A_1+
\vectorentry{\vect{y}}{2}A_2+
\vectorentry{\vect{y}}{3}A_3+
\cdots+
\vectorentry{\vect{y}}{n}A_n
\end{align*}
%
Then
%
\begin{align*}
\zerovector
&=\vect{b}-\vect{b}\\
&=
\left(
\vectorentry{\vect{x}}{1}A_1+
\vectorentry{\vect{x}}{2}A_2+
\cdots+
\vectorentry{\vect{x}}{n}A_n
\right)
-
\left(
\vectorentry{\vect{y}}{1}A_1+
\vectorentry{\vect{y}}{2}A_2+
\cdots+
\vectorentry{\vect{y}}{n}A_n
\right)\\
&=
\left(\vectorentry{\vect{x}}{1}-\vectorentry{\vect{y}}{1}\right)A_1+
\left(\vectorentry{\vect{x}}{2}-\vectorentry{\vect{y}}{2}\right)A_2+
\cdots+
\left(\vectorentry{\vect{x}}{n}-\vectorentry{\vect{y}}{n}\right)A_n
\end{align*}
%
This is a relation of linear dependence (\acronymref{definition}{RLDCV}) on a linearly independent set (the columns of $A$).  So the scalars {\em must} all be zero,
%
\begin{align*}
\vectorentry{\vect{x}}{1}-\vectorentry{\vect{y}}{1}&=0
&
\vectorentry{\vect{x}}{2}-\vectorentry{\vect{y}}{2}&=0
&
\dots&
&
\vectorentry{\vect{x}}{n}-\vectorentry{\vect{y}}{n}&=0
\end{align*}
%
Rearranging these equations yields the statement that $\vectorentry{\vect{x}}{i}=\vectorentry{\vect{y}}{i}$, for $1\leq i\leq n$.  However, this is exactly how we define vector equality (\acronymref{definition}{CVE}), so $\vect{x}=\vect{y}$ and the system has only one solution.