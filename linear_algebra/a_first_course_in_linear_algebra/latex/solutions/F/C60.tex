%%%%(c)
%%%%(c)  This file is a portion of the source for the textbook
%%%%(c)
%%%%(c)    A First Course in Linear Algebra
%%%%(c)    Copyright 2004 by Robert A. Beezer
%%%%(c)
%%%%(c)  See the file COPYING.txt for copying conditions
%%%%(c)
%%%%(c)
Remember that every computation must be done with arithmetic in the field, reducing any intermediate number outside of $\set{0,\,1,\,2,\,3,\,4}$ to its remainder after division by 5.\par
%
The matrix inverse can be found with \acronymref{theorem}{CINM} (and we discover along the way that $A$ is nonsingular).  The inverse is
%
\begin{align*}
\inverse{A}&=
\begin{bmatrix}
 1 & 1 & 3 & 1 \\
 3 & 4 & 1 & 4 \\
 1 & 4 & 0 & 2 \\
 3 & 0 & 1 & 0
\end{bmatrix}
\end{align*}
%
Then by an application of \acronymref{theorem}{SNCM} the (unique) solution to the system will be
%
\begin{align*}
\inverse{A}\vect{b}
&=
\begin{bmatrix}
 1 & 1 & 3 & 1 \\
 3 & 4 & 1 & 4 \\
 1 & 4 & 0 & 2 \\
 3 & 0 & 1 & 0
\end{bmatrix}
\colvector{3\\3\\2\\0}
&=
\colvector{2\\3\\0\\1}
\end{align*}
%
