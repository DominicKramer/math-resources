%%%%(c)
%%%%(c)  This file is a portion of the source for the textbook
%%%%(c)
%%%%(c)    A First Course in Linear Algebra
%%%%(c)    Copyright 2004 by Robert A. Beezer
%%%%(c)
%%%%(c)  See the file COPYING.txt for copying conditions
%%%%(c)
%%%%(c)
Since any symmetric matrix is of the form
\begin{align*}
\begin{bmatrix} a & b & c \\ b & d & e \\ c & e & f \end{bmatrix} 
&= 
\begin{bmatrix} a & 0 & 0 \\ 0 & 0 & 0 \\ 0 & 0 & 0 \end{bmatrix} + 
\begin{bmatrix} 0 & b & 0 \\ b & 0 & 0 \\ 0 & 0 & 0 \end{bmatrix} + 
\begin{bmatrix} 0 & 0 & c \\ 0 & 0 & 0 \\ c & 0 & 0 \end{bmatrix} + 
\begin{bmatrix} 0 & 0 & 0 \\ 0 & d & 0 \\ 0 & 0 & 0 \end{bmatrix} + 
\begin{bmatrix} 0 & 0 & 0 \\ 0 & 0 & e \\ 0 & e & 0 \end{bmatrix} + 
\begin{bmatrix} 0 & 0 & 0 \\ 0 & 0 & 0 \\ 0 & 0 & f \end{bmatrix},
\end{align*}
Any symmetric matrix is a linear combination of the linearly independent vectors in set $T$ below, so that $\spn{T} = Y$:
\begin{equation*}
T = \set{
\begin{bmatrix} 1 & 0 & 0 \\ 0 & 0 & 0 \\ 0 & 0 & 0 \end{bmatrix},
\begin{bmatrix} 0 & 1 & 0 \\ 1 & 0 & 0 \\ 0 & 0 & 0 \end{bmatrix},
\begin{bmatrix} 0 & 0 & 1 \\ 0 & 0 & 0 \\ 1 & 0 & 0 \end{bmatrix}, 
\begin{bmatrix} 0 & 0 & 0 \\ 0 & 1 & 0 \\ 0 & 0 & 0 \end{bmatrix}, 
\begin{bmatrix} 0 & 0 & 0 \\ 0 & 0 & 1 \\ 0 & 1 & 0 \end{bmatrix},
\begin{bmatrix} 0 & 0 & 0 \\ 0 & 0 & 0 \\ 0 & 0 & 1 \end{bmatrix}
}
\end{equation*}
(Something to think about:  How do we know that these matrices are linearly independent?)
