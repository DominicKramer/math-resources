%%%%(c)
%%%%(c)  This file is a portion of the source for the textbook
%%%%(c)
%%%%(c)    A First Course in Linear Algebra
%%%%(c)    Copyright 2004 by Robert A. Beezer
%%%%(c)
%%%%(c)  See the file COPYING.txt for copying conditions
%%%%(c)
%%%%(c)
This is an equality of sets, so we want to establish two subset conditions (\acronymref{definition}{SE}).\par
%
First, show $\csp{A}\subseteq\rng{T}$.  Choose $\vect{y}\in\csp{A}$.  Then by \acronymref{definition}{CSM} and \acronymref{definition}{MVP} there is a vector $\vect{x}\in\complex{n}$ such that $A\vect{x}=\vect{y}$.  Then
%
\begin{align*}
\lt{T}{\vect{x}}
&=A\vect{x}&&\text{Definition of $T$}\\
&=\vect{y}
\end{align*}
%
This statement qualifies $\vect{y}$ as a member of $\rng{T}$ (\acronymref{definition}{RLT}), so $\csp{A}\subseteq\rng{T}$.\par
%
Now, show $\rng{T}\subseteq\csp{A}$.  Choose $\vect{y}\in\rng{T}$.  Then by \acronymref{definition}{RLT}, there is a vector $\vect{x}$ in $\complex{n}$ such that $\lt{T}{\vect{x}}=\vect{y}$.  Then
%
\begin{align*}
A\vect{x}
&=\lt{T}{\vect{x}}&&\text{Definition of $T$}\\
&=\vect{y}
\end{align*}
%
So by \acronymref{definition}{CSM} and \acronymref{definition}{MVP}, $\vect{y}$ qualifies for membership in $\csp{A}$ and so $\rng{T}\subseteq\csp{A}$.\par
