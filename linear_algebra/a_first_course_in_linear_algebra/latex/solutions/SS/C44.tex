%%%%(c)
%%%%(c)  This file is a portion of the source for the textbook
%%%%(c)
%%%%(c)    A First Course in Linear Algebra
%%%%(c)    Copyright 2004 by Robert A. Beezer
%%%%(c)
%%%%(c)  See the file COPYING.txt for copying conditions
%%%%(c)
%%%%(c)
Form a linear combination, with unknown scalars, of $S$ that equals $\vect{y}$,
%
\begin{equation*}
a_1\colvector{-1 \\ 2 \\ 1}+
a_2\colvector{ 3 \\ 1 \\ 2}+
a_3\colvector{ 1 \\ 5 \\ 4}+
a_4\colvector{-6 \\ 5 \\ 1}
=
\colvector{-5\\3\\0}
\end{equation*}
%
We want to know if there are values for the scalars that make the vector equation true since that is the definition of membership in $\spn{S}$.  By \acronymref{theorem}{SLSLC} any such values will also be solutions to the linear system represented by the augmented matrix,
%
\begin{equation*}
\begin{bmatrix}
 -1 & 3 & 1 & -6 & -5 \\
 2 & 1 & 5 & 5 & 3 \\
 1 & 2 & 4 & 1 & 0
\end{bmatrix}
\end{equation*}
%
Row-reducing the matrix yields,
%
\begin{equation*}
\begin{bmatrix}
 \leading{1} & 0 & 2 & 3 & 2 \\
 0 & \leading{1} & 1 & -1 & -1 \\
 0 & 0 & 0 & 0 & 0
\end{bmatrix}
\end{equation*}
%
From this we  see that the system of equations is consistent (\acronymref{theorem}{RCLS}), and has a infinitely many solutions.  Any solution will provide a linear combination of the vectors in $R$ that equals $\vect{y}$. So $\vect{y}\in S$, for example,
%
\begin{equation*}
(-10)\colvector{-1 \\ 2 \\ 1}+
(-2)\colvector{ 3 \\ 1 \\ 2}+
(3)\colvector{ 1 \\ 5 \\ 4}+
(2)\colvector{-6 \\ 5 \\ 1}
=
\colvector{-5\\3\\0}
\end{equation*}