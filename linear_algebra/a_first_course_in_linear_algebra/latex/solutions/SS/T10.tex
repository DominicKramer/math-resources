%%%%(c)
%%%%(c)  This file is a portion of the source for the textbook
%%%%(c)
%%%%(c)    A First Course in Linear Algebra
%%%%(c)    Copyright 2004 by Robert A. Beezer
%%%%(c)
%%%%(c)  See the file COPYING.txt for copying conditions
%%%%(c)
%%%%(c)
This is an equality of sets, so \acronymref{definition}{SE} applies.\par
%
First show that $X=\spn{\set{\vect{v}_1,\,\vect{v}_2}}\subseteq
\spn{\set{\vect{v}_1,\,\vect{v}_2,\,5\vect{v}_1+3\vect{v}_2}}=Y$.\\
%
Choose $\vect{x}\in X$.  Then $\vect{x}=a_1\vect{v}_1+a_2\vect{v}_2$ for some scalars $a_1$ and $a_2$.  Then,
%
\begin{equation*}
\vect{x}=a_1\vect{v}_1+a_2\vect{v}_2=a_1\vect{v}_1+a_2\vect{v}_2+0(5\vect{v}_1+3\vect{v}_2)
\end{equation*}
%
which qualifies $\vect{x}$ for membership in $Y$, as it is a linear combination of $\vect{v}_1,\,\vect{v}_2,\,5\vect{v}_1+3\vect{v}_2$.\par
%
Now show the opposite inclusion, $Y=\spn{\set{\vect{v}_1,\,\vect{v}_2,\,5\vect{v}_1+3\vect{v}_2}}\subseteq\spn{\set{\vect{v}_1,\,\vect{v}_2}}=X$.\\
%
Choose $\vect{y}\in Y$.  Then there are scalars $a_1,\,a_2,\,a_3$ such that
%
\begin{equation*}
\vect{y}=a_1\vect{v}_1+a_2\vect{v}_2+a_3(5\vect{v}_1+3\vect{v}_2)
\end{equation*}
%
Rearranging, we obtain,
%
\begin{align*}
\vect{y}&=a_1\vect{v}_1+a_2\vect{v}_2+a_3(5\vect{v}_1+3\vect{v}_2)\\
%
&=a_1\vect{v}_1+a_2\vect{v}_2+5a_3\vect{v}_1+3a_3\vect{v}_2&&\text{\acronymref{property}{DVAC}}\\
%
&=a_1\vect{v}_1+5a_3\vect{v}_1+a_2\vect{v}_2+3a_3\vect{v}_2&&\text{\acronymref{property}{CC}}\\
%
&=(a_1+5a_3)\vect{v}_1+(a_2+3a_3)\vect{v}_2&&\text{\acronymref{property}{DSAC}}\\
%
\end{align*}
%
This is an expression for $\vect{y}$ as a linear combination of $\vect{v}_1$ and $\vect{v}_2$, earning $\vect{y}$ membership in $X$.
%
Since $X$ is a subset of $Y$, and vice versa, we see that $X=Y$, as desired.

