%%%%(c)
%%%%(c)  This file is a portion of the source for the textbook
%%%%(c)
%%%%(c)    A First Course in Linear Algebra
%%%%(c)    Copyright 2004 by Robert A. Beezer
%%%%(c)
%%%%(c)  See the file COPYING.txt for copying conditions
%%%%(c)
%%%%(c)
Begin with a matrix representation of $R$, any matrix representation, but use the same basis for both instances of $S_{22}$.  We'll choose a basis that makes it easy to compute vector representations in $S_{22}$.
%
\begin{equation*}
B=\set{
\begin{bmatrix} 1 & 0 \\ 0 & 0 \end{bmatrix},\,
\begin{bmatrix} 0 & 1 \\ 1 & 0 \end{bmatrix},\,
\begin{bmatrix} 0 & 0 \\ 0 & 1 \end{bmatrix}
}
\end{equation*}
%
Then the resulting matrix representation of $R$  (\acronymref{definition}{MR}) is
%
\begin{equation*}
\matrixrep{R}{B}{B}=
\begin{bmatrix}
 -5 & 2 & -3 \\
 -12 & 5 & -6 \\
 6 & -2 & 4
\end{bmatrix}
\end{equation*}
%
Now, compute the eigenvalues and eigenvectors of this matrix, with the goal of diagonalizing the matrix (\acronymref{theorem}{DC}),
%
\begin{align*}
\lambda&=2
&
\eigenspace{\matrixrep{R}{B}{B}}{2}&=\spn{\set{\colvector{-1\\-2\\1}}}\\
%
\lambda&=1
&
\eigenspace{\matrixrep{R}{B}{B}}{1}&=\spn{\set{\colvector{-1\\0\\2},\,\colvector{1\\3\\0}}}\\
\end{align*}
%
The three vectors that occur as basis elements for these eigenspaces will together form a linearly independent set (check this!).  So these column vectors may be employed in a matrix that will diagonalize the matrix representation.  If we ``un-coordinatize'' these three column vectors relative to the basis $B$, we will find three linearly independent elements of $S_{22}$ that are eigenvectors of the linear transformation $R$ (\acronymref{theorem}{EER}).  A matrix representation relative to this basis of eigenvectors will be diagonal, with the eigenvalues ($\lambda=2,\,1$) as the diagonal elements.  Here we go,
%
\begin{align*}
\vectrepinv{B}{\colvector{-1\\-2\\1}}&=
(-1)\begin{bmatrix} 1 & 0 \\ 0 & 0 \end{bmatrix}+
(-2)\begin{bmatrix} 0 & 1 \\ 1 & 0 \end{bmatrix}+
1\begin{bmatrix} 0 & 0 \\ 0 & 1 \end{bmatrix}
=
\begin{bmatrix}
-1 & -2 \\-2 & 1
\end{bmatrix}\\
%
\vectrepinv{B}{\colvector{-1\\0\\2}}&=
(-1)\begin{bmatrix} 1 & 0 \\ 0 & 0 \end{bmatrix}+
0\begin{bmatrix} 0 & 1 \\ 1 & 0 \end{bmatrix}+
2\begin{bmatrix} 0 & 0 \\ 0 & 1 \end{bmatrix}
=
\begin{bmatrix}
-1 & 0 \\ 0 & 2
\end{bmatrix}\\
%
\vectrepinv{B}{\colvector{1\\3\\0}}&=
1\begin{bmatrix} 1 & 0 \\ 0 & 0 \end{bmatrix}+
3\begin{bmatrix} 0 & 1 \\ 1 & 0 \end{bmatrix}+
0\begin{bmatrix} 0 & 0 \\ 0 & 1 \end{bmatrix}
=
\begin{bmatrix}
1 & 3 \\ 3 & 0
\end{bmatrix}\\
%
\end{align*}
%
So the requested basis of $S_{22}$, yielding a diagonal matrix representation of $R$, is
%
\begin{equation*}
\set{
\begin{bmatrix}
-1 & -2 \\-2 & 1
\end{bmatrix}\,\
%
\begin{bmatrix}
-1 & 0 \\ 0 & 2
\end{bmatrix},\,
%
\begin{bmatrix}
1 & 3 \\ 3 & 0
\end{bmatrix}%
}
\end{equation*}