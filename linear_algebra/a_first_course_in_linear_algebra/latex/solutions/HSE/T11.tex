%%%%(c)
%%%%(c)  This file is a portion of the source for the textbook
%%%%(c)
%%%%(c)    A First Course in Linear Algebra
%%%%(c)    Copyright 2004 by Robert A. Beezer
%%%%(c)
%%%%(c)  See the file COPYING.txt for copying conditions
%%%%(c)
%%%%(c)
%%%%(c)
%%%%(c)  This file is a portion of the source for the textbook
%%%%(c)
%%%%(c)    A First Course in Linear Algebra
%%%%(c)    Copyright 2004 by Robert A. Beezer
%%%%(c)
%%%%(c)  See the file COPYING.txt for copying conditions
%%%%(c)
%%%%(c)
If the first system is homogeneous, then the zero vector is in the solution set of the system.  (See the proof of \acronymref{theorem}{HSC}.)\par
%
Since the two systems are equivalent, they have equal solutions sets (\acronymref{definition}{ESYS}).  So the zero vector is in the solution set of the second system.  By \acronymref{exercise}{HSE.T10} the presence of the zero vector in the solution set implies that the system is homogeneous.\par
%
So if any one of two equivalent systems is homogeneous, then they both are homogeneous.