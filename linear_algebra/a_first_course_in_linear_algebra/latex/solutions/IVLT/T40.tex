%%%%(c)
%%%%(c)  This file is a portion of the source for the textbook
%%%%(c)
%%%%(c)    A First Course in Linear Algebra
%%%%(c)    Copyright 2004 by Robert A. Beezer
%%%%(c)
%%%%(c)  See the file COPYING.txt for copying conditions
%%%%(c)
%%%%(c)
Since $\compose{S}{T}$ is invertible, by \acronymref{theorem}{ILTIS} $\compose{S}{T}$ is injective and therefore has a trivial kernel by \acronymref{theorem}{KILT}.  Then
%
\begin{align*}
\krn{T}
&\subseteq\krn{\compose{S}{T}}&&\text{\acronymref{exercise}{ILT.T15}}\\
&=\set{\zerovector}&&\text{\acronymref{theorem}{KILT}}
\end{align*}
%
Since $T$ has a trivial kernel, by \acronymref{theorem}{KILT}, $T$ is injective.  Also,
%
\begin{align*}
\rank{T}
&=\dimension{U}-\nullity{T}&&\text{\acronymref{theorem}{RPNDD}}\\
&=\dimension{U}-0&&\text{\acronymref{theorem}{NOILT}}\\
&=\dimension{V}&&\text{Hypothesis}
\end{align*}
%
Since $\rng{T}\subseteq V$, \acronymref{theorem}{EDYES} gives $\rng{T}=V$, so by \acronymref{theorem}{RSLT}, $T$ is surjective.  Finally, by \acronymref{theorem}{ILTIS}, $T$ is invertible.\par
%
Since $\compose{S}{T}$ is invertible, by \acronymref{theorem}{ILTIS} $\compose{S}{T}$ is surjective and therefore has a full range by \acronymref{theorem}{RSLT}.  Then
%
\begin{align*}
W
&=\rng{\compose{S}{T}}&&\text{\acronymref{theorem}{RSLT}}\\
&\subseteq\rng{S}&&\text{\acronymref{exercise}{SLT.T15}}
\end{align*}
%
Since $\rng{S}\subseteq W$ we have $\rng{S}=W$ and by \acronymref{theorem}{RSLT}, $S$ is surjective.  By an application of \acronymref{theorem}{RPNDD} similar to the first part of this solution, we see that $S$ has a trivial kernel, is therefore injective (\acronymref{theorem}{KILT}), and thus invertible (\acronymref{theorem}{ILTIS}).