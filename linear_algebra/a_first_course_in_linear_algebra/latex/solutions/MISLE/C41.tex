%%%%(c)
%%%%(c)  This file is a portion of the source for the textbook
%%%%(c)
%%%%(c)    A First Course in Linear Algebra
%%%%(c)    Copyright 2004 by Robert A. Beezer
%%%%(c)
%%%%(c)  See the file COPYING.txt for copying conditions
%%%%(c)
%%%%(c)
The coefficient matrix of this system of equations is
%
\begin{equation*}
A=
\begin{bmatrix}
 1 & 2 & -1 \\
 2 & 5 & -1 \\
 -1 & -4 & 0
\end{bmatrix}
\end{equation*}
%
and the vector of constants is $\vect{b}=\colvector{-3\\-4\\2}$.  So by \acronymref{theorem}{SLEMM} we can convert the system to the form $A\vect{x}=\vect{b}$.   Row-reducing this matrix yields the identity matrix so by \acronymref{theorem}{NMRRI} we know $A$ is nonsingular.  This allows us to apply \acronymref{theorem}{SNCM} to find the unique solution as 
%
\begin{equation*}
\vect{x}
=
\inverse{A}\vect{b}
=
\begin{bmatrix}
 -4 & 4 & 3 \\
 1 & -1 & -1 \\
 -3 & 2 & 1
\end{bmatrix}
\colvector{-3\\-4\\2}
=
\colvector{2\\-1\\3}
%
\end{equation*}
%
Remember, you can check this solution easily by evaluating the matrix-vector product  $A\vect{x}$ (\acronymref{definition}{MVP}).