%%%%(c)
%%%%(c)  This file is a portion of the source for the textbook
%%%%(c)
%%%%(c)    A First Course in Linear Algebra
%%%%(c)    Copyright 2004 by Robert A. Beezer
%%%%(c)
%%%%(c)  See the file COPYING.txt for copying conditions
%%%%(c)
%%%%(c)
The derivative of $p(x) = a + bx + cx^2$ is $p^\prime(x) = b + 2cx$.  
Thus, if $p \in R$, then $p^\prime(0) = b + 2c(0) = 0$, 
so we must have $b = 0$.  We see that we can rewrite $R$ as 
$R = \setparts{p(x) = a + cx^2}{a, c\in\complexes}$.
A linearly independent set that spans $R$ is $B = \set{1,x^2}$, and $B$ is a basis of $R$. 


