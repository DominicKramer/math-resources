%%%%(c)
%%%%(c)  This file is a portion of the source for the textbook
%%%%(c)
%%%%(c)    Abstract Algebra: Theory and Applications
%%%%(c)    by Thomas W. Judson
%%%%(c)
%%%%(c)    Sage Material
%%%%(c)    Copyright 2011 by Robert A. Beezer
%%%%(c)
%%%%(c)  See the file COPYING.txt for copying conditions
%%%%(c)
%%%%(c)
These exercises are about investigating basic properties of the integers, something we will frequently do when investigating groups.  Use the editing capabilities of a Sage worksheet to annotate and explain your work.
\begin{sageverbatim}\end{sageverbatim}
%
\sageexercise{1}%
Use the \verb?next_prime()? command to construct two different 8-digit prime numbers.
\begin{sageverbatim}\end{sageverbatim}
%
\sageexercise{2}%
Use the \verb?.is_prime()? method to verify that your primes are really prime.
\begin{sageverbatim}\end{sageverbatim}
%
\sageexercise{3}%
Verify that the greatest common divisor of your two primes is 1.
\begin{sageverbatim}\end{sageverbatim}
%
\sageexercise{4}%
Find two integers that make a ``linear combination'' of your primes equal to 1.  Include a verification of your result.
\begin{sageverbatim}\end{sageverbatim}
%
\sageexercise{5}%
Determine a factorization into powers of primes for $b=4\,598\,037\,234$.
\begin{sageverbatim}\end{sageverbatim}
%
\sageexercise{6}%
Write statements that show that $b$ is (i) divisible by 7, (ii) not divisible by 11.  Your statements should simply return \verb?True? or \verb?False? and be sufficiently general that a different value of $b$ or different candidate prime divisors could be easily used by making just one substitution for each changed value.
\begin{sageverbatim}\end{sageverbatim}
%
