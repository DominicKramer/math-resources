%%%%(c)
%%%%(c)  This file is a portion of the source for the textbook
%%%%(c)
%%%%(c)    Abstract Algebra: Theory and Applications
%%%%(c)    by Thomas W. Judson
%%%%(c)
%%%%(c)    Sage Material
%%%%(c)    Copyright 2011 by Robert A. Beezer
%%%%(c)
%%%%(c)  See the file COPYING.txt for copying conditions
%%%%(c)
%%%%(c)
The following exercises are less about cosets and subgroups, and more about using Sage as an experimental tool.  They are designed to help you become both more efficient, and more expressive, as you write commands in Sage.  We will have many opportunities to work with cosets and subgroups in the coming chapters.\par
%
These exercises do not contain much guidance, and get more challenging as they go.  They are designed to explore, or confirm, results presented in this chapter.  You should answer each one with a single (complicated) line of Sage that concludes by outputting \verb?True?.\par
%
When you check integers below for divisibility, recognize that \verb?range()? produces plain integers, which are quite simple in their functionality.  However, the \verb?srange()? command produces Sage integers, which have many more capabilities.  (See the last exercise for an example.)
\begin{sageverbatim}\end{sageverbatim}
%
\sageexercise{1}%
Use \verb?.subgroups()? to find an example of a group $G$ and an integer $m$, so that (a) $m$ divides the order of $G$, and (b) $G$ has no subgroup of order $m$.  (Do not use the group $A_4$ for $G$, since this is in the text.)  Provide a single line of Sage code that has all the logic to produce the desired $m$ as its output.  Here is a very simple example that might help you structure your answer.
%
\begin{sageexample}
sage: a = 5
sage: b = 10
sage: c = 6
sage: d = 13
sage: a.divides(b)
True
sage: not (b in [c,d])
True
sage: a.divides(b) and not (b in [c,d])
True
\end{sageexample}
%
\begin{sageverbatim}\end{sageverbatim}
%
\sageexercise{2}%
Verify the truth of Fermat's Little Theorem (either variant) for your own choice of a single number for the base $a$ (or $b$), and for $p$ assuming the value of every prime number between $100$ and $1000$.\par
%
Build up a solution slowly --- make a list of powers (start with just a few primes), then make a list of powers reduced by modular arithmetic, then a list of comparisons with  the predicted value, then a check on all these logical values resulting from the comparisons.  This is a useful strategy for many similar problems.  Eventually you will write a single line that performs the verification by eventually printing out \verb?True?.  Here are some more hints about useful functions.
%
\begin{sageexample}
sage: a = 20
sage: b = 6
sage: a.mod(b)
2
sage: prime_range(50, 100)
[53, 59, 61, 67, 71, 73, 79, 83, 89, 97]
sage: all([True, True, True, True])
True
sage: all([True, True, False, True])
False
\end{sageexample}
%
\begin{sageverbatim}\end{sageverbatim}
%
\sageexercise{3}%
Verify that the group of units mod $n$ has order $n-1$ when $n$ is prime, again for all primes between $100$ and $1000$.  As before, your output should be simply \verb?True?, just once indicating that the statement about the order is true for all the primes examined.  As before, build up your solution slowly, and with a smaller range of primes in the beginning.  Express your answer as a single line of Sage code.
\begin{sageverbatim}\end{sageverbatim}
%
\sageexercise{4}%
Verify Euler's Theorem for all values of $0<n<100$ and for $1\leq a \leq n$.  This will require nested \verb?for? statements with a conditional.  Again, here's a small example that might be helpful for constructing your one-line of Sage code.  Note the use of \verb?srange()? in this example.
%
\begin{sageexample}
sage: [a/b for a in srange(9) for b in srange(1,a) if gcd(a,b)==1]
[2, 3, 3/2, 4, 4/3, 5, 5/2, 5/3, 5/4, 6, 6/5,
 7, 7/2, 7/3, 7/4, 7/5, 7/6, 8, 8/3, 8/5, 8/7]
\end{sageexample}
%
\begin{sageverbatim}\end{sageverbatim}
%

